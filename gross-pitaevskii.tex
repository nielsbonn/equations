\documentclass[11pt,a4paper,draft,DIV11]{scrartcl}	%twoside
%\usepackage[a4paper,hmargin=2.8cm,vmargin=2.8cm]{geometry}
\usepackage[utf8x]{inputenc}
\usepackage[bookmarksnumbered=true]{hyperref}
\usepackage[color]{showkeys}
\usepackage{scrtime}
\usepackage{amsmath,amsthm,amsfonts,amssymb}
\usepackage{dsfont,wasysym,mathtools}
\usepackage{gitinfo}
\usepackage{nccmath}

% Footer:
\usepackage{fancyhdr}
\pagestyle{fancy}
\fancyhf{}
\renewcommand{\headrulewidth}{0pt}
\rfoot{Page \thepage}
\lfoot{Commit \gitAbbrevHash \ on \gitCommitterIsoDate \ (\jobname.tex)}

% Set enumerate environment using roman counters
\renewcommand{\theenumi}{{\rm (\roman{enumi})}}
\renewcommand{\labelenumi}{\theenumi}

% Nicer marginpar with smaller font
\let\oldmarginpar\marginpar
\renewcommand\marginpar[1]{\-\oldmarginpar[\raggedleft\footnotesize #1]%
  {\raggedright\footnotesize #1}}

%%%%%%%%%%%%%%%%%%%%%%%%% Theorems etc %%%%%%%%%%%%%%%%%%%%%%%%%%%%%%%%%%%%%%%%%

\newtheorem{thm}{Theorem}[section]
\newtheorem{cor}[thm]{Corollary}
\newtheorem{prp}[thm]{Proposition}
\newtheorem{lem}[thm]{Lemma}
\newtheorem{dfn}[thm]{Definition}
\newtheorem{exm}[thm]{Example}

\newtheorem*{rem}{Remark}
\newtheorem*{rems}{Remarks}
\newtheorem*{hyp}{Hypothesis}

%%%%%%%%%%%%%%%%%%%%%%%% Gustavo's aliases %%%%%%%%%%%%%%%%%%%%%%%%%%%%%%%%%%%%%

\newcommand{\R}{\mathds{R}}
\newcommand{\N}{\mathcal{N}}
\newcommand{\K}{\mathcal{K}}

%%%%%%%%%%%%%%%%%%%%%%%% Notation %%%%%%%%%%%%%%%%%%%%%%%%%%%%%%%%%%%%%%%%%%%%%%

\newcommand{\ad}{\operatorname{ad}}	% abbreviation for iterated commutators
\newcommand{\fock}{\mathcal{F}}		% fock space symbol
\newcommand{\di}{\textrm{d}}		% differential (for integrals)
\newcommand{\Lcal}{\mathcal{L}}		% calligraphic L
\newcommand{\Ncal}{\mathcal{N}}		% calligraphic N
\newcommand{\Kcal}{\mathcal{K}}		% calligraphic N
\newcommand{\Vcal}{\mathcal{V}}		% calligraphic V
\newcommand{\Hcal}{\mathcal{H}}		% calligraphic H
\newcommand{\Ocal}{\mathcal{O}}		% big-O, order-of
\newcommand{\tilV}{\tilde{\mathcal{V}}_N}		% symbol for the smeared potential energy part of the hamiltonian
\newcommand{\tilK}{\tilde{\mathcal{K}}}		% smeared kinetic energy
% estlist
\newcommand{\estlist}[2]{\emph{\vspace{.3em}\\Line \ref{l#1}, summand #2:}}
\newcommand{\nestlist}[2]{line \ref{l#1}, summand #2}
\newcommand{\Nestlist}[2]{Line \ref{l#1}, summand #2}
%
\newcommand{\hc}{\mbox{h.c.}}		%hermitian conjugate
\newcommand{\scal}[2]{\big<#1,#2\big>} % scalaer product
\newcommand{\cc}[1]{\overline{#1}}	% complex conjugate
\newcommand{\Rbb}{\mathbb{R}}		% real numbers
\newcommand{\Cbb}{\mathbb{C}}		% complex numbers
\newcommand{\Nbb}{\mathbb{N}}		% natural numbers
\renewcommand{\Re}{\operatorname{Re}\,} 	%RealPart
\renewcommand{\Im}{\operatorname{Im}\,} 	%ImaginaryPart
\newcommand{\norm}[1]{\lVert#1\rVert}	%Norm
\newcommand{\ev}[1]{\big<#1\big>}	%expectation value
\newcommand{\ph}{\varphi_t^{(N)}}	% solution of N-dependent Hartree equation
\newcommand{\phdot}{\dot{\varphi}_t^{(N)}}	% time derivative of solution of N-dependent Hartree equation
\newcommand{\phddot}{\ddot{\varphi}_t^{(N)}}	% second time derivative of solution of N-de Hartree equaution
\newcommand{\sqn}{\sqrt{N}}		% square root of N
\newcommand{\project}[1]{\lvert #1 \big>\big< #1\rvert}	% orthogonal projection operator
\newcommand{\Tr}{\operatorname{Tr}}	% Trace
\newcommand{\dxyNV}{\frac{1}{2}\int \di x\di y N V_N(x-y)} % abbreviation for the one-half integral dx dy NV_N(x-y) which appears everywhere
\newcommand{\dxyV}{\frac{1}{2}\int \di x\di y V_N(x-y)} % abbreviation for the one-half integral dx dy V_N(x-y) which appears everywhere
\newcommand{\tLcal}{\tilde \Lcal_N(t)}
\newcommand{\tLcalo}{\tilde \Lcal_N(0)}
\newcommand{\xn}{\mathbf{x}}		% x = (x_1. dots x_N)
\newcommand{\gradone}{\nabla_2}
\newcommand{\HS}{_{\textrm{HS}}}

\newcommand{\be}[1]{\begin{equation}\label{eq:#1}}	%begin equation with label
\newcommand{\ee}{\end{equation}}
\newcommand{\bd}{\begin{displaymath}}			% abbreviation begin displaymath
\newcommand{\ed}{\end{displaymath}}

\newcommand{\todo}[1]{}

\newcommand{\tagg}[1]{ \stepcounter{equation} \tag{\theequation} \label{eq:#1} } % add tag and label in align*-environments

\newcommand{\eqr}[1]{\eqref{eq:#1}}			%eqref with prefix :eq

%%%%%%%%%%%%%%%%%%%%%%%%% main content %%%%%%%%%%%%%%%%%%%%%%%%%%%%%%%


%\allowdisplaybreaks

\author{Institute of Applied Mathematics, University of Bonn\\ Endenicher Allee 60, 53115 Bonn, Germany}
\title{Rate of Convergence towards the Gross-Pitaevskii limit of interacting Bose-Einstein condensate}

\begin{document}
\maketitle

\begin{abstract}
\textbf{\sffamily Abstract.} The non-linear Gross-Pitaevskii equation describes the macroscopic dynamics of Bose-Einstein condensates in the limit of large number of particles $N$. In this paper we show that the microscopic quantum mechanical evolution converges towards the Gross-Pitaevskii dynamics if the two-particle interaction potential is scaled as $N^3 V(N(x_i-x_j))$. 
Our method is based on using a time-dependent Bogoliubov transformation to implement a short-scale structure on coherent initial states, similar to squeezed coherent states known from quantum optics. The Gross-Pitaevskii limit can then be treated as a mean-field limit. We obtain explicit estimates on the rate of convergence of $k$-particle reduced density matrices.
\end{abstract}

\section{Introduction and Main Result}
\label{s:intro}
We consider a system of $N$ identical bosons with repulsive interaction. On the Hilbert space $L^2_s(\Rbb^{3N})$ of wave functions symmetric w.\,r.\,t.\ permutation of the $N$ variables this system is described by the Hamiltonian
\bd
H_N = \sum_{j=1}^N (-\Delta_{x_j}) + \frac{1}{N}\sum_{i<j}^N N^3 V(N(x_i-x_j)).
\ed
We assume that the unscaled potential $V \in C^\infty_c(\Rbb^3)$ is \marginpar{for us, $V \in C_c$ seems ok. What about scatt. equ. properties?}
non-negative and spherically symmetric. The scaling is chosen in such a way that for $N \to \infty$ formally $N^3 V(N(x_i-x_j)) \to \delta(x_i-x_j) b_0$, where $b_0 = \int V$.
We use the short-hand $V_N(x) = N^2 V(Nx)$.

We are interested in the time evolution of factorized initial wave functions, which we see as a model for a Bose-Einstein condensate in a trap:
\bd
\psi_N(x_1,\ldots x_N) = \prod_{j=1}^N \varphi(x_j),
\ed
where $\varphi \in L^2(\Rbb^3)$ is a one-particle wave function normalized to $\norm{\varphi}_{L^2} = 1$. The time evolution is given by the solution to the Schr\"odinger equation 
\bd
i \partial_t \psi_{N,t} = H_N \psi_{N,t}\quad \mbox{with inital data } \psi_{N,0} = \psi_N.
\ed

The function $f_N = 1 - w_N: \Rbb^3 \to \Rbb$ is the solution to the zero-energy scattering equation
\begin{equation}
\label{eq:scatt}
\left(-\Delta + \frac{1}{2}V_N \right) f_N = 0
\end{equation}
with boundary condition $\lim_{\lvert x\rvert \to \infty} f_N(x) = 1$. The scattering length $a_0$ of the potential $V$ is given by
\bd
8\pi a_0 = \int \di x NV_N(x)f_N(x) = \int \di x V(x)f(x),
\ed
where we write $f$ for $f_N$ with $N=1$.

We expect that in the limiting case of large $N$, the dynamics of the
Bose-Einstein condensate is well-described by
\be{approx_dyn}
\psi_{N,t}(x_1,\ldots x_N) \simeq \prod_{j=1}^N \varphi_t(x_j),
\ee
where $\varphi_t$ is the solution to the Gross-Pitaevskii equation
\[
i \partial_t \varphi_t = - \Delta \varphi_t + 8\pi a_0 \lvert \varphi_t\rvert^2 \varphi_t\quad \mbox{with initial data } \varphi_0 = \varphi.
\]

To make precise in which sense \eqr{approx_dyn} holds we have to introduce reduced density matrices, also called marginal densities.
%Let us write
%\[
%\gamma_{N,t}(x_1, \ldots x_N; y_1, \ldots y_N) = \psi_{N,t}(x_1,\ldots x_N) \cc{\psi_{N,t}(y_1,\ldots y_N)}.
%\]
The $k$-particle reduced density matrix associated with a state $\psi_{N,t} \in L^2_s(\Rbb^{3N})$ is the positive trace-class operator on $L^2(\Rbb^{3k})$ given by the integral kernel
\be{red_density}
\begin{split}
& \gamma_{N,t}^{(k)}(x_1,\ldots x_k;y_1,\ldots y_k) \\
&  := \int \di x_{k+1} \cdots \di x_N\,\psi_{N,t}(x_1,\ldots x_k,x_{k+1},\ldots x_N) \cc{\psi_{N,t}(y_1,\ldots y_k,x_{k+1},\ldots x_N)}.
\end{split}
\ee
The integration corresponds to taking the partial trace or `tracing out' $N-k$ particles. It is obvious that $\Tr \gamma_{N,t}^{(k)} = 1$ if $\norm{\psi_{N,t}} = 1$.

We denote the projection on the state $\varphi_t \in L^2(\Rbb^3)$ by the Dirac bra-ket notation $\project{\varphi_t}$.

We can now formulate a theorem estimating the difference of the expectation values of one-particle observables w.\,r.\,t.\ the Schr\"odinger evolution and the effective evolution.

\begin{thm}[Time Evolution of Factorized States] \label{thm:main_factorized}
 Let $\psi_N(x_1, \ldots x_N) = \prod_{j=1}^N \varphi(x_j)$ for some $\varphi \in H^4(\Rbb^3)$ with $\norm{\varphi}_{L^2} = 1$. Denote by $\psi_{N,t} = e^{-it H_N}\psi_N$ the solution to the Schr\"odinger equation with initial data $\psi_{N,0}= \psi_N$, and let $\gamma_{N,t}^{(1)}$ be the one-particle reduced density matrix associated with $\psi_{N,t}$.
Let $\varphi_t$ be the solution to the Gross-Pitaevskii equation with initial data $\varphi_0 = \varphi$.

 Then there exist constants $C$, $K_1$ and $K_2$, which do not depend on $N$
 and $t$, such that for any Hilbert-Schmidt operator $J$ on $L^2(\Rbb^3)$ (a one-particle observable)
\bd
\left\lvert \Tr\Big(J \gamma_{N,t}^{(1)} \Big) - \Tr\Big( J \project{\varphi_t}  \Big) \right\rvert \leq \norm{J}\HS \frac{C}{N^{1/4}}e^{K_1 e^{K_2 t}}.
\ed
\end{thm}

\begin{rems}
\begin{enumerate}
\item It is a general fact \cite{RS2009} that $\Tr \big\lvert \gamma_{N}^{(1)} - \project{\varphi} \big\rvert \leq 2 \norm{\gamma_N^{(1)} - \project{\varphi}}\HS$ for any reduced density matrix $\gamma^{(1)}_N$. As the Hilbert-Schmidt operators form a Hilbert space with scalar product $\scal{A}{B}\HS = \Tr A^\ast B$ we have 
\[\norm{\gamma_N^{(1)} - \project{\varphi}}\HS \leq \sup_{\norm{J}\HS=1} \Tr \left( J \big( \gamma^{(1)}- \project{\varphi} \big) \right).\]
 Thus we have actually proven convergence in the trace norm,
\bd
\Tr \left\lvert \gamma_{N}^{(1)} - \project{\varphi} \right\rvert \leq \frac{C}{N^{1/4}}e^{K_1 e^{K_2 t}}.
\ed
\item It is a general fact \cite{KP2009} that a bound on the convergence of the one-particle reduced density matrix implies a bound on the convergence of the $k$-particle reduced density matrix. We thus obtain
\bd
\Tr \left\lvert \gamma_{N}^{(k)} - \project{\varphi}^{\otimes k} \right\rvert \leq C\sqrt{ \frac{k}{ N^{1/4} } }e^{K_1 e^{K_2 t}/2}.
\ed
\item In the second-quantized setting we are going to work in it is more natural to consider the time evolution of squeezed coherent states instead of factorized states. For squeezed coherent states a better rate of convergence can be obtained, see Theorem \ref{thm:main_squeezed}.
\end{enumerate}
\end{rems}

\emph{Heuristics of our approach:} 
The method of our proof is based on the study of the evolution of coherent
states. It is known \cite{RS2009, CLS2011} that for solutions to the Schr\"odinger equation with Hamiltonian with potential $U^2 \apprle (1-\Delta)$,
\[
H_N = \sum_{j=1}^N (-\Delta_{x_j}) + \frac{1}{N}\sum_{i<j}^N U(x_i-x_j),
\]
and with factorized initial data, the reduced density matrices convergence in trace-norm to a solution of the Hartree equation
\[
i \partial_t \varphi_t = -\Delta \varphi_t + \left(U \ast \lvert \varphi_t\rvert^2 \right) \varphi_t,
\]
that is
\[
\Tr \left\lvert \gamma_{N,t}^{(1)} - \project{\varphi_t} \right\rvert \leq \frac{C}{N}e^{Kt}.
\]
Notice that for $U = N V_N \to b_0
\delta$ we obtain a Gross-Pitaevskii equation with coupling constant $b_0$.
However, from \cite{ESY2010} it is known that the correct coupling constant
is $8 \pi a_0$ and not $b_0$. (Notice that the result in that
paper is based on a compactness argument and does not give any information
about the rate of convergence.) This behaviour has its origin in a singular short-scale structure of the $N$-body
quantum state $\psi_{N,t}$. The short-scale structure is given \cite{EMS2009} by the solution to the two-particle scattering equation \eqref{eq:scatt}.

Our strategy now is to compare the evolution of coherent states with the effective evolution given by the modified Hartree equation
\be{modHartreeEqu}
i \partial_t \ph = -\Delta \ph + \left(N f_N V_N \ast \lvert \ph\rvert^2 \right) \ph,
\ee
which has the correct limiting equation.
Using a time-dependent Bogoliubov transformation we can implement the short-scale structure in the coherent states. This corresponds to the use of squeezed coherent states known from quantum optics \cite{Loudon}. Mathematically, the Bogoliubov transformation explicitly exhibits cancellations in the generator of fluctuation dynamics around the effective evolution.
We complete the proof by showing that solutions of the modified Hartree equation converge to solutions of the Gross-Pitaevskii equation with coupling constant $8\pi a_0$.
\newline

\emph{Overview of the field}: References to BBGKY, Grillakis-Machedon-Margetis \cite{GMM2010}, Pickl. Comparison of scaling. TODO.\newline

\emph{Organization of the paper:}
In section \ref{s:fock} we explain how the $N$-particle system can be embedded into Fock space so that coherent states can be used.

In section \ref{s:bogoliubov} we introduce the Bogoliubov transformation.

In section \ref{s:fluctuations} we introduce the dynamics of fluctuations around the effective evolution. Subsection \ref{ss:generator} is dedicated to the calculation of the generator of fluctuation dynamics, where the Bogoliubov transform is used to explicitly exhibit cancellations. Subsection \ref{ss:number} explains how the number of fluctuations can be bounded and subsection \ref{ss:apriori} states some weak a-priori estimates.

Section \ref{s:expectation} is dedicated to estimating expectation values $\Tr\big(J \gamma_{N,t}^{(1)}\big)$ in terms of the number of fluctuations; then the main theorem and a similar theorem on the evolution of squeezed coherent states is proven.

Appendix \ref{s:generatorestimates} lists estimates for the summand of the generator of fluctuations. Appendix \ref{s:pde} is dedicated to proving uniform in $N$ regularity estimates for solutions of the Hartree and Gross-Pitaevskii equation and shows that solutions of the Hartree equation converge to solutions of the Gross-Pitaevskii equation. Appendix \ref{s:bogbounds} lists some further estimates on the integral kernels appearing in the Bogoliubov transformation.
\newline

\emph{Notation:} We use the symbol $C$ throughout the paper for constants which might be redefined from line to line, but which are always independent of $N$ and of time $t$. We sometimes comment explicitly on additional properties of constants $C$.
The notation
\[
  A \apprle B
\]
means that $A \le C B$ for some constant $C$.

Notice that we use $\sup_x$ as a short-hand for $\operatorname{ess\,sup}_{x\in \Rbb^3}$.

In accordance with \eqr{no1} we have $\norm{p_x}_{L^2} = \left( \int \di y\, \lvert p(y,x\rvert^2 \right)^{1/2}$. It should be clear from the context where $\norm{\cdot}_{L^2}$ means the $L^2(\Rbb^3)$ or $L^2(\Rbb^3 \times \Rbb^3)$ norm, similarly it should be clear which norm is meant by $\norm{\cdot}$ (usually Fock space or one of the $L^2$-norms).
For any integral kernel $p$ the gradient with respect to the second variable is denoted by $\nabla_2 p$.

Notice also that though the integral kernels appearing in our calculation (like $k$, $s$, $r$ and $p$, see below) may depend on $t$ through $\ph$, we do not denote their $t$-dependence explicitly. Some lemmata about the integral kernels are formulated for general $\varphi$ where the modified Hartree equation does not play any role.

We commonly denote Hilbert-Schmidt operators and their integral kernels by the same letter.

\section{Fock Space Representation}
\label{s:fock}
The bosonic Fock space over $L^2(\R^3)$ is the Hilbert space
\[
  \mathcal{F} = \bigoplus_{n \ge 0} L^2(\R^3)^{\otimes_s n} = \mathds{C}
  \oplus \bigoplus_{n \ge 1} L^2_s(\R^{3n}),
\]
with the convention that $L^2(\R^3)^{\otimes_s 0} = \mathds{C}$. Here
$L^2_s(\R^{3n})$ is the subspace of $L^2(\R^{3n})$ consisting of all functions
that are symmetric with respect to arbitrary permutations of the $n$
variables. Vectors in $\mathcal{F}$ are sequences $\psi = (\psi^{(n)})_{n
\ge 0}$ of $n$-particle wave functions $\psi^{(n)} \in L^2_s(\R^{3n})$. The
inner product on $\mathcal{F}$ is defined as
\begin{align*}
  \langle \psi_1, \psi_2 \rangle & = \sum_{n \ge 0} \langle \psi_1^{(n)},
  \psi_2^{(n)} \rangle_{L^2(\R^{3n})} \\
  & = \overline{\psi_1^{(0)}} \psi_2^{(0)} + \sum_{n \ge 1} \int dx_1 \cdots
  dx_n \overline{\psi_1^{(n)}}(x_1, \dots, x_n) \psi_2^{(n)}(x_1, \dots, x_n).
\end{align*}
An $N$-particle state with wave function $\psi_N$ is described on
$\mathcal{F}$ by a sequence $(\psi^{(n)})_{n \ge 0}$ where $\psi^{(n)} =
0$ for all $n \neq N$ and $\psi^{(N)} = \psi_N$. The vector $(1, 0, 0, \dots
) \in \mathcal{F}$ is called the vacuum, and is denoted by $\Omega$.


For $f \in L^2(\R^3)$, the creation operator $a^*(f)$ and the annihilation
operator $a(f)$ on $\mathcal{F}$ are defined as
\[
  \begin{split}
    (a^*(f) \psi)^{(n)}(x_1, \dots, x_n) & = \frac{1}{\sqrt{n}} \sum_{j=1}^n
    f(x_j) \psi^{(n-1)}(x_1, \dots, x_{j-1}, x_{j+1}, \dots, x_n), \\
    (a(f) \psi)^{(n)}(x_1, \dots, x_n) & = \sqrt{n+1} \int dx \,
    \overline{f(x)} \psi^{(n+1)}(x, x_1, \dots, x_n).
  \end{split}
\]
The operators $a^*(f)$ and $a(f)$ are unbounded, densely defined and closed.
Note that $a^*(f)$ is linear in $f$, while $a(f)$ is anti-linear. The creation
operator $a^*(f)$ is the adjoint of the annihilation operator $a(f)$, and
they satisfy the canonical commutation relations
\[
  [a(f), a^*(g)] = \langle f, g \rangle_{L^2} \qquad \text{and} \qquad [a(f),
  a(g)] = [a^*(f), a^*(g)] = 0
\]
for $f,g \in L^2(\R^3)$. It is useful to introduce the self-adjoint operator
\[
  \phi(f) = a^*(f) + a(f).
\]


We will make use of operator-valued distributions $a_x^*$ and $a_x$, with $x
\in \R^3$, defined so that
\[
  a^*(f) = \int \di x\, f(x) a^*_x \qquad \text{and} \qquad a(f) = \int \di
  x\, \cc{f(x)} a_x
\]
for $f \in L^2(\R^3)$. For these distributions, the canonical commutation
relations assume the form
\[
  [a_x, a_y^*] = \delta(x-y) \qquad \text{and} \qquad [a_x, a_y] = [a_x^*,
  a_y^*] = 0,
\]
where $\delta$ is the Dirac delta distribution.


The number of particles operator $\mathcal{N}$ on $\mathcal{F}$ is defined as
$(\N \psi)^{(n)} = n \psi^{(n)}$. Eigenvectors of $\N$ are vectors of the form
$(0, \dots, 0, \psi^{(m)}, 0, \dots )$ with a fixed number of particles.


The Hamiltonian $\Hcal_N$ on $\mathcal{F}$ is defined as $(\Hcal_N
\psi)^{(n)} = \Hcal_N^{(n)} \psi^{(n)}$ with
\todo{Double sum inconsistent with single sum in other places.}
\[
  \Hcal_N^{(n)} = \sum_{j=1}^n (-\Delta_{x_j}) + \sum_{j=1}^n
  \sum_{i=1}^{j-1} V_N(x_i - x_j).
\]
By definition, the Hamiltonian $\Hcal_N$ leaves sectors of $\fock$ with a
fixed number of particles invariant, and $\Hcal_N$ commutes with $\N$.
Clearly $\Hcal_N^{(N)} = H_N$, therefore
\[
e^{-it \Hcal_N} (0,\dots,0, \psi^{(N)},0,\dots) = (0,\dots,0,
e^{-itH_N}\psi^{(N)},0,\dots).
\]
Thus we have embedded the time evolution of the $N$ boson system in Fock space.

Using the distributions $a_x^*$ and $a_x$, we can express the operators $\N$
and $\Hcal_N$ as
\[
  \N = \int dx \, a_x^* a_x \quad \mbox{and} \quad
  \Hcal_N = \K + \mathcal{V}_N
\]
with
\[
  \K = \int dx \, \nabla_x a_x^* \nabla_x a_x \qquad \text{and} \qquad
  \mathcal{V}_N = \frac{1}{2} \int dx dy \, V_N(x-y) a_x^* a_y^* a_y a_x.
\]
We call $\K$ the kinetic energy operator, and $\mathcal{V}_N$ the potential
energy operator.

We can extend the definition of one-particle reduced density matrices to states $\psi \in \fock$ by defining its kernel as
\be{fock_density}
\gamma_{\psi}^{(1)}(x;y) := \frac{1}{\scal{\psi}{\Ncal \psi}} \scal{\psi}{a^\ast_y a_x \psi}.
\ee
For $\psi$ in the $N$-particle subspace of $\fock$ this definition is equivalent to definition \eqr{red_density}.

The following standard lemma provides some useful bounds to control creation and
annihilation operators in terms of the number of particles operator and the
kinetic energy operator.


\begin{lem} \label{l:a}
  Let $f \in L^2(\R^3)$. Then, for any $\psi \in
  \mathcal{F}$,
  \begin{equation} \label{aNorm}
    \begin{aligned}
      \norm{a(f)\psi} & \leq \norm{f}_{L^2} \norm{\Ncal^{1/2}\psi}, \\
      \norm{a^*(f)\psi} & \leq \norm{f}_{L^2} \norm{(\Ncal+1)^{1/2}\psi}, \\
      \norm{\phi(f) \psi} & \leq 2 \norm{f}_{L^2} \norm{(\N+1)^{1/2} \psi}.
    \end{aligned}
  \end{equation}
\end{lem}

For $f \in L^2(\R^3)$, we define the Weyl operator on $\mathcal{F}$ as
\[
  W(f) = e^{a^*(f) - a(f)}.
\]
A state of the form $W(f)\Omega$ is called a coherent state.


In the following lemma we collect some well-known properties of Weyl operators
and coherent states.


\begin{lem}[Weyl Operators and Coherent States] \label{l:W}
  Let $f, g \in L^2(\R^3)$.
  \begin{enumerate}
    \item \label{l:W1} The Weyl operator satisfies the relations
      \[
        W(f) W(g) = W(g) W(f) e^{-2i \Im \langle f, g \rangle_{L^2}} = W(f+g)
        e^{-i \Im \langle f, g \rangle_{L^2}}.
      \]
    \item \label{l:W2} The operator $W(f)$ is unitary on $\mathcal{F}$ and
      \[
        W(f)^* = W(f)^{-1} = W(-f).
      \]
    \item \label{l:W3} We have
      \[
        W(f)^* a_x W(f) = a_x + f(x) \qquad \text{and} \qquad W(f)^* a_x^*
        W(f) = a_x^* + \overline{f(x)}.
      \]
    \item \label{l:W4} Coherent states are
      eigenvectors of annihilation operators:
      \[
        a_x W(f)\Omega = f(x) W(f)\Omega \qquad \text{and} \qquad a(g) W(f)\Omega
        = \langle g, f \rangle_{L^2} W(f)\Omega.
      \]
\iffalse    \item \label{l:W5} The expectation of the number of particles in the
      coherent state $\psi(f)$ is given by $\| f \|_{L^2}^2$, that is,
      \[
        \langle \psi(f), \N^2 \psi(f) \rangle = \| f \|_{L^2}^2.
      \]
      Also the variance of the number of particles in $\psi(f)$ is given by
      $\| f \|_{L^2}$ (the distribution of $\N$ is poisson), that is,
      \[
        \langle \psi(f), \N^2 \psi(f) \rangle - \langle \psi(f), \N \psi(f)
        \rangle^2 = \| f \|_{L^2}^2.
      \]
    \item \label{l:W6} Coherent states are normalized but not orthogonal to
      each other. In fact,
      \[
        \langle \psi(f), \psi(g) \rangle = e^{-\frac{1}{2} (\| f \|_{L^2}^2 +
        \| g \|_{L^2}^2 - 2 \langle f, g \rangle_{L^2} )} \qquad \text{so
        that} \qquad |\langle \psi(f), \psi(g) \rangle| = e^{-\frac{1}{2} \| f
        - g \|_{L^2}^2}.
      \]\fi
  \end{enumerate}
\end{lem}


\section{Bogoliubov Transformation}
\label{s:bogoliubov}
Consider an integral kernel $f\colon \Rbb^3\times\Rbb^3 \to \Cbb$. To simplify the
notation we write
\be{no1}
  f_y(x) = f(x,y)
\ee
so that
\[
  a(f_y) = a(f(\,\cdot\,,y)) \qquad \text{and} \qquad a^*(f_y) =
  a^*(f(\,\cdot\,,y)).
\]
For notational convenience, we slightly abuse notation and write
\[
  a_y + a(f_y) = a(\delta_y + f_y).
\]
%
For $k \in L^2(\R^3 \times \R^3)$, define the operators $B(k)$ and $T(k)$ on
$\mathcal{F}$ as
\[
  B(k) = \frac{1}{2} \int dxdy \, (k(x,y) a_x^* a_y^* - \overline{k(x,y)} a_x
  a_y)
\]
and
\[
  T(k) = e^{B(k)}.
\]


\begin{lem}[Bogoliubov Transformation] \label{l:bt}
  Let $k \in L^2(\R^3 \times \R^3)$ with $\nabla_2 k \in L^2(\R^3 \times
  \R^3)$.
  \begin{itemize}
  \item[(i)] The operator $T(k)$ is unitary on $\mathcal{F}$ and
  \[
    T(k)^* = T(k)^{-1} = T(-k).
  \]
  \item[(ii)] The creation and annihilation operators transform as
  \be{no4}
    T(k)^* a_y T(k) = a(c_y) + a^*(s_y) \qquad \text{and} \qquad T(k)^*
    a_y^* T(k) = a^*(c_y) + a(s_y)
  \ee
  with integral kernels $c(x,y)$ and $s(x,y)$ given by
  \todo{Why $(x,y)$ in some places and not in others? Why we need to
  evaulate the functions?}
  \begin{alignat*}{2}
    c(x,y) & = \delta(x-y) + p(x,y), & \qquad p & = \sum_{n=1}^\infty \frac{1}{(2n)!}
    \big( k \overline{k} \big)^n, \\
    s(x,y) & = k(x,y) + r(x,y), & \qquad r & = \sum_{n=1}^\infty \frac{1}{(2n+1)!}
    \big( k \overline{k} \big)^n k,
  \end{alignat*}
  where the product in the power series is the convolution of integral
  kernels, that is $fg(x,y) = \int dz \, f(x,z) g(z,y)$. Furthermore
  \begin{alignat*}{2}
    \| p \|_{L^2} & \le e^{\| k \|_{L^2}}, \qquad & \| \nabla_2 p
    \|_{L^2} & \le e^{\| k \|_{L^2}} \| \nabla_2 (k
    \overline{k}) \|_{L^2}, \\
    \| r \|_{L^2} & \le e^{\| k \|_{L^2}}, \qquad & \| \nabla_2 r
    \|_{L^2} & \le e^{\| k \|_{L^2}} \| \nabla_2 (\overline{k} k)
    \|_{L^2}, \\
    \| s \|_{L^2} & \le e^{\| k \|_{L^2}}, \qquad & \| \nabla_2 s
    \|_{L^2} & \le e^{\| k \|_{L^2}} \| \nabla_2 k \|_{L^2}.
  \end{alignat*}
  \end{itemize}
\end{lem}
\begin{rem}
Notice that the integral kernels $c$ and $s$ are defined by the series of the hyperbolic cosine respectively the hyperbolic sine. 
\end{rem}


Let $f$ be the solution of the zero-energy scattering equation
\[
  \left( -\Delta + \frac{1}{2} V \right) f = 0
\]
with boundary condition $\lim_{|x|\to\infty} f(x) = 1$. We will write
\[
  f = 1 - w
\]
with $\lim_{|x|\to\infty} w(x) = 0$. The scattering length of $V$ is defined
as
\[
  a_0 = \lim_{|x| \to \infty} w(x)|x|.
\]
Since $V$ is supported inside a ball of radius $R$, we have
\[
  f(x) = 1 - \frac{a_0}{|x|} \qquad \text{for } |x| \ge R,
\]
where we have chosen $R \in \Rbb$ such that $\operatorname{supp} \ V \subset \{ x \in \R^3: |x| \le R \}$.
From the zero-energy scattering equation we also have the identity
\[
  \int dx \, V(x) f(x) = 8 \pi a_0.
\]
By scaling, the scattering length of the potential $V_N(x) = N^2 V(Nx)$ is
$a_N = a_0/N$, and the zero-energy scattering equation for the potential $V_N$
is
\begin{equation} \label{eq:scattering}
  \left( -\Delta + \frac{1}{2} V_N \right) f_N = 0,
\end{equation}
where $f_N = 1 - w_N$ with $w_N(x) = w(Nx)$.
% Note that $w_N(x) = a_N/|x|$ for $|x| \ge R/N$.


\begin{lem}[Zero-energy Scattering Equation \cite{ESY2010}] \label{l:w}
  Let $V \in C_c^\infty(\R^3)$ with $V \ge 0$, and suppose that $V$ is
  spherically symmetric with scattering length $a_0$. Let $f_N = 1 - w_N$ be
  the solution of the zero-energy scattering equation \eqref{eq:scattering}
  with scaled potential $V_N$. Then, the following hold with constants
  uniform in $N$.
  \begin{enumerate}
    \item \label{l:w1} There exists a constant $C_0 > 0$, which depends on
    the unscaled potential $V$, such that
      \[
        C_0 \le 1 - w_N(x) \le 1 \qquad \text{for all} \ x \in \R^3.
      \]
    \item \label{l:w2} Let $R$ be such that $\text{supp} \ V \subset \{ x
    \in \R^3 : |x| \le R \}$. Then,
      \[
        w_N(x) = \frac{a_0}{N|x|} \qquad \text{for all} \ x \in \R^3 \
        \text{with} \ |x| > R/N.
      \]
    \item \label{l:w3} It follows from \ref{l:w1} and \ref{l:w2} that
      \[
        |w_N(x)| \le \max\{a_0, R\} \frac{1}{N|x|} \qquad \text{for all} \ x
        \in \R^3.
      \]
    \item \label{l:w4} There exists a constant $C$, depending only on
      $V$, such that, for all $x \in \R^3$,
      \[
        \lvert\nabla w_N(x)\rvert \le C N \qquad \mbox{and} \qquad |\nabla w_N(x)| \le \frac{C a_0}{N |x|^2}.% \qquad |\nabla w_N(x)| \le
       % \frac{C \rho}{|x|}, \qquad |\nabla^2 w_N(x)| \le \frac{C \rho}{N^2
       % |x|^2}.		%% Here equations which were mentioned in ESY were left away.
      \]
    \item \label{l:w5} It follows from \ref{l:w4} that
      \[
        |\nabla w_N(x)| \le C(1+ a_0) \frac{2N}{N^2|x|^2 + 1}
        \qquad \text{for all} \ x \in \R^3.
      \]
  \end{enumerate}
\end{lem}


\begin{lem} \label{l:kernels}
  Let $\varphi \in H^1(\R^3)$ and let $f_N=1-w_N$ be the solution of the
  zero-energy scattering equation. Suppose that
  \[
    k(x,y) = - N w_N(x-y) \varphi(x) \varphi(y).
  \]
  Then, there exists a constant $C$, uniform in $N$, that depends only on the
  unscaled potential~$V$, such that: 
  \begin{enumerate}
    \item \label{k}
      $        \| k \|_{L^2} \le C \| \varphi \|_{H^1}^2, \qquad \| \nabla_2 k
        \|_{L^2} \le C \| \varphi \|_{H^1}^2 \sqrt{N}, \qquad \| \nabla_2 (k
        \overline{k}) \|_{L^2} \le C \| \varphi \|_{H^1}^4.      $
    \item \label{kr} For almost all $x,y \in \R^3$,
      \[
        |k(x,y)| \leq N |\varphi(x)| |\varphi(y)|, \qquad |r(x,y)| \le e^{C \|
        \varphi \|_{H^1}^2} |\varphi(x)| |\varphi(y)|.
      \] 
    \item \label{sup} Let $p$ and $s$ be the integral kernels defined in Lemma
      \ref{l:bt}, and suppose further that $\varphi \in H^2(\R^3)$. Then,
      \[
        \sup_{x \in \R^3} \, \norm{p_x}_{L^2} \apprle e^{C \| \varphi
        \|_{H^1}^2} \norm{\varphi}_{H^2}, \qquad \sup_{x \in \R^3}
        \norm{s_x}_{L^2} \apprle e^{C \| \varphi \|_{H^1}^2}
        \norm{\varphi}_{H^2}.
      \]
  \end{enumerate}
\end{lem}


\begin{prp} \label{p:wphi}
  Let $\varphi \in H^1(\R^3)$ and let $f_N=1-w_N$ be the solution of the
  zero-energy scattering equation. Then, for all $x,y \in \R^3$,
  \begin{align*}
    \norm{N w_N(x-\cdot)\varphi(\cdot)}_{L^2}^2 = \int dz \, N^2 w_N(x-z)^2 |\varphi(z)|^2 & \le C \| \nabla \varphi
    \|_{L^2}^2, \\
    %\norm{N w_N(x-\cdot)w_N(\cdot - y)\varphi(\cdot)}_{L^2}^2 & = 
\int dz \, N^2 w_N(x-z) w_N(z-y) |\varphi(z)|^2 & \le C \| \nabla \varphi
    \|_{L^2}^2, \\
    \int dx dy \, N^2 |\nabla w_N(x-y)|^2 |\varphi(x)|^2 |\varphi(y)|^2 & \le
    C N \| \varphi \|_{H^1}^2,
  \end{align*}
  where $C$ is a constant, uniform in $N$, that depends only on the unscaled
  potential $V$.
\end{prp}


\begin{proof}[Proof of Proposition \ref{p:wphi}]
  Using Lemma \ref{l:w}\ref{l:w3} and Hardy's inequality, we find that
  \bd
    \int dz \, N^2 w_N(x-z)^2 |\varphi(z)|^2  \le C_1^2 \int dz \,
    |x-z|^{-2} |\varphi(z)|^2 \leq 4 C_1^2 \| \nabla \varphi \|_{L^2}^2,
  \ed
  where $C_1 = \max\{a_0, R\}$. This proves
  the first estimate.


  Now, by H\"older's inequality and the above estimate,
  \begin{align*}
    & \int dz \, N^2 w_N(x-z) w_N(z-y) |\varphi(z)|^2 \\
    & \le \left( \int dz \, N^2 w_N(x-z)^2 |\varphi(z)|^2 \right)^{1/2} \left(
    \int dz \, N^2 w_N(z-y)^2 |\varphi(z)|^2 \right)^{1/2} \le 4C_1^2 \|
    \nabla \varphi \|_{L^2}^2.
  \end{align*}
  This proves the second estimate.


  Finally, by Young's inequality, Sobolev's inequality, and Lemma
  \ref{l:w}\ref{l:w5},
  \begin{align*}
    \int dx dy \, N^2 |\nabla w_N(x-y)|^2 |\varphi(x)|^2 |\varphi(y)|^2 & \apprle
    N^2 \| | \nabla w_N|^2 \|_{L^1} \| \varphi \|_{L^4}^4 \\
    & \apprle C_2 N \| \varphi \|_{H^1}^4 \int dx \, \frac{N^3}{(|Nx|^2 +
    1)^2} \apprle C_2 N \| \varphi \|_{H^1}^4,
  \end{align*}
  where $C_2$ is a constant from Lemma \ref{l:w}, that depends only on $V$ and
  $a_0$. This proves the third estimate and concludes the proof of the
  proposition.
\end{proof}

\begin{proof}[Proof of Lemma \ref{l:bt}]
  (i) The operator $B(k)$ is densely defined and anti-self-adjoint.
  Hence, by the spectral theorem, the unitary operator $T(k) = e^{B(k)}$ is
  well-defined and $T(k)^* = T(k)^{-1} = T(-k)$.
\iffalse
  Let $\chi$ be the characteristic function, and consider a real number $M >
  0$. We give a proof of the formulae on the subspace $\chi(\N \le M)
  \mathcal{F}$, and then we indicate how to extend it to the whole
  $\mathcal{F}$. First, note that $B(k)$ and $a(f)$ are bounded operators on
  $\chi(\N \le M) \mathcal{F}$. Recall the formula
  \[
    e^X Y e^{-X} = Y + [X,Y] + \frac{1}{2!} [X,[X,Y]] + \cdots
  \]
\fi

  (ii) Observing that $[a_w a_z, a_y] = [a_w^* a_z^*, a_y^*] = 0$ and
  \[
    [a_w a_z, a_y^*] = \delta(w-y) a_z + \delta(z-y) a_w \qquad \text{and}
    \qquad [a_w^* a_z^*, a_y] = -\delta(y-w) a_z^* - \delta(y-z) a_w^*,
  \]
  one easily finds the expression
  \begin{align*}
    T(k)^* a_y T(k) & = e^{-B(k)} a_y e^{B(k)} = a_y - [B(k), a_y] +
    \frac{1}{2!} [B(k), [B(k), a_y]] + \cdots \\
    & = a_y + \int dx \, \sum_{n=1}^\infty \frac{1}{(2n)!} \big( 
    \overline{k} k \big)^n(x,y) a_x + \int dx \, \sum_{n=0}^\infty
    \frac{1}{(2n+1)!} \big( k \overline{k} \big)^n k(x,y) a_x^* \\
    & = a_y + \int dx \, \overline{p(x,y)} a_x + \int dx \, s(x,y) a_x^* \\
    & = a_y + a(p_y) + a(s_y) = a(c_y) + a(s_y).
  \end{align*}


  We now prove the estimates for the integral kernels $p$, $r$ and $s$. 
  Let $fg$ be the convolution of integral kernels $f, g \in L^2(\Rbb^3\times\Rbb^3)$. Then, by H\"older's inequality,
  \[
    \| f g \|_{L^2} \le \| f \|_{L^2} \| g \|_{L^2}.
  \]
  Hence, applying this to the series for $p$, we get
  \[
    \| p \|_{L^2} \le \sum_{n=1}^\infty \frac{1}{(2n)!} \| k \|_{L^2}^{2n} \le
    e^{\| k \|_{L^2}},
  \]
  and similarly $\| r \|_{L^2} \le e^{\| k \|_{L^2}}$ and $\| s \|_{L^2} \le
  e^{\| k \|_{L^2}}$. Furthermore, since the series for $p$, $r$ and $s$ are
  absolutely convergent, we can write
  \begin{align*}
    \nabla_2 p & = \frac{1}{2} \nabla_2 (k \overline{k}) + \left[
    \sum_{n=2}^\infty \frac{1}{(2n)!} (k \overline{k})^{n-1} \right]
    \nabla_2 (k \overline{k}) , \\
    \nabla_2 r & = \left[ \sum_{n=1}^\infty \frac{1}{(2n+1)!} (k
    \overline{k})^{n-1} k \right] \nabla_2 (\overline{k} k), \\
    \nabla_2 s & = \nabla_2 k + \left[ \sum_{n=1}^\infty \frac{1}{(2n+1)!}
    (k \overline{k})^n \right] \nabla_2 k.
  \end{align*}
  Thus, similarly as above,
  \begin{displaymath}
    \| \nabla_2 p \|_{L^2} \le e^{\| k \|_{L^2}} \| \nabla_2 (k
    \overline{k}) \|_{L^2},\  
    \| \nabla_2 r \|_{L^2} \le e^{\| k \|_{L^2}} \| \nabla_2
    (\overline{k} k) \|_{L^2},\ 
    \| \nabla_2 s \|_{L^2} \le e^{\| k \|_{L^2}} \| \nabla_2 k
    \|_{L^2}.
  \end{displaymath}
This completes the proof of the lemma.
\end{proof}

\begin{proof}[Proof of Lemma \ref{l:kernels}]
\hspace{-0.3em} %weird, without hspace the ``Proof'' disappears
\ref{k} We will repeatedly apply Proposition \ref{p:wphi}, Sobolev's
  inequality, Leibniz' rule, and the triangle inequality. We begin with
  \[
    \| k \|_{L^2}^2 = \int dx \, |\varphi(x)|^2 \int dy \, N^2 w_N(x-y)^2
    |\varphi(y)|^2 \le C \| \varphi \|_{L^2}^2 \| \nabla \varphi \|_{L^2}^2
    \apprle C \| \varphi \|_{H^1}^4
  \]
  and
  \begin{align*}
    \| \nabla_2 k \|_{L^2} & \le \| N w_N(x-y) \varphi(x) \nabla \varphi(y)
    \|_{L^2(dx dy)} + \| N \nabla w_N(x-y) \varphi(x) \varphi(y) \|_{L^2(dx dy)} \\
    & \le C \| \nabla \varphi \|_{L^2}^2 + C \sqrt{N} \| \varphi \|_{H^1}^2
    \apprle C \sqrt{N} \| \varphi \|_{H^1}^2.
  \end{align*}
  This proves the first two estimates in \ref{k}. Now,
  \[
    \| \nabla_2 (k \overline{k}) \|_{L^2} \le \| I \|_{L^2} + \| J \|_{L^2},
  \]
  where, by Proposition \ref{p:wphi},
  \[
    \| I \|_{L^2}^2 = \int dx \, |\varphi(x)|^2 \int dy \, |\nabla
    \varphi(y)|^2 \left| \int dz \, N^2 w_N(x-z) w_N(z-y) |\varphi(z)|^2
    \right|^2 \apprle C^2 \| \varphi \|_{H^1}^8,
  \]
  and, using Lemma \ref{l:w} \ref{l:w3}-\ref{l:w4}, and Hardy's inequality,
  \begin{align*}
    \| J \|_{L^2}^2 & = \int dx \, |\varphi(x)|^2 \int dy \, |\varphi(y)|^2
    \left| \int dz \, N^2 w_N(x-z) \nabla w_N(z-y) |\varphi(z)|^2 \right|^2 \\
& \apprle \int \di x\lvert\varphi(x)\rvert^2 \int \di y \lvert \varphi(y)\rvert^2
\left\lvert \int \di z \frac{\lvert\varphi(z)\rvert^2}{\lvert x-z\rvert
\lvert z-y\rvert^2} \right\rvert^2 \\
& = \int \di y\di z_1 \di z_2 \lvert \varphi(y)\rvert^2 \frac{\lvert \varphi(z_1)\rvert^2}{\lvert y-z_1\rvert^2} \frac{\lvert\varphi(z_2)\rvert^2}{\lvert y-z_2\rvert^2} \int \di x \frac{\lvert \varphi(x)\rvert^2}{\lvert z_1-x\rvert \lvert z_2 - x\rvert} \\
& \apprle \int \di y \lvert \varphi(y)\rvert^2 \int \di z_1 \frac{\lvert \varphi(z_1)\rvert^2}{\lvert y-z_1\rvert^2} \int \di z_2 \frac{\lvert \varphi(z_2)\rvert^2}{\lvert y-z_2\rvert^2} \norm{\nabla \varphi}_{L^2}^2\\
& \apprle C \norm{\varphi}_{H^1}^8.
  \end{align*}
  Therefore
  \[
    \| \nabla_2 (k \overline{k}) \|_{L^2} \le \| I \|_{L^2} + \| J \|_{L^2}
    \apprle C \| \varphi \|_{H^1}^4.
  \]
  This completes the proof of part \ref{k}.

  \ref{kr} By Lemma \ref{l:w}\ref{l:w1}, we have $w_N \leq 1$. Hence
  \[
    |k(x,y)| = N w_N(x-y) |\varphi(x)| |\varphi(y)| \leq N |\varphi(x)|
    |\varphi(y)|.
  \]
  This proves the first estimate in \ref{kr}.
   
  To bound $\lvert r(x,y)\rvert$ observe that, by H\"older's inequality,
  Proposition \ref{p:wphi}, and part \ref{k},
  \begin{align*}
    & |( k \overline{k})^n k(x,y)| = | k \overline{k} ( k \overline{k})^{n-1}
    k(x,y)| \\
    & = |\varphi(x)| |\varphi(y)| \, \left| \int dz_1 dz_2 \, N^2 w_N(x-z_1)
    w_N(z_2-y) \varphi(z_1) \varphi(z_2) \overline{k}(k
    \overline{k})^{n-1}(z_1,z_2) \right| \\
    & \le |\varphi(x)| |\varphi(y)| \, \| \overline{k} (k \overline{k})^{n-1}
    \|_{L^2} \\
    & \quad \times \left( \int dz_1 N^2 w_N(x-z_1)^2 |\varphi(z_1)|^2 \int
    dz_2 \, N^2 w_N(z_2-y)^2 |\varphi(z_2)|^2 \right)^{1/2} \\
    & \le |\varphi(x)| |\varphi(y)| C \| \nabla \varphi \|_{L^2}^2 \| k
    \|_{L^2}^{2n-1} \le |\varphi(x)| |\varphi(y)| C^{2n} \| \varphi
    \|_{H^1}^{4n}.
  \end{align*}
  Thus,
  \[
    |r(x,y)| \le \sum_{n=1}^\infty \frac{1}{(2n+1)!} |( k \overline{k})^n k
    (x,y)|\le |\varphi(x)| |\varphi(y)| e^{C \| \varphi \|_{H^1}^2}.
  \]
  This completes the proof of part \ref{kr}.


  \ref{sup} We give a proof of the estimate for $\sup_{y \in \R^3} \| s_y
  \|_{L^2}$. The proof for $\sup_{y \in \R^3} \| p_y \|_{L^2}$ is similar.
  First, by H\"older's inequality and Proposition \ref{p:wphi},
  \begin{align*}
    |(k \overline{k})^n k(x,y)| & = \left| \varphi(y) \int dz \, N w_N(z-y)
    \varphi(z) (k \overline{k})^n(x,z) \right| \\
    & \le |\varphi(y)| \left( \int dz \, N^2 w_N(z-y)^2 |\varphi(z)|^2
    \right)^{1/2} \| (k \overline{k})^n(x,\, \cdot\,) \|_{L^2} \\
    & \le C \| \nabla \varphi \|_{L^2} | \varphi(y)| \, \| (k \overline{k}
    )^n(x,\, \cdot\,) \|_{L^2}.
  \end{align*}
  Hence, using part \ref{k},
  \[
    \| s_y \|_{L^2} \le \sum_{n=0}^\infty \frac{\| (k \overline{k})^n k(\,
    \cdot\,,y) \|_{L^2}}{(2n+1)!} \le |\varphi(y)| \sum_{n=0}^\infty
    \frac{C^{2n+1} \| \varphi \|_{H^1}^{2(2n+1)}}{(2n+1)!} \le e^{C \|
    \varphi\|_{H^1}^2} |\varphi(y)|.
  \]
  Therefore, by Sobolev's inequality,
  \[
    \sup_{y \in \R^3} \| s_y \|_{L^2} \le e^{C \| \varphi \|_{H^1}^2} \sup_{y
    \in \R^3} |\varphi(y)| \apprle e^{C \| \varphi \|_{H^1}^2} \| \varphi
    \|_{H^2}.
  \]
  This proves part \ref{sup} and completes the proof of the lemma.
\end{proof}


\section{Dynamics of Quantum Fluctuations}
\label{s:fluctuations}
We use the short-hand $T_t = T(k)$ with
\bd
k(x,y) = - N w_N(x-y) \ph(x) \ph(y),
\ed
where $\ph$ is a solution of the modified Hartree equation. We also write $W_t = W(\sqrt{N}\ph)$.
We define the time evolution of quantum fluctuations by
\bd
U_N(t) := T^\ast_t W^\ast_t e^{-it \Hcal_N} W_0 T_0.
\ed
To simplify the notation we sometimes write $U_t := U_N(t)$.

\subsection{The Generator of Quantum Fluctuations}
\label{ss:generator}
The generator $\Lcal_N(t)$ of quantum fluctuations is defined by
\bd
\Lcal_N(t) U_N(t) = i \partial_t U_N(t).
\ed
We calculate that
\begin{align*}
\Lcal_N(t) 	& = (i \partial_t T^*_t) T_t + T^*_t \big( (i \partial_t W^*_t) W_t + W^*_t \Hcal_N W_t \big) T_t \\
		& = (i \partial_t T^*_t) T_t + T^*_t \Lcal^{(0)}_N(t) T_t
\end{align*}
where, using $(i \partial_t W^*_t) W_t = i a(\sqrt{N} \phdot) - i a^\ast(\sqrt{N}\phdot) + N \Im \scal{\ph}{\phdot}$ and Lemma \ref{l:W} \ref{l:W3},
\be{no3}
\begin{split}
\Lcal^{(0)}_N(t) & = \Kcal + \Vcal_N \\
		& \quad + N^{1/2} \left[  a^*\left( (w_N N V_N \ast \lvert \ph \rvert^2)\ph \right) + \hc  \right] \\
		& \quad + N^0	    \left[  \frac{1}{2}\int \di x \di y\, NV_N(x-y)\left( \cc{\ph(x)} \cc{\ph(y)} a_y a_x + \hc \right) \right] \\
		& \quad + N^0	    \left[  \int \di x \di y\, NV_N(x-y)\left( \lvert \ph(x) \rvert^2 a^*_y a_y + \cc{\ph(x)} \ph(y) a^*_y a_x \right) \right] \\
		& \quad + N^{-1/2}\left[  \int \di x \di y\, NV_N(x-y) \left( \cc{\ph(x)} a^*_y a_y a_x + \hc \right)  \right] + b(N,t).
\end{split}
\ee
The phase $b(N,t) \in \Rbb$, like all phases appearing in $\Lcal_N(t)$, will
be dropped from now on without any further comment. This is possible because $\scal{U_N(t)T^\ast_0 \Psi}{(\Ncal+1)U_t T_0^\ast \Psi} = \scal{\tilde U_N(t)T^\ast_0 \Psi}{(\Ncal+1)\tilde U_t T_0^\ast \Psi}$ for all $\tilde U_N(t) = e^{i\theta_N(t)} U_N(t)$ with $\theta_N(t) \in \Rbb$.


In \eqr{no3}, we made use of a cancellation among the terms which are linear in creation and annihilation operators due to $\ph$ satisfying the modified Hartree equation \eqr{modHartreeEqu}.
%\bd
%i\partial_t \ph = -\Delta \ph + \left(f_N N V_N \ast \lvert \ph \rvert^2 \right) \ph.
%\ed
This should be compared to \cite{RS2009}, with the important difference being that due to the factor $f_N$ in the modified Hartree equation, in our case this cancellation is incomplete. This is essential for ensuring the correct coupling constant $8\pi a_0$ instead of $b_0$ in the Gross-Pitaevskii equation and ensures important cancellations (see below). These cancellations are revealed through the Bogoliubov transformation, to be calculated now.

We first identify cancellations between linear terms originating from the transform \eqr{no4}, $T(k)^* a_y T(k) = a(c_y) + a^*(s_y)$, of linear terms and of linear terms originating from normal-ordering the transform of cubic terms
\begin{align*}
T^*_t \Lcal^{(0)}_N(t) T_t \qquad \\
= \int \di x\, T^*_t a^*_x T_t \bigg[ &   \frac{1}{2}(-\Delta_x T^*_t a_x T_t)  + \frac{1}{4}\int \di y V_N(x-y) T^*_t a^*_y a_y a_x T_t \\
& + N^{1/2} \ph(x) \int \di y\, w_N N V_N(x-y) \lvert \ph(y) \rvert^2 \tagg{linearterm} \\
& + \frac{1}{2} \int \di y\, NV_N(x-y)  \ph(x) \ph(y)  T^*_t a^*_y T_t \\
& + \frac{1}{2} \int \di y\, NV_N(x-y) \left(  \lvert \ph(y)\rvert^2 T^*_t a_x T_t + \ph(x) \cc{\ph(y)} T^*_t a_y T_t  \right) \\
& + N^{-1/2} \int \di y\, NV_N(x-y) \cc{\ph(y)} T^*_t a_x a_y T_t  \bigg]\tagg{cubicterm} \\
+ \hc \qquad \quad
\end{align*}
In line \eqr{cubicterm}, we use $T^\ast_t a_x a_y T_t = \left( a(c_x)+a^\ast
(s_x) \right)\left( a(c_y) + a^\ast(s_y) \right)$ and normal-order these
terms, thereby creating a commutator $[a(c_x),a^\ast(s_y)] = \scal{c_x}{s_y}$. Now $\scal{c_x}{s_y} = \scal{\delta_x + p_x}{k_y + r_y} = k(x,y) + r(x,y) + \scal{p_x}{s_y}$, and the contribution of $k(x,y)$ cancels line \eqr{linearterm}. The result is
\begin{align*}
T^*_t \Lcal^{(0)}_N(t) T_t \qquad \\
= \int \di x\, T^*_t a^*_x T_t \bigg[ &   \frac{1}{2}(-\Delta_x T^*_t a_x T_t)  + \frac{1}{4}\int \di y V_N(x-y) T^*_t a^*_y a_y a_x T_t \\
& + \frac{1}{2} \int \di y\, NV_N(x-y)  \ph(x) \ph(y)  T^*_t a^*_y T_t \\
& + \frac{1}{2} \int \di y\, NV_N(x-y) \left(  \lvert \ph(y)\rvert^2 T^*_t a_x T_t + \ph(x) \cc{\ph(y)} T^*_t a_y T_t  \right) \\
& + N^{-1/2} \int \di y\, NV_N(x-y) \cc{\ph(y)} \bigg( r(x,y) + \scal{p_x}{s_y} + \\
& \qquad \qquad \hspace{0.7cm}   a^*(s_x) a^*(s_y) + a^*(s_x) a(c_y)  + a^*(s_y) a(c_x) + a(c_x) a(c_y)  \bigg)  \bigg] \\
+ \hc \qquad \quad
\end{align*}
As the next step, we expand the Bogoliubov transform everywhere using
\eqr{no4}, then multiply out all the terms and normal-order them. Notice that many terms appear twice, e.\,g.\ in the hermitian conjugate or with $x$ and $y$ interchanged. We obtain
\begin{align*}
& T^*_t \Lcal_N^{(0)}(t) T_t = \\ 
& \frac{1}{2} \int \di x\, \left[ a^*(c_x) a(-\Delta_x c_x) + \boxed{2 a^*(c_x) a^*(-\Delta_x s_x)} + a^*(-\Delta_x s_x) a(s_x) \right]\\% \tagg{cancellation_kinetic} \\
& + \frac{1}{2}\int \di x \di y\, NV_N(x-y) \times \\
& \times \Big[   \frac{1}{2N}\bigg( a^*(c_x) a^*(c_y) a(c_y) a(c_x) + 4 a^*(c_x) a^*(c_y) a^*(s_x) a(c_y) \\
				      & \qquad\qquad + 2 a^*(c_x) a^*(c_y) a^*(s_y) a^*(s_x) + 2 a^*(c_x) a^*(s_x) a(s_y) a(c_y) \\
				      & \qquad\qquad + 2 a^*(c_x) a^*(s_y) a(s_y) a(c_x) + 4 a^*(c_x) a^*(s_y) a^*(s_x) a(s_y) \\
				      & \qquad\qquad + a^*(s_y) a^*(s_x) a(s_x) a(s_y) \bigg) \\
& + \frac{1}{N}\bigg(   \boxed{a^*(c_x)a^*(c_y) \scal{c_y}{s_x}} + a^*(c_x) a(c_y) \scal{s_y}{s_x}\\% \tagg{cancellation_normalorder} \\
			& \qquad\quad + a^*(c_x) a(s_y) \scal{c_y}{s_x} + a^*(c_x) a(c_x) \scal{s_y}{s_y} \\
			& \qquad\quad + 2 a^*(c_x) a^*(s_x) \scal{s_y}{s_y} + 2a^*(c_x)a^*(s_y) \scal{s_y}{s_x} \\
			& \qquad\quad + a^*(c_y) a(s_x) \scal{c_y}{s_x} +  a^*(s_y) a(s_y) \scal{s_x}{s_x}\\
			& \qquad\quad + a^*(s_y) a^*(s_x) \scal{s_x}{c_y} + a^*(s_y) a(s_x) \scal{s_y}{s_x}   \bigg) \\
& + \ph(x)\ph(y) \Big( \boxed{a^*(c_x) a^*(c_y)} + 2 a^*(c_x) a(s_y) +a(s_x) a(s_y) \Big)\\% \tagg{cancellation_standard} \\
& + \ph(x) \cc{\ph(y)} \Big( a^*(c_x) a(c_y) + 2 a^*(c_x) a^*(s_y) + a^*(s_y) a(s_x) \Big) \\
& + \lvert \ph(y) \rvert^2 \Big( a^*(c_x) a(c_x) + 2 a^*(c_x) a^*(s_x) + a^*(s_x) a(s_x) \Big) \\
& + \frac{2}{\sqrt{N}}\cc{\ph(y)} \bigg(    a^*(c_x) a^*(s_x) a^*(s_y) + a^*(c_x) a^*(s_x) a(c_y) + a^*(s_x) a^*(s_y) a(s_x)\\
					    & \qquad\qquad\qquad\quad + a^*(c_x) a^*(s_y) a(c_x) + a^*(c_x) a(c_x) a(c_y)+ a^*(s_x) a(s_x) a(c_y) \\
					    & \qquad\qquad\qquad\quad + a^*(s_y) a(s_x) a(c_x) + a(s_x) a(c_x) a(c_y)    \bigg) \\
& + \frac{2}{\sqrt{N}}\cc{\ph(y)} \bigg(    a^*(s_x) \scal{s_x}{s_y} + a^*(s_y) \scal{s_x}{s_x}  + a(c_y) \scal{s_x}{s_x} + a(c_x) \scal{s_x}{s_y} \\
					    & \qquad\qquad\qquad\quad + a^*(c_x)r(x,y) + a^*(c_x)\scal{p_x}{s_y} + a(s_x)r(x,y) + a(s_x)\scal{p_x}{s_y}		\bigg)    \Big]\\
& + \hc
\end{align*}
We proceed to identify a cancellation among the boxed terms. We exhibit the
cancellation by expanding $c = \delta + p$ and $s = k + r$ and applying the
product rule to $-\Delta_x k(y,x)$. Thus, the left hand side of the zero-energy scattering equation \eqr{scattering} appears. However, there are some (more regular) remainder terms left, see \eqref{l7}, \eqref{l8}, \eqref{l14} and \eqref{l18} in the final result for the generator below.

The final result is
\begin{align}
& T^*_t \Lcal_N^{(0)}(t) T_t = \nonumber \\ 
& \frac{1}{2} \int \di x\, \bigg[ a^*(c_x) \int \di y\, a^*_y \Big(4 N \nabla w_N(x-y) \nabla_x \ph(x) \ph(y) \label{l7}\\
& \qquad\qquad \qquad\qquad \qquad \ \	+ 2Nw_N(x-y) \Delta_x \ph(x) \ph(y) - 2\Delta_x r(y,x) \Big) \label{l8}\\
& \qquad\qquad 			+ a^*(c_x) a(-\Delta_x c_x) + a^*(-\Delta_x s_x) a(s_x) \bigg] \label{l9} \\
& + \frac{1}{2}\int \di x \di y\, NV_N(x-y) \times \nonumber \\
& \times \Big[   \frac{1}{2N}\bigg( a^*(c_x) a^*(c_y) a(c_y) a(c_x) + 4 a^*(c_x) a^*(c_y) a^*(s_x) a(c_y) \label{l10}\\
				      & \qquad\qquad + 2 a^*(c_x) a^*(c_y) a^*(s_y) a^*(s_x) + 2 a^*(c_x) a^*(s_x) a(s_y) a(c_y) \label{l11}\\
				      & \qquad\qquad + 2 a^*(c_x) a^*(s_y) a(s_y) a(c_x) + 4 a^*(c_x) a^*(s_y) a^*(s_x) a(s_y) \label{l12}\\
				      & \qquad\qquad + a^*(s_y) a^*(s_x) a(s_x) a(s_y) \bigg) \label{l13}\\
& + \frac{1}{N}\bigg(   a^*(c_x) a^*(c_y) \Big( r(y,x) + \scal{p_y}{s_x} \Big) + a^*(c_x) a^*(p_y) k(y,x) \label{l14} \\
      & \qquad\quad + a^*(c_x) a(c_y) \scal{s_y}{s_x} + a^*(s_y) a(s_y) \scal{s_x}{s_x} + a^*(s_y) a(s_x) \scal{s_y}{s_x} \label{l15}\\
      & \qquad\quad + a^*(c_x) a(s_y) \scal{c_y}{s_x} + a^*(c_x) a(c_x) \scal{s_y}{s_y} + a^*(s_y) a^*(s_x) \scal{s_x}{c_y} \label{l16}\\
      & \qquad\quad + 2a^*(c_x) a^*(s_x) \scal{s_y}{s_y} + 2a^*(c_x)a^*(s_y) \scal{s_y}{s_x} + a^*(c_y) a(s_x) \scal{c_y}{s_x}    \bigg) \label{l17}\\
& + \ph(x)\ph(y) \Big( a^*(c_x) a^*(p_y) + 2 a^*(c_x) a(s_y) +a(s_x) a(s_y) \Big) \label{l18}\\
& + \ph(x) \cc{\ph(y)} \Big( a^*(c_x) a(c_y) + 2 a^*(c_x) a^*(s_y) + a^*(s_y) a(s_x) \Big) \label{l19}\\
& + \lvert \ph(y) \rvert^2 \Big( a^*(c_x) a(c_x) + 2 a^*(c_x) a^*(s_x) + a^*(s_x) a(s_x) \Big) \label{l20}\\
& + \frac{2}{\sqrt{N}}\cc{\ph(y)} \bigg(    a^*(c_x) a^*(s_x) a^*(s_y) + a^*(c_x) a^*(s_x) a(c_y) \label{l21}\\
					    & \qquad\qquad\qquad + a^*(c_x) a^*(s_y) a(c_x) + a^*(c_x) a(c_x) a(c_y) + a^*(s_y) a(s_x) a(c_x) \label{l22}\\
					    & \qquad\qquad\qquad + a^*(s_x) a^*(s_y) a(s_x) + a^*(s_x) a(s_x) a(c_y) + a(s_x) a(c_x) a(c_y)  \bigg) \label{l23}\\
& + \frac{2}{\sqrt{N}}\cc{\ph(y)} \bigg(    a^*(s_x) \scal{s_x}{s_y} + a^*(s_y) \scal{s_x}{s_x}  + a(c_y) \scal{s_x}{s_x} + 							a(c_x) \scal{s_x}{s_y} \label{l24}\\
					    & \qquad\qquad\qquad + a^*(c_x)r(x,y) + a^*(c_x)\scal{p_x}{s_y} + a(s_x)r(x,y) + 			a(s_x)\scal{p_x}{s_y}		\bigg)    \Big] \label{l25}\\
&+ \hc \nonumber
\end{align}

\begin{rem}The cancellations enable us to give estimates for all terms of the generator, see appendix \ref{ch:generatorestimates}. If we had not identified the cancellation between the boxed terms, there would be contributions of order $N^1$.
\end{rem}


\subsection{Estimating the number of fluctuations}
\label{ss:number}
The estimates in this section hold for the expectation value $\ev{\cdot} =
\scal{\psi}{\cdot\,\psi}$ in any state $\psi \in \fock$. Especially, $C$ here stands for constants which are also independent of $\psi$. 
We define
\bd
\tilV := \frac{1}{4} \int \di x\di y\, V_N(x-y) a^\ast(c_x) a^\ast(c_y)a(c_y)a(c_x).
\ed
(Notice that through the $t$-dependence of $c$, the operator $\tilV$ is time-dependent. However, as this does not play any role in our proof, we do not indicate this time-dependence and $\tilV$ should always be read as belonging to the same time $t$ as the other objects in the equation where it appears.)

\begin{lem}
\label{lem:kvbounds}
There exists a constant $C$ such that
\bd
\ev{\Kcal + \tilV} \leq 4 \ev{\Lcal_N(t)} + C \norm{\ph}_{H^2}^4 \ev{\frac{1}{N}\Ncal^2 + \Ncal + 1}.
\ed
As $\Kcal$ and $\tilV$ are non-negative operators, the same upper bound holds for each of $\Kcal$ and $\tilV$ individually.
\end{lem}
\begin{proof}
We know that $\Lcal_N(t) = (i\partial_t T^\ast_t)T_t + T^\ast_t
\Lcal_N^{(0)}(t)T_t$ with $T^\ast_t \Lcal_N^{(0)}(t)T_t$ written out in
\eqref{l7} through \eqref{l25}. We now write only line \eqref{l9} and the
first summand in line \eqref{l10} explicitly, and all the other terms we collect as $\Lcal_{\textrm{rem}}$:
\begin{align*}
\Lcal_N(t) & = \frac{1}{2} \int \di x\, a^\ast(c_x) a(-\Delta_x c_x) + \frac{1}{2}\int \di x\, a^\ast(-\Delta_x s_x)a(s_x) \\
& \quad + \frac{1}{2}\int \di x\di y\, NV_N(x-y) \frac{1}{2N} a^\ast(c_x) a^\ast(c_y) a(c_y) a(c_x) + \Lcal_{\textrm{rem}}.
\end{align*}
We define $\tilK := \int \di x\, a^\ast(c_x) a(-\Delta_x c_x)$ and observe that
\bd
\ev{\frac{1}{2}\int \di x\, a^\ast(-\Delta_x s_x)a(s_x)} = \frac{1}{2}\int \di x\, \norm{a(\nabla_x s_x)\psi}^2 \geq 0,
\ed
so by leaving away this non-negative term we obtain
\be{Lcalbound}
\ev{\Lcal_N(t)} \geq \ev{\frac{1}{2}\tilK + \tilV} - \lvert\ev{\Lcal_{\textrm{rem}}}\rvert.
\ee
We now collect the estimates for lines \eqref{l7}, \eqref{l8} and \eqref{l10} through \eqref{l25} from appendix \ref{ch:generatorestimates} and also use Lemma \ref{lm:timederivative} to obtain an upper bound for $\lvert\ev{\Lcal_{\textrm{rem}}}\rvert$, i.\,e.\ a lower bound for $-\lvert\ev{\Lcal_{\textrm{rem}}}\rvert$.
(Notice that $\norm{\ph}_{H^2} \geq \norm{\ph}_{L^2} =1$ and use $\frac{1}{N} \leq 1$, $\frac{1}{\sqrt{N}} \leq 1$ where they are not needed to controll a $\Ncal^2$, and use that $1 \leq 1/\varepsilon$.)
\bd
\lvert\ev{\Lcal_{\textrm{rem}}}\rvert \leq C \norm{\ph}_{H^2}^2 \left( \frac{1}{\varepsilon}\ev{\frac{1}{N}\Ncal^2 + \Ncal + 1} + \varepsilon\ev{\Kcal+\tilV} \right).
\ed
Hence using \eqr{Lcalbound} we conclude that
\bd
\ev{\Lcal_N(t)} \geq \ev{\frac{1}{2}\tilK+\tilV} - C\norm{\ph}_{H^2}^2 \left( \frac{1}{\varepsilon}\ev{\frac{1}{N}\Ncal^2 +\Ncal+1} + \varepsilon \ev{\Kcal+\tilV} \right).
\ed
Combining this with $\ev{\tilK} \geq \ev{\Kcal} - C\left(\varepsilon\ev{\Kcal} + \frac{1}{\varepsilon}\ev{\Ncal} \right)$ from \eqref{eq:bound1A}, we arrive at
\bd
\ev{\Lcal_N(t)} \geq  \ev{\frac{1}{2}\Kcal+\tilV} - C\norm{\ph}_{H^2}^2 \left( \frac{1}{\varepsilon}\ev{\frac{1}{N}\Ncal^2 +\Ncal+1} + \varepsilon \ev{\Kcal+\tilV} \right).
\ed
Now choose $\varepsilon = 1/(4C\norm{\ph}_{H^2}^2)$ (with $C$ having the same value as in the previous equation) to conclude
\bd
\ev{\Lcal_N(t)} + 4\left( C \norm{\ph}_{H^2}^2 \right)^2
\ev{\frac{1}{N}\Ncal^2 + \Ncal + 1} \geq
\ev{\frac{1}{4}\Kcal+\frac{3}{4}\tilV} \geq \frac{1}{4} \ev{\Kcal + \tilV}.
\ed
This completes the proof.
\end{proof}

For later reference we explicitly state the following simple Corollary.
\begin{lem}
\label{lem:Llowerbound}
There exists a constant $C$ such that
\bd
\ev{\Lcal_N(t)} \geq -C \norm{\ph}_{H^2}^4 \ev{\frac{1}{N}\Ncal^2 + \Ncal + 1}.
\ed 
\end{lem}
\begin{proof}
As $\ev{\Kcal} \geq 0$ and $\ev{\tilV} \geq 0$, this lemma is an easy corollary of lemma \ref{lem:kvbounds}.
\end{proof}

\begin{lem}
\label{lem:ldotbounds} There exists a constant $C$ such that
 \bd
  \lvert \ev{\dot \Lcal_N(t)} \rvert \leq C \norm{\ph}_{H^4}^8 \ev{\frac{1}{N}\Ncal^2 + \Ncal + 1 + \Kcal + \tilV}.
 \ed
\end{lem}
\begin{proof} Recall that $\dot \Lcal_N(t) = \partial_t\big((i\partial_t T^\ast_t) T_t \big) + \partial_t\big( T^\ast_t \Lcal_N^{(0)}(t) T_t \big)$.

Combining Lemma \ref{lm:timederivative} and Lemma \ref{lm:kbounds}, we obtain the estimate
\bd
\lvert\ev{\partial_t\big((\partial_t T^\ast_t) T_t \big)}\rvert \leq C \norm{\ph}_{H^4}^4 \ev{\Ncal+1}.
\ed
The term $\ev{\partial_t\big( T^\ast_t \Lcal_N^{(0)}(t) T_t \big)}$ can be estimated by the same methods which are used for estimating $\ev{ T^\ast_t \Lcal_N^{(0)}(t) T_t}$ in appendix \ref{ch:generatorestimates}, but now additionally the quantities
\begin{align*}
& \norm{\gradone \dot p},\ \norm{\gradone \dot r},\ \norm{\dot s},\ \norm{\dot p},\ \sup_x \norm{\dot p_x},\ \sup_x \norm{\dot s_x},\ \mbox{Sobolev norms of } \phdot, \\
& \int \di x\di y V_N(x-y) \lvert \dot r(x,y)\rvert^2,\ \int \di x \di y V_N(x-y) \lvert \dot s(x,y)\rvert^2/N
\end{align*}
are used and then estimated using the Lemmata \ref{lem:dottedests1}, \ref{lem:dottedests2} and \ref{lem:dottedests3}.

(Notice that in $\partial_t\big( T^\ast_t \Lcal_N^{(0)}(t) T_t\big)$ there are about four times as many terms as in $T^\ast_t \Lcal_N^{(0)}(t) T_t$ because of the product rule, but many of them are easier to estimate because $\partial_t a^\ast(c_x) = a^\ast(\dot p_x)$ is regular while $a^\ast(c_x)$ is singular.)
\end{proof}


\begin{lem}
\label{lem:lncommutatorbound}
There exists a constant $C$ such that
 \bd
  \lvert \ev{[\Lcal_N(t),\Ncal]} \rvert \leq C \norm{\ph}_{H^2}^2 \ev{\frac{1}{N}\Ncal^2 + \Ncal + 1 + \Kcal + \tilV}.
 \ed
\end{lem}
\begin{proof}
Recall that
\[\Lcal_N(t) = (i \partial_t T^\ast_t)T_t + T^\ast_t \Lcal_N^{(0)}(t)T_t.\]

First we calculate $[T^\ast_t \Lcal_N^{(0)}(t)T_t,\Ncal]$. Summands which contain an equal number of annihilation and creation operators commute with $\Ncal$ and thus drop out. The other summands get numerical prefactors, but except for that stay unchanged. %(Because of these prefactors, the sign of the $+\hc$-part changes, but this is not important for our estimates.)
Thus we can use the estimates from appendix \ref{ch:generatorestimates}
with $\varepsilon = 1$ to bound all summands of the commutator $[T^\ast_t \Lcal_N^{(0)}(t)T_t,\Ncal]$.

Second we expand $(\partial_t T^\ast_t)T_t$ as a series according to
\bd
\left(\partial_t e^{-B(t)} \right) e^{B(t)} = - \dot B(t) +
\frac{1}{2!}[B(t),\dot B(t)] - \frac{1}{3!}[B(t),[B(t),\dot B(t)]] + \dots
\ed
From this we can see that $[(\partial_t T^\ast_t)T_t,\Ncal]$ is of the same form as $(\partial_t T^\ast_t)T_t$, the only difference being that the particle-number conserving summands of the series drop out. Thus  $[(\partial_t T^\ast_t)T_t,\Ncal]$ can be estimated in the same way as $(\partial_t T^\ast_t)T_t$ in Lemma \ref{lm:timederivative}.
\end{proof}


\begin{prp}[Bound on Number of Fluctuations]
\label{prp:Nbound}
Let $\Psi \in \fock$ with $\scal{T^\ast_0 \Psi}{\Lcal_N(0)T^\ast_0 \Psi}
\leq C$ and  $\scal{\Psi}{\left(\frac{1}{N}\Ncal^2 + \Ncal \right)\Psi} \leq
C$ for some constant $C$ (independent of $N$). Then
\bd
\scal{U_t T^\ast_0 \Psi}{\left(\Ncal+1\right)U_t T^\ast_0 \Psi} \leq C e^{K_1 e^{K_2 t}}
\ed
for some constants $C$, $K_1$ and $K_2$ (all independent of $N$ and $t$).
\end{prp}
\begin{proof}
By Lemma \ref{lem:Llowerbound}, Corollary \ref{cor:N2} and Theorem \ref{t:pdes}, we obtain
\bd
\scal{U_t T^\ast_0 \Psi}{\Lcal_N(t) U_t T^\ast_0 \Psi} \geq - C_1 e^{M_1 t}\scal{U_t T^\ast_0 \Psi}{(\Ncal+1)U_t T^\ast_0 \Psi},
\ed
with constants $C_1$ and $M_1$ (which we keep fixed in this proof). Therefore the operator
\bd
\tLcal := \Lcal_N(t) + C_1 e^{M_1 t}(\Ncal+1)
\ed
has non-negative expectation value $\scal{U_t T^\ast_0 \Psi}{\tLcal U_t T^\ast_0 \Psi} \geq 0$, so it is sufficient to prove an upper bound for $u(t) := \scal{U_t T^\ast_0 \Psi}{(\tLcal+\Ncal+1)U_t T^\ast_0 \Psi}$. We want to apply Gronwall's Lemma, so we calculate the time derivative:
\begin{align*}
 u'(t) & = \scal{U_t T^\ast_0 \Psi}{\dot \Lcal_N(t)U_t T^\ast_0 \Psi} + C_1 M_1 e^{M_1 t}\scal{U_t T^\ast_0 \Psi}{(\Ncal+1)U_t T^\ast_0 \Psi} \\
& \quad + (C_1 e^{M_1 t}+1)\scal{U_t T^\ast_0 \Psi}{i[\Lcal_N(t),\Ncal]U_t T^\ast_0 \Psi}.
\end{align*}
These expectation values can be bounded using Lemma \ref{lem:ldotbounds}, Lemma \ref{lem:lncommutatorbound} and Theorem \ref{t:pdes}:
\bd
u'(t) \leq C e^{M t} \scal{U_t T^\ast_0 \Psi}{\left(\frac{1}{N}\Ncal^2+\Ncal+1+\Kcal+\tilV \right)U_t T^\ast_0 \Psi}.
\ed
By Lemma \ref{lem:kvbounds} and Corollary \ref{cor:N2} we can eliminate first $\Kcal + \tilV$ and then $\Ncal^2/N$ at the cost of $\Lcal_N(t)$ appearing. Then we use Theorem \ref{t:pdes} to estimate the $H^n$ norms. Then we add non-negative summands $\Ncal+1$ with an appropriate exponential prefactor to convert the $\Lcal_N(t)$ on the rhs into $\tLcal$ and arrive at
\bd
u'(t) \leq \tilde C e^{\tilde M t} u(t).
\ed
We now apply Gronwall's Lemma to conclude that
\bd
u(t) \leq u(0) \exp(K_1 e^{K_2 t})
\ed
for some constants $K_1$ and $K_2$ (both independent of $N$). The factor $u(0)$ is given by
\bd
u(0) = \scal{T^\ast_0 \Psi}{(\tLcalo + \Ncal+1)T^\ast_0 \Psi}.
\ed
By Proposition \ref{p:TNT}, we know that $\scal{T^\ast_0
\Psi}{(\Ncal+1)T^\ast_0 \Psi} \leq C\scal{\Psi}{(\Ncal+1)\Psi}$.
So using the assumption we conclude that $u(0) \leq C$ for some constant $C$.
\end{proof}

\iffalse
\begin{lem}
\label{lem:tilV}\marginpar{not needed for factorized states, but might become helpful for more general initial states}
There exists a constant $C$ such that
\bd
\ev{\tilV} \leq C \ev{\Vcal_N} + C \norm{\ph}_{H^2}^2\ev{\frac{1}{N}(\Ncal+1)^2}.
\ed
 Therefore we have the following example satisfying the assumptions of Proposition \ref{prp:Nbound}:
$\Psi = T_0 \tilde \Psi$ where $\tilde \Psi$ satisfies $\scal{\tilde \Psi}{\Hcal_N \tilde \Psi} \leq C$. 
\end{lem}
\begin{proof}
Plug in $a(c_x) = a_x + a(p_x)$, multiply out and use Hoelder's inequality.
\end{proof}\fi

\subsection{A-priori estimates}
\label{ss:apriori}
For $\varphi \in L^2(\Rbb^3)$ and $k \in L^2(\Rbb^3\times \Rbb^3)$, we use the short-hands
\[
  W = W(\sqrt{N} \varphi) \qquad \text{and} \qquad T = T(k)
\]
if the possible time-dependence does not play any role.

Consider two operators $A$ and $B$ on the Fock space $\mathcal{F}$. The
notation $ A \le B$ means that $\langle \psi, A \psi \rangle \le \langle \psi, B \psi \rangle$ for
all $\psi \in \mathcal{F}$.


\begin{lem} \label{l:ap}
There exists a constant $C$, % uniform in $N$ and $t$,
      that depends only on $\| \varphi \|_{H^1}$ and $V$, such that
        \[
          U_t^* \N^2 U_t \le C \big( N U_t^* \N U_t + N (\N+1) + (\N+1)^2
          \big).
        \]
\end{lem}


The proof of Lemma \ref{l:ap} is based on two propositions, which we prove
first.


\begin{prp} \label{p:TNT}
  Let $k \in L^2(\R^3 \times \R^3)$ with $\nabla_2 k \in L^2(\R^3 \times
  \R^3)$. Then there exists a constant $C$, that depends only on $\| k \|_{L^2}$, such that:
  \begin{align}
    T^* \N T & \le C (\N+1), \label{TNT} \tag{i} \\
%    T^* \K T & \le C (\K + \| \nabla_2 k \|_{L^2}^2), \label{TKT} \tag{ii} \\
    T^* \N^2 T & \le C (\N+1)^2. \label{TN2T} \tag{ii}% \\
%    \phi(f) \K \phi(f) & \le C_1 \big( \| f \|_{L^2}^2 \K (\N+1) + \| \nabla f \|_{L^2}^2 (\N + 1) \big). \label{fKf} \tag{iv}
  \end{align}
  Moreover, the same inequalities hold with $T$ replaced by $T^*$.
\end{prp}

\begin{rem}
One can similarly prove that $T^* \K T \le C (\K + \| \nabla_2 k \|_{L^2}^2)$, where for $k(x,y) = -N w_N(x-y) \varphi(x)\varphi(y)$ we have $\| \nabla_2 k \|_{L^2}^2$ of order $N$. Compared to Proposition \ref{p:TNT} \eqref{TNT} this shows that the two-particle correlations implemented by $T$ are small compared to the number of particles, but have a large influence on the kinetic energy. This is due to the fact that short-scale fluctuations make for a large derivative of the wave-function. 
\end{rem}

\begin{prp} \label{p:f}
  Let $\varphi \in L^2(\R^3)$ with $\| \varphi \|_{L^2} = 1$, and let $k \in
  L^2(\R^3 \times \R^3)$. The following pull-through formulae hold:
  \begin{align}
    \N W(\sqrt{N} \varphi)^* & = W(\sqrt{N} \varphi)^* (\N - \sqrt{N}
    \phi(\varphi) + N), \tag{i} \label{f1} \\
    \N W(\sqrt{N} \varphi) & = W(\sqrt{N} \varphi) (\N + \sqrt{N}
    \phi(\varphi) + N), \tag{ii} \label{f2} \\
    W(\sqrt{N} \varphi)^* \phi(\varphi) & = (\phi(\varphi) + 2 \sqrt{N})
    W(\sqrt{N} \varphi)^*, \tag{iii} \label{f3} \\
    \phi(\varphi) T(k) & = T(k) \phi(C \varphi + S \overline{\varphi}),
    \tag{iv} \label{f4} 
  \end{align}
  where $C$ and $S$ are integral operators on $L^2(\R^3)$, with integral
  kernels $c$ and $s$ as in Lemma \ref{l:bt}.
\end{prp}

\begin{cor}
\label{cor:N2}
If $\Psi \in \fock$ with a constant $C_1$ such that $\scal{\Psi}{\left( \frac{1}{N}\Ncal^2+\Ncal \right)\Psi} \leq C_1 \scal{\Psi}{\Psi}$, then there exists a constant $C_2$ such that
\bd
\frac{1}{N}\scal{U_t T^\ast_0 \Psi}{\Ncal^2 U_t T^\ast_0 \Psi} \leq C_2 \scal{U_t T^\ast_0 \Psi}{\left(\Ncal+1\right)U_t T^\ast_0 \Psi}.
\ed
\end{cor}
\begin{proof}[Proof of Corollary \ref{cor:N2}]
This is evident from Lemma \ref{l:ap} and Proposition \ref{p:TNT}.
\end{proof}

\begin{proof}[Proof of Proposition \ref{p:f}]
  We give only an outline of the proof. Recall that $\phi(\varphi) =
  a^*(\varphi) + a(\varphi)$. Then parts \eqref{f1}, \eqref{f2} and \eqref{f3}
  follow easily by a brief calculation using parts \ref{l:W1} and \ref{l:W3}
  of Lemma \ref{l:W}. Similarly, part \eqref{f4} follows from Lemma
  \ref{l:bt}.
\end{proof}


\begin{proof}[Proof of Proposition \ref{p:TNT}]
  We first remark that it is clear that all the steps in the proof hold with
  $T$ replaced by $T^*$ because $T^* a_x T = a(c_x) + a^*(s_x)$ and $T a_x T^*
  = a(c_x) - a^*(s_x)$ differ only by a minus sign.


  \eqref{TNT} Write
  \[
    \langle \psi, T^* \N T \psi \rangle = \int dx \, \langle T^* a_x T \psi,
    T^* a_x T \psi \rangle = \int dx \, \| (a_x + a(p_x) + a^*(s_x)) \psi
    \|^2.
  \]
  \iffalse and
  \[
    \langle \psi, T^* \K T \psi \rangle = \int dx \, \langle T^* \nabla_x a_x
    T \psi, T^* \nabla_x a_x T \psi \rangle = \int dx \, \| (\nabla a_x +
    a(\nabla_x p_x) + a^*(\nabla_x s_x)) \psi \|^2.
  \]\fi
  Then, by Cauchy-Schwarz inequality, and Lemma \ref{l:a},
  \begin{align*}
    \langle \psi, T^* \N T \psi \rangle & \apprle \int dx \, \| a_x \psi \|^2
    + \int dx \, \| a(p_x) \psi \|^2 + \int dx \, \| a^*(s_x) \psi \|^2 \\
    & \apprle \langle \psi, \N \psi \rangle + \int dx \, \| p_x \|_{L^2}^2
    \langle \psi, \N \psi \rangle + \int dx \, \| s_x \|_{L^2}^2 \langle
    \psi, (\N+1) \psi \rangle \\
    & \apprle (1 + \| p \|_{L^2}^2 + \| s \|_{L^2}^2) \langle \psi, (\N+1) \psi
    \rangle.
  \end{align*}
  \iffalse and
  \begin{align*}
    \langle \psi, T^* \K T \psi \rangle & \apprle \int dx \, \| \nabla_x a_x
    \psi \|^2 + \int dx \, \| a(\nabla_x p_x) \psi \|^2 + \int dx \, \|
    a^*(\nabla_x s_x) \psi \|^2 \\
    & = \langle \psi, \K \psi \rangle + \int dx \, \| a(\nabla_x p_x) \psi
    \|^2 + \int dx \, \| a(\nabla_x s_x) \psi \|^2 + \| \nabla_2 s \|_{L^2}^2
    \\
    & \apprle (1 + \| p \|_{L^2}^2 + \| s \|_{L^2}^2) \langle \psi, \K \psi
    \rangle + \| \nabla_2 s \|_{L^2}^2.
  \end{align*}\fi
  Furthermore by Lemma \ref{l:bt}
  \begin{equation} \label{c1}
    \| p \|_{L^2}^2 + \| s \|_{L^2}^2 \le 2 e^{2 \| k \|_{L^2}}.%  \| \nabla_2 s \|_{L^2}^2 \le e^{C_1 (1+\|k \|_{L^2})} \| \nabla_2 k \|_{L^2}^2.
  \end{equation}
  All this together proves parts \eqref{TNT}.
  

  \eqref{TN2T} Write
  \begin{align*}
    & \langle \psi, T^* \N^2 T \psi \rangle = \int dxdy \, \langle \psi, T^* a_x^* a_x a_y^* a_y T \psi \rangle \\
    & = \int dx \, \langle \psi, T^* a_x^* \N a_x T \psi \rangle + \langle
    \psi, T^* \N T \psi \rangle \\
    & = \int dx \, \langle (a_x + a(p_x) + a^*(s_x)) \psi, T^* \N T (a_x +
    a(p_x) + a^*(s_x)) \psi \rangle + \langle \psi, T^* \N T \psi \rangle.
  \end{align*}
  Then, applying part \eqref{TNT} and Cauchy-Schwarz inequality, and using the
  pull-through formula $a_x \N^{1/2} = (\N+1)^{1/2} a_x$ and Lemma \ref{l:a},
  we obtain
  \begin{align*}
    & \langle \psi, T^* \N^2 T \psi \rangle \\
    & \apprle C \int dx \, \| (\N+1)^{1/2} (a_x + a(p_x) +
    a^*(s_x)) \psi \|^2 + C \langle \psi, (\N+1) \psi \rangle \\
    & \apprle C \int dx \, (\| a_x \N^{1/2} \psi \|^2 + \| a(p_x)
    \N^{1/2} \psi \|^2 + \| a^*(s_x) (\N+2)^{1/2} \psi \|^2 ) + C \langle
    \psi, (\N+1) \psi \rangle \\
    & \apprle C(1 + \| p \|_{L^2}^2 + \| s \|_{L^2}^2) \langle \psi, (\N+1)^2
    \psi \rangle.
  \end{align*}
  Recalling \eqref{c1}, this proves part \eqref{TN2T}.
\iffalse
  \eqref{fKf} By Cauchy-Schwarz inequality, a brief calculation, and Lemma
  \ref{l:a},
  \begin{align*}
    \langle \psi, \phi(\varphi) \K \phi(\varphi) \psi \rangle & = \| \K^{1/2}
    (a^*(\varphi) + a(\varphi) ) \psi \|^2 \\
    & \apprle \langle \psi, a(\varphi) \K a^*(\varphi) \psi \rangle + \langle
    \psi, a^*(\varphi) \K a(\varphi) \psi \rangle \\
    & = \int dx \, \| a^*(\varphi) \nabla_x a_x \psi \|^2 + \int dx \, \|
    a(\varphi) \nabla_x a_x \psi \|^2 + \int dx \, |\nabla \varphi(x)|^2 \\
    & \quad - 2 \Re \int dx dy \, \langle \nabla \varphi(x) a_y \psi, f(y)
    \nabla_x a_x \psi \rangle \\
    & \apprle \| \varphi \|_{L^2}^2 \int dx \, \| (\N + 1)^{1/2} \nabla_x a_x
    \psi \|^2 + \int dx \, | \nabla \varphi(x) |^2 \\
    & \quad + \int dx dy \, | \nabla \varphi(x)|^2 \| a_y \psi \|^2 + \int dx
    dy \, |f(y)|^2 \| \nabla_x a_x \psi \|^2 \\
    & = \| \varphi \|_{L^2}^2 \langle \psi, \K (\N + 1) \psi \rangle + \|
    \nabla \varphi \|_{L^2}^2 \langle \psi, (\N + 1) \psi \rangle.
  \end{align*}
  This proves part \eqref{fKf} and completes the proof of the proposition.\fi
\end{proof}


\begin{proof}[Proof of Lemma \ref{l:ap}]
  Observe that, by Lemmata \ref{l:bt}, \ref{l:kernels}\ref{k} and Theorem
  \ref{t:pdes}\eqref{H1}, there exists a constant $C$, %uniform in $N$ and $t$,
  that depends only on $\| \varphi \|_{H^1}$ and $V$, such that
  \begin{equation} \label{c2}
    %\begin{gathered}
      \| k \|_{L^2} \le C %, \qquad \| \nabla_2 k \|_{L^2} \le C \sqrt{N}, \qquad \| \nabla_2 (k \overline{k}) \|_{L^2} \le C. \\
\quad\text{and}\quad
      \| p \|_{L^2}^2 + \| s \|_{L^2}^2 \le C%, \qquad \| \nabla \ph \|_{L^2}^2 + \| \nabla_2 p \|_{L^2}^2 \le C, \qquad \| \nabla_2 s \|_{L^2}^2 \le C N.
    %\end{gathered}
  \end{equation}
  By parts \eqref{TNT} and \eqref{TN2T} of Proposition \ref{p:TNT},
  \begin{equation}
    \label{ep4}
    \begin{split}
      \langle U_t \psi, \N^2 U_t \psi \rangle & = \langle T_t U_t \psi, T_t
      \N^2 T_t^* T_t U_t \psi \rangle \\
      & \le C \langle T_t U_t \psi, (\N+1)^2 T_t U_t \psi \rangle \\
      & \le C \langle T_t U_t \psi, \N^2 T_t U_t \psi \rangle + 2 C \langle
      U_t \psi, \N U_t \psi \rangle + C\scal{\psi}{\psi}. \\
    \end{split}
  \end{equation} 
  Now, using the formulae in Proposition \ref{p:f} and noting that $\N$
  commutes with $\mathcal{H}_N$,
  \begin{align*}
    & \langle T_t U_t \psi, \N^2 T_t U_t \psi \rangle \\
    & = \langle T_t U_t \psi, \N^2 W_t^* e^{-it \mathcal{H}_N} W_0 T_0 \psi \rangle
    \\
    & = \langle T_t U_t \psi, \N W_t^* \left(\N - \sqrt{N} \phi(\ph) + N\right)
    e^{-it \mathcal{H}_N} W_0 T_0 \psi \rangle \\
    & = N \langle T_t U_t \psi, \N T_t U_t \psi \rangle - \sqrt{N} \langle T_t
    U_t \psi, \N \left( \phi(\ph) + 2\sqrt{N}\right) W_t^* e^{-it\mathcal{H}_N} W_0 T_0 
    \psi \rangle \\
    & \quad + \langle T_t U_t \psi, \N W_t^* e^{-it\mathcal{H}_N} W_0 \left(\N +
    \sqrt{N} \phi(\varphi) + N\right) T_0 \psi \rangle \\
    & = - \sqrt{N} \langle T_t U_t \psi, \N \phi(\ph) W_t^*
    e^{-it\mathcal{H}_N} W_0 T_0 \psi \rangle + \sqrt{N} \langle T_t U_t \psi, \N
    W_t^* e^{-it\mathcal{H}_N} W_0 T_0 \phi(C \varphi + S \overline{\varphi}) \psi
    \rangle \\
    & \quad + \langle T_t U_t \psi, \N W_t^* e^{-it\mathcal{H}_N} W_0 \N T_0 \psi
    \rangle \\
    & = - \sqrt{N} \langle \N T_t U_t \psi, \phi(\ph) T_t U_t \psi
    \rangle + \sqrt{N} \langle \N T_t U_t \psi, W_t^* e^{-it\mathcal{H}_N} W_0 T_0 \phi(C \varphi + S
    \overline{\varphi}) \psi \rangle \\
    & \quad + \langle \N T_t U_t \psi, W_t^* e^{-it\mathcal{H}_N} W_0 \N T_0 \psi \rangle.
  \end{align*}
  Hence, applying Cauchy-Schwarz inequality and Lemma \ref{l:a}, and observing
  that the operator norm of $P$ and $S$ (the integral operators with kernel $p$ resp.\ $s$) is controlled by their Hilbert-Schmidt
  norm, we get
  \begin{align*}
    & \langle T_t U_t \psi, \N^2 T_t U_t \psi \rangle \\
    & \le \| \N T_t U_t \psi \| \left( \sqrt{N} \| \phi(\ph) T_t U_t \psi \|
    + \sqrt{N} \| \phi(C \varphi + S \overline{\varphi}) \psi \| + \| \N T_0
    \psi \| \right) \\
    & \le \| \N T_t U_t \psi \| \Big( 2 \sqrt{N} \| \ph \|_{L^2} \|
    (\N+1)^{1/2} T_t U_t \psi \| \\
    & \quad + 2 \sqrt{N} ( 1 + \| p \|_{L^2} + \| s \|_{L^2}) \| \varphi
    \|_{L^2} \| (\N+1)^{1/2} \psi \| + \| \N T_0 \psi \| \Big) \\
    & = \frac{1}{2}\| \N T_t U_t \psi \| \Big( 4 \sqrt{N} \| \ph \|_{L^2} \|
    (\N+1)^{1/2} T_t U_t \psi \| \\
    & \quad + 4 \sqrt{N} ( 1 + \| p \|_{L^2} + \| s \|_{L^2}) \| \varphi
    \|_{L^2} \| (\N+1)^{1/2} \psi \| +  2 \| \N T_0 \psi \| \Big),
\intertext{then by using $ab \leq a^2 + b^2$ we get}
    & \le 3\frac{1}{4} \| \N T_t U_t \psi \|^2 + 16N \| (\N + 1)^{1/2} T_t U_t
    \psi \|^2 \\
    & \quad + 16N (1 + \| p \|_{L^2} + \| s \|_{L^2})^2 \| (\N + 1)^{1/2}
    \psi \|^2 + 4\| \N T_0 \psi \|^2.
  \end{align*}
  We move $\langle T_t U_t \psi, \N^2 T_t U_t \psi \rangle$ to the lhs and then use Proposition \ref{p:TNT}, so recalling \eqref{c2} we get
  \begin{align*}
    & \langle T_t U_t \psi, \N^2 T_t U_t \psi \rangle \\
    & \le C \Big( N \langle U_t \psi, T_t^* (\N+1) T_t U_t \psi \rangle + N
    \langle \psi, (\N+1) \psi \rangle + \langle \psi, T^*_0 \N^2 T_0 \psi \rangle
    \Big) \\
    & \le C \Big( N \langle U_t \psi, (\N+1) U_t \psi \rangle + N \langle
    \psi, (\N+1) \psi \rangle + \langle \psi, (\N+1)^2 \psi \rangle \Big),
  \end{align*}
  for some constants $C$ that depend only on $\| \varphi \|_{H^1}$
  and $V$. Combining this with \eqref{ep4}, we obtain
  \begin{displaymath}
    \langle U_t \psi, \N^2 U_t \psi \rangle 
    % \le C \langle T_t U_t \psi, \N^2 T_t U_t \psi \rangle + 2 C^2
    %\langle U_t \psi, \N U_t \psi \rangle + C. \\
     \le C \Big( N \langle U_t \psi, \N U_t \psi \rangle + N \langle
    \psi, (\N+1) \psi \rangle + \langle \psi, (\N+1)^2 \psi \rangle \Big).
  \end{displaymath}
  This concludes the proof.
\end{proof}



\section{Convergence of Expectation Values}
\label{s:expectation}
Let $P_N$ be the projection on the $N$-particle subspace of Fock space.
\begin{lem}
\label{lem:TrJ}
Let $\Psi \in \fock$ and assume there is a constant $C_1$ such that $\scal{\Psi}{\left(\frac{1}{N}\Ncal^2 + \Ncal\right)\Psi} \leq C_1 \scal{\Psi}{\Psi}$. Let $J$ be a Hilbert-Schmidt operator on $L^2(\Rbb^3)$. Then there exists a constant $C_2$ (independent of $N$ and $t$) such that
\bd
\left\lvert \Tr\left(J \gamma_{N,t}^{(1)} \right) - \Tr\left( J \project{\ph}  \right) \right\rvert \leq \frac{d_N}{\sqrt{N}}C_2 \norm{J}\HS \Big( \scal{U_N(t) T^\ast_0 \Psi}{\left( \Ncal+1 \right)U_N(t) T^\ast_0 \Psi} \Big)^{1/2},
\ed
where $d_N = \norm{P_N W_0\Psi}^{-1}$ and $\gamma_{N,t}^{(1)} = \gamma_{\psi_{N,t}}^{(1)}$ is the one-particle reduced density matrix associated with the state $\psi_{N,t} = e^{-it \Hcal_N} d_N P_N W_0 \Psi \in \fock$.
%
%Remark: $\ph$ needn't be a solution of the Hartree equation here.
\end{lem}
\begin{proof}
Recall the definition \eqr{fock_density} of the reduced density matrix. 

We start by rewriting $\Tr\left(J \gamma_{N,t}^{(1)} \right)$ in an appropriate way. The operator $P_N$ commutes with $e^{-it \Hcal_N}$ and with $a^\ast_x a_y$, so because of $P_N^2 = P_N$ one of the projections $P_N$ can be eliminated. We then insert $W^\ast_t W_t$ and apply Lemma \ref{l:W} \ref{l:W3}:
\begin{align}
& \Tr\left(J\gamma_{N,t}^{(1)}\right) = \int \di x \di y\, J(x;y) \gamma_{N,t}^{(1)}(y;x) \nonumber\\
& = \frac{d_N^2}{N} \int \di x\di y J(x,y) \scal{e^{-it\Hcal_N}P_N W_0 \Psi}{ a^\ast_x a_y e^{-it\Hcal_N t}P_N W_0 \Psi} \nonumber\\
& = \frac{d_N^2}{N} \int \di x\di y J(x,y) \scal{e^{-it\Hcal_N}P_N W_0 \Psi}{ a^\ast_x a_y e^{-it\Hcal_N t}W_0 \Psi} \nonumber\\
& = \frac{d_N^2}{N} \int \di x\di y J(x,y) \scal{W^\ast_t e^{-it\Hcal_N}P_N W_0 \Psi}{\left(a^\ast_x + \sqrt{N} \cc{\ph(x)} \right) \times \nonumber\\
& \qquad\qquad\qquad\quad \times \left(a_y + \sqrt{N} \ph(y) \right) W^\ast_t e^{-it\Hcal_N t}W_0 \Psi} \nonumber\\
& = \frac{d_N^2}{N} \int \di x\di y J(x,y) \scal{W^\ast_t e^{-it\Hcal_N}P_N W_0 \Psi}{a^\ast_x a_y W^\ast_t e^{-it \Hcal_N} W_0 \Psi} \label{eq:quad}\\
& \quad + \frac{d_N^2}{N} \int \di x\di y J(x,y) \sqrt{N} \ph(y) \scal{W^\ast_t e^{-it\Hcal_N} P_N W_0 \Psi}{a^\ast_x W^\ast_t e^{-it \Hcal_N} W_0 \Psi} \label{eq:lin1} \\
& \quad + \frac{d_N^2}{N} \int \di x\di y J(x,y) \sqrt{N} \cc{\ph(x)} \scal{W^\ast_t e^{-it\Hcal_N}P_N W_0 \Psi}{a_y W^\ast_t e^{-it\Hcal_N} W_0 \Psi} \label{eq:lin2} \\
& \quad + \frac{d_N^2}{N} \int \di x\di y J(x,y) N \cc{\ph(x)}\ph(y) \scal{W^\ast_t e^{-it\Hcal_N} P_N W_0 \Psi}{W^\ast_t e^{-it\Hcal_N} W_0 \Psi}. \label{eq:gamma}
\end{align}
The last line, \eqref{eq:gamma}, is the expectation value of $J$ in the one-particle state given by the Gross-Pitaevskii equation:
\begin{align*}
& \frac{d_N^2}{N} \int \di x\di y J(x,y) N \cc{\ph(x)}\ph(y) \scal{W^\ast_t e^{-it\Hcal_N} P_N W_0 \Psi}{W^\ast_t e^{-it\Hcal_N} W_0 \Psi} \\
& = \frac{d_N^2}{N} \int \di x\di y J(x,y) N \cc{\ph(x)}\ph(y) \norm{P_N W_0 \Psi}^2 \\
& = \int \di x\di y J(x,y) \ph(y) \cc{\ph(x)} = \Tr\left( J\project{\ph} \right).
\end{align*}
Summands \eqref{eq:quad} - \eqref{eq:lin2} have to be estimated. We start with line \eqref{eq:lin1}:
\begin{align*}
& \left\lvert \frac{d_N^2}{N} \int \di x\di y J(x,y) \sqrt{N} \ph(y) \scal{W^\ast_t e^{-it\Hcal_N} P_N W_0 \Psi}{a^\ast_x W^\ast_t e^{-it \Hcal_N} W_0 \Psi} \right\rvert \\
& \leq \frac{d_N^2}{N} \int \di y \sqrt{N} \lvert \ph(y)\rvert \norm{W^\ast_t e^{-it \Hcal_N} P_N W_0 \Psi} \norm{ a^\ast(J_y)W^\ast_t e^{-it \Hcal_N} W_0 \Psi} \\
& \leq \frac{d_N}{\sqrt{N}} \int \di y \lvert \ph(y)\rvert \norm{J_y}_{L^2} \norm{(\Ncal+1)^{1/2}W^\ast_t e^{-it\Hcal_N}W_0 \Psi} \\
& \leq \frac{d_N}{\sqrt{N}} \norm{\ph}_{L^2} \norm{J}_{\textrm{HS}} \Big( \scal{W^\ast_t e^{-it \Hcal_N}W_0 \Psi}{(\Ncal+1)W^\ast_t e^{-it\Hcal_N} W_0 \Psi} \Big)^{1/2} \\
& \leq \frac{d_N}{\sqrt{N}} \norm{J}_{\textrm{HS}} \Big(C \scal{U_t T^\ast_0 \Psi}{(\Ncal+1)U_t T^\ast_0 \Psi} \Big)^{1/2},
\end{align*}
where in the last step Proposition \ref{p:TNT} (\ref{TNT}) has been used in the form of $\Ncal \leq C T_t (\Ncal+1)T^\ast_t$.
Line \eqref{eq:lin2} can be estimated in the same way as line \eqref{eq:lin1}.

We now estimate line \eqref{eq:quad}. Recall that $\di\Gamma(J) := \int \di x\di y J(x,y) a^\ast_x a_y$ can be estimated by $\norm{\di\Gamma(J)\zeta} \leq \norm{J}_{\textrm{HS}} \norm{\Ncal \zeta}$ for all $\zeta \in \fock$. Then
\begin{align*}
& \left\lvert \frac{d_N^2}{N} \int \di x\di y J(x,y) \scal{W^\ast_t e^{-it\Hcal_N} P_N W_0 \Psi}{a^\ast_x a_y W^\ast_t e^{-it \Hcal_N}W_0 \Psi} \right\rvert \\
%& = \frac{d_N^2}{N} \scal{W^\ast_t e^{-it\Hcal_N}P_N W_0 \Psi}{\di\Gamma(J) W^\ast_t e^{-it\Hcal_N}W_0 \Psi} \\
& \leq \frac{d_N^2}{N} \norm{W^\ast_t e^{-it\Hcal_N}P_N W_0 \Psi} \norm{\di\Gamma(J)W^\ast_t e^{-it \Hcal_N}W_0 \Psi} \\
& \leq \frac{d_N^2}{N} d_N^{-1} \norm{J}_{\textrm{HS}} \norm{\Ncal W^\ast_t e^{-it\Hcal_N}W_0 \Psi} \\
& = \frac{d_N}{\sqrt{N}} \norm{J}_{\textrm{HS}} \left( \frac{1}{N} \scal{W^\ast_t e^{-it\Hcal_N}W_0 \Psi}{\Ncal^2 W^\ast_t e^{-it\Hcal_N}W_0 \Psi} \right)^{1/2} \\
& \leq \frac{d_N}{\sqrt{N}} \norm{J}_{\textrm{HS}} \left( \frac{C}{N} \scal{U_t T^\ast_0 \Psi}{(\Ncal+1)^2 U_t T^\ast_0 \Psi} \right)^{1/2}.
\end{align*}
In the last step, Proposition \ref{p:TNT} (\ref{TN2T}) has been used. The estimate is completed using Corollary \ref{cor:N2} to control the expectation of $\Ncal^2/N$ through $\Ncal+1$.
\end{proof}

\begin{lem}
\label{lem:phNtoph}
Let $\ph$ be the solution to the modified Hartree equation with initial data $\varphi_0 = \varphi$ and let $\varphi_t$ be the solution to the Gross-Pitaevskii equation with initial data $\varphi_0 = \varphi$.

Then there exists a constant $C$ such that for any Hilbert-Schmidt operator $J$ on $L^2(\Rbb^3)$ the following estimate holds:
\bd
\left\lvert \Tr\left(J \project{\ph}\right) - \Tr\left(J \project{\varphi_t}\right) \right\rvert \leq \norm{J}_{\textrm{HS}} \frac{C}{N} e^{e^{Kt}}.
\ed
\end{lem}
\begin{proof}
Using H\"older's inequality, we calculate that
\begin{align*}
& \left\lvert \int \di x \di y J(x,y) \ph(y) \cc{\ph(x)} - \int \di x\di y J(x,y) \varphi_t(y) \cc{\varphi_t(x)} \right\rvert \\ 
& \leq \int \di x \di y \lvert J(x,y)\rvert \lvert \ph(y) - \varphi_t(y)\rvert \lvert \ph(x)\rvert + \int \di x \di y \lvert J(x,y) \rvert \lvert \varphi_t(y)\rvert \lvert \ph(x)-\varphi_t(x)\rvert \\
& \leq \norm{\ph}_{L^2} \norm{J}_{\textrm{HS}} \norm{\ph - \varphi_t}_{L^2} + \norm{\varphi_t}_{L^2} \norm{J}_{\textrm{HS}} \norm{\ph - \varphi_t}_{L^2}. 
\end{align*}
The proof is finished noticing that $\norm{\varphi_t}_{L^2} = \norm{\ph}_{L^2} = 1$ and, by Lemma \ref{lem:phitN2phit}, $\norm{\ph - \varphi_t}_{L^2} \leq C e^{e^{KT}}/N$.
\end{proof}

\begin{proof}[Proof of Theorem \ref{thm:main_factorized}]
We choose $\Psi = \Omega$ in Lemma \ref{lem:TrJ}. Then $d_N P_N W_0 \Psi = \varphi^{\otimes N}$ and $d_N = \frac{\sqrt{N!}}{e^{-N/2}N^{N/2}} \simeq N^{1/4}$.
We want to apply Proposition \ref{prp:Nbound} %. Clearly $\scal{\Omega}{\left(\frac{1}{N}\Ncal^2 + \Ncal\right)\Omega} = 0$ satisfies the assumption of the proposition.
so we have to check the condition on 
\bd
\scal{T^\ast_0 \Omega}{\Lcal_N(0)T^\ast_0 \Omega} = \scal{\Omega}{\left(T_0(i\partial_t T^\ast_t)\rvert_{t=0} + \Lcal_N^{(0)}(0)\right)\Omega}.
\ed
According to \eqref{eq:no3} $\Lcal_N^{(0)}(0)$ is normal-ordered and thus $\scal{\Omega}{\Lcal_N^{(0)}(0)\Omega} = 0$ (recall that the scalar term in $\Lcal_N^{(0)}(0)$ was dropped because it's just  a phase in the unitary evolution). The expectation value $\scal{\Omega}{T_0(i\partial_t T^\ast_t)\rvert_{t=0} \Omega}$ can be bounded by Lemma \ref{lm:timederivative} with $t=0$.

The proof is completed by employing Lemma \ref{lem:phNtoph}.
\end{proof}

To expose the naturalness of squeezed coherent states as initial data we prove the following theorem. (This theorem is not necessary for the proof of the main result.)
\begin{thm}[Time Evolution of Squeezed Coherent States] \label{thm:main_squeezed}
 Let $\psi_{N} = W(\sqrt{N} \varphi) T_0 \Omega$ for some $\varphi \in H^4(\Rbb^3)$ with $\norm{\varphi}_{L^2} = 1$. Denote by $\psi_{N,t} = e^{-it \Hcal_N}\psi_N$ the solution to the second-quantized Schr\"odinger equation in Fock space
\[i \partial_t \psi_{N,t} = \Hcal_N \psi_{N,t},\quad \mbox{with inital data } \psi_{N,0} = \psi_N.\]
Let $\Gamma_{N,t} := \gamma_{\psi_{N,t}}^{(1)}$ be the corresponding one-particle reduced density matrix.
Let $\varphi_t$ be the solution to the Gross-Pitaevskii equation with initial data $\varphi_0 = \varphi$.

 Then there exist constants $C$, $K_1$ and $K_2$, which do not depend on $N$ and $t$, such that for any Hilbert-Schmidt operator $J$ on $L^2(\Rbb^3)$
\bd
\left\lvert \Tr\left(J \Gamma_{N,t} \right) - \Tr\left( J \project{\varphi_t}  \right) \right\rvert \leq \norm{J}\HS \frac{C}{\sqrt{N}}e^{K_1 e^{K_2 t}}.
\ed
\end{thm}
\begin{proof} Notice first that $e^{-it \Hcal_N}$ commutes with $\Ncal$. Thus
using Proposition \ref{p:f} (\ref{f2}) and (\ref{f4}) and equation \eqr{no4} we easily calculate that $\scal{\psi_{N,t}}{\Ncal \psi_{N,t}} = \norm{s}^2 + N$. Thus $1/\scal{\psi_{N,t}}{\Ncal \psi_{N,t}} \leq 1/N$.

The remaining proof follows the strategy of the proof of Theorem \ref{thm:main_factorized} but the projection $P_N$ is now absent. We expand
\begin{align}
& \Gamma_{N,t}(y;x) = \scal{\psi_{N,t}}{a^\ast_x a_y \psi_{N,t}}/\scal{\psi_{N,t}}{\Ncal \psi_{N,t}} \nonumber \\
& = \scal{W^\ast_t e^{-i \Hcal_N t} W_0 T_0 \Omega}{a^\ast_x a_y W^\ast_t e^{-i\Hcal_N t} W_0 T_0 \Omega}\left(\norm{s}^2 + N\right)^{-1} \label{eq:squeez1} \\
& \quad + \sqrt{N} \ph(y) \scal{W^\ast_t e^{-i\Hcal_N t} W_0 T_0 \Omega}{a^\ast_x W^\ast_t e^{-i \Hcal_N t} W_0 T_0 \Omega}\left(\norm{s}^2 + N\right)^{-1} \label{eq:squeez3}\\
& \quad + \sqrt{N} \cc{\ph(x)} \scal{W^\ast_t e^{-i\Hcal_N t} W_0 T_0 \Omega}{a_y W^\ast_t e^{-i \Hcal_N t} W_0 T_0 \Omega} \left(\norm{s}^2 + N\right)^{-1} \label{eq:squeez2}\\
& \quad + N \ph(y) \cc{\ph(x)} \scal{W^\ast_t e^{-i\Hcal_N t} W_0 T_0 \Omega}{W^\ast_t e^{-i\Hcal_N t} W_0 T_0 \Omega}\left(\norm{s}^2 + N\right)^{-1}. \label{eq:squeez4}
\end{align}
Clearly line \eqref{eq:squeez4} is equal to $\frac{N}{\norm{s}^2+N} \ph(y) \cc{\ph(x)}$. Using H\"older's inequality
\[
\left\lvert \frac{N}{\norm{s}^2+N} \int \di x \di y\, J(x,y) \ph(y) \cc{\ph(x)} - \Tr\left( J \project{\ph}  \right) \right\rvert \leq \frac{C}{N} \norm{J}\HS,
\]
and according to Lemma \ref{lem:phNtoph}
\bd
\left\lvert \Tr\left(J \project{\ph}\right) - \Tr\left(J \project{\varphi_t}\right) \right\rvert \leq \norm{J}_{\textrm{HS}} \frac{C}{N} e^{e^{Kt}}.
\ed

We have to prove bounds for lines \eqref{eq:squeez1} and \eqref{eq:squeez3}; the proof for line \eqref{eq:squeez2} is similar to line \eqref{eq:squeez3}.
For line \eqref{eq:squeez3} we have the estimate
\begin{align*}
& \left\lvert \frac{\sqrt{N}}{\norm{s}^2+N} \int \di x\di y\, J(x,y) \ph(y) \scal{W^\ast_t e^{-i\Hcal_N t}W_0 T_0 \Omega}{a^\ast_x W^\ast_t e^{-i\Hcal_N t}W_0 T_0 \Omega}  \right\rvert \\
& \leq \frac{\sqrt{N}}{\norm{s}^2+N}\left\lvert \int \di y\, \ph(y) \norm{a^\ast(J_y)W^\ast_t e^{-i\Hcal_N t} W_0 T_0 \Omega} \right\rvert\\
& \leq \frac{C}{\sqrt{N}} \int \di y \lvert \ph(y)\rvert \norm{J_y}_{L^2} \norm{(\Ncal+1)^{1/2} W^\ast_t e^{-i\Hcal_N t}W_0 T_0 \Omega} \\
& \leq \frac{C}{\sqrt{N}} \norm{\ph}_{L^2} \norm{J}\HS \scal{W^\ast_t e^{-i\Hcal_N t}W_0 T_0 \Omega}{(\Ncal+1)W^\ast_t e^{-i\Hcal_N t}W_0 T_0 \Omega} \\
& \leq \frac{C}{\sqrt{N}} \norm{J}\HS \scal{U_t \Omega}{(\Ncal+1)U_t \Omega},
\end{align*}
where in the last step Proposition \ref{p:TNT} (\ref{TNT}) has been used.
For line \eqref{eq:squeez1} we have the estimate
\begin{align*}
& \left\lvert \frac{1}{\norm{s}^2+N} \int \di x\di y\, J(x,y) \scal{W^\ast_t e^{-i\Hcal_N t}W_0 T_0 \Omega}{a^\ast_x a_y W^\ast_t e^{-i\Hcal_N t}W_0 T_0 \Omega} \right \rvert \\
& \leq \frac{1}{\norm{s}^2+N} \norm{W^\ast_t e^{-i\Hcal_N t}W_0 T_0 \Omega}\norm{\di\Gamma(J) W^\ast_t e^{-i\Hcal_N t}W_0 T_0 \Omega} \\
& \leq \frac{1}{\sqrt{N}} \norm{J}\HS \left( \frac{\scal{W^\ast_t e^{-i\Hcal_N t}W_0 T_0 \Omega}{\Ncal^2 W^\ast_t e^{-i\Hcal_N t}W_0 T_0 \Omega}}{N} \right)^{1/2} \\
& \leq \frac{C}{\sqrt{N}} \norm{J}\HS \left( \frac{\scal{U_t \Omega}{(\Ncal+1)^2 U_t \Omega}}{N} \right)^{1/2} \\
& \leq \frac{C}{\sqrt{N}} \norm{J}\HS \left( \scal{U_t \Omega}{(\Ncal+1) U_t \Omega} \right)^{1/2},
\end{align*}
where in the last two steps Proposition \ref{p:TNT} (\ref{TN2T}) and Corollary \ref{cor:N2} were used.
According to Proposition \ref{prp:Nbound} $\scal{U_t \Omega}{\left(\Ncal + 1\right) U_t \Omega} \leq C e^{K_1 e^{K_2 t}}$, so the proof is complete.
\end{proof}

%%%%%%%%%%%%%%%%%%%%%%%%%%%% APPENDIX %%%%%%%%%%%%%%%%%%%%%%%%%%%%%%%%%%%

\appendix
\section{Estimates for the terms of the generator}
\label{ch:generatorestimates}
\label{s:generatorestimates}
The estimates in this section hold for the expectation value $\ev{\cdot} = \scal{\psi}{\cdot\,\psi}$ in any state $\psi \in \fock$. Especially, $C$ stands for constants independent of $\psi$ (and, as always, of $N$ and $t$).

In this section it is assumed that
\[k(x,y) = - N w_N(x-y)\ph(x)\ph(y)\]
with $\ph$ the solution to the modified Hartree equation with initial data $\varphi^{(N)}_0 = \varphi$.

For convenience, we remind the reader of the simple but repeatedly useful estimate
\bd
\int \di x\, \norm{a(c_x)\psi}^2 \leq 2(1+\norm{p}^2)\scal{\psi}{\Ncal\psi} \leq C\ev{\Ncal}.
\ed
We now have the following list of estimates for the terms of the generator.\newline
\begin{fleqn}[0.5em]	% move formulas to left side in this section, make it look less chaotic.
\estlist{7}{1} (here, the cancellation is important!)
Using integration by parts to move the derivative from $\nabla_x w_N(x-y) = - \nabla_y w_N(x-y)$ to $\ph(y) a_y$ we estimate
\begin{align*}
& \lvert \scal{\psi}{\frac{1}{2}\int \di x\, 2a^\ast(c_x) \int \di y\,a^\ast_y N\nabla_x w_N(x-y)\nabla_x \ph(x) 2\ph(y) \psi}\rvert \\
& \leq 2\left\lvert \int \di x \di y\, Nw_N(x-y) \nabla_x \ph(x)  \nabla_y \ph(y) \scal{a_y\psi}{a^\ast_x\psi} \right\rvert \\
& \quad + 2\left\lvert \int \di x \di y\,  Nw_N(x-y) \nabla_x \ph(x) \ph(y)  \scal{\nabla_y a_y\psi}{a^\ast_x \psi} \right\rvert \\
& \quad + 2\left\lvert \int \di x \di y\, Nw_N(x-y) \nabla_x  \ph(x) \nabla_y \ph(y) \scal{a_y \psi}{a^\ast(p_x)\psi} \right\rvert \\
& \quad + 2\left\lvert \int \di x \di y\, Nw_N(x-y) \nabla_x \ph(x) \ph(y) \scal{\nabla_y a_y\psi}{a^\ast(p_x)\psi} \right\rvert \\
&\leq 2\int \di y\,\lvert \nabla_y \ph(y)\rvert \norm{a_y \psi} \norm{a^\ast\big(Nw_N(\cdot -y)\nabla\ph(\cdot)\big)\psi} \\
& \quad + 2\int \di y\,\lvert \ph(y)\rvert \norm{\nabla_y a_y\psi} \norm{a^\ast\big(Nw_N(\cdot-y)\nabla\varphi(\cdot)\big)\psi} \\
& \quad + \int \di x\di y\, \frac{C}{\lvert x-y\rvert} \lvert \nabla_x\ph(x)\rvert \lvert \nabla_y \ph(y)\rvert \norm{a_y \psi} \norm{p_x} \norm{(\Ncal+1)^{1/2}\psi} \\
& \quad + \int \di x\di y\, \frac{C}{\lvert x-y\rvert} \lvert \nabla_x\ph(x)\rvert \lvert \ph(y)\rvert \norm{\nabla_y a_y\psi} \norm{p_x} \norm{(\Ncal+1)^{1/2}\psi} \\
&\leq C \norm{\Delta\ph}_{L^2} \norm{(\Ncal+1)^{1/2}\psi} \int \di y \lvert \nabla_y \ph(y) \rvert \norm{a_y \psi} \\
& \quad + C \norm{\Delta \ph}_{L^2} \norm{(\Ncal+1)^{1/2}\psi} \int \di y \frac{1}{\sqrt{\varepsilon}} \lvert \ph(y)\rvert \sqrt{\varepsilon} \norm{\nabla_y a_y\psi} \\
& \quad + C \norm{(\Ncal+1)^{1/2}\psi} \left(\int \di x\di y \lvert \nabla_x\ph(x)\rvert^2 \frac{\lvert\nabla_y \ph(y)\rvert^2}{\lvert x-y \rvert^2} \right)^{1/2} \left(\int \di x\di y \norm{a_y \psi}^2 \norm{p_x}^2 \right)^{1/2} \\
& \quad + C \norm{(\Ncal+1)^{1/2}\psi} \left(\int \di x\di y \lvert \nabla_x\ph(x)\rvert^2 \frac{\lvert \ph(y)\rvert^2}{\lvert x-y\rvert^2} \right)^{1/2} \left(\int \di x\di y \norm{\nabla_y a_y\psi}^2 \norm{p_x}^2 \right)^{1/2} \\
& \leq C \norm{\Delta \ph}_{L^2} \norm{(\Ncal+1)^{1/2}\psi} \norm{\nabla \ph}_{L^2} \norm{\Ncal^{1/2}\psi} \\
& \quad + C\norm{\Delta \ph}_{L^2} \frac{1}{\sqrt{\varepsilon}} \norm{(\Ncal+1)^{1/2}\psi}  \sqrt{\varepsilon}\norm{\Kcal^{1/2}\psi} \norm{\ph}_{L^2} \\
& \quad + C \norm{(\Ncal+1)^{1/2}\psi} \norm{\nabla \ph}_{L^2} \norm{\Delta \ph}_{L^2} \norm{p} \norm{\Ncal^{1/2}\psi} \\
& \quad + C \frac{1}{\sqrt{\varepsilon}}\norm{(\Ncal+1)^{1/2}\psi} \sqrt{\varepsilon}\norm{\Kcal^{1/2}\psi} \norm{\nabla \ph}_{L^2}^2 \norm{p} \\
& \leq C \norm{\ph}_{H^2} \left( (1+\frac{1}{\varepsilon})\ev{\Ncal+1} + \varepsilon\ev{\Kcal} \right) + C \left( \frac{1}{\varepsilon}\ev{\Ncal+1} + \varepsilon \ev{\Kcal} \right).
\end{align*}
\estlist{8}{1} (here, the cancellation is important!)
\begin{align*}
&\lvert \scal{\psi}{\frac{1}{2}\int \di x\di y\, 2a^\ast(c_x) a^\ast_y N w_N(x-y) \Delta_x \ph(x) \ph(y) \psi}\rvert\\
& \leq \int \di x\, \norm{a(c_x)\psi} \lvert \Delta_x \ph(x)\rvert \norm{a^\ast\big(Nw_N(x-\cdot)\ph(\cdot)\big)\psi} \\
& \leq \int \di x\, \norm{a(c_x)\psi} \lvert \Delta_x \ph(x)\rvert \norm{N w_N(x-\cdot)\ph(\cdot)}_{L^2} \norm{(\Ncal+1)^{1/2}\psi} \\
& \leq C \norm{\nabla\ph}_{L^2} \norm{(\Ncal+1)^{1/2} \psi} \left( \int \di x\, \lvert \Delta_x \ph(x)\rvert^2 \right)^{1/2} \left( \int \di x\, \norm{a(c_x)\psi}^2 \right)^{1/2} \\
& \leq C \norm{\ph}_{H^1} \norm{\ph}_{H^2} \ev{\Ncal+1} \\
& \leq C \norm{\ph}_{H^2} \ev{\Ncal+1},
\end{align*}
where $\norm{Nw_N(x-\cdot)\ph(\cdot)}_{L^2} \leq C\norm{\nabla \ph}_{L^2}$ was used.\\
\estlist{8}{2}
For this estimate, it is useful to know that
\begin{equation} \label{eq:bound1B}                                     
\begin{split}
 & \scal{\psi}{\frac{1}{2}\int \di x a^\ast(c_x) a(-\Delta_x c_x)\psi} = \int \di x\, \norm{\nabla_x a(c_x)\psi}^2 \\
 & \leq 2 \int \di x\, \norm{\nabla_x a_x \psi}^2 + 2\int \di x\, \norm{a(\nabla_x p_x)\psi}^2 \leq 2\ev{\Kcal} + 2\int \di x\, \norm{\nabla_x p_x}^2 \norm{\Ncal^{1/2}\psi}^2 \\
 & = 2\ev{\Kcal}+ 2\norm{\gradone p}^2 \ev{\Ncal} \leq C\ev{\Kcal + \Ncal}.
\end{split}
\end{equation}
Using \eqref{eq:bound1B} and integration by parts we get
\begin{align*}
& \lvert \scal{\psi}{\frac{1}{2}\int \di x\, 2a^\ast(c_x) \int \di y\, (-\Delta_x r(y,x))\psi} \rvert \\
& = \lvert \scal{\psi}{\int \di x\, \nabla_x a^\ast(c_x) \int \di y\, a^\ast_y \nabla_x r(y,x) \psi}\rvert \\
& \leq \int \di x\, \norm{\nabla_x a(c_x)\psi} \norm{a^\ast(\nabla_x r_x)\psi} \\
& \leq \left( \varepsilon \int \di x\, \norm{\nabla_x a(c_x) \psi}^2 \right)^{1/2} \left( \int \di x\, \norm{\nabla_x r_x}^2 \right)^{1/2} \frac{1}{\sqrt{\varepsilon}} \norm{(\Ncal+1)^{1/2}\psi} \\
& \leq \norm{\gradone r} \left( \varepsilon 2\ev{\Kcal} + \varepsilon 2 \norm{\gradone p}^2 \ev{\Ncal} + \frac{1}{\varepsilon}\ev{\Ncal+1} \right) \\
& \leq C \left( \varepsilon \ev{\Kcal} + (\varepsilon+\frac{1}{\varepsilon})\ev{\Ncal+1} \right).
\end{align*}
\estlist{9}{1}
First notice that
\begin{align*}
 & \scal{\psi}{\frac{1}{2}\int \di x\, a^\ast(c_x) a(-\Delta_x c_x)\psi} = \frac{1}{2}\int \di x \scal{a(\nabla_x c_x)\psi}{a(\nabla_x c_x)\psi} \\
& = \frac{1}{2}\int \di x \scal{\big( \nabla_x a_x + a(\nabla_x p_x) \big)\psi}{\big(  \nabla_x a_x + a(\nabla_x p_x) \big) \psi} \\
& = \scal{\psi}{\frac{1}{2}\Kcal \psi} + \frac{1}{2}\int \di x\scal{\nabla_x
a_x \psi}{a(\nabla_x p_x)\psi} \\
& \quad + \frac{1}{2}\int \di x\, \scal{a(\nabla_x p_x)\psi}{\nabla_x a_x \psi} + \frac{1}{2}\int \di x\, \norm{a(\nabla_x p_x)\psi}^2.
\end{align*}
Leaving away the last summand, which is non-negative, we get
\begin{equation}
\label{eq:bound1A}
\begin{split}
 \ev{\Kcal} & \leq \scal{\psi}{\int \di x\, a^\ast(c_x) a(-\Delta_x c_x)\psi} + 2\int \di x\, \sqrt{\varepsilon} \norm{\nabla_x a_x \psi} \frac{1}{\sqrt{\varepsilon}} \norm{a(\nabla_x p_x)\psi} \\
& \leq \scal{\psi}{\int \di x\, a^\ast(c_x) a(-\Delta_x c_x)\psi} + 2\left( \varepsilon \ev{\Kcal} + \frac{1}{\varepsilon}  \norm{\gradone p}^2 \ev{\Ncal}\right) \\
& \leq \scal{\psi}{\int \di x\,a^\ast(c_x) a(-\Delta_x c_x)\psi} + C\left( \varepsilon \ev{\Kcal} + \frac{1}{\varepsilon}\ev{\Ncal} \right).
\end{split}
\end{equation}
\estlist{9}{2}
We start with an elementary Lemma.
\begin{lem} \label{l:agradnorm}
 Let $g \in H^1(\Rbb^3)$ and $\psi \in \fock$. Then $\norm{a(\nabla g)\psi}  \leq \norm{g}_{L^2} \norm{\Kcal^{1/2}\psi}$.
\end{lem}
\begin{proof}
Write in terms of operator-valued distributions and use integration by parts. 
\end{proof}
Using the product rule and the identity $-\nabla_x w_N(y-x) = \nabla_y w_N(y-x)$ we get (where the $\nabla$ in the second summand of the first line has to be understood with respect to the 'dot'-variable)
\begin{align*}
 a(\nabla_x s_x) & = \cc{\ph(x)} a\big(-Nw_N(\cdot-x)\nabla\ph(\cdot)\big) + \cc{\ph(x)}\,a\Big(\nabla\big(N w_N(\cdot-x)\ph(\cdot)\big)\Big) \\
 & + \cc{\nabla_x \ph(x)} a\big(-Nw_N(\cdot-x)\ph(\cdot)\big) + a(\nabla_x r_x).
\end{align*}
Using $(a+b+c+d)^2 \leq 4(a^2+b^2+c^2+d^2)$ and Lemma \ref{l:agradnorm} we now conclude
\begin{align*}
 & \scal{\psi}{\frac{1}{2}\int \di x\, a^\ast(-\Delta_x s_x) a(s_x)\psi} = \frac{1}{2}\int \di x\, \norm{a(\nabla_x s_x)\psi}^2 \\
& \leq \frac{1}{2} \int \di x \Big( \lvert \ph(x)\rvert \norm{a\big(-Nw_N(x-\cdot)\nabla\ph(\cdot)\big)\psi} \\
& \qquad\qquad\quad + \lvert \ph(x)\rvert \norm{a\big(\nabla(-N w_N(x-\cdot)\ph(\cdot))\big)\psi} \\
& \qquad\qquad\quad + \lvert \nabla_x \ph(x)\rvert \norm{a\big(-Nw_N(x-\cdot)\ph(\cdot)\big)\psi} + \norm{a(\nabla_x r_x)\psi}  \Big)^2 \\
& \leq 2\int \di x\, \Big( \lvert \ph(x)\rvert^2 \norm{-N w_N(x-\cdot)\nabla\ph(\cdot)}_{L^2}^2 \norm{\Ncal^{1/2}\psi}^2  \\
& \qquad\qquad\quad + \lvert \ph(x)\rvert^2 \norm{-Nw_N(x-\cdot) \ph(\cdot)}_{L^2}^2 \norm{\Kcal^{1/2}\psi}^2  \\
& \qquad\qquad\quad + \lvert \nabla_x \ph(x)\rvert^2 \norm{-N w_N(x-\cdot) \ph(\cdot)}_{L^2}^2 \norm{\Ncal^{1/2}\psi}^2 + \norm{\nabla_x r_x}^2 \norm{\Ncal^{1/2}\psi}^2 \Big) \\
& \leq 2 \int \di x\, \Big( \lvert \ph(x)\rvert^2 \norm{\Delta \ph}_{L^2}^2 C \norm{\Ncal^{1/2}\psi}^2 + \lvert \ph(x)\rvert^2 \norm{\nabla \ph}_{L^2}^2 C \norm{\Kcal^{1/2}\psi}^2 \\
& \qquad\qquad\quad + \lvert \nabla_x \ph(x)\rvert^2 \norm{\nabla \ph}_{L^2}^2 C \norm{\Ncal^{1/2}\psi}^2 + \norm{\nabla_x r_x}^2 \norm{\Ncal^{1/2}\psi}^2 \Big)\\
& \leq C \norm{\ph}_{H^2}^2 \ev{\Ncal} + C\ev{\Kcal} + C\ev{\Ncal}. \tagg{bound2B}
\end{align*}
Here we made use of the following estimates: $\norm{-N w_N(x-\cdot) \nabla\ph(\cdot)}_{L^2}^2 \leq C\norm{\Delta \ph}_{L^2}$, $\norm{-N w_N(x-\cdot)\ph(\cdot)}^2 \leq C \norm{\nabla \ph}^2$  and $ \norm{\gradone r} \leq C$.\newline

\emph{In proving the following estimates, the important ideas are:}
\begin{itemize}
 \item As $c_x = \delta_x + p_x$, $a^*(c_x)$ is only defined as an operator-valued distribution. Thus each $a^\ast(c_x)$ has to be moved to the other argument of the scalar product as $a(c_x)$ which is a well-defined operator.
\item We use Cauchy-Schwarz so that at most two operators $a$ or $a^\ast$ act on each $\psi$. Usually following Cauchy-Schwarz we apply H\"older's inequality in the form
\bd
\begin{split}
& \int \di x \di y\, V_N(x-y) f(x,y) g(x,y) = \int \di x \di y\, \sqrt{V_N(x-y)}^2 f(x,y) g(x,y)\\
&  \leq \int \di x\di y\, V_N(x-y) \lvert f(x,y)\rvert^2 + \int \di x\di y\, V_N(x-y) \lvert g(x,y)\rvert^2.
\end{split}
\ed
 \item For summands containing $a(c_x) a(c_y) \psi$ we insert $\sqrt{\varepsilon} \frac{1}{\sqrt{\varepsilon}}$ and then use H\"older's inequality so that $\varepsilon\ev{\tilV}$ is obtained.
 \item The operators $a^\ast(s_x)$, $a(s_x)$, $a^\ast(p_x)$ and $a(p_x)$ are well-defined and can be estimated using Lemma \ref{l:a}.
 \item We can pull out $\sup_x \norm{s_x}^2$, $\sup_x \norm{p_x}^2$ and $\norm{\ph}_\infty$ from the integrals; then we can use that $\int \di y\, NV_N(x-y) = b_0$ is independent of $x$ (and of $N$).
\item For terms containing three or four operators $a$ or $a^\ast$ our upper bounds use $\Ncal^2$, so a prefactor $1/N$ is necessary, cf.\ Corollary \ref{cor:N2}. 
\end{itemize}
%
\estlist{10}{1}
By definition of $\tilV$
\bd
\scal{\psi}{\frac{1}{2}\int \di x\di y\, NV_N(x-y)\frac{1}{2N} a^\ast(c_x) a^\ast(c_y) a(c_y) a(c_x)\psi} = \ev{\tilV}.
\ed
\estlist{10}{2}
\begin{align*}
 & \lvert \scal{\psi}{\dxyNV \frac{1}{2N} 4 a^\ast(c_x)a^\ast(c_y) a^\ast(s_x)a(c_y)\psi}\rvert \\
& \leq \int \di x\di y V_N(x-y) \frac{\sqrt{\varepsilon}}{2} \norm{a(c_y) a(c_x)\psi} \frac{2}{\sqrt{\varepsilon}} \norm{a^\ast(s_x)a(c_y)\psi} \\
& \leq \frac{\varepsilon}{4} \int \di x\di y V_N(x-y) \scal{\psi}{a^\ast(c_x)a^\ast(c_y)a(c_y)a(c_x)\psi} + \frac{4}{\varepsilon} \int \di x\di y V_N(x-y) \norm{a^\ast(s_x)a(c_y)\psi}^2 \\
& \leq \varepsilon \ev{\tilV} + \frac{4}{\varepsilon}\int \di x\di y V_N(x-y) \norm{s_x}^2 \norm{a(c_y)\Ncal^{1/2}\psi}^2 \\
& \leq \varepsilon \ev{\tilV} + \frac{4}{\varepsilon} \sup_x \norm{s_x}^2 \int \di y\, \norm{a(c_y)\Ncal^{1/2}\psi}^2 \frac{b_0}{N}\\
& \leq \varepsilon \ev{\tilV} + \frac{4}{\varepsilon} \sup_x \norm{s_x}^2 \frac{b_0}{N} 2 (1+\norm{p}^2)\ev{\Ncal^2} \\
& \leq \varepsilon \ev{\tilV} + C\norm{\ph}_{H^2}^2 \frac{1}{\varepsilon} \frac{1}{N}\ev{\Ncal^2}.
\end{align*}
\estlist{11}{1}
\begin{align*}
& \lvert \scal{\psi}{\dxyNV \frac{1}{2N}2a^\ast(c_x)a^\ast(c_y) a^\ast(s_y) a^\ast(s_x) \psi}\rvert \\
& \leq \frac{1}{2}\int \di x\di y\, V_N(x-y)  \norm{a(c_y)a(c_x)\psi} \norm{a^\ast(s_y)a^\ast(s_x)\psi} \\
& \leq \varepsilon 2 \ev{\tilV} + \frac{1}{2\varepsilon} \int \di x\di y\, V_N(x-y) \norm{a^\ast(s_y)a^\ast(s_x)\psi}^2 \\
& \leq \varepsilon 2 \ev{\tilV} + \frac{1}{2\varepsilon} \int \di x\di y\, V_N(x-y) \norm{s_y}^2 \norm{s_x}^2 \ev{(\Ncal+2)^2} \\
& \leq \varepsilon 2 \ev{\tilV} + \frac{1}{2\varepsilon} \sup_x \norm{s_x}^2 \int \di y\, \norm{s_y}^2 \int \di x\, V_N(x-y) \ev{(\Ncal+2)^2} \\
& \leq \varepsilon 2\ev{\tilV} + C\norm{\ph}_{H^2}^2 \frac{1}{\varepsilon} \frac{1}{N}\ev{(\Ncal+2)^2}.
\end{align*}
\estlist{11}{2}
\begin{align*}
& \lvert \scal{\psi}{\dxyNV \frac{1}{2N} 2 a^\ast(c_x) a^\ast(s_x) a(s_y) a(c_y)\psi}\rvert \\
& \leq \frac{1}{2}\int \di x\di y\, V_N(x-y) \norm{a(s_x) a(c_x) \psi} \norm{a(s_y) a(c_y)\psi} \\
& \leq \int \di x\di y\, V_N(x-y) \norm{s_y}^2 \norm{a(c_y) \Ncal^{1/2}\psi}^2\\
& \leq \sup_y \norm{s_y}^2 \int \di y\, \norm{a(c_y)\Ncal^{1/2}\psi}^2 \int \di x\, V_N(x-y)\\
& \leq C \norm{\ph}_{H^2}^2 \frac{1}{N}\ev{\Ncal^2}.
\end{align*}
Here we made use of the fact that both summands originating from the Hoelder inequality are equal, as can be seen by interchanging the integration variables $x$ and $y$.\newline
\emph{\vspace{.3em}\\\Nestlist{12}{1}, \nestlist{12}{2} and \nestlist{13}{1}:} can be estimated in the same way as \nestlist{11}{2}.\newline
\estlist{14}{1} (here, cancellation is important!)
\begin{align*}
& \lvert \scal{\psi}{\dxyNV \frac{1}{N}a^\ast(c_x) a^\ast(c_y) \left( r(y,x)+\scal{p_y}{s_x} \right)\psi} \rvert \\
& \leq \varepsilon \dxyV \norm{a(c_y)a(c_x)\psi}^2 + \frac{1}{\varepsilon} \dxyV \lvert r(y,x)+\scal{p_y}{s_x} \rvert^2 \\
& \leq \varepsilon 2\ev{\tilV} + \frac{1}{\varepsilon} \int \di x\di y\, V_N(x-y) \left( C\lvert\ph(x)\rvert^2\lvert\ph(y)\rvert^2 +\norm{p_y}^2\norm{s_x}^2 \right) \\
& \leq \varepsilon 2\ev{\tilV} + \frac{C}{\varepsilon} \norm{\ph}_\infty^2 \int \di x \di y\, \lvert \ph(y)\rvert^2 V_N(x-y) + \frac{1}{\varepsilon} \sup_x\norm{s_x}^2 \int \di x\di y\, \norm{p_y}^2 V_N(x-y) \\
& \leq \varepsilon 2\ev{\tilV} + C\norm{\ph}_{H^2}^2 \frac{1}{\varepsilon N}.
\end{align*}
\estlist{14}{2}
\begin{align*}
& \lvert \scal{\psi}{\dxyNV \frac{1}{N} a^\ast(c_x) a^\ast(p_y) k(y,x)\psi} \rvert \\
& \leq \dxyV \norm{a(c_x)\psi} \norm{a^\ast(p_y)\psi} \lvert k(y,x)\rvert \\
\intertext{and using that $\lvert k(y,x)\rvert \leq N\lvert \ph(x)\rvert\lvert \ph(y)\rvert$ by Lemma \ref{l:kernels} \ref{kr} we get}
& \leq \dxyNV \norm{a(c_x)\psi}^2 \lvert\ph(x)\rvert^2 + \dxyNV \norm{a^\ast(p_y)\psi}^2 \lvert \ph(y)\rvert^2\\
& \leq \frac{1}{2} \norm{\ph}_\infty^2 \left( b_0 \int \di x\,\norm{a(c_x)\psi}^2 + b_0 \int \di y\,\norm{a^\ast(p_y)\psi}^2 \right) \\
& \leq C \norm{\ph}_{H^2}^2 \ev{\Ncal+1}.
\end{align*}
\estlist{15}{1}
\begin{align*}
& \lvert \scal{\psi}{\dxyNV \frac{1}{N}a^\ast(c_x)a(c_y) \scal{s_y}{s_x}\psi} \rvert \\
%& \leq \dxyV \lvert \scal{s_y}{s_x}\rvert \norm{a(c_x)\psi} \norm{a(c_y)\psi} \\
& \leq \dxyV \norm{s_y} \norm{s_x} \norm{a(c_x)\psi} \norm{a(c_y)\psi} \\
& \leq \int \di x \di y\, V_N(x-y) \norm{s_y}^2 \norm{a(c_y)\psi}^2 \\
%\leq & \frac{b_0}{N} \sup_y \norm{s_y}^2 \int \di x \norm{a(c_x)\psi}^2 \\
& \leq C \norm{\ph}_{H^2}^2 \frac{1}{N}\ev{\Ncal}. 
\end{align*}
\emph{\vspace{.3em}\\\Nestlist{15}{2} and \nestlist{15}{3}:} like \nestlist{15}{1}.\newline
\estlist{16}{1}
\begin{align*}
& \lvert \scal{\psi}{\dxyNV \frac{1}{N}a^\ast(c_x) a(s_y) \scal{c_y}{s_x}\psi} \rvert \\
& \leq \dxyV \norm{a(c_x)\psi} \norm{a(s_y)\psi} \lvert s(y,x)+\scal{p_y}{s_x} \rvert \\
& \leq \frac{C}{2}\int \di x\di y N V_N(x-y) \norm{a(c_x)\psi}^2 \lvert \ph(x)\rvert^2 + \frac{C}{2}\int \di x\di y N V_N(x-y) \norm{a(s_y)\psi}^2 \lvert \ph(y)\rvert^2\\
& \quad + \dxyV \norm{a(c_x)\psi}^2 \norm{s_x}^2 + \dxyV \norm{a(s_y)\psi}^2 \norm{p_y}^2 \\
& \leq \frac{C}{2} \norm{\ph}_\infty^2 \int \di x\, \norm{a(c_x)\psi}^2 \int \di y\, NV_N(x-y) + \frac{C}{2} \norm{\ph}_\infty^2 \int \di y\, \norm{a(s_y)\psi}^2 b_0 \\
& \quad + \frac{1}{2}\sup_x\norm{s_x}^2 \int \di x\, \norm{a(c_x)\psi}^2 \frac{b_0}{N} + \frac{1}{2} \sup_y \norm{p_y}^2 \int \di y\, \norm{a(s_y)\psi}^2 \frac{b_0}{N} \\
& \leq \frac{C b_0}{2} \norm{\ph}_{H^2}^2 \left(1+\frac{1}{N}\right) \left( \int \di x\,\norm{a(c_x)\psi}^2 + \int \di y\,\norm{a(s_y)\psi}^2 \right) \\
& \leq C\norm{\ph}_{H^2}^2 \left(1+\frac{1}{N}\right)\ev{\Ncal}.
\end{align*}
\estlist{16}{2}
\begin{align*}
& \lvert \scal{\psi}{\dxyNV \frac{1}{N} a^\ast(c_x) a(c_x)\scal{s_y}{s_y}\psi} \rvert \\
& \leq \dxyV \norm{s_y}^2 \norm{a(c_x)\psi}^2 \\
& \leq \frac{1}{2} \sup_y \norm{s_y}^2 \int \di x\, \norm{a(c_x)\psi}^2 \int \di y\, V_N(x-y) \\
& \leq C\norm{\ph}_{H^2}^2 \frac{1}{N}\ev{\Ncal}.
\end{align*}
\emph{\vspace{.3em}\\\Nestlist{16}{3}:} like \nestlist{16}{1}.\newline
\estlist{17}{1}
\begin{align*}
 & \lvert \scal{\psi}{\dxyNV \frac{1}{N}2 a^\ast(c_x)a^\ast(s_x) \scal{s_y}{s_y}\psi} \rvert \\
& \leq \int \di x\di y\, V_N(x-y) \norm{s_y}^2 \lvert \scal{a(c_x)\psi}{a^\ast(s_x)\psi}\rvert \\
& \leq \sup_y \norm{s_y}^2 \int \di x\, \frac{b_0}{N} \norm{a(c_x)\psi}^2 + \sup_y \norm{s_y}^2 \int \di x\, \frac{b_0}{N} \norm{a^\ast(s_x)\psi}^2 \\
& \leq C \norm{\ph}_{H^2}^2 \frac{1}{N}\ev{\Ncal+1}.
\end{align*}
\emph{\vspace{.3em}\\\Nestlist{17}{2}:} like \nestlist{17}{1}.\newline
\estlist{17}{3}
\begin{align*}
& \lvert \scal{\psi}{\dxyNV \frac{1}{N} a^\ast(c_y) a(s_x) \scal{c_y}{s_x} \psi} \rvert \\
& \leq \dxyV \lvert \scal{c_y}{s_x} \rvert \norm{a(c_y)\psi} \norm{a(s_x)\psi} \\
& \leq \frac{C}{2}\int \di x\di y N V_N(x-y) \norm{a(c_y)\psi}^2 \lvert\ph(y)\rvert^2 + \frac{C}{2}\int \di x\di y N V_N(x-y) \norm{a(s_x)\psi}^2 \lvert \ph(x)\rvert^2 \\
& \quad + \dxyV \norm{a(c_y)\psi}^2 \norm{p_y}^2 + \dxyV \norm{a(s_x)\psi}^2 \norm{s_x}^2 \\
& \leq C \norm{\ph}_{H^2}^2 \left(1+\frac{1}{N}\right)\ev{\Ncal}.
\end{align*}
\estlist{18}{1} (here, cancellation is important!)
\begin{align*}
& \lvert \scal{\psi}{\dxyNV \ph(x)\ph(y) a^\ast(c_x) a^\ast(p_y)\psi} \rvert \\
& \leq \dxyNV \lvert\ph(x)\rvert \lvert\ph(y)\rvert \norm{a(c_x)\psi} \norm{a^\ast(p_y)\psi}\\
& \leq \dxyNV \lvert \ph(x)\rvert^2 \norm{a(c_x)\psi}^2 + \dxyNV \lvert\ph(y)\rvert^2 \norm{a^\ast(p_y)\psi}^2 \\
& \leq C\norm{\ph}_{H^2}^2 \ev{\Ncal+1}. 
\end{align*}
%\estlist{18}{2} %29.2
%\begin{align*}
%& \lvert \scal{\psi}{\dxyNV \ph(x)\ph(y) 2a^\ast(c_x)a(s_y)\psi} \rvert \\
%& \leq \dxyNV \lvert\ph(x)\rvert^2 \norm{a(c_x)\psi}^2 + \dxyNV \lvert \ph(y)\rvert^2 \norm{a(s_y)\psi}^2 \\
%& \leq C\norm{\ph}_{H^2}^2 \ev{\Ncal+1}.
%\end{align*}
\emph{\vspace{.3em}\\\Nestlist{18}{2} through \nestlist{20}{3}:} like \nestlist{18}{1}.\newline
\estlist{21}{1}
\begin{align*}
 & \lvert \scal{\psi}{\dxyNV \frac{2}{\sqrt{N}}\cc{\ph(y)}a^\ast(c_x)a^\ast(s_x)a^\ast(s_y)\psi} \rvert \\
& \leq \int \di x\di y\, NV_N(x-y) \lvert\ph(y)\rvert \norm{a(c_x)\psi}\frac{1}{\sqrt{N}}\norm{a^\ast(s_x)a^\ast(s_y)\psi} \\
& \leq \norm{\ph}_\infty^2 \int \di x\di y\, NV_N(x-y) \norm{a(c_x)\psi}^2 + \int \di x\di y\, NV_N(x-y) \frac{1}{N} \norm{a^\ast(s_x)a^\ast(s_y)\psi}^2 \\
& \leq \norm{\ph}_{H^2}^2 C \ev{\Ncal} + \frac{1}{N}\ev{(\Ncal+2)^2} \int \di x\di y\, NV_N(x-y) \norm{s_x}^2 \norm{s_y}^2 \\
& \leq C\norm{\ph}_{H^2}^2 \left( \ev{\Ncal} + \frac{1}{N}\ev{(\Ncal+2)^2} \right).
\end{align*}
\emph{\vspace{.3em}\\\Nestlist{21}{2} and \nestlist{22}{1}:} like \nestlist{21}{1}.\newline
\estlist{22}{2}
\begin{align*}
 & \lvert \scal{\psi}{\dxyNV \frac{2}{\sqrt{N}}\cc{\ph(y)} a^\ast(c_x)a(c_x)a(c_y)\psi} \rvert \\
& \leq \int \di x\di y\, NV_N(x-y) \lvert \ph(y)\rvert \frac{1}{\sqrt{\varepsilon}} \norm{a(c_x)\psi} \norm{a(c_x)a(c_y)\psi} \frac{1}{\sqrt{N}}\sqrt{\varepsilon} \\
& \leq \frac{1}{\varepsilon}\int \di x\di y\, NV_N(x-y) \lvert \ph(y)\rvert^2
  \norm{a(c_x)\psi}^2 \\
& \quad  + \varepsilon \int \di x\di y\, V_N(x-y) \scal{\psi}{a^\ast(c_y)a^\ast(c_x)a(c_x)a(c_y) \psi} \\
& \leq C\norm{\ph}_{H^2}^2 \frac{1}{\varepsilon}\ev{\Ncal} + \varepsilon 4\ev{\tilV}.
\end{align*}
\estlist{22}{3}
\begin{align*}
& \lvert \scal{\psi}{\dxyNV \frac{2}{\sqrt{N}} \cc{\ph(y)}a^\ast(s_y)a(s_x)a(c_x)\psi} \rvert \\
& \leq \int \di x\di y\, NV_N(x-y) \lvert \ph(y)\rvert \norm{a(s_y)\psi} \norm{a(s_x)a(c_x)\psi}\frac{1}{\sqrt{N}} \\
& \leq \int \di x \di y\, NV_N(x-y) \lvert\ph(y)\rvert^2 \norm{a(s_y)\psi}^2 + \int \di x\di y\, NV_N(x-y) \norm{a(s_x)a(c_x)\psi}^2 \frac{1}{N} \\
& \leq \norm{\ph}_\infty^2 b_0 \norm{s}^2 \ev{\Ncal} + \frac{1}{N}b_0 \sup_x\norm{s_x}^2 \int \di x\, \norm{a(c_x)\Ncal^{1/2}\psi}^2 \\
& \leq C \norm{\ph}_{H^2}^2 \left( \ev{\Ncal} + \frac{1}{N}\ev{\Ncal^2}\right). 
\end{align*}
\emph{\vspace{.3em}\\\Nestlist{23}{1} and \nestlist{23}{2}:} %34
like \nestlist{22}{2}.\newline
\estlist{23}{3}
\begin{align*}
& \lvert \scal{\psi}{\dxyNV \frac{2}{\sqrt{N}} \cc{\ph(y)} a(s_x) a(c_x) a(c_y)\psi} \rvert \\
& \leq \int \di x\di y\, NV_N(x-y) \lvert \ph(y)\rvert \frac{1}{\sqrt{\varepsilon}} \norm{a^\ast(s_x)\psi} \norm{a(c_x)a(c_y)\psi}\frac{1}{\sqrt{N}}\sqrt{\varepsilon} \\
& \leq \int \di x\di y\, NV_N(x-y) \lvert \ph(y)\rvert^2
  \norm{a^\ast(s_x)\psi}^2\frac{1}{\varepsilon} \\
& \quad + \int \di x\di y\, NV_N(x-y) \frac{1}{N} \scal{\psi}{a^\ast(c_y)a^\ast(c_x)a(c_x)a(c_y)\psi}\varepsilon \\
& \leq \frac{1}{\varepsilon} \norm{\ph}_\infty^2 b_0 \int \di x\, \norm{s_x}^2 \ev{\Ncal+1} + \varepsilon 4 \ev{\tilV}\\
& \leq \varepsilon 4 \ev{\tilV} + C\norm{\ph}_{H^2}^2 \frac{1}{\varepsilon}\ev{\Ncal+1}.
\end{align*}
\estlist{24}{1}
\begin{align*}
 & \lvert \scal{\psi}{\dxyNV \frac{2}{\sqrt{N}}\cc{\ph(y)}a^\ast(s_x)\scal{s_x}{s_y}\psi} \rvert \\
& \leq \int \di x\di y\, NV_N(x-y) \lvert \ph(y)\rvert \frac{1}{\sqrt{N}} \norm{s_x} \norm{s_y} \norm{\psi} \norm{a(s_x)\psi} \\
& \leq \int \di x\di y\, NV_N(x-y) \lvert \ph(y)\rvert \frac{1}{\sqrt{N}} \norm{s_x}^2 \norm{s_y} \norm{\Ncal^{1/2}\psi} \norm{\psi} \\
& \leq \sup_x \norm{s_x}^2 \frac{1}{\sqrt{N}} \int \di x\di y\, NV_N(x-y) \lvert \ph(y)\rvert \norm{s_y} \ev{\Ncal+1} \\
& \leq \sup_x \norm{s_x}^2 \frac{1}{\sqrt{N}} b_0 \ev{\Ncal+1} \left( \int \di y\, \lvert \ph(y)\rvert^2 \right)^{1/2} \left( \int \di y\, \norm{s_y}^2 \right)^{1/2} \\
& \leq C\norm{\ph}_{H^2}^2 \frac{1}{\sqrt{N}}\ev{\Ncal+1}.
\end{align*}
\emph{\vspace{.3em}\\\Nestlist{24}{2} through \nestlist{25}{4}:} like \nestlist{24}{1}.\newline
This completes the list of estimates for the terms of $T^*_t \Lcal_N^{(0)}(t) T_t$.\vspace{1em}
%%%%%%%%%%%%%%%%%%%%%%%%%%%%%%%%%%%%%%%%%
\end{fleqn}

We still have to give a bound for the $(\partial_t T^*_t)T_t$-part of the generator. We will expand $(\partial_t T^*_t)T_t$ as a series and bound the summands individually, using the following lemma.
\begin{lem}
\label{lm:Bbound}
Let $f_1, f_2 \in L^2(\Rbb^3\times \Rbb^3)$. Then for all $\psi \in \fock$
 \[
\lvert \scal{\psi}{\frac{1}{2} \int \di x\di y \left( f_1(x,y) a^\ast_x a^\ast_y + f_2(x,y) a_x a_y \right) \psi} \rvert \leq \scal{\psi}{(\Ncal+1)\psi} \frac{\norm{f_1}+\norm{f_2}}{2}.
\]
and
\[
 \lvert \scal{\psi}{\frac{1}{2} \int \di x\di y \left( f_1(x,y) a^\ast_x
a_y + f_2(x,y) a_x a^\ast_y \right)  \psi} \rvert
 \leq \scal{\psi}{\Ncal\psi} \frac{\norm{f_1}+\norm{f_2}}{2} + \left\lvert \int f_2(x,x) \di x \right\rvert \frac{\scal{\psi}{\psi}}{2}.
\]
\end{lem}
\begin{proof} Using H\"older's inequality we prove the first estimate as follows:
 \begin{align*}
  & \lvert \scal{\psi}{\frac{1}{2} \int \di x\di y \left( f_1(x,y) a^\ast_x a^\ast_y + f_2(x,y) a_x a_y \right) \psi} \rvert \\
& \leq \frac{1}{2} \sum_{i=1}^2 \lvert \scal{\psi}{\int \di x\di y\, f_i(x,y) a^\ast_x a^\ast_y \psi }\rvert \\
& \leq \frac{1}{2} \sum_{i=1}^2 \int \di y \norm{a_y \psi} \norm{a^\ast(f_i(\cdot,y))\psi} \\
& \leq \frac{1}{2} \sum_{i=1}^2 \left( \int \di y_1\, \norm{a_{y_1}\psi}^2 \int \di y_2\, \norm{f_i(\cdot,y_2)}^2 \norm{(\Ncal+1)^{1/2} \psi}^2 \right)^{1/2} \\
& \leq \ev{\Ncal+1} \frac{\norm{f_1}+\norm{f_2}}{2}.
 \end{align*}
To prove the second estimate we use the CCR and H\"older's inequality:
 \begin{align*}
& \lvert \scal{\psi}{\frac{1}{2} \int \di x\di y \left( f_1(x,y) a^\ast_x a_y + f_2(x,y) a_x a^\ast_y \right)  \psi} \rvert\\
& \leq \frac{1}{2} \lvert \scal{\psi}{\int \di x\di y\, f_1(x,y) a^\ast_x a_y \psi} \rvert + \frac{1}{2}\lvert \scal{\psi}{\int \di x \di y\, f_2(x,y) a^\ast_y a_x \psi}\rvert \\
&\quad + \frac{1}{2} \lvert \scal{\psi}{\int \di x \di y\, f_2(x,y) \delta(x-y) \psi}\rvert \\
& \leq \frac{1}{2} \int \di x\, \norm{a_x \psi} \norm{a\big(f_1(x,\cdot)\big)\psi} + \frac{1}{2} \int \di y\, \norm{a_y\psi} \norm{a\big(f_2(\cdot,y)\big)\psi} + \frac{1}{2} \left\lvert \int f_2(x,x) \di x \right\rvert \scal{\psi}{\psi} \\
& \leq \frac{1}{2} \left( \int \di x_1 \norm{a_{x_1}\psi}^2 \int \di x_2 \norm{f_1(x_2,\cdot)}^2 \norm{\Ncal^{1/2}\psi}^2 \right)^{1/2}\\
& \quad + \frac{1}{2} \left( \int \di x_1 \norm{a_{x_1}\psi}^2 \int \di x_2 \norm{f_2(\cdot,x_2)}^2 \norm{\Ncal^{1/2}\psi}^2 \right)^{1/2} + \frac{1}{2}\left\lvert \int f_2(x,x) \di x\right\rvert \scal{\psi}{\psi} \\
& \leq \frac{1}{2}\ev{\Ncal} \left(\left(\int \di x \norm{f_1(x,\cdot)}^2 \right)^{1/2} + \left(\int \di x \norm{f_2(\cdot,x)}^2 \right)^{1/2} \right) + \frac{1}{2}\left\lvert \int f_2(x,x) \di x\right\rvert \scal{\psi}{\psi}. 	\qedhere
 \end{align*}
\end{proof}

We now calculate and estimate the functions $f_i$ which will appear in the series for $(\partial_t T^*_t)T_t$. Recall that for two operators $A$ and $B$
\[
 \ad^0_A(B) = B \qquad \mbox{and} \qquad \ad^n_A(B) = [A,\ad^{n-1}_A(B)] \mbox{ for } n \geq 1.
\]


\begin{lem}
\label{lm:highercommutators}
 For each $n \in \Nbb$ and each $i \in \{1,2\}$, there exists $f_{n,i} \in L^2(\Rbb^3 \times \Rbb^3)$ such that
\begin{itemize}
 \item for $n$ even we have
\bd
\ad^n_B(\dot B) = \frac{1}{2} \int \di x\di y\left( f_{n,1}(x,y) a^\ast_y a^\ast_x + f_{n,2}(x,y) a_x a_y \right)
\ed
and $\norm{f_{n,i}} \leq \norm{\dot k} (2\norm{k})^n$; 
 \item for $n$ odd we have
\bd
\ad^n_B(\dot B) = \frac{1}{2} \int \di x\di y\left( f_{n,1}(x,y) a^\ast_x a_y + f_{n,2}(x,y) a_x a^\ast_y \right)
\ed
and the following bounds hold: $\norm{f_{n,i}} \leq \norm{\dot k} (2\norm{k})^n$ and $\int \lvert f_{n,i}(x,x)\rvert \di x \leq \norm{\dot k} (2\norm{k})^n$. 
\end{itemize}
\end{lem}
\begin{proof} The proof is by induction in $n$. In the inductive step, we have to treat the cases of even $n$ and odd $n$ separately (but the calculations are similar).\\
\underline{Basis $n=0$:} The estimate stated in the even case is clearly fulfilled for
\bd
\ad^0_B(\dot B) = \dot B = \frac{1}{2}\int \di x\di y\left( \dot k(x,y) a^\ast_x a^\ast_y - \cc{\dot k(x,y)} a_x a_y \right).
\ed
\underline{Inductive step, starting from even $n$:}\\
We calculate by using the CCR and appropriately renaming integration variables that
\begin{align*}
& \ad^{n+1}_B(\dot B) = [B,\ad^n_B(\dot B)] \\
& = \left[\frac{1}{2} \int \di x\di y\left( k(x,y)a^\ast_x a^\ast_y - \cc{k(x,y)}a_x a_y \right), \frac{1}{2}\int \di x\di y\left( f_{n,1}(x,y) a^\ast_x a^\ast_y + f_{n,2}(x,y) a_x a_y \right)\right] \\
& = \frac{1}{2} \int \di x\di z \left(f_{n+1,1}(x,z) a^\ast_x a_z + f_{n+1,2}(x,z) a_x a^\ast_z \right),
\end{align*}
where
\begin{equation}
\label{eq:even}
\begin{split}
f_{n+1,1}(x,z) & = -\frac{1}{2} \int \di y \left( k(x,y) \left( f_{n,2}(z,y) + f_{n,2}(y,z) \right) + \cc{k(y,z)}\left( f_{n,2}(x,y) + f_{n,2}(y,x)\right) \right)\\
f_{n+1,2}(x,z) & = -\frac{1}{2} \int \di y \left( k(y,z) \left( f_{n,1}(x,y) + f_{n,1}(y,x) \right) + \cc{k(x,y)}\left( f_{n,1}(z,y) + f_{n,1}(y,z)\right) \right), 
\end{split}
\end{equation}
so, as $n+1$ is odd, we have the correct expression for $\ad^{n+1}_B(\dot B)$. We still have to check for validity of the estimates.
Clearly
\be{normnorm}
\begin{split}
\norm{f_{n+1,1}}_{L^2} & \leq \frac{1}{2} \bigg( \norm{\int \di y\, k(x,y) f_{n,2}(z,y)}_{L^2(dxdz)} + \norm{\int \di y\, k(x,y) f_{n,2}(y,z)}_{L^2(dxdz)} \\
& \quad + \norm{\int \di y\, \cc{k(y,z)}f_{n,2}(x,y)}_{L^2(dxdz)} + \norm{\int \di y\, \cc{k(y,z)} f_{n,2}(y,x)}_{L^2(dxdz)} \bigg),
\end{split}
\ee
Using H\"older's inequality we get
\begin{align*}
& \norm{\int \di y\,k(x,y) f_{n,2}(z,y)}^2_{L^2(dxdz)} = \int \di x\di z \left\lvert \int \di y\,k(x,y) f_{n,2}(z,y) \right\rvert^2 \\
& \leq \int \di x \di z \int \di y_1\,\lvert k(x,y_1) \rvert^2 \int \di y_2\,\lvert f_{n,2}(z,y_2) \rvert^2 = \norm{k}^2 \norm{f_{n,2}}^2. 
\end{align*}
The other three summands in \eqr{normnorm} obey the same bound, so by making use of the inductive hypothesis, we obtain
\bd
\norm{f_{n+1,1}}_{L^2} \leq \norm{\dot k} (2\norm{k})^{n+1},
\ed
which was to be proven. The calculation for $\norm{f_{n+1,2}}_{L^2}$ works the same way. We still have to proof the second estimate of the lemma, so using H\"older's inequality we calculate that
\[
\int \lvert f_{n+1,1}(x,x) \rvert \di x \leq 2\norm{k} \norm{f_{n,2}} \leq \norm{\dot k} (2\norm{k})^{n+1}
\]
and in the same way $\int \lvert f_{n+1,2}(x,x) \rvert \di x \leq \norm{\dot k}(2\norm{k})^{n+1}$.\vspace{1em}\\
\underline{Inductive step, starting from odd $n$:}\\
We calculate that% (by using the CCR and appropriately renaming integration variables)
\begin{align*}
& \ad^{n+1}_B(\dot B)\\
& = \left[ \frac{1}{2}\int \di x \di y \left( k(x,y)a^\ast_x a^\ast_y - \cc{k(x,y)} a_x a_y \right) , \frac{1}{2}\int \di x\di y \left( f_{n,1}(x,y) a^\ast_x a_y + f_{n,2}(x,y) a_x a^\ast_y \right) \right] \\
& = \frac{1}{2} \int \di x\di z \big( a^\ast_x a^\ast_z f_{n+1,1}(x,z) + a_x a_z f_{n+1,2}(x,z) \big)
\end{align*}
where
\begin{equation}
\label{eq:odd}
\begin{split}
f_{n+1,1}(x,z) & = - \int \di y\, k(x,y)\left( f_{n,1}(z,y) + f_{n,2}(y,z) \right) \\
f_{n+1,2}(x,z) & = - \int \di y\, \cc{k(x,y)}\left( f_{n,1}(y,z) + f_{n,2}(z,y) \right),
\end{split}
\end{equation}
so, as $n+1$ is even, we have the correct expression for $\ad^{n+1}_B(\dot B)$. Validity of the estimate for $\norm{f_{n+1,1}}$ and $\norm{f_{n+1,2}}$ again follows by H\"older's inequality and the inductive hypothesis.
\end{proof}

The operator $\partial_t\left((\partial_t T^*_t)T_t\right)$ can be bounded by applying $\partial_t$ to all summands of the series of $(\partial_t T^*_t)T_t$ individually and then employing lemma \ref{lm:Bbound} and the following lemma. %The second estimate is only needed for odd $n$.
\begin{lem}
\label{lm:deldelt}
For all $n \in \Nbb$ and all $i \in \{1,2\}$, we have the estimates
\bd
\norm{\dot f_{n,i}} \leq 2n \norm{\dot k}^2 (2\norm{k})^{n-1} + \norm{\ddot k}(2\norm{k})^n.
\ed
For odd $n$, we have the estimate
\bd
\int \lvert \dot f_{n,i}(x,x) \rvert \di x \leq 2n \norm{\dot k}^2 (2\norm{k})^{n-1} + \norm{\ddot k}(2\norm{k})^n .
\ed
\end{lem}
\begin{proof}
The first estimate is proved by induction. The proof of the basis case is trivial. For the inductive step, we again have to treat the cases of even and odd $n$ separately: We employ \eqref{eq:even} respectively \eqref{eq:odd}, use H\"older's inequality, then make use of $\norm{f_{n,i}} \leq \norm{\dot k} (2\norm{k})^n$ and finally apply the inductive hypothesis.

To prove the second estimate, we use \eqref{eq:even} to express $f_{n,i}$ in terms of the $f_{n-1,i}$ and then use H\"older's inequality and apply the first estimate.
\end{proof}

We now combine the bounds on the individual summands to obtain an estimate for the series.
\begin{lem}
\label{lm:timederivative}
There exists a constant $C$ such that for all $t \in \Rbb$ and all $\psi \in \fock$
 \bd
  \lvert \scal{\psi}{(\partial_t T^*_t)T_t \psi} \rvert \leq \norm{\dot k} e^{2\norm{k}} \scal{\psi}{(\Ncal+1)\psi} \leq C \norm{\ph}_{H^2}^2 \scal{\psi}{(\Ncal+1)\psi}.
 \ed
Furthermore
 \bd
    \lvert \scal{\psi}{\partial_t \big( \left(\partial_t T^\ast_t\right) T_t \big) \psi} \rvert \leq \left( 2\norm{\dot k}^2 + \norm{\ddot k} \right) e^{2\norm{k}} \scal{\psi}{(\Ncal+1)\psi}.
 \ed
\end{lem}
\begin{proof}
The first estimate is proved by expanding the time derivative as a series,
\bd
\left(\partial_t e^{-B(t)} \right) e^{B(t)} = - \dot B(t) +
\frac{1}{2!}[B(t),\dot B(t)] - \frac{1}{3!}[B(t),[B(t),\dot B(t)]] + \dots = \sum_{n=0}^\infty \frac{(-1)^{n+1}}{(n+1)!} \ad^n_B(\dot B),
\ed
and then using the Lemmata \ref{lm:highercommutators}, \ref{lm:Bbound} and \ref{lm:kbounds}.\\
For the second estimate, differentiate the summands individually (i.\,e. place a dot on the respective $f_{n,i}$) and then estimate all summands like before, now using lemma \ref{lm:deldelt}.
\end{proof}


\section{Hartree and Gross-Pitaevskii equations}
\label{s:pde}

In this section we prove uniform (i.\,e.\ independent of $N$) $H^n$
regularity for the solutions of the modified Hartree equation and the
Gross-Pitaevskii equation. We also prove that solutions $\varphi_t^{(N)}$ of
the modified Hartree equation converge to solutions $\varphi_t$ of the
Gross-Pitaevskii equation as $N \to \infty$. We finally collect some
estimates on the regularity of $\phdot$ and $\phddot$.


\subsection{Uniform $H^n$-regularity}
\label{ss:regularity}


\begin{thm} \label{t:pdes}% \label{c:reg2}
  Let $\varphi \in H^n(\R^3)$ with $n \ge 2$ and $\| \varphi \|_{L^2} = 1$.
  Suppose that $f \in L^\infty(\R^3)$ and $V \in C_c^\infty(\R^3)$ with $fV \ge
  0$. For $N \ge 1$, consider the solution $\ph \in H^1(\R^3)$ of the
  modified Hartree equation
  \[
    i \partial_t \varphi_t^{(N)} = - \Delta \varphi_t^{(N)} + (N f_N V_N *
    |\varphi_t^{(N)}|^2) \varphi_t^{(N)}
  \]
  with initial data $\varphi^{(N)}_0 = \varphi$, where $f_N V_N(x) = N^2
  f(Nx)V(Nx)$. Consider also the solution $\varphi_t \in H^1(\R^3)$ of the
  Gross-Pitaevskii equation
  \[
    i \partial_t \varphi_t = - \Delta \varphi_t + 8 \pi a_0 |\varphi_t|^2
    \varphi_t
  \]
  with initial data $\varphi_0 = \varphi$, where $8 \pi a_0 = \int f V$.
  Then, there exist constants $C_1$ and $K_n$, depending only on $\| \varphi
  \|_{H^1}$ and $\| fV \|_{L^1}$, and a constant $C_n$, depending only on
  $\| fV \|_{L^1}$ and $\| \varphi \|_{H^j}$ for $1 \le j \le n$, such that,
  for all $N \ge 1$ and $t \ge 0$,
  \begin{alignat}{2}
    \| \varphi_t^{(N)} \|_{H^1} & \le C_1, & \qquad \| \varphi_t \|_{H^1} &
    \le C_1, \tag{i} \label{H1} \\
    \| \varphi_t^{(N)} \|_{H^n} & \le C_n e^{K_n t}, & \qquad \| \varphi_t
    \|_{H^n} & \le C_n e^{K_n t}. \tag{ii}
  \end{alignat}
  Furthermore, there exist constants $C$ and $K$, depending only on $\|
  \varphi \|_{H^1}$, $\| \varphi \|_{H^2}$ and $\| fV \|_{L^1}$, such that,
  for all $N \geq 1$ and $t \geq 0$,
  \begin{equation}
    \| \varphi_t^{(N)} - \varphi_t \|_{L^2} \le \frac{C}{N} e^{e^{K t}}.
    \tag{iii}
  \end{equation}
\end{thm}


\begin{proof}
  This theorem is a combined restatement of Proposition \ref{p:reg1},
  Corollary \ref{c:regn}, and Lemma \ref{lem:phitN2phit}, which are proved
  below.
\end{proof}


The initial value problem for the Hartree equation
\[
  i \partial_t \varphi_t = - \Delta \varphi_t + (V * |\varphi_t|^2)
  \varphi_t
\]
with initial data $\varphi_0 = \varphi$ is known to be globally well-posed
in $H^1(\R^3)$. This follows from the conservation of mass $\| \varphi
\|_{L^2}$ and energy
\[
  \mathcal{E}(\varphi) = \int dx \, |\nabla \varphi(x)|^2 + \frac{1}{2} \int
  dx \, (V * |\varphi|^2)(x) |\varphi(x)|^2.
\]
The exact same statement holds for the Gross-Pitaevskii equation
\[
  i \partial_t \varphi_t = - \Delta \varphi_t + 8 \pi a_0 |\varphi_t|^2
  \varphi_t
\]
with initial data $\varphi_0 = \varphi$, mass $\| \varphi \|_{L^2}$, and
energy
\[
  \mathcal{E}_{GP}(\varphi) = \int dx \, |\nabla \varphi(x)|^2 + 4 \pi a_0
  \int dx \, |\varphi(x)|^4,
\]
where $8 \pi a_0 = \int f V$.


We are mainly interested in what we call the modified Hartree equation, in
which the potential $V$ is given by $N f_N V_N$. As we need regularity
estimates that are independent of $N$, we must not use any norm of
$N f_N V_N$ other than
the $L^1$-norm, which is given by $\norm{N f_N V_N}_{L^1} = \| fV \|_{L^1}$.
The Gross-Pitaevskii equation can be formally seen as the limiting equation of the modified Hartree
equation as $N f_N V_N \to 8 \pi a_0 \delta$ when $N \to \infty$.


The next proposition provides an upper bound for the energy of both partial
differential equations.


\begin{prp} \label{p:energy}
  Let $\varphi \in H^1(\R^3)$ and $V \in L^1(\R^3)$. Then,
  \[
    \mathcal{E}(\varphi) \apprle \| \varphi \|_{H^1}^2 + \| V \|_{L^1} \|
    \varphi \|_{H^1}^4 \qquad \text{and} \qquad \mathcal{E}_{GP}(\varphi)
    \apprle \| \varphi \|_{H^1}^2 + |a_0| \| \varphi \|_{L^2} \| \varphi
    \|_{H^1}^3.
  \]
\end{prp}


\begin{proof}
  By Young's inequality, Sobolev's inequality, and H\"older's inequality, 
  \[
    \int dx \, (V * |\varphi|^2)(x) |\varphi(x)|^2 \le \| V \|_{L^1} \|
    \varphi^2 \|_{L^2}^2 = \| V \|_{L^1} \| \varphi \|_{L^4}^4 \apprle \| V
    \|_{L^1} \| \varphi \|_{H^1}^4
  \]
  and
  \[
    \int dx \, |\varphi(x)|^4 \le \| \varphi \|_{L^2} \| \varphi^3 \|_{L^2} =
    \| \varphi \|_{L^2} \| \varphi \|_{L^6}^3 \apprle \| \varphi \|_{L^2} \|
    \varphi \|_{H^1}^3.
  \]
  Observing that $\| \varphi \|_{H^1}^2 = \| \varphi \|_{L^2}^2 + \| \nabla
  \varphi \|_{L^2}^2$, we obtain the desired estimates.
\end{proof}


Using the above estimates, we obtain an upper bound for the $H^1$-norm of
solutions that holds uniformly in time.


\begin{prp} \label{p:reg1}
  Let $\varphi \in H^1(\R^3)$ and $V \in L^1(\R^3)$ with $V \ge 0$. Consider
  the solution $\varphi_t \in H^1(\R^3)$ of the Hartree equation
  \[
    i \partial_t \varphi_t = - \Delta \varphi_t + (V * |\varphi_t|^2)
    \varphi_t
  \]
  with initial data $\varphi_0 = \varphi$. Then, there exists a constant
  $B_1$, depending only on $\| \varphi \|_{H^1}$ and $\| V \|_{L^1}$, such
  that, for all $t \geq 0$,
  \[
    \norm{\varphi_t}_{H^1} \leq B_1. 
  \]
  Furthermore, the constant $B_1$ increases monotonically with respect to
  $\| \varphi \|_{H^1}$. The exact same statement holds for solutions of the
  Gross-Pitaevskii equation with $\| V \|_{L^1}$ replaced by $8 \pi a_0$.
\end{prp}


\begin{proof}
  Recall that $V \ge 0$. By conservation of mass and energy, and by
  Proposition \ref{p:energy},
  \[
    \| \varphi_t \|_{H^1}^2 \le \| \varphi_t \|_{L^2}^2 +
    \mathcal{E}(\varphi_t) = 1 + \mathcal{E}(\varphi) \apprle \| \varphi
    \|_{H^1}^2 + \| V \|_{L^1} \| \varphi \|_{H^1}^4 \eqqcolon B_1^2.
  \]
  Clearly the constant $B_1$ increases monotonically with respect to $\|
  \varphi \|_{H^1}$. This proves the first part of the proposition. The
  proof of the estimate for solutions of the Gross-Pitaevskii equation is
  similar.
\end{proof}


It turns out that the initial value problems for the Hartree and
Gross-Pitaevskii equations are also well-posed in $H^n(\R^3)$ for $n \ge 2$.
The next proposition provides an upper bound for the $H^n$-norm of solutions
that holds locally in time.


\begin{prp} \label{p:lregn}
  Let $\varphi \in H^n(\R^3)$ with $n \geq 2$, and let $V \in L^1(\R^3)$
  with $V \ge 0$. Consider the solution $\varphi_t \in H^1(\R^3)$ of the
  Hartree equation
  \[
    i \partial_t \varphi_t = - \Delta \varphi_t + (V * |\varphi_t|^2)
    \varphi_t
  \]
  with initial data $\varphi_0 = \varphi$. Then, there exists a constants $T
  > 0$, depending only on $\| \varphi_0 \|_{H^1}$, $\| V \|_{L^1}$ and $n$,
  such that
  \begin{equation}\label{eq:preregularity}
    \sup_{t \in [0,T]} \| \varphi_t \|_{H^n} \le C \| \varphi_0 \|_{H^n} +
    \sup_{t \in [0,T]} \| \varphi_t \|_{H^{n-1}}^3,
  \end{equation}
  where $C$ is a constant that depends only on $n$. Furthermore, the
  constant $T$ decreases monotonically with respect to $\| \varphi_0
  \|_{H^1}$. The exact same statement holds for solutions of the
  Gross-Pitaevskii equation with $\| V \|_{L^1}$ replaced by $8 \pi a_0$.
\end{prp}

We shall shortly prove this proposition. Before doing that, we first prove
that by combining it with the global estimates for the $H^1$-norm, we obtain
bounds for the $H^n$-norm that hold globally in time.


\begin{cor}[$H^n$-Regularity] \label{c:regn}
  Under the hypothesis of Proposition \ref{p:lregn}, there exist constants
  $B_n$ and $K_n$, where $B_n$ depends only on $\| \varphi \|_{H^j}$ for $1
  \le j \le n$, $\| V \|_{L^1}$ and $n$, and $K_n$ depends only on $\|
  \varphi \|_{H^1}$, $\| V \|_{L^1}$ and $n$, such that, for all $t \ge 0$,
  \[
    \| \varphi_t \|_{H^n} \le B_n e^{K_n t}.
  \]
  The exact same statement holds for solutions of the Gross-Pitaevskii
  equation with $\| V \|_{L^1}$ replaced by $8 \pi a_0$.
\end{cor}


\begin{proof}
The proof is based on the fact that $T$ in Proposition \ref{p:lregn} depends only on the $H^1$-norm: As the $H^1$-norm is bounded uniformly in time, \eqref{eq:preregularity} can be iterated. 
  For simplicity, we will omit the dependence of the constants with respect
  to $\| V \|_{L^1}$ and $n$.

  The proof is by induction in $n$. First observe that, from Proposition
  \ref{p:reg1}, it follows trivially that $\| \varphi_t \|_{H^1} \le B_1
  e^{K_1 t}$ for some constants $B_1$ and $K_1$ that depend only on $\|
  \varphi \|_{H^1}$. This proves the desired estimate for $n=1$.

  Suppose that $\| \varphi_t \|_{H^{n-1}} \le B_{n-1} e^{K_{n-1} t}$, where
  $B_{n-1}$ is a constant that depends only on $\| \varphi \|_{H^j}$ for $1
  \le j \le n-1$, and $K_{n-1}$ is a constant that depends only on $\|
  \varphi \|_{H^1}$. By Proposition \ref{p:reg1}, we have $\| \varphi_t
  \|_{H^1} \le B_1$ for all $t \ge 0$.
%
 Thus by Proposition \ref{p:lregn} there exist a constant $T>0$, depending 
  only on $B_1$ and $n$, such that \eqref{eq:preregularity} holds on arbitrary intervals of length $T$. Now for
 any $t > 0$ there exists an integer $j$, depending
  only on $B_1$ and $n$, such that $(j-1)T < t \le jT$. Therefore,
%
  \begin{align*}
    \| \varphi_t \|_{H^n} & \le \sup_{s \in [(j-1)T, jT]} \| \varphi_s
    \|_{H^n} \\
    & \le C \| \varphi_{(j-1)T} \|_{H^n} + \sup_{s \in [(j-1)T, jT]} \|
    \varphi_s \|_{H^{n-1}}^3 \\
    & \le C \| \varphi_{(j-1)T} \|_{H^n} + B_{n-1}^3 e^{3 K_{n-1} j T},
  \end{align*}
  where $C$ is a constant that depends only on $n$. Since $\|
  \varphi_{(j-2)T} \|_{H^1} < B_1$, similarly we have
  \[
    \| \varphi_{(j-1)T} \|_{H^n} \le \sup_{s \in [(j-2)T, (j-1)T]} \|
    \varphi_s \|_{H^n} \le C \|
    \varphi_{(j-2)T} \|_{H^n} + B_{n-1}^3 e^{3 K_{n-1} (j-1)T}.
  \]
  Thus we can iterate the above procedure $j$ times to cover the interval
  $[0,t]$ and obtain
  \begin{align*}
    \| \varphi_t \|_{H^n} & \le C^j \| \varphi \|_{H^n} + (1+C) B_{n-1}^3
    \sum_{l=0}^j e^{3 K_{n-1} (j-l)T} \\
    & \le \| \varphi \|_{H^n} (C^{2 T^{-1}})^{jT/2} + (1+C) B_{n-1}^3
    (1-e^{-3K_{n-1} T})^{-1} e^{6 K_{n-1} jT/2} \\
    & \le B_n e^{K_n t},
  \end{align*}
  for some constant $B_n$ depending only on $\| \varphi \|_{H^j}$ for $1 \le
  j \le n$, and some constant $K_n$ depending only on $\| \varphi \|_{H^1}$.
  This completes the induction step and the proof of the corollary.
\end{proof}


We next prove Proposition \ref{p:lregn}. Its proof is based on well-known
Strichartz estimates for the free Schr\"odinger evolution and the following
lemma, which we prove first.


For $T > 0$ set
\[
  L_t^q L_x^r = L^q([0,T], L^r(\R^3)).
\]


\begin{lem} \label{l:interp}
  Let $V \in L^1(\R^3)$, $f \in L_t^{q_1} L_x^{r_1}$ and $g \in L_t^{q_2}
  L_x^{r_2}$ with $q_j, r_j \in [1,\infty]$ for $j \in \{1,2\}$. Then, for any $s \in [1, \infty]$, $(V * f)
  g \in L_t^q L_x^r$ with $q^{-1} = s^{-1} + q_1^{-1} + q_2^{-1}$ and $r^{-1}
  = r_1^{-1} + r_2^{-1}$. Furthermore
  \[
    \| (V * f)g \|_{L_t^q L_x^r} \le \| V \|_{L^1} T^{1/s} \| f \|_{L_t^{q_1}
    L_x^{r_1}} \| g \|_{L_t^{q_2} L_x^{r_2}}.
  \]
\end{lem}


\begin{proof}
  Applying H\"older's inequality in space, and then in time, we find that
  \[
    \| (V * f) g \|_{L_t^q L_x^r} \le \| V * f \|_{L_t^u L_x^{r_1}} \| g
    \|_{L_t^{q_2} L_x^{r_2}}
  \]
  with $q^{-1} = u^{-1} + q_2^{-1}$ and $r^{-1} = r_1^{-1} + r_2^{-1}$. Now,
  applying Young's inequality in space, H\"older's inequality in time, and
  observing that $V$ is time-independent, we get
  \[
    \| V * f \|_{L_t^u L_x^{r_1}} \le \| V \|_{L^1} T^{1/s} \| f \|_{L_t^{q_1}
    L_x^{r_1}}
  \]
  with $u^{-1} = s^{-1} + q_1^{-1}$. Combining all this we obtain the desired
  result.
\end{proof}


\begin{proof}[Proof of Proposition \ref{p:lregn}]
  The Hartree equation in integral form is
  \[
    \varphi_t = e^{it\Delta} \varphi - i \int_0^t ds \, e^{i(t-s)\Delta} (V *
    |\varphi_s|^2) \varphi_s.
  \]
  Differentiating this equation we find that
  \[
    \partial^\alpha \varphi_t = e^{it \Delta} \partial^\alpha \varphi - i
    \int_0^t ds \, e^{i(t-s) \Delta} \sum_{\beta \le \alpha} \sum_{\nu \le
    \beta} \binom{\alpha}{\beta} \binom{\beta}{\nu} \big( V * (\partial^\nu
    \overline{\varphi_s} \partial^{\beta - \nu} \varphi_s) \big)
    \partial^{\alpha - \beta} \varphi_s.
  \]
  Here $\alpha$ is a three-dimensional multi-index of non-negative integers,
  and we are using standard multi-index notation. The length of a multi-index
  $\gamma$ is defined as $|\gamma| = \gamma_1 + \gamma_2 + \gamma_3$.


  The $L_t^\infty L_x^2$-norm of the above expression can be controlled using
  Strichartz estimates for the free Schr\"odinger evolution $e^{it\Delta}$
  (see \cite[Theorem 1.2]{KT}). By these estimates, triangle inequality, Lemma
  \ref{l:interp}, and H\"older's inequality,
  \begin{align}
    & \| \partial^\alpha \varphi_{(\cdot)} \|_{L_t^\infty L_x^2} - \|
    \partial^\alpha \varphi \|_{L^2} \notag \\
    & \le \sum_{\beta \le \alpha} \sum_{\nu \le \beta} \binom{\alpha}{\beta}
    \binom{\beta}{\nu} \| \big( V * (\partial^\nu
    \overline{\varphi_{(\cdot)}} \partial^{\beta - \nu} \varphi_{(\cdot)})
    \big) \partial^{\alpha - \beta} \varphi_{(\cdot)} \|_{L_t^2 L_x^{6/5}}
    \notag \\
    & \le \| V \|_{L^1} T^{1/2} \sum_{\beta \le \alpha} \sum_{\nu \le \beta}
    \binom{\alpha}{\beta} \binom{\beta}{\nu} \sup_{t \in [0,T]} \|
    \partial^\nu \varphi_t \|_{L^{p_1}} \| \partial^{\beta - \nu} \varphi_t
    \|_{L^{p_2}} \sup_{t \in [0,T]} \| \partial^{\alpha - \beta} \varphi_t
    \|_{L^{p_3}} \label{Dphi}
  \end{align}
  with $p_1^{-1} + p_2^{-1} + p_3^{-1} = 5/6$. Note that $(p_1, p_2, p_3)$ can
  be chosen differently for each term in the summation. We will appropriatedly
  choose it to be either $(2,6,6)$, $(6,2,6)$ or $(6,6,2)$, as we describe
  next.


  Consider a term in \eqref{Dphi}, and consider the order of each partial
  derivative in it: $|\nu|$, $|\beta-\nu|$ and $|\alpha-\beta|$. Then choose
  $p_j = 2$ for one of the $p_j$'s corresponding to the derivatives of highest
  order, and choose the other two $p_j$'s equal to $6$. For the factors with
  $p_j = 6$, use the estimate
  \[
    \| \partial^\gamma \varphi_t \|_{L^6} \le \| \varphi_t \|_{W^{|\gamma|,6}}
    \apprle \| \varphi_t \|_{H^{|\gamma|+1}},
  \]
  which holds for any multi-index $\gamma$. This follows from the definition
  of Sobolev norm and Sobolev's inequality. For the factor with $p_j = 2$,
  simply note that
  \[
    \| \partial^\gamma \varphi_t \|_{L^2} \le \| \varphi_t \|_{H^{|\gamma|}}.
  \]
  We will estimate all terms in \eqref{Dphi} using these two inequalities.


  Let $k$ be an integer such that $1 \le k \le |\alpha|$, and consider a
  term in \eqref{Dphi} whose factor of highest order derivative has order
  $k$. Let $\gamma_1$ and $\gamma_2$ be the multi-indexes in the derivatives
  of the other two factors in this term. Note that $|\gamma_1 + \gamma_2| =
  |\alpha|-k$. By proceeding as described in the above paragraph, we
  conclude that the term under consideration is bounded by a constant times
  \begin{equation} \label{quant}
    \sup_{t \in [0,T]} \| \varphi_t \|_{H^{|\gamma_1|+1}} \sup_{t \in [0,T]}
    \| \varphi_t \|_{H^{|\gamma_2|+1}} \sup_{t \in [0,T]} \| \varphi_t
    \|_{H^k}.
  \end{equation}
  Observe that, by varying $k$ in the range $1 \le k \le |\alpha|$, we
  exhaust all terms in \eqref{Dphi}. We next estimate \eqref{quant} by
  considering different values of $k$ separately. 


  If $|\alpha|=1$ and $k=1$, then $|\gamma_1|=|\gamma_2|=0$, so that
  \eqref{quant} is bounded by
  \[
    \sup_{t \in [0,T]} \| \varphi_t \|_{H^1}^3.
  \]
  Let $|\alpha|=2$. If $k=1$, then either $|\gamma_1|=1$ and $|\gamma_2|=0$,
  or $|\gamma_1|=0$ and $|\gamma_2|=1$. If $k=2$, then $|\gamma_1| =
  |\gamma_2| = 0$. In all these cases \eqref{quant} is bounded by
  \[
    \sup_{t \in [0,T]} \| \varphi_t \|_{H^1}^2 \sup_{t \in [0,T]} \|
    \varphi_t \|_{H^2}.
  \]
  If $|\alpha| = 3$ and $k = 1$, then $|\gamma_1|=|\gamma_2|=1$, and thus
  \eqref{quant} is bounded by
  \[
    \sup_{t \in [0,T]} \| \varphi_t \|_{H^2}^2 \sup_{t \in [0,T]} \|
    \varphi_t \|_{H^1}.
  \]
  If $|\alpha| \ge 3$ and $2 \le k < |\alpha|$, then $|\gamma_1| \le
  |\alpha| - 2 $ and $|\gamma_2| \le |\alpha|-2$, and \eqref{quant} is
  bounded by
  \[
    \sup_{t \in [0,T]} \| \varphi_t \|_{H^{|\alpha|-1}}^3.
  \]
  If $|\alpha| \ge 3$ and $k=|\alpha|$, then $|\gamma_1|=|\gamma_2|=0$, so
  that \eqref{Dphi} is bounded by
  \[
    \sup_{t \in [0,T]} \| \varphi_t \|_{H^1}^2 \sup_{t \in [0,T]} \|
    \varphi_t \|_{H^{|\alpha|}}.
  \]
  We have considered all possible values for $|\alpha|$ and $k$.
  
  
  Combining the above estimates with \eqref{Dphi}, we find that, for any
  $\alpha$ obeying $0 \le |\alpha| \le n$,
  \begin{equation} \label{dal}
    \begin{split}
      & \sup_{t \in [0,T]} \| \partial^\alpha \varphi_t \|_{L^2} \\
      & \le \| \partial^\alpha \varphi \|_{L^2} + C_1 4^{|\alpha|} \| V
      \|_{L^1} T^{1/2} \Big[ \sup_{t \in [0,T]} \| \varphi_t \|_{H^1}^2
      \sup_{t \in [0,T]} \| \varphi_t \|_{H^{|\alpha|}} + \sup_{t \in [0,T]}
      \| \varphi_t \|_{H^{|\alpha|-1}}^3 \Big],
    \end{split}
  \end{equation}
  where $C_1$ is a universal constant. By Proposition \ref{p:reg1}, there is a
  constant $B_1$, depending only on $\| V \|_{L^1}$ and monotonically
  increasing on $\| \varphi \|_{H^1}$, such that $\| \varphi_t \|_{H^1} \le
  B_1$. Thus, by summing \eqref{dal} over $|\alpha| \le n$, we get
  \[
    \sup_{t \in [0,T]} \| \varphi_t \|_{H^n} \le C_3 \| \varphi \|_{H^n} +
    C_4 T^{1/2} \Big[ \sup_{t \in [0,T]} \| \varphi_t \|_{H^n} + \sup_{t \in
    [0,T]} \| \varphi_t \|_{H^{n-1}}^3 \Big],
  \]
  for some constant $C_3$ depending only on $n$, and some constant $C_4$
  depending only on $\| V \|_{L^1}$ and monotonically increasing on $\|
  \varphi \|_{H^1}$ and $n$. Therefore, by choosing $T > 0$ such that $C_4
  T^{1/2} \le 1/2$, we obtain
  \[
    \sup_{t \in [0,T_n]} \| \varphi_t \|_{H^n} \le 2 C_3 \| \varphi \|_{H^n}
    + \sup_{t \in [0,T_n]} \| \varphi_t \|_{H^{n-1}}^3.
  \]
  This is the desired estimate. Observe that $T$ depends only on $\| V
  \|_{L^1}$, $\| \varphi \|_{H^1}$ and $n$, and decreases monotonically with
  respect to $\| \varphi \|_{H^1}$.
\end{proof}


\subsection{$\ph$ converges to $\varphi_t$ in $L^2$}


The next lemma says that solutions $\varphi_t^{(N)}$ of the modified Hartree
equation converge to solutions $\varphi_t$ of the Gross-Pitaevskii equation
as $N \to \infty$.


\begin{lem} \label{lem:phitN2phit}
  Let $\varphi \in H^2(\R^3)$ with $\| \varphi \|_{L^2} = 1$. Suppose that
  $f \in L^\infty(\R^3)$ and $V \in C_c^\infty(\R^3)$ with $fV \ge 0$. For $N \ge
  1$, consider the solution $\varphi_t^{(N)} \in H^1(\R^3)$ of the modified
  Hartree equation
  \[
    i \partial_t \varphi_t^{(N)} = - \Delta \varphi_t^{(N)} + (N f_N V_N *
    |\varphi_t^{(N)}|^2) \varphi_t^{(N)}
  \]
  with initial data $\varphi^{(N)}_0 = \varphi$, where $f_N V_N(x) = N^2
  f(Nx)V(Nx)$. Consider also the solution $\varphi_t \in H^1(\R^3)$ of the
  Gross-Pitaevskii equation
  \[
    i \partial_t \varphi_t = - \Delta \varphi_t + 8 \pi a_0 |\varphi_t|^2
    \varphi_t
  \]
  with initial data $\varphi_0 = \varphi$, where $8 \pi a_0 = \int f V$.
  Then, for all $N \ge 1$ and $t \ge 0$,
  \[
    \| \varphi_t^{(N)} - \varphi_t \|_{L^2} \le \frac{C}{N} e^{e^{K t}},
  \]
  where $C$ and $K$ are constants that depend only on $\| \varphi \|_{H^1}$,
  $\| \varphi \|_{H^2}$ and $\| fV \|_{L^1}$.
\end{lem}


\begin{proof}
  Observing that $\langle \Delta \varphi_t, \varphi_t^{(N)} \rangle = \langle
  \varphi_t, \Delta \varphi_t^{(N)} \rangle$, we easily find
  \begin{equation}\label{ddt}
    \begin{aligned}
      \partial_t \| \varphi_t - \varphi_t^{(N)} \|_{L^2}^2 & = -2 \Im \langle
      \varphi_t, \big(N f_N V_N * |\varphi_t^{(N)}|^2 - 8 \pi a_0 |\varphi_t|^2\big)
      \varphi_t^{(N)} \rangle \\
      & = 2 \Im \langle \varphi_t, \big(N f_N V_N * |\varphi_t^{(N)}|^2 - 8 \pi
      a_0 |\varphi_t|^2\big) (\varphi_t - \varphi_t^{(N)}) \rangle \\
      & = 2 \Im \langle \varphi_t, \big(N f_N V_N * |\varphi_t|^2 - 8 \pi a_0
      |\varphi_t|^2\big) (\varphi_t - \varphi_t^{(N)}) \rangle \\
      & \quad + 2 \Im \langle \varphi_t, N f_N V_N * \big(|\varphi_t^{(N)}|^2 -
      |\varphi_t|^2\big) (\varphi_t - \varphi_t^{(N)}) \rangle.
    \end{aligned}
  \end{equation}
  In order to apply Gronwall's inequality, we next estimate each term in this
  expression.


  By H\"older's and Sobolev's inequalities,
  \begin{align*}
    & |\langle \varphi_t, \big(N f_N V_N * |\varphi_t|^2 - 8 \pi a_0
    |\varphi_t|^2\big) (\varphi_t - \varphi_t^{(N)}) \rangle| \\
    & \qquad \le \| \varphi_t \|_{L^6} \| \big(N f_N V_N * |\varphi_t|^2 - 8 \pi
    a_0 |\varphi_t|^2\big) (\varphi_t - \varphi_t^{(N)}) \|_{L^{6/5}} \\
    & \qquad \apprle \| \varphi_t \|_{H^1} (\| \varphi_t \|_{L^2} + \|
    \varphi_t^{(N)} \|_{L^2}) \| N f_N V_N * |\varphi_t|^2 - 8 \pi a_0
    |\varphi_t|^2 \|_{L^3}.
  \end{align*}
  By triangle inequality, Sobolev's inequality, and Young's inequality,
  \begin{align*}
    & |\langle \varphi_t, N f_N V_N * \big(|\varphi_t^{(N)}|^2 -
    |\varphi_t|^2\big) (\varphi_t - \varphi_t^{(N)}) \rangle| \\
    & \le \int \di x\, |\varphi_t(x)| |\varphi_t(x) - \varphi_t^{(N)}(x)|
    \int \di y\, N f_N V_N(x-y) \big(|\varphi_t^{(N)}(y)| +
    |\varphi_t(y)|\big) |\varphi_t^{(N)}(y) - \varphi_t(y)| \\
    & \le \| \varphi_t \|_{L^\infty} ( \| \varphi_t^{(N)} \|_{L^\infty} + \|
    \varphi_t \|_{L^\infty} ) \int dx dy \, |\varphi_t(x) -
    \varphi_t^{(N)}(x)| N f_N V_N(x-y) |\varphi_t^{(N)}(y) - \varphi_t(y)|
    \\
    & \apprle \| \varphi_t \|_{H^2} ( \| \varphi_t^{(N)} \|_{H^2} + \|
    \varphi_t \|_{H^2} ) \| fV \|_{L^1} \| \varphi_t^{(N)} - \varphi_t
    \|_{L^2}^2.
  \end{align*}
  Thus, substituting all this into \eqref{ddt}, and using Corollary
  \ref{c:regn} and Proposition \ref{p:reg1} to estimate $\| \varphi_t^{(N)}
  \|_{H^1}$, $\|
  \varphi_t^{(N)} \|_{H^2}$, $\| \varphi_t \|_{H^1}$ and $\| \varphi_t
  \|_{H^2}$, we obtain
  \begin{equation} \label{endproof2}
    \partial_t \| \varphi_t^{(N)} - \varphi_t \|_{L^2}^2 \le C e^{Kt} \|
    \varphi_t^{(N)} - \varphi_t \|_{L^2}^2 + C \| N f_N V_N * |\varphi_t|^2
    - 8 \pi a_0 |\varphi_t|^2 \|_{L^3},
  \end{equation}
  where $C$ and $K$ are constants that depend only on $\| \varphi \|_{H^1}$,
  $\| \varphi \|_{H^2}$ and $\| fV \|_{L^1}$. We are left to estimating the
  second term in \eqref{endproof2}.


  Write
  \begin{align*}
    N f_N V_N * |\varphi_t|^2(x) - 8 \pi a_0 |\varphi_t|^2(x) & = \int dy
    \big( |\varphi_t(x-y)|^2 - |\varphi_t(x)|^2 \big) N^3 fV(Ny) \\
    & = \int dz \big( |\varphi_t(x-z/N)|^2 - |\varphi_t(x)|^2 \big) fV(z),
  \end{align*}
  and let $R$ be such that $\text{supp} \ V \subset \{ x \in \R^3 : |x| \le
  R \}$. By Minkowski's, H\"older's, and Sobolev's inequalities,
  \begin{align*}
    \| N f_N V_N * |\varphi_t|^2 - 8 \pi a_0 |\varphi_t|^2 \|_{L^3} & \le
    \int dz \, \| |\varphi_t(\, \cdot \, -z/N)|^2 - |\varphi_t|^2 \|_{L^3}
    |fV(z)| \\
    & \le \| fV \|_{L^1} \sup_{|z| \le R} \| |\varphi_t(\, \cdot \, -
    z/N)|^2 - |\varphi_t|^2 \|_{L^3}.
  \end{align*}
  Given $\varepsilon = 1/N$, there exists $\psi_t \in C^\infty(\R^3)$ such
  that $\| \varphi_t - \psi_t \|_{H^2} < 1/N$. Hence, by H\"older's
  inequality, Sobolev's inequality and an $\varepsilon/3$-argument, the mean
  value theorem (with some constant $0 \le c \le 1$), and Sobolev's
  inequality again,
  \begin{align*}
    \| |\varphi_t(\, \cdot \, - z/N)|^2 - |\varphi_t|^2 \|_{L^3} & \le 2 \|
    \varphi_t \|_{L^6} \| |\varphi_t(\, \cdot \, - z/N)| - |\varphi_t|
    \|_{L^6} \\
    & \apprle \| \varphi_t \|_{H^1} \big( 1/N + \| |\psi_t(\, \cdot \, - z/N)|
    - |\psi_t| \|_{L^6} \big) \\
    & \apprle \| \varphi_t \|_{H^1} \big( 1/N + |z|/N \| \nabla |\psi_t(\,
    \cdot \, - c z/N)| \|_{L^6} \big) \\
    & \apprle \| \varphi_t \|_{H^1} \big( 1/N + |z|/N \| \psi_t \|_{H^2}
    \big) \\
    & \apprle \| \varphi_t \|_{H^1} \big( 1/N + |z|/N^2 + |z|/N \| \varphi_t
    \|_{H^2} \big).
  \end{align*}
  Therefore, using again Corollary \ref{c:regn} and Proposition \ref{p:reg1},
  \begin{align*}
    & \| N f_N V_N * |\varphi_t|^2 - 8 \pi a_0 |\varphi_t|^2 \|_{L^3} \\
    & \apprle \| fV \|_{L^1} \| \varphi_t \|_{H^1} \Big( \frac{1}{N} +
    \frac{R}{N^2} + \frac{R}{N} \| \varphi_t \|_{H^2} \Big) \le
    \frac{C_1}{N}(1 + \| \varphi_t \|_{H^2}) \le \frac{C_2}{N} e^{Kt},
  \end{align*}
  where $C_2$ is a constant that depends only on $\| \varphi \|_{H^1}$, $\|
  \varphi \|_{H^2}$ and $\| fV \|_{L^1}$. Substituting this into
  \eqref{endproof2}, we obtain
  \[
    \partial_t \| \varphi_t^{(N)} - \varphi_t \|_{L^2}^2 \le C_3 e^{Kt} \|
    \varphi_t^{(N)} - \varphi_t \|_{L^2}^2 + \frac{C_3}{N} e^{Kt}.
  \]
  Finally, by Gronwall's inequality (with reminder),
  \begin{align*}
    \| \varphi_t^{(N)} - \varphi_t \|_{L^2} & \le e^{\int_0^t C_3 e^{Ks} ds}
    \Big( \| \varphi_0^{(N)} - \varphi_0 \|_{L^2} + \int_0^t \frac{C_3}{N}
    e^{Ks} ds \big) \\
    & = e^{K^{-1} C_3 e^{Kt}} \Big( \| \varphi_0^{(N)} - \varphi_0 \|_{L^2}
    + \frac{C_3}{KN} e^{Kt} \Big) \le \frac{C_4}{N} e^{e^{K_1 t}},
  \end{align*}
  for some constants $C_4$ and $K_1$ depending only on $\| \varphi
  \|_{H^1}$, $\| \varphi \|_{H^2}$ and $\| fV \|_{L^1}$.
\end{proof}


\subsection{Regularity of $\phdot$ and $\phddot$}
\label{sec:phdotreg}
\label{ss:phdotreg}
\begin{lem} \label{lm:phdotregularity}
Let $\varphi \in H^4(\R^3)$ with $\| \varphi \|_{L^2} = 1$. Suppose that
  $f \in L^\infty(\R^3)$ and $V \in C_c^\infty(\R^3)$ with $fV \ge 0$. For $N \ge
  1$, consider a solution $\varphi_t^{(N)} \in H^1(\R^3)$ of the modified
  Hartree equation
  \[
    i \partial_t \varphi_t^{(N)} = - \Delta \varphi_t^{(N)} + (N f_N V_N *
    |\varphi_t^{(N)}|^2) \varphi_t^{(N)}
  \]
  with initial data $\varphi^{(N)}_0 = \varphi$, where $f_N V_N(x) = N^2
  f(Nx)V(Nx)$.

Then there exists a constant $C$ such that:
\begin{enumerate}
 \item $\norm{\phdot}_{L^2} \leq C \norm{\ph}_{H^2}^2.$ % first step: \leq \norm{\ph}_{H^2} + 8\pi a_0 \norm{\ph}_{H^2}^2
\item For $i, j \in \{1,2,3\}$
\bd
\norm{\nabla \phdot}_{L^2} \leq C \norm{\ph}_{H^3}^2, \quad
\norm{\partial_i \partial_j \phdot}_{L^2} \leq C \norm{\ph}_{H^4}^3, \quad \norm{\phdot}_\infty \leq C \norm{\ph}_{H^4}^3.
\ed
\item Furthermore $\norm{\phddot}_{L^2} \leq C \norm{\ph}_{H^4}^4$.
\end{enumerate}
\end{lem}
\begin{proof}(i) According to the modified Hartree equation
\be{no11}
\norm{\phdot}_{L^2} \leq \norm{\ph}_{H^2} + \norm{\left(N f_N V_N \ast \lvert \ph\rvert^2 \right)\ph}_{L^2}.
\ee
Now we calculate
\begin{align*}
\norm{\left(N f_N V_N \ast \lvert \ph\rvert^2 \right)\ph}_{L^2}^2 & = \int \di x\, \lvert \ph(x)\rvert^2 \left\lvert \int \di y\, N f_NV_N(x-y) \lvert \ph(y)\rvert^2 \right\rvert^2 \\
& \leq \int \di x\, \lvert \ph(x)\rvert^2 \left\lvert \norm{\ph}_\infty^2 \int \di y\, N f_N V_N(x-y) \right\rvert^2 \\
& \leq \norm{\ph}_{L^2}^2 \norm{\ph}_\infty^4 \norm{fV}_{L^1}^2 \leq C \norm{\ph}_{H^2}^4,
\end{align*}
where in the last step a Sobolev inequality and $\norm{\ph}_{L^2} = 1$ was used.
This combined with \eqr{no11} proves (i).

(ii)
Applying the gradient to the modified Hartree equation, the product rule yields
\begin{align}
i \nabla \phdot & = - \nabla \Delta \ph + \left( N f_N V_N \ast \lvert \ph \rvert^2 \right) \nabla \ph \label{eq:firstline}\\
& \quad + \left( N f_N \ast \cc{\ph} \nabla \ph \right) \ph + \left( N f_N V_N \ast \cc{\nabla \ph} \ph \right) \ph. \label{eq:secondline}
\end{align}
Clearly, $\norm{\nabla \Delta \ph}_{L^2} \leq \norm{\ph}_{H^3}$. Now for line \eqref{eq:firstline}, we have the estimate
\begin{align*}
\norm{\left( N f_N V_N \ast \lvert \ph \rvert^2 \right)\nabla \ph}_{L^2}^2 & = \int \di x \lvert \nabla \ph(x)\rvert^2 \left\lvert \int \di y N f_N V_N(y) \lvert\ph(x-y) \rvert^2\right\rvert^2\\
& \leq \int \di x \lvert \nabla \ph(x)\rvert^2 \norm{\ph}_\infty^4 \left\lvert \int \di y N f_N V_N(y) \right\rvert^2 \\
& = \norm{\nabla \ph}_{L^2}^2 \norm{\ph}_\infty^4 \norm{fV}_{L^1}^2. 
\end{align*}
Similarly, we show that the $L^2$-norms of the summands in line \eqref{eq:secondline} are both bounded by $C \norm{\ph}_{H^2}^2 \norm{\ph}_{H^3}^2$. We then conclude that
\[
\norm{\nabla \phdot}_{L^2} \leq \norm{\ph}_{H^3} + C \norm{\nabla \ph}_{L^2} \norm{\ph}_\infty^2 + C \norm{\ph}_{H^2} \norm{\ph}_{H^3} \leq C \norm{\ph}_{H^3}^2.
\]
The bound for $\norm{\partial_i \partial_j \phdot}_{L^2}$ is proven by the same strategy.
Then by a Sobolev inequality $\norm{\phdot}_\infty \leq C \norm{\phdot}_{H^2} \leq C \norm{\ph}_{H^4}^3$.

(iii) To prove that $\phddot$ is in $L^2$, we apply $i\partial_t$ to the modified Hartree equation and get
\begin{align*}
ii\phddot & = - \Delta i \phdot + \left( N f_N V_N \ast \cc{\ph}i \phdot \right) \ph \\
& \quad + \left( N f_N V_N \ast \cc{(-i\phdot)} \ph \right) \ph + \left( N f_N V_N \ast \lvert\ph\rvert^2 \right) i\phdot.
\end{align*}
Now for $i\phdot$ we plug in the modified Hartree equation and arrive at
\begin{align*}
ii \phddot & = \Delta^2 \ph - \Delta \left( (N f_N V_N \ast \lvert \ph\rvert^2)\ph \right) + \left( N f_N V_N \ast \lvert \ph\rvert^2 \right)^2 \ph \\
& + (N f_N V_N \ast \cc{\ph}(-\Delta \ph))\ph + 2 \left[ N f_N V_N \ast \left( \lvert \ph\rvert^2 (N f_N V_N \ast \lvert \ph\rvert^2) \right) \right] \ph \\
& + (N f_N V_N \ast \cc{(-\Delta \ph)} \ph)\ph + \left( N f_N V_N \ast \lvert \ph \rvert^2 \right)(-\Delta \ph).
\end{align*}
The $L^2$-norms of the summands on the right hand side can be estimated by the same methods as in (i) and (ii), e.\,g.\ 
\begin{align*}
 \norm{\left( N f_N V_N \ast \lvert \ph\rvert^2 \right) \Delta \ph}^2 & = \int \di x \lvert \Delta \ph(x)\rvert^2 \left\lvert \int \di y\, N f_N V_N(y) \lvert \ph(x-y) \rvert^2 \right\rvert^2 \\
& \leq \norm{\ph}_\infty^4 \norm{fV}_{L^1}^2 \norm{\Delta \ph}_{L^2}^2 \leq C \norm{\ph}_{H^4}^6. \qedhere
\end{align*}
%(Actually, the $H^2$-norm is enough for this term, but we need the $H^4$-norm anyway for some of the other summands.)
\end{proof}

\section{Bounds for the Bogoliubov Transformation}
\label{s:bogbounds}
%
\begin{lem} \label{lm:kbounds}
Let $\varphi \in H^4(\R^3)$ with $\| \varphi \|_{L^2} = 1$. Suppose that
  $f \in L^\infty(\R^3)$ and $V \in C_c^\infty(\R^3)$ with $fV \ge 0$. For $N \ge
  1$, consider a solution $\varphi_t^{(N)} \in H^1(\R^3)$ of the modified
  Hartree equation
  \[
    i \partial_t \varphi_t^{(N)} = - \Delta \varphi_t^{(N)} + (N f_N V_N *
    |\varphi_t^{(N)}|^2) \varphi_t^{(N)}
  \]
  with initial data $\varphi^{(N)}_0 = \varphi$, where $f_N V_N(x) = N^2
  f(Nx)V(Nx)$. Let $k(x,y) = -N w_N(x-y) \ph(x) \ph(y)$.

Then there exists a constant $C$ such that:
\bd
\norm{\dot k} \leq C \norm{\ph}_{H^2}^2
 \quad \mbox{and} \quad
\norm{\ddot k} \leq C \norm{\ph}_{H^4}^4.
\ed
\end{lem}
\begin{proof} (i) We use Hardy's inequality and Lemma \ref{lm:phdotregularity} to estimate
\begin{align*}
\norm{\dot k}^2 & = \int \di x\di y\, \left\lvert \frac{\di}{\di t} \left( -N w_N(x-y) \ph(x)\ph(y) \right) \right\rvert^2 \\
& \leq 2 \int \di x\di y\, \lvert N w_N(x-y) \phdot(x) \ph(y) \rvert^2 + 2 \int \di x \di y\, \lvert N w_N(x-y) \ph(x) \phdot(y) \rvert^2 \\
& \leq 4 \left( \max\{R,a_0\} \right)^2 \int \di x\di y\, \lvert \phdot(x)\rvert^2 \frac{\lvert\ph(y)\rvert^2}{\lvert x-y\rvert^2} \\
& \leq C \int \di x\di y\, \lvert \phdot(x)\rvert^2 \lvert \nabla_y \ph(y)\rvert^2 \\
& \leq C \norm{\phdot}_{L^2}^2 \norm{\ph}_{H^1}^2 \leq C \norm{\ph}_{H^2}^4.
\end{align*}
(ii) First, we use $\lvert N w_N(x-y) \rvert \leq \max \{R,a_0\} \lvert x-y\rvert^{-1}$ and Hardy's inequality to show
\bd
\norm{\ddot k} \leq C \norm{\phddot}_{L^2} + C \norm{\phdot}_{L^2} \norm{\nabla \phdot}_{L^2}.
\ed
This quantities are now easily bounded using Lemma \ref{lm:phdotregularity}.
\end{proof}

\begin{lem}
\label{lem:dottedests1}
Let $p$ and $s$ be the integral kernels defined in Lemma \ref{l:bt}. Then, the following estimates hold:
\begin{enumerate}
\item If $k_t \in L^2(\Rbb^3\times\Rbb^3)$ is an arbitrary time-dependent integral kernel then
\bd
\norm{\dot p} \leq \norm{\dot k} \cosh(\norm{k}) \quad \mbox{and} \quad \norm{\dot s} \leq \norm{\dot k} \cosh(\norm{k}).
\ed
\item If $k(x,y) = - N w_N(x-y) \ph(x) \ph(y)$ with $\ph$ a solution of the modified Hartree equation with initial data $\varphi \in H^3(\Rbb^3)$, then there exists a constant $C$ such that
\bd
\norm{\dot p} \leq C \norm{\ph}_{H^2}^2 \quad \mbox{and} \quad \norm{\dot s} \leq C \norm{\ph}_{H^2}^2
\ed
and
\bd
\sup_x \norm{\dot p_x}_{L^2} \leq C \norm{\ph}_{H^4}^3 \quad \mbox{and} \quad \sup_x \norm{\dot s_x}_{L^2} \leq C \norm{\ph}_{H^4}^3.
\ed
\end{enumerate}
\end{lem}
\begin{proof}
We give only an outline of the proof.

(i) Differentiate the summands of the series individually and make use of the estimate $\norm{fg}_{L^2} \leq \norm{f}_{L^2} \norm{g}_{L^2}$ for $f,g \in L^2(\Rbb^3\times \Rbb^3)$.

(ii) For the first pair of bounds, use (i) and employ the estimates $\norm{k} \leq C$ and $\norm{\dot k} \leq C \norm{\ph}_{H^2}^2$ (see Lemma \ref{lm:kbounds}).

For the second pair of bounds, notice that for $f_i \in L^2(\Rbb^3 \times \Rbb^3)$, $i=1,\dots n$, by H\"older's inequality we have the estimate
\begin{align*}
\norm{\left[ f_1 f_2 \cdots f_n\right]_x}^2 & = \int \di y \left\lvert \int \di z_1 \cdots \di z_{n-1}\ f_1(y,z_1) f_2(z_1,z_2) \cdots f_n(z_{n-1},x) \right\rvert^2 \\
& \leq \norm{f_1}^2 \norm{f_2}^2 \cdots \norm{f_{n-1}}^2 \int \di z_{n-1} \lvert f_n(z_{n-1},x) \rvert^2 
\end{align*}
(notice that on the lhs $\norm{\cdot}= \norm{\cdot}_{L^2(\Rbb^3)}$, while on the rhs $\norm{\cdot} = \norm{\cdot}_{L^2(\Rbb^3\times\Rbb^3)}$).
For $f_n = k$ or $f_n = \cc{k}$ we use Hardy's inequality to prove $\int \di z_{n-1} \lvert f_n(z_{n-1},x) \rvert^2 \leq C \lvert \ph(x) \rvert^2 \norm{\nabla \ph}^2$. For $f_n = \cc{\dot k}$, again by Hardy's inequality, we have the estimate $\int \di z_{n-1} \lvert f_n(z_{n-1},x) \rvert^2 \leq C\lvert \ph(x)\rvert^2 \norm{\nabla \phdot}^2 + C \lvert \phdot(x)\rvert^2 \norm{\nabla \ph}^2$. Then use the Lemmata \ref{lm:phdotregularity} and \ref{lm:kbounds} and the fact that $\norm{\ph}_{H^1} \leq C$ by Corollary \ref{c:regn}.
\end{proof}

\begin{lem}
\label{lem:dottedests2}
Let $k(x,y) = -N w_N(x-y) \ph(x) \ph(y)$, where $\ph$ is a solution of the modified Hartree equation with initial data $\varphi \in H^3(\Rbb^3)$. Then there exists a constant $C$ such that
\begin{enumerate}
 \item $\int \di x \di y \frac{1}{N} V_N(x-y) \lvert \dot k(x,y)\rvert^2 \leq C \norm{\ph}_{H^3}^4$, 
 \item $\int \di x\di y\, V_N(x-y) \lvert \dot r(x,y)\rvert^2 \leq \frac{C}{N} \norm{\ph}_{H^3}^4 \norm{\ph}_{H^2}^2$,
 \item $\int \di x\di y\frac{1}{N} V_N(x-y) \lvert \dot s(x,y)\rvert^2 \leq C \norm{\ph}_{H^3}^4 + \frac{C}{N^2} \norm{\ph}_{H^3}^4 \norm{\ph}_{H^2}^2$.
\end{enumerate}
%Notice that in (i) and (iii) compared to (ii) an additional $1/N$ is required because $s$ is more singular than $r$.
\end{lem}
\begin{proof} (iii) is a trivial corollary from (i) and (ii).

(i) The calculation works as follows:
\begin{align*}
& \int \di x\di y \frac{1}{N}V_N(x-y) \lvert \dot k(x,y)\rvert^2 \\
& = \int \di x\di y \frac{1}{N} V_N(x-y) \lvert -N w_N(x-y) \phdot(x) \ph(y) - N w_N(x-y) \ph(x) \phdot(y)\rvert^2 \\
& \leq \frac{4}{N} \int \di x\di y V_N(x-y) \lvert N w_N(x-y) \rvert^2 \lvert \phdot(x)\rvert^2 \lvert \ph(y)\rvert^2 \\
& = 4 \int \di x \di y NV_N(x-y) w_N(x-y)^2 \lvert \phdot(x)\rvert^2 \lvert \ph(y)\rvert^2 \\
& \leq 4 \norm{N V_N w_N^2}_{L^1} \norm{\lvert\phdot\rvert^2}_{L^2} \norm{\lvert \ph\rvert^2}_{L^2}.
\end{align*}
In the last step Young's inequality was used. Using $\lvert w_N\rvert \leq 1$ we get $\norm{N V_N w_N^2}_{L^1} \leq \norm{V}_{L^1} = b_0$.
Furthermore, by a Sobolev inequality and Lemma \ref{lm:phdotregularity}, we get
\bd
\norm{\lvert \phdot\rvert^2}_{L^2} = \norm{\phdot}_{L^4}^2 \leq C \norm{\phdot}_{H^1}^2 \leq C(\norm{\ph}_{H^2}^2 + \norm{\ph}_{H^3}^2)^2 \leq C \norm{\ph}_{H^3}^4.
\ed
Also by a Sobolev inequality $\norm{\lvert \ph\rvert^2}_{L^2} \leq C$ because $\norm{\ph}_{H^1} \leq C$. Putting all this together estimate (i) is obtained.

(ii) By H\"older's inequality, one proves that for integral kernels $f_1, \dots f_n \in L^2(\Rbb^3\times\Rbb^3)$
\[
\lvert (f_1 \cdots f_N)(x,y)\rvert \leq \left( \int \di z \lvert f_1(x,z)\rvert^2 \right)^{1/2} \left( \int \di z \lvert f_n(z,y)\rvert^2\right)^{1/2} \norm{f_2} \cdots \norm{f_{n-1}}.
\]
We now apply this to get
\begin{align*}
& \lvert \dot r(x,y) \rvert = \left\lvert \sum_{n=1}^\infty \frac{1}{(2n+1)!} \partial_t \left(k(\cc k k)^n \right)(x,y)\right\rvert \\
& \leq \sum_{n=1}^\infty \frac{1}{(2n+1)!} \bigg\lvert\Big( \dot k(\cc k k)^n + \sum_{j=1}^{n-1} k(\cc k k)^{j-1}(\cc{\dot k}k + \cc k \dot k)(\cc k k)^{n-j} + k (\cc k k)^{n-1} \cc{\dot k} k + k(\cc{k}k)^{n-1}\cc{k}\dot k \Big)(x,y)\bigg\rvert \\
& \leq \left(\int \di z \lvert \dot k(x,z)\rvert^2 \right)^{1/2} \left( \int \di z \lvert k(z,y)\rvert^2 \right)^{1/2} \sum_{n=1}^\infty \frac{\norm{k}^{2n-1}}{(2n+1)!} \\
& \quad + \norm{\dot k} \left( \int \di z \lvert k(x,z)\rvert^2 \right)^{1/2} \left( \int \di z \lvert k(z,y)\rvert^2 \right)^{1/2} \sum_{n=1}^\infty \frac{\norm{k}^{2n-2}}{(2n+1)!} (2n-1) \tagg{no2}\\
& \quad + \left( \int \di z \lvert k(x,z)\rvert^2 \right)^{1/2} \left( \int \di z \lvert \dot k(z,y)\rvert^2 \right)^{1/2} \sum_{n=1}^\infty \frac{\norm{k}^{2n-1}}{(2n+1)!}.
\end{align*}
As $\norm{k} \leq C$ the series appearing here are bounded by a constant $C$ (clearly, the series are summable). From Lemma \ref{lm:kbounds} we know that $\norm{\dot k} \leq C \norm{\ph}_{H^2}^2$. By Proposition \ref{p:wphi} we get
\bd
\int \di z \lvert k(x,z)\rvert^2 \leq \lvert \ph(x)\rvert^2 C \norm{\nabla \varphi}_{L^2}^2 \leq C \lvert \ph(x)\rvert^2.
\ed
and
\bd
\int \di z \lvert \dot k(x,z)\rvert^2 \leq \lvert \ph(x)\rvert^2 C \norm{\nabla \phdot}_{L^2}^2 + \lvert \phdot(x)\rvert^2 C \norm{\nabla \ph}_{L^2}^2.
\ed
These estimates and the estimate \eqr{no2} are now used to conclude
\begin{align*}
& \int \di x \di y V_N(x-y) \lvert \dot r(x,y)\rvert^2 \\
& \leq C \int \di x \di y V_N(x-y) \int \di z_1 \lvert \dot k(x,z_1)\rvert^2 \int \di z_2 \lvert k(z_2,y)\rvert^2 \\
& \quad + C \norm{\ph}_{H^2}^4 \int \di x \di y V_N(x-y) \int \di z_1 \lvert k(x,z_1)\rvert^2 \int \di z_2 \lvert k(z_2,y)\rvert^2 \\
& \leq C \norm{\nabla \phdot}_{L^2}^2 \int \di x \di y V_N(x-y) \lvert \ph(x)\rvert^2 \lvert \ph(y)\rvert^2 \\
& \quad + C \norm{\nabla \ph}_{L^2}^2 \int \di x \di y V_N(x-y) \lvert \phdot(x)\rvert^2 \lvert \ph(y)\rvert^2 \\
& \quad + C \norm{\ph}_{H^2}^4 \int \di x \di y V_N(x-y) \lvert \ph(x)\rvert^2 \lvert \ph(y)\rvert^2 \\
& \leq C \norm{\nabla \phdot}_{L^2}^2 \norm{\ph}_\infty^2 \frac{b_0}{N} \norm{\ph}_{L^2}^2 + C \norm{\nabla \ph}_{L^2}^2 \norm{\ph}_\infty^2 \frac{b_0}{N} \norm{\phdot}_{L^2}^2\\
& \quad + C \norm{\ph}_{H^2}^4 \norm{\ph}_\infty^2 \frac{b_0}{N} \norm{\ph}_{L^2}^2.
\end{align*}
In the last step we pulled out $\norm{\ph}_\infty^2$ from all the integrals. Now using the Sobolev inequality $\norm{\ph}_\infty \leq C \norm{\ph}_2$ and then Lemma \ref{lm:phdotregularity} the proof is complete.
\end{proof}

\begin{lem}
\label{lem:dottedests3}
Let $k(x,y) = -N w_N(x-y) \ph(x)\ph(y)$ where $\ph$ is a solution to the modified Hartree equation with initial data $\varphi \in H^3(\Rbb^3)$. Then there exists a constant $C$ such that
\bd
\norm{\gradone \dot p} \leq C \norm{\ph}_{H^3}^2 \quad \mbox{and} \quad \norm{\gradone \dot r} \leq C \norm{\ph}_{H^3}^2.
\ed
\end{lem}
\begin{proof}
We only give the proof for $\gradone \dot p$ here; the proof for $\gradone \dot r$ works similarly and will be omitted.

Recalling the definition of $p$ we have
\bd
\gradone \dot p = \frac{\partial}{\partial_t} \left( \sum_{n=1}^\infty (k \cc k)^{n-1} \gradone(k \cc k) \frac{1}{(2n)!} \right).
\ed
Using the product rule for the time derivative and Lemma \ref{l:kernels} \ref{k} we obtain
\begin{align*}
\norm{\gradone \dot p} & \leq \sum_{n=2}^\infty \frac{1}{(2n)!} \sum_{j=0}^{n-2} \norm{k}^{2j} \norm{\partial_t(k \cc k)} \norm{k}^{2(n-2-j)} \norm{\gradone (k \cc k)} \\
& \quad + \sum_{n=1}^\infty \frac{1}{(2n)!} \norm{k}^{2(n-1)} \norm{\gradone \partial_t (k \cc k)} \\
& \leq C \norm{\dot k} \norm{\gradone(k \cc k)} + C \norm{\gradone(k \cc{\dot k})} + C \norm{\gradone (\dot k \cc k)},
\end{align*}
where in the last step, summable series depending on $\norm{k}$ were absorbed in $C$.

We are left with the task of giving an estimate for $\norm{\gradone(k \cc{\dot k})}$ (an estimate for $\norm{\gradone (\dot k \cc k)}$ is proven in the same way). We start by applying the product rule:
\begin{align}
& \norm{\gradone(k \cc{\dot k})}^2 \nonumber \\
& = \int \di x \di y \bigg\lvert \int \nabla_y\left( N w_N(z-y) \cc{\phdot(y)} \cc{\ph(z)} + N w_N(z-y) \cc{\ph(y)} \cc{\phdot(z)} \right) \times \nonumber \\
& \qquad\qquad\qquad \times N w_N(x-z) {\ph(z)} {\ph(x)} \di z \bigg\rvert^2 \nonumber \\
& \leq 4 \int \di x \di y \left\lvert \int N \nabla w_N(z-y) \cc{\phdot(y)} \lvert\ph(z)\rvert^2 {\ph(x)} N w_N(x-z) \di z \right\rvert^2 \label{eq:kk1} \\
& \quad + 4 \int \di x \di y \left\lvert N w_N(z-y) \cc{\nabla_y \phdot(y)} \lvert \ph(z)\rvert^2 {\ph(x)} N w_N(x-z) \di z \right\rvert^2 \label{eq:kk2}\\
& \quad + 4 \int \di x \di y \left\lvert N \nabla w_N(z-y) \cc{\ph(y)} \cc{\phdot(z)} {\ph(z)} {\ph(x)} N w_N(x-z) \di z \right\rvert^2 \label{eq:kk3}\\
& \quad + 4 \int \di x \di y \left\lvert N w_N(z-y) \cc{\nabla_y \ph(y)} \cc{\phdot(z)} {\ph(z)} {\ph(x)} N w_N(x-z) \right\rvert^2. \label{eq:kk4}
\end{align}

We now prove estimates for these four terms. For the summands \eqref{eq:kk2} and \eqref{eq:kk4} we apply Hardy's inequality; both terms are bounded by $C \norm{\nabla \phdot}_{L^2}^2$.
%For \eqref{eq:kk1} and \eqref{eq:kk3} it is a non-trivial task to find an estimate because $\nabla w_N$ has a stronger singularity than $w_N$.
We will now show how to estimate line \eqref{eq:kk1}.
\begin{align*}
\eqref{eq:kk1} & \leq C \int \di x \di y \left\lvert \int \di z \frac{1}{\lvert y-z\rvert^2} \cc{\phdot(y)} \lvert \ph(z)\rvert^2 {\ph(x)} \frac{1}{\lvert z-x\rvert} \right\rvert^2 \\
& = C \int \di x \di y \di z_1 \di z_2 \frac{\lvert\phdot(y)\rvert^2 \lvert \ph(x)\rvert^2 \lvert \ph(z_1)\rvert^2 \lvert \ph(z_2)\rvert^2}{\lvert z_1-x\rvert \lvert z_2 -x\rvert \lvert y-z_1\rvert^2 \lvert y-z_2\rvert^2} \\
& = C \int \di y \lvert\phdot(y)\rvert^2 \int \di z_1 \di z_2 \frac{\lvert \ph(z_1)\rvert^2 \lvert \ph(z_2)\rvert^2}{\lvert y-z_1\rvert^2 \lvert y-z_2\rvert^2} \int \di x \frac{\lvert \ph(x)\rvert^2}{\lvert z_1-x\rvert \lvert z_2 - x\rvert}.
\intertext{Using H\"older's inequality and then Hardy's inequality the integral w.\,r.\,t.\ $x$ is bounded by $C \norm{\nabla \ph}_{L^2}^2$. Using $\norm{\nabla \ph}_{L^2}^2 \leq C$ we get}
\eqref{eq:kk1} & \leq C \int \di y \lvert\phdot(y)\rvert^2 \int \di z_1 \frac{\lvert \ph(z_1)\rvert^2 }{\lvert y-z_1\rvert^2 } \int \di z_2 \frac{\lvert \ph(z_2)\rvert^2}{\lvert y-z_2\rvert^2}
\intertext{and by Hardy's inequality, this is bounded by}
& \leq C \int \di y \lvert \phdot(y)\rvert^2 \norm{\nabla \ph}_{L^2}^2 \norm{\nabla \ph}_{L^2}^2 \leq C \norm{\phdot}_{L^2}^2.
\end{align*}
The calculation for \eqref{eq:kk3} works similarly and yields the bound $C \norm{\nabla \phdot}_{L^2}^2$.
\end{proof}
\cite{*} CITE ALL
\bibliographystyle{plain}
\bibliography{gross-pitaevskii}
\end{document}
