%\documentclass[11pt,a4paper,twoside,headsepline]{scrartcl}
\documentclass[11pt,a4paper]{scrartcl} % Added by GBdO
\usepackage[a4paper,hmargin=2.8cm,vmargin=2.8cm]{geometry} % Added by GBdO
%\usepackage[utf8x]{inputenc}
%\usepackage{showkeys}
\usepackage[color,notref,notcite]{showkeys} % Added by GBdO
\usepackage[12h=false]{scrtime}
\usepackage{amsmath, amsthm, amssymb}
\usepackage{dsfont} % added by GBdO

% Original header
%\usepackage{scrpage2}
%\ihead{Gross-Pitaevskii Equation}
%\chead{Page: \thepage}
%\ohead{Version: \today, \thistime}
%\pagestyle{scrheadings}

% How about this footer?
\usepackage{fancyhdr}
\pagestyle{fancy}
\fancyhf{}
\renewcommand{\headrulewidth}{0pt}
\lfoot{{\footnotesize GP equation}}
\cfoot{\thepage}
\rfoot{{\footnotesize \today, \thistime}}

% Nicer marginpar with smaller font
\let\oldmarginpar\marginpar
\renewcommand\marginpar[1]{\-\oldmarginpar[\raggedleft\footnotesize #1]%
  {\raggedright\footnotesize #1}}

%%%%%%%%%%%%%%%%%%%%%%%%% Theorems etc %%%%%%%%%%%%%%%%%%%%%%%%%%%%%%%%%%%%%%%%%
\newtheorem{thm}{Theorem}[section]
\newtheorem{cor}[thm]{Corollary}
\newtheorem{prp}[thm]{Proposition}
\newtheorem{lem}[thm]{Lemma}
\newtheorem{dfn}[thm]{Definition}
\newtheorem{exm}[thm]{Example}

\newtheorem*{rem}{Remark}
\newtheorem*{hyp}{Hypothesis}

%%%%%%%%%%%%%%%%%%%%%%%% Gustavo's aliases %%%%%%%%%%%%%%%%%%%%%%%%%%%%%%%%%%%%%
\newcommand{\R}{\mathds{R}}
\newcommand{\N}{\mathcal{N}}
\newcommand{\K}{\mathcal{K}}

%%%%%%%%%%%%%%%%%%%%%%%% Notation %%%%%%%%%%%%%%%%%%%%%%%%%%%%%%%%%%%%%%%%%%%%%%
\newcommand{\di}{\textrm{d}}		% differential (for integrals)
\newcommand{\Lcal}{\mathcal{L}}		% calligraphic L
\newcommand{\Ncal}{\mathcal{N}}		% calligraphic N
\newcommand{\Kcal}{\mathcal{K}}		% calligraphic N
\newcommand{\Vcal}{\mathcal{V}}		% calligraphic V
\newcommand{\Hcal}{\mathcal{H}}		% calligraphic H
\newcommand{\Ocal}{\mathcal{O}}		% big-O, order-of
\newcommand{\hc}{\mbox{h.c.}}		%hermitian conjugate
\newcommand{\scal}[2]{\left<#1,#2\right>} % scalaer product
\newcommand{\cc}[1]{\overline{#1}}	% complex conjugate
\newcommand{\Rbb}{\mathbb{R}}		% real numbers
\newcommand{\Cbb}{\mathbb{C}}		% complex numbers
\newcommand{\Nbb}{\mathbb{N}}		% natural numbers
\renewcommand{\Re}{\operatorname{Re}} 	%RealPart
\renewcommand{\Im}{\operatorname{Im}} 	%ImaginaryPart
\newcommand{\norm}[1]{\lVert#1\rVert}	%Norm
\newcommand{\ev}[1]{\left<#1\right>}	%expectation value
\newcommand{\ph}{\varphi_t^{(N)}}	% solution of N-dependent Hartree equation
\newcommand{\phdot}{\dot{\varphi}_t^{(N)}}	% time derivative of solution of N-dependent Hartree equation
\newcommand{\sqn}{\sqrt{N}}		% square root of N
\newcommand{\project}[1]{\lvert #1 \big>\big< #1\rvert}	% orthogonal projection operator
\newcommand{\Tr}{\operatorname{Tr}}	% Trace


\newcommand{\be}[1]{\begin{equation}\label{eq:#1}}	%begin equation with label
\newcommand{\ee}{\end{equation}}
\newcommand{\bd}{\begin{displaymath}}			% abbreviation begin displaymath
\newcommand{\ed}{\end{displaymath}}

\newcommand{\tagg}[1]{ \stepcounter{equation} \tag{\theequation} \label{eq:#1} } % add tag and label in align*-environments

\newcommand{\eqr}[1]{\eqref{eq:#1}}			%eqref with prefix :eq

%%%%%%%%%%%%%%%%%%%%%%%%% main content %%%%%%%%%%%%%%%%%%%%%%%%%%%%%%%

%\allowdisplaybreaks    % Let's try to avoid this. GBdO.


\begin{document}


\section{Fock space representation}


\marginpar{I think that is = instead of := in this case}
\begin{equation}
  \label{a}
  \begin{aligned}
    a^*(f) & = \int \di x\, f(x) a^*_x, \\
      a(f) & = \int \di x\, \cc{f(x)} a_x.
  \end{aligned}
\end{equation}


\begin{lem}
  \label{l:a}
  Let $f \in L^2(\R^3)$. Then, for any $\psi \in \mathcal{F}$,
  \begin{equation}
    \label{aNorm}
    \begin{aligned}
      \norm{a(f)\psi} & \leq \norm{f}_{L^2} \norm{\Ncal^{1/2}\psi}, \\
      \norm{a^*(f)\psi} & \leq \norm{f}_{L^2} \norm{(\Ncal+1)^{1/2}\psi}, \\
      \norm{\phi(f) \psi} & \leq 2 \norm{f}_{L^2} \norm{(\N+1)^{1/2} \psi}.
    \end{aligned}
  \end{equation}
\end{lem}


Second quantization
\bd
A(f) := a^*(f) - a(f), \quad W(f) := \exp(A(f))
\ed
Estimates
\marginpar{Did you already use this?}
\bd
\mbox{For } (\nabla a)(f) := \int \di x\, \cc{f(x)}\nabla_x a_x, \mbox{ we have } \norm{(\nabla a)(f)\psi} \leq \norm{f}_{L^2} \norm{\Kcal^{1/2}\psi}.
\ed


Hamiltonian etc.
\bd
\Kcal := \int \di x\, \nabla_x a^*_x \nabla_x a_x,
\quad
\Vcal_N := \frac{1}{2}\int\di x \di y\, V_N(x-y) a^*_x a^*_y a_y a_x
\ed
\bd
V_N(x) := N^2V(Nx), \quad \Hcal := \Kcal + \Vcal_N, \quad \Ncal := \int \di x\, a^*_x a_x
\ed


\fbox{Combo Lemma for Weyl operator} \label{l:W}


\section{Zero-energy scattering equation}


Zero-energy scattering equation
\be{scatteringequation}
\left(-\Delta + \frac{1}{2}V_N \right)(1-w_N) = 0
\ee
\bd
w_N(x) = w(Nx), \quad f_N(x) = 1-w_N(x) = f(Nx)
\ed
\bd
U_N(t) := T^*_t W^*_t e^{- i t \Hcal_N} W_0 T_0,\quad W_t := W(\sqn \ph)
\ed

\begin{lem}[Properties of the scattering equation \cite{ESY2006}]
 \bd
\lvert N w_N(x)\rvert \leq \frac{\max\{a_0,R\}}{\lvert x\rvert}\quad \forall x \in \Rbb^3
\ed
\bd
0 < C_0 \leq w_N(x) \leq 1\quad \forall x \in \Rbb^3, \quad C_0 = C_0(V) \mbox{ uniform in } N
\ed
\bd
\lvert \nabla w_N(x) \rvert \leq \frac{c a_0}{N \lvert x\rvert^2}\quad \forall x \in \Rbb^3, \quad c \mbox { is universal constant}
\ed
\end{lem}


\section{Bogoliubov transformation}


%\marginpar{Sure. If no objections, commenting this out last time...}
%Remark: In some places e.\,g.\ lemma \ref{lem:kvbounds}, we should work backwards, from the proofs of the theorems to the proof of the lemmata, so that we know how exactly to formulate the lemmata.


Bogoliubov transformation
\bd
k(x,y) = k_t(x,y) := -N w_N(x-y) \ph(x) \ph(y) = k(y,x)
\ed
\bd
B(k_t) = \frac{1}{2}\int \di x \di y\, \left( \cc{k(x,y)} a_x a_y - k(x,y) a^*_x a^*_y\right) (!), \quad T_t := e^{-B(k_t)}
\ed
\bd
s(x,y) := k(x,y) + \frac{1}{3!}(k \cc k k)(x,y) + \dots = \sum_{n=0}^\infty \frac{(k \cc k)^n k}{(2n+1)!}(x,y) =: k(x,y) + r(x,y)
\ed
\bd
c(x,y) := \delta(x-y) + \frac{1}{2!}(k \cc k)(x,y) + \dots = \sum_{n=0}^\infty \frac{(k \cc k)^n}{(2n)!}(x,y) =: \delta(x-y) + p(x,y)
\ed
the power series of $\sinh$ and $\cosh$, where the product is convolution of integral kernels, i.\,e. $(k \cc k)(x,y) = \int \di z\, k(x,z) \cc{k(z,y)}$.
\bd
a^\ast(q_y) := \int \di z\, q(z,y) a^*_z = a^*(q(\cdot,y)) , \quad a(q_y) := \int \di z\, \cc{q(z,y)} a_z = a(q(\cdot,y))
\ed


\begin{lem}[Bogoliubov transformation \cite{GMM2010}]
\label{lem:bogoliubovtransformation}
 \bd
 T_t^* a^*_x T_t = \int \left( \cc{s(y,x)} a_y + c(y,x) a^*_y \right) \di y = a^*(c_x) + a(s_x).
 \ed
\bd
 T_t^* a_x T_t = a(c_x) + a^*(s_x).
\ed
\end{lem}


\begin{lem}[Formula in Fock space \cite{RS2009}]
 \bd
\left(\partial_t e^{C(t)} \right) e^{-C(t)} = \dot C(t) + \frac{1}{2!}[C(t),\dot C(t)]+ \frac{1}{3!}[C(t),[C(t),\dot C(t)]] + \dots
 \ed
%\bd
%e^{C(t)} H e^{-C(t)} = H + [C(t),H] + \frac{1}{2!}[C(t),[C(t),H]] + \dots
%\ed
\bd
 W^*(f) a_x W(f) = a_x + f(x), \quad W^*(f)a^*_x W(f) = a^*_x + \cc{f(x)}
\ed
\bd
[a_x,a^*_y] = \delta(x-y), \quad [a(f),a^*(g)] = \scal{f}{g}
\ed
\end{lem}


\section{The generator}
The generator $\Lcal_N(t)$ is defined by
\bd
\Lcal_N(t) U_N(t) = i \partial_t U_N(t).
\ed
We calculate that
\begin{align*}
\Lcal_N(t) 	& = (i \partial_t T^*_t) T_t + T^*_t \left( (i \partial_t W^*_t) W_t + W^*_t \Hcal_N W_t \right) T_t \\
		& =: (i \partial_t T^*_t) T_t + T^*_t \Lcal^{(0)}_N(t) T_t
\end{align*}
where
\begin{align*}
& \Lcal^{(0)}_N(t) = \Kcal + \Vcal_N \\
		& + N^{1/2} \left(  a^*\left( (w_N N V_N \ast \lvert \ph \rvert^2)\ph \right) + \hc  \right) \\
		& + N^0	    \left(  \frac{1}{2}\int \di x \di y\, NV_N(x-y)\left( \cc{\ph(x)} \cc{\ph(y)} a_y a_x + \hc \right) \right) \\
		& + N^0	    \left(  \int \di x \di y\, NV_N(x-y)\left( \lvert \ph(x) \rvert^2 a^*_y a_y + \cc{\ph(x)} \ph(y) a^*_y a_x \right) \right) \\
		& + N^{-1/2}\left(  \int \di x \di y\, NV_N(x-y) \left( \cc{\ph(x)} a^*_y a_y a_x + \hc \right)  \right) \\
		& + N b(N,t),
\end{align*}
with a phase (which, like all phases, will be dropped from now on without any further comment)
\bd
b(N,t) = \norm{\nabla \ph}_{L^2}^2 - \Im \scal{\ph}{\phdot} + \frac{1}{2}\int \di x \di y\, NV_N(x-y) \lvert \ph(x)\rvert^2 \lvert \ph(y) \rvert^2.
\ed
Here we have made use of a cancellation in the term $\Ocal({N^{1/2}})$ (term which is linear in creation/annihilation operators) due to $\ph$ satisfying the modified Hartree equation
\bd
i\partial_t \ph = -\Delta \ph + \left(f_N N V_N \ast \lvert \ph \rvert^2 \right) \ph.
\ed
Compare \cite{RS2009}, but notice that due to the factor $f_N$ in the Hartree equation, in our case the cancellation is incomplete. This is essential for ensuring the correct coupling constant $8\pi a_0$ and ensures further cancelation with the cubic terms. This cancelation is revealed through the Bogoliubov transformation, to be calculated now.

We now identify cancellations between linear terms stemming from the transformation of linear terms and linear terms stemming from normal-ordering transformed cubic operators:
\begin{align*}
& T^*_t \Lcal^{(0)}_N(t) T_t = \int \di x\, T^*_t a^*_x T_t \bigg[   \frac{1}{2}(-\Delta_x T^*_t a_x T_t) \\
& + \frac{1}{4}\int \di y V_N(x-y) T^*_t a^*_y a_y a_x T_t \\
& + N^{1/2} \ph(x) \int \di y\, w_N N V_N(x-y) \lvert \ph(y) \rvert^2 \tagg{linearterm} \\
& + \frac{1}{2} \int \di y\, NV_N(x-y)  \ph(x) \ph(y)  T^*_t a^*_y T_t \\
& + \frac{1}{2} \int \di y\, NV_N(x-y) \left(  \lvert \ph(y)\rvert^2 T^*_t a_x T_t + \ph(x) \cc{\ph(y)} T^*_t a_y T_t  \right) \\
& + N^{-1/2} \int \di y\, NV_N(x-y) \cc{\ph(y)} T^*_t a_x a_y T_t  \bigg]\tagg{cubicterm} \\
& + \hc
\end{align*}
Between the term \eqr{linearterm} and \eqr{cubicterm}, after normal ordering and plugging in $c = \delta + p$ and $s = k + r$, we find that the linear term is cancelled up to two remainder terms (the $N^{1/2}$-term is completely cancelled, i.\,e.\ line \eqr{linearterm}). Result:
\begin{align*}
& T^*_t \Lcal^{(0)}_N(t) T_t = \int \di x\, T^*_t a^*_x T_t \bigg[   \frac{1}{2}(-\Delta_x T^*_t a_x T_t) \\
& + \frac{1}{4}\int \di y V_N(x-y) T^*_t a^*_y a_y a_x T_t \\
& + \frac{1}{2} \int \di y\, NV_N(x-y)  \ph(x) \ph(y)  T^*_t a^*_y T_t \\
& + \frac{1}{2} \int \di y\, NV_N(x-y) \left(  \lvert \ph(y)\rvert^2 T^*_t a_x T_t + \ph(x) \cc{\ph(y)} T^*_t a_y T_t  \right) \\
& + N^{-1/2} \int \di y\, NV_N(x-y) \cc{\ph(y)} \bigg( r(x,y) + \scal{p_x}{s_y} + \\
& \qquad \qquad  a^*(s_x) a^*(s_y) + a^*(s_x) a(c_y)  + a^*(s_y) a(c_x) + a(c_x) a(c_y)  \bigg)  \bigg] \\
& + \hc
\end{align*}
Now we apply Lemma \ref{lem:bogoliubovtransformation} everywhere and expand all the terms, normal order all terms, and observe that many terms appear twice (maybe in the hermitian conjugate or with $x$ and $y$ interchanged):
\begin{align*}
& T^*_t \Lcal_N^{(0)}(t) T_t = \\ 
& \frac{1}{2} \int \di x\, \left[ a^*(c_x) a(-\Delta_x c_x) + \boxed{2 a^*(c_x) a^*(-\Delta_x s_x)} + a^*(-\Delta_x s_x) a(s_x) \right] \tagg{cancellation_kinetic} \\
& + \frac{1}{2}\int \di x \di y\, NV_N(x-y) \times \\
& \times \Big[   \frac{1}{2N}\bigg( a^*(c_x) a^*(c_y) a(c_y) a(c_x) + 4 a^*(c_x) a^*(c_y) a^*(s_x) a(c_y) \\
				      & \qquad\qquad + 2 a^*(c_x) a^*(c_y) a^*(s_y) a^*(s_x) + 2 a^*(c_x) a^*(s_x) a(s_y) a(c_y) \\
				      & \qquad\qquad + 2 a^*(c_x) a^*(s_y) a(s_y) a(c_x) + 4 a^*(c_x) a^*(s_y) a^*(s_x) a(s_y) \\
				      & \qquad\qquad + a^*(s_y) a^*(s_x) a(s_x) a(s_y) \bigg) \\
& + \frac{1}{N}\bigg(   \boxed{a^*(c_x)a^*(c_y) \scal{c_y}{s_x}} + a^*(c_x) a(c_y) \scal{s_y}{s_x} \tagg{cancellation_normalorder} \\
			& \qquad\qquad + a^*(c_x) a(s_y) \scal{c_y}{s_x} + a^*(c_x) a(c_x) \scal{s_y}{s_y} \\
			& \qquad\qquad + 2 a^*(c_x) a^*(s_x) \scal{s_y}{s_y} + 2a^*(c_x)a^*(s_y) \scal{s_y}{s_x} \\
			& \qquad\qquad + a^*(c_y) a(s_x) \scal{c_y}{s_x} +  a^*(s_y) a(s_y) \scal{s_x}{s_x}\\
			& \qquad\qquad + a^*(s_y) a^*(s_x) \scal{s_x}{c_y} + a^*(s_y) a(s_x) \scal{s_y}{s_x}   \bigg) \\
& + \ph(x)\ph(y) \Big( \boxed{a^*(c_x) a^*(c_y)} + 2 a^*(c_x) a(s_y) +a(s_x) a(s_y) \Big) \tagg{cancellation_standard} \\
& + \ph(x) \cc{\ph(y)} \Big( a^*(c_x) a(c_y) + 2 a^*(c_x) a^*(s_y) + a^*(s_y) a(s_x) \Big) \\
& + \lvert \ph(y) \rvert^2 \Big( a^*(c_x) a(c_x) + 2 a^*(c_x) a^*(s_x) + a^*(s_x) a(s_x) \Big) \\
& + \frac{2}{\sqrt{N}}\cc{\ph(y)} \bigg(    a^*(c_x) a^*(s_x) a^*(s_y) + a^*(c_x) a^*(s_x) a(c_y) + a^*(s_x) a^*(s_y) a(s_x)\\
					    & \qquad\qquad + a^*(c_x) a^*(s_y) a(c_x) + a^*(c_x) a(c_x) a(c_y)+ a^*(s_x) a(s_x) a(c_y) \\
					    & \qquad\qquad + a^*(s_y) a(s_x) a(c_x) + a(s_x) a(c_x) a(c_y)    \bigg) \\
& + \frac{2}{\sqrt{N}}\cc{\ph(y)} \bigg(    a^*(s_x) \scal{s_x}{s_y} + a^*(s_y) \scal{s_x}{s_x}  + a(c_y) \scal{s_x}{s_x} + a(c_x) \scal{s_x}{s_y} \\
					    & \qquad\qquad + a^*(c_x)r(x,y) + a^*(c_x)\scal{p_x}{s_y} + a(s_x)r(x,y) + a(s_x)\scal{p_x}{s_y}		\bigg)    \Big] + \hc
\end{align*}
We now proceed to identify a cancellation between the second summand in line \eqr{cancellation_kinetic}, the first summand in line \eqr{cancellation_normalorder} and the first summand in line \eqr{cancellation_standard}. To see the cancellation, we expand $c = \delta + p$ and $s = k + r$ and use the product rule for the Laplacian $-\Delta_x k(y,x)$, then notice that the lhs of the zero-energy scattering equation \eqr{scatteringequation} appears. There are however 6 remainder terms left (see final result for the generator below).

Then the final result for the generator is:
\begin{align}
& T^*_t \Lcal_N^{(0)}(t) T_t = \nonumber \\ 
& \frac{1}{2} \int \di x\, \bigg[ a^*(c_x) \int \di y\, a^*_y \Big( N \nabla w_N(x-y) \nabla_x \ph(x) 2 \ph(y) \\
& \qquad\qquad \qquad\qquad						+ Nw_N(x-y) \Delta_x \ph(x) \ph(y) - \Delta_x r(y,x) \Big) \\
& \qquad\qquad 			+ a^*(c_x) a(-\Delta_x c_x) + a^*(-\Delta_x s_x) a(s_x) \bigg] \\
& + \frac{1}{2}\int \di x \di y\, NV_N(x-y) \times \nonumber \\
& \times \Big[   \frac{1}{2N}\bigg( a^*(c_x) a^*(c_y) a(c_y) a(c_x) + 4 a^*(c_x) a^*(c_y) a^*(s_x) a(c_y) \\
				      & \qquad\qquad + 2 a^*(c_x) a^*(c_y) a^*(s_y) a^*(s_x) + 2 a^*(c_x) a^*(s_x) a(s_y) a(c_y) \\
				      & \qquad\qquad + 2 a^*(c_x) a^*(s_y) a(s_y) a(c_x) + 4 a^*(c_x) a^*(s_y) a^*(s_x) a(s_y) \\
				      & \qquad\qquad + a^*(s_y) a^*(s_x) a(s_x) a(s_y) \bigg) \\
& + \frac{1}{N}\bigg(   a^*(c_x) a^*_y \Big( r(y,x) + \scal{p_y}{s_x} \Big) + a^*(c_x) a^*(p_y) \Big( s(y,x) + \scal{p_y}{s_x} \Big) \\
      & \qquad\qquad + a^*(c_x) a(c_y) \scal{s_y}{s_x} + a^*(s_y) a(s_y) \scal{s_x}{s_x} + a^*(s_y) a(s_x) \scal{s_y}{s_x}\\
      & \qquad\qquad + a^*(c_x) a(s_y) \scal{c_y}{s_x} + a^*(c_x) a(c_x) \scal{s_y}{s_y} + a^*(s_y) a^*(s_x) \scal{s_x}{c_y}\\
      & \qquad\qquad + 2a^*(c_x) a^*(s_x) \scal{s_y}{s_y} + 2a^*(c_x)a^*(s_y) \scal{s_y}{s_x} + a^*(c_y) a(s_x) \scal{c_y}{s_x}    \bigg) \\
& + \ph(x)\ph(y) \Big( a^*(c_x) a^*(p_y) + 2 a^*(c_x) a(s_y) +a(s_x) a(s_y) \Big) \\
& + \ph(x) \cc{\ph(y)} \Big( a^*(c_x) a(c_y) + 2 a^*(c_x) a^*(s_y) + a^*(s_y) a(s_x) \Big) \\
& + \lvert \ph(y) \rvert^2 \Big( a^*(c_x) a(c_x) + 2 a^*(c_x) a^*(s_x) + a^*(s_x) a(s_x) \Big) \\
& + \frac{2}{\sqrt{N}}\cc{\ph(y)} \bigg(    a^*(c_x) a^*(s_x) a^*(s_y) + a^*(c_x) a^*(s_x) a(c_y) \\
					    & \qquad\qquad + a^*(c_x) a^*(s_y) a(c_x) + a^*(c_x) a(c_x) a(c_y) + a^*(s_y) a(s_x) a(c_x)\\
					    & \qquad\qquad + a^*(s_x) a^*(s_y) a(s_x) + a^*(s_x) a(s_x) a(c_y) + a(s_x) a(c_x) a(c_y)  \bigg) \\
& + \frac{2}{\sqrt{N}}\cc{\ph(y)} \bigg(    a^*(s_x) \scal{s_x}{s_y} + a^*(s_y) \scal{s_x}{s_x}  + a(c_y) \scal{s_x}{s_x} + 							a(c_x) \scal{s_x}{s_y} \\
					    & \qquad\qquad + a^*(c_x)r(x,y) + a^*(c_x)\scal{p_x}{s_y} + a(s_x)r(x,y) + 			a(s_x)\scal{p_x}{s_y}		\bigg)    \Big] + \hc
\end{align}
The cancellations enable us to give estimates for all terms, only using $\Ncal$, $\Ncal^2$ and $\Ncal \Kcal$, never $\Kcal^2$, which we expect to be too large due to the singular short scale structure of the two-particle correlations.

\section{Estimates for the terms of the generator}
ToDo: Type all the estimates here.

\section{A-priori estimates}
\begin{lem}
 \bd
  \norm{k} \leq C, \quad \norm{\nabla k} \leq \sqrt{N}
 \ed
 \bd
  \lvert p(x,y) \rvert \leq C \lvert \ph(x) \rvert\cdot \lvert \ph(y) \rvert, \quad \norm{p} \leq C, \quad \norm{\nabla p} \leq C
 \ed
\bd
\mbox{maybe } \sup_x \norm{s_x}_2^2 \leq C \mbox{ (can be used ?)}
\ed
\end{lem}

\begin{lem}
\bd
\Ncal \leq C T^*_t (\Ncal + 1) T_t, \quad \Ncal^2 \leq C T^*_t (\Ncal + 1)^2 T_t, \quad T^*_t (\Ncal + 1) T_t \leq \Ncal
\ed
\end{lem}


\begin{lem}
\label{lem:nsquaredbound}
For $\ev{\cdot} = \scal{U_N(t)\Omega}{\cdot U_N(t)\Omega}$ or maybe easier $\ev{\cdot} = \scal{U_N(t)T^*\Omega}{\cdot U_N(t)T^*\Omega}$, we have
 \bd
 \frac{1}{N}\ev{\Ncal^2} \leq C \ev{\Ncal}, \quad \frac{1}{N}\ev{\Ncal \Kcal} \leq \ev{\Kcal}
 \ed
\end{lem}

\begin{lem}
 \bd
  \ev{(\partial_t T^*_t)T_t} \leq C \ev{\Ncal}
 \ed
\end{lem}
\begin{proof}
remark: contains only quadratic terms $a^* a$ and $a a$. coefficients are complicated.
\end{proof}



\section{Estimating the number of fluctuations}
\begin{lem}
\bd
-c\ev{\Ncal} \leq \ev{\Lcal}
\ed 
\end{lem}

\begin{lem}
\label{lem:kvbounds}
 \bd
\ev{\Kcal} \leq  C \ev{\Lcal + \Ncal}
\ed
and for terms in $T^*_t \Vcal T_t$ of the form $A^* A$, where $A$ is quadratic in annihilation and creation operators, we also need bounds by $\Lcal$.
\end{lem}

\begin{lem}
\label{lem:ldotbounds}
 \bd
  \frac{\di}{\di t} \ev{\Lcal} = \ev{\dot \Lcal} \leq C \ev{\Lcal + \Ncal}
 \ed
\end{lem}

\begin{lem}
\label{lem:lncommutatorbound}
 \bd
  \frac{\di}{\di t}\ev{\Ncal} = \ev{[\Lcal,\Ncal]} \leq C \left( \ev{N} + \frac{1}{N}\ev{\Ncal^2} + \frac{1}{N} \ev{\Ncal \Kcal} \right)
 \ed
(rhs can have also other terms for which lemma \ref{lem:kvbounds} holds)
\end{lem}
\begin{proof}
We have
\begin{align*}
& [T^*_t \Lcal_N^{(0)}(t) T_t,\Ncal] = \\ 
& \int \di x\, \bigg[ a(c_x) \int \di y\, a_y \Big( N \nabla w_N(x-y) \cc{\nabla_x \ph(x)} 2 \cc{\ph(y)} \\
& \qquad\qquad \qquad\qquad						+ Nw_N(x-y) \cc{\Delta_x \ph(x) \ph(y)} - \cc{\Delta_x r(y,x)} \Big) \bigg]\\
& + \frac{1}{2}\int \di x \di y\, NV_N(x-y) \times \\
& \times \Big[   \frac{4}{N}\bigg( a^*(c_y)a(s_x)a(c_y)a(c_x) + a(s_x)a(s_y)a(c_y)a(c_x) + a^*(s_y)a(s_x)a(s_y)a(c_x) \bigg) \\
& + \frac{2}{N}\bigg(  a(c_x) a_y \Big( \cc{r(y,x)} + \scal{s_x}{p_y} \Big) + a(c_x) a(p_y) \Big( \cc{s(y,x)} + \scal{s_x}{p_y} \Big) \\
			& \qquad\qquad + 2 a(c_x) a(s_x) \scal{s_y}{s_y} + 2 a(c_x)a(s_y) \scal{s_x}{s_y} + a(s_y) a(s_x) \scal{c_y}{s_x} \bigg)\\
& + \ph(x)\ph(y) 2\Big(  a(c_x) a(p_y) + a(s_x) a(s_y) \Big) \\
& + \ph(x) \cc{\ph(y)} 4 a(c_x) a(s_y) \qquad \qquad + \lvert \ph(y) \rvert^2 4 a(c_x) a(s_x) \\
& + \frac{2}{\sqrt{N}}\cc{\ph(y)} \bigg(   3 a(s_y)a(s_x)a(c_x) + a^*(c_y)a(s_x)a(c_x) \\
					    & \qquad\qquad + a^*(c_x)a(s_y)a(c_x) + a^*(c_x) a(c_x) a(c_y) + a^*(s_y) a(s_x) a(c_x)\\
					    & \qquad\qquad + a^*(s_x)a(s_y)a(s_x) + a^*(s_x) a(s_x) a(c_y) + 3 a(s_x) a(c_x) a(c_y) \bigg) \\
& + \frac{2}{\sqrt{N}}\cc{\ph(y)} \bigg(    a(s_x) \scal{s_y}{s_x} + a(s_y) \scal{s_x}{s_x}  + a(c_y) \scal{s_x}{s_x} + a(c_x) \scal{s_x}{s_y} \\
					    & \qquad\qquad + a(c_x)\cc{r(x,y)} + a(c_x)\scal{s_y}{p_x} + a(s_x)r(x,y) + a(s_x)\scal{p_x}{s_y}  \bigg)    \Big] - \hc
\end{align*}
\end{proof}


\begin{prp}
 $N$-independent bound on $\ev{\Ncal}$ by employing all the above and Gronwall.
\end{prp}
\begin{proof}
 By lemma \ref{lem:lncommutatorbound}, lemma \ref{lem:nsquaredbound} and lemma \ref{lem:kvbounds}, $\ev{[\Lcal,\Ncal]}$ is bounded by $\ev{\Lcal+\Ncal}$. So by lemma \ref{lem:ldotbounds}, the sum $\Lcal+C \Ncal$ is bounded above, and for $C$ large enough, it is positive, so we get a bound for $\Ncal$.
\end{proof}


\section{Main result}
\begin{thm}
 Convergence of reduced density matrix for Bogoliubov state.
\end{thm}

\begin{thm}
 Convergence of reduced density matrix for factorized initial state.
\end{thm}


\section{Uniform regularity for the Hartree equation}
\begin{lem}[Uniform $H^2$-regularity]
 XYZ
\end{lem}
\begin{lem}[Convergence of $\ph$ to $\varphi_t$ in $L^2$]
 XZY
\end{lem}

\begin{lem}
 Convergence of $\Tr \lvert \project{\ph} - \project{\varphi_t} \rvert$.
\end{lem}


\bibliography{gross-pitaevskii}
\bibliographystyle{plain}

\end{document}


