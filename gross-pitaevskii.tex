%\documentclass[11pt,a4paper,twoside,headsepline]{scrartcl}
\documentclass[11pt,a4paper]{scrartcl}
\usepackage[a4paper,hmargin=2.8cm,vmargin=2.8cm]{geometry}
\usepackage[utf8x]{inputenc}
%\usepackage[bookmarksnumbered=true]{hyperref}
\usepackage[color,notref,notcite]{showkeys}
\usepackage{scrtime}
\usepackage{amsmath,amsthm,amsfonts,amssymb}
\usepackage{dsfont,wasysym}

% Original header
%\usepackage{scrpage2}
%\ihead{Gross-Pitaevskii Equation}
%\chead{Page: \thepage}
%\ohead{Version: \today, \thistime}
%\pagestyle{scrheadings}

% How about this footer?
\usepackage{fancyhdr}
\pagestyle{fancy}
\fancyhf{}
\renewcommand{\headrulewidth}{0pt}
\lfoot{{\footnotesize GP equation}}
\cfoot{\thepage}
\rfoot{{\footnotesize \today, \thistime}}

% Set enumerate environment using roman counters
\renewcommand{\theenumi}{{\rm (\roman{enumi})}}
\renewcommand{\labelenumi}{\theenumi}

% Nicer marginpar with smaller font
\let\oldmarginpar\marginpar
\renewcommand\marginpar[1]{\-\oldmarginpar[\raggedleft\footnotesize #1]%
  {\raggedright\footnotesize #1}}

%%%%%%%%%%%%%%%%%%%%%%%%% Theorems etc %%%%%%%%%%%%%%%%%%%%%%%%%%%%%%%%%%%%%%%%%

\newtheorem{thm}{Theorem}[section]
\newtheorem{cor}[thm]{Corollary}
\newtheorem{prp}[thm]{Proposition}
\newtheorem{lem}[thm]{Lemma}
\newtheorem{dfn}[thm]{Definition}
\newtheorem{exm}[thm]{Example}

\newtheorem*{rem}{Remark}
\newtheorem*{hyp}{Hypothesis}

%%%%%%%%%%%%%%%%%%%%%%%% Gustavo's aliases %%%%%%%%%%%%%%%%%%%%%%%%%%%%%%%%%%%%%

\newcommand{\R}{\mathds{R}}
\newcommand{\N}{\mathcal{N}}
\newcommand{\K}{\mathcal{K}}

%%%%%%%%%%%%%%%%%%%%%%%% Notation %%%%%%%%%%%%%%%%%%%%%%%%%%%%%%%%%%%%%%%%%%%%%%

\newcommand{\ad}{\operatorname{ad}}	% abbreviation for iterated commutators
\newcommand{\fock}{\mathcal{F}}		% fock space symbol
\newcommand{\di}{\textrm{d}}		% differential (for integrals)
\newcommand{\Lcal}{\mathcal{L}}		% calligraphic L
\newcommand{\Ncal}{\mathcal{N}}		% calligraphic N
\newcommand{\Kcal}{\mathcal{K}}		% calligraphic N
\newcommand{\Vcal}{\mathcal{V}}		% calligraphic V
\newcommand{\Hcal}{\mathcal{H}}		% calligraphic H
\newcommand{\Ocal}{\mathcal{O}}		% big-O, order-of
\newcommand{\tilV}{\tilde{\mathcal{V}}_N}		% symbol for the smeared potential energy part of the hamiltonian
\newcommand{\tilK}{\tilde{\mathcal{K}}}		% smeared kinetic energy
\newcommand{\estlist}[2]{\underline{Line \ref{l#1}, summand #2:}}
\newcommand{\hc}{\mbox{h.c.}}		%hermitian conjugate
\newcommand{\scal}[2]{\big<#1,#2\big>} % scalaer product
\newcommand{\cc}[1]{\overline{#1}}	% complex conjugate
\newcommand{\Rbb}{\mathbb{R}}		% real numbers
\newcommand{\Cbb}{\mathbb{C}}		% complex numbers
\newcommand{\Nbb}{\mathbb{N}}		% natural numbers
\renewcommand{\Re}{\operatorname{Re}\,} 	%RealPart
\renewcommand{\Im}{\operatorname{Im}\,} 	%ImaginaryPart
\newcommand{\norm}[1]{\lVert#1\rVert}	%Norm
\newcommand{\ev}[1]{\big<#1\big>}	%expectation value
\newcommand{\ph}{\varphi_t^{(N)}}	% solution of N-dependent Hartree equation
\newcommand{\phdot}{\dot{\varphi}_t^{(N)}}	% time derivative of solution of N-dependent Hartree equation
\newcommand{\phddot}{\ddot{\varphi}_t^{(N)}}	% second time derivative of solution of N-de Hartree equaution
\newcommand{\sqn}{\sqrt{N}}		% square root of N
\newcommand{\project}[1]{\lvert #1 \big>\big< #1\rvert}	% orthogonal projection operator
\newcommand{\Tr}{\operatorname{Tr}}	% Trace
\newcommand{\dxyNV}{\frac{1}{2}\int \di x\di y N V_N(x-y)} % abbreviation for the one-half integral dx dy NV_N(x-y) which appears everywhere
\newcommand{\dxyV}{\frac{1}{2}\int \di x\di y V_N(x-y)} % abbreviation for the one-half integral dx dy V_N(x-y) which appears everywhere

\newcommand{\be}[1]{\begin{equation}\label{eq:#1}}	%begin equation with label
\newcommand{\ee}{\end{equation}}
\newcommand{\bd}{\begin{displaymath}}			% abbreviation begin displaymath
\newcommand{\ed}{\end{displaymath}}

\newcommand{\tagg}[1]{ \stepcounter{equation} \tag{\theequation} \label{eq:#1} } % add tag and label in align*-environments

\newcommand{\eqr}[1]{\eqref{eq:#1}}			%eqref with prefix :eq

%%%%%%%%%%%%%%%%%%%%%%%%% main content %%%%%%%%%%%%%%%%%%%%%%%%%%%%%%%


%\allowdisplaybreaks


\begin{document}


\tableofcontents


\newpage


\section{Fock space representation}


The bosonic Fock space over $L^2(\R^3)$ is the Hilbert space
\[
  \mathcal{F} = \bigoplus_{n \ge 0} L^2(\R^3)^{\otimes_s n} = \mathds{C}
  \oplus \bigoplus_{n \ge 1} L^2_s(\R^{3n}),
\]
with the convention that $L^2(\R^3)^{\otimes_s 0} = \mathds{C}$. Here
$L^2_s(\R^{3n})$ is the subspace of $L^2(\R^{3n})$ consisting of all functions
that are symmetric with respect to arbitrary permutations of the $n$
variables. Vectors in $\mathcal{F}$ are sequences $\psi = \{ \psi^{(n)} \}_{n
\ge 0}$ of $n$-particle wave functions $\psi^{(n)} \in L^2_s(\R^{3n})$. The
inner product on $\mathcal{F}$ is defined as
\begin{align*}
  \langle \psi_1, \psi_2 \rangle & = \sum_{n \ge 0} \langle \psi_1^{(n)},
  \psi_2^{(n)} \rangle_{L^2(\R^{3n})} \\
  & = \overline{\psi_1^{(0)}} \psi_2^{(0)} + \sum_{n \ge 1} \int dx_1 \cdots
  dx_n \overline{\psi_1^{(n)}}(x_1, \dots, x_n) \psi_2^{(n)}(x_1, \dots, x_n).
\end{align*}
An $N$-particle state with wave function $\psi_N$ is described on
$\mathcal{F}$ by a sequence $\{ \psi^{(n)} \}_{n \ge 0}$ where $\psi^{(n)} =
0$ for all $n \neq N$ and $\psi^{(N)} = \psi_N$. The vector $\{ 1, 0, 0, \dots
\} \in \mathcal{F}$ is called the vacuum, and is denoted by $\Omega$.


For $f \in L^2(\R^3)$, the creation operator $a^*(f)$ and the annihilation
operator $a(f)$ on $\mathcal{F}$ are defined as
\[
  \begin{split}
    (a^*(f) \psi)^{(n)}(x_1, \dots, x_n) & = \frac{1}{\sqrt{n}} \sum_{j=1}^n
    f(x_j) \psi^{(n-1)}(x_1, \dots, x_{j-1}, x_{j+1}, \dots, x_n), \\
    (a(f) \psi)^{(n)}(x_1, \dots, x_n) & = \sqrt{n+1} \int dx \,
    \overline{f(x)} \psi^{(n+1)}(x, x_1, \dots, x_n).
  \end{split}
\]
The operators $a^*(f)$ and $a(f)$ are unbounded, densely defined, and closed.
Note that $a^*(f)$ is linear in $f$, while $a(f)$ is anti-linear. The creation
operator $a^*(f)$ is the adjoint of the annihilation operator $a(f)$, and they
satisfy the canonical commutation relations
\[
  [a(f), a^*(g)] = \langle f, g \rangle_{L^2} \qquad \text{and} \qquad [a(f),
  a(g)] = [a^*(f), a^*(g)] = 0
\]
for $f,g \in L^2(\R^3)$. It is useful to introduce the self-adjoint operator
\[
  \phi(f) = a^*(f) + a(f).
\]


We will make use of operator-valued distributions $a_x^*$ and $a_x$, with $x
\in \R^3$, defined so that
\[
  a^*(f) = \int \di x\, f(x) a^*_x \qquad \text{and} \qquad a(f) = \int \di
  x\, \cc{f(x)} a_x
\]
for $f \in L^2(\R^3)$. For these distributions, the canonical commutation
relations assume the form
\[
  [a_x, a_y^*] = \delta(x-y) \qquad \text{and} \qquad [a_x, a_y] = [a_x^*,
  a_y^*] = 0,
\]
where $\delta$ is the Dirac delta distribution.


The number of particles operator $\mathcal{N}$ on $\mathcal{F}$ is defined as
$(\N \psi)^{(n)} = n \psi^{(n)}$. Eigenvectors of $\N$ are vectors of the form
$\{0, \dots, 0, \psi^{(m)}, 0, \dots \}$ with a fixed number of particles.


The Hamiltonian $\Hcal_N$ on $\mathcal{F}$ is defined as $(\Hcal_N
\psi)^{(n)} = \Hcal_N^{(n)} \psi^{(n)}$ with
\[
  \Hcal_N^{(n)} = \sum_{j=1}^n \Delta_{x_j} + \sum_{i<j}^n V_N(x_i - x_j).
\]
By definition, the Hamiltonian $\Hcal_N$ leaves sectors of $\mathcal{F}$ with
a fixed number of particles invariant. This is just a restatement of the fact
that $\Hcal_N$ and $\N$ commute.


Using the distributions $a_x^*$ and $a_x$, we can express the operators $\N$
and $\Hcal_N$ as
\[
  \N = \int dx \, a_x^* a_x
\]
and
\[
  \Hcal_N = \int dx \, \nabla_x a_x^* \nabla_x a_x + \frac{1}{2} \int dx dy \,
  V_N(x-y) a_x^* a_y^* a_y a_x.
\]
The operator
\[
  \K = \int dx \, \nabla_x a_x^* \nabla_x a_x
\]
is called the kinetic energy operator.


The following lemma provides some useful bounds to control creation and
annihilation operators in terms of the number of particles operator $\N$ and
the kinetic energy operator $\K$.


\begin{lem} \label{l:a}
  Let $f \in L^2(\R^3)$ and $g \in H^1(\R^3)$. Then, for any $\psi \in
  \mathcal{F}$,
  \begin{equation} \label{aNorm}
    \begin{aligned}
      \norm{a(f)\psi} & \leq \norm{f}_{L^2} \norm{\Ncal^{1/2}\psi}, \\
      \norm{a^*(f)\psi} & \leq \norm{f}_{L^2} \norm{(\Ncal+1)^{1/2}\psi}, \\
      \norm{\phi(f) \psi} & \leq 2 \norm{f}_{L^2} \norm{(\N+1)^{1/2} \psi}
    \end{aligned}
  \end{equation}
and
\be{agradnorm}
\norm{a(\nabla g)\psi}  \leq \norm{g}_{L^2} \norm{\Kcal^{1/2}\psi}.
\ee
\end{lem}


Hamiltonian etc.
\bd
\Kcal := \int \di x\, \nabla_x a^*_x \nabla_x a_x,
\quad
\Vcal_N := \frac{1}{2}\int\di x \di y\, V_N(x-y) a^*_x a^*_y a_y a_x
\ed
\bd
V_N(x) := N^2V(Nx), \quad \Hcal := \Kcal + \Vcal_N, \quad \Ncal := \int \di
x\, a^*_x a_x
\ed


\section{Weyl operator and coherent states}


For $f \in L^2(\R^3)$, we define the Weyl operator as
\[
  W(f) = e^{A(f)}
\]
with
\[
  A(f) = a^*(f) - a(f).
\]


\begin{lem}[Weyl operator and coherent states] \label{l:W}
  Let $f, g \in L^2(\R^3)$. The following hold.
  \begin{enumerate}
    \item \label{lW1} The Weyl operator satisfies the relations
      \[
        W(f) W(g) = W(g) W(f) e^{-2i \Im \langle f, g \rangle_{L^2}} = W(f+g)
        e^{-i \Im \langle f, g \rangle_{L^2}}.
      \]
    \item \label{lW2} The operator $W(f)$ is unitary on $\mathcal{F}$ and
      \[
        W(f)^* = W(f)^{-1} = W(-f).
      \]
    \item \label{lW3} We have
      \[
        W(f)^* a_x W(f) = a_x + f(x) \qquad \text{and} \qquad W(f)^* a_x^* W(f)
        = a_x^* + \overline{f(x)}.
      \]
    \item \label{lW4} It follows from \ref{lW3} that coherent states are
      eigenvectors of anihilation operators:
      \[
        a_x \psi(f) = f(x) \psi(f) \qquad \text{so that} \qquad a(g) \psi(f)
        = \langle g, f \rangle_{L^2} \psi.
      \]
    \item \label{lW5} The expectation of the number of particles in the
      coherent state $\psi(f)$ is given by $\| f \|_{L^2}^2$, that is,
      \[
        \langle \psi(f), \N^2 \psi(f) \rangle = \| f \|_{L^2}^2.
      \]
      Also the variance of the number of particles in $\psi(f)$ is given by
      $\| f \|_{L^2}$ (the distribution of $\N$ is poisson), that is,
      \[
        \langle \psi(f), \N^2 \psi(f) \rangle - \langle \psi(f), \N \psi(f)
        \rangle^2 = \| f \|_{L^2}^2.
      \]
    \item \label{lW6} Coherent states are normalized but not orthogonal to
      each other. In fact,
      \[
        \langle \psi(f), \psi(g) \rangle = e^{-\frac{1}{2} (\| f \|_{L^2}^2 +
        \| g \|_{L^2}^2 - 2 \langle f, g \rangle_{L^2} )} \qquad \text{so
        that} \qquad |\langle \psi(f), \psi(g) \rangle| = e^{-\frac{1}{2} \| f
        - g \|_{L^2}^2}.
      \]
    \end{enumerate}
\end{lem}


\section{Zero-energy scattering equation}


Consider $V \in C_c^\infty(\R^3)$, and let $f$ be the solution of the
zero-energy scattering equation
\[
  \left( -\Delta + \frac{1}{2} V \right) f = 0
\]
with normalization $\lim_{|x|\to\infty} f(x) = 1$. We will write
\[
  f = 1 - w
\]
with $\lim_{|x|\to\infty} w(x) = 0$. The scattering length $a_0$ of $V$ is
defined as
\[
  a_0 = \lim_{|x| \to \infty} w(x)|x|.
\]
Since $V$ has compact support, we have
\[
  f(x) = 1 - \frac{a_0}{|x|} \qquad \text{for } |x| \ge R,
\]
where $R$ is such that $\text{supp }V \subset \{ x \in \R^3 \; | \;\; |x| \le
R \}$. From the zero-energy scattering equation, we also have the identity
\[
  \int dx \, V(x) f(x) = 8 \pi a_0.
\]
By scaling, the scattering length of the potential $V_N(x) = N^2 V(Nx)$ is
$a_N = a_0/N$, and the zero-energy scattering equation for the potential $V_N$
is
\begin{equation} \label{eq:scatteringequation}
  \left( -\Delta + \frac{1}{2} V_N \right) f_N = 0,
\end{equation}
where $f_N(x) = 1 - w_N(x)$ with $w_N(x) = w(Nx)$. Note that $w_N(x) =
a_N/|x|$ for $|x| \ge R/N$.


\begin{lem}[Zero-energy scattering equation \cite{ESY2010}] \label{l:w}
\marginpar{We might clean up this lemma if we don't use everything}
  Let $V \in C_c^\infty(\R^3)$ with $V \ge 0$, and suppose that $V$ is
  spherical symmetric with scattering length $a_0$. Let
  \[
    \rho = \sup_{r \ge 0} r^2 V(r) + \int_0^\infty dr \, r V(r),
  \]
  and consider the solution $f_N = 1-w_N$ of the zero-energy scattering
  equation with scaled potential $V_N$. Then, the following hold with
  constants uniform in $N$.
  \begin{enumerate}
    \item \label{lw1} There exists a constant $C_0 > 0$, which depends on the
      unscaled potential $V$, such that
      \[
        C_0 \le 1 - w_N(x) \le 1 \qquad \text{for all } x \in \R^3.
      \]
    \item \label{lw2} Let $R$ be such that $\text{supp }V \subset \{ x \in
      \R^3 \; | \;\; |x| \le R \}$. Then,
      \[
        w_N(x) = \frac{a_0}{N|x|} \qquad \text{for all } x \in \R^3 \text{
        with } |x| > R/N.
      \]
    \item \label{lw3} From \ref{lw1} and \ref{lw2} it follows that
      \[
        |w_N(x)| \le \max\{a_0, R\} \frac{1}{N|x|} \qquad \text{for all } x \in
        \R^3.
      \]
    \item \label{lw4} There exist constants $C_1$ and $C_2$, depending only on
      $V$, such that
      \[
        |\nabla w_N(x)| \le C_1 N \qquad \text{and} \qquad | \nabla^2 w_N(x)|
        \le C_2 N^2
      \]
      for all $x \in \R^3$. Moreover, there exists a universal constant
      $C$, such that
      \[
        |\nabla w_N(x)| \le \frac{C a_0}{N |x|^2}, \qquad |\nabla w_N(x)| \le
        \frac{C \rho}{|x|}, \qquad |\nabla^2 w_N(x)| \le \frac{C \rho}{N^2
        |x|^2}
      \]
      for all $x \in \R^3$.
    \item \label{lw5} From \ref{lw4} it follows that
      \[
        |\nabla w_N(x)| \le \max\{ C_1, C a_0 \} \frac{2N}{N^2|x|^2 + 1}
        \qquad \text{for all } x \in \R^3.
      \]
  \end{enumerate}
\end{lem}


\section{Bogoliubov transformation}


Let $f(x,y)$ be a function of two variables. To simplify the notation we write
\[
  f_x(y) = f(x,y)
\]
so that
\[
  a(f_x) = a(f(x, \,\cdot\,)) \qquad \text{and} \qquad a^*(f_x) = a^*(f(x,
  \,\cdot\,)).
\]
For notational convenience, it will be useful to slightly abuse notation and
write
\[
  a_x + a(f_x) = a(\delta_x + f_x).
\]


For $k \in L^2(\R^3 \times \R^3)$, define the operators $B(k)$ and $T(k)$ on
$\mathcal{F}$ as
\[
  B(k) = \frac{1}{2} \int dxdy \, (k(x,y) a_x^* a_y^* - \overline{k(x,y)} a_x
  a_y)
\]
and
\[
  T(k) = e^{B(k)}.
\]


\begin{lem}[Bogoliubov transformation] \label{l:bt}
  Let $k \in L^2(\R^3 \times \R^3)$.
  \begin{enumerate}
    \item \label{l:bt1} The operator $T(k)$ is unitary on $\mathcal{F}$ and
      \[
        T(k)^* = T(k)^{-1} = T(-k).
      \]
    \item \label{l:bt2} We have
      \[
        T(k)^* a_x T(k) = a(c_x) + a^*(s_x) \qquad \text{and} \qquad T(k)^*
        a_x^* T(k) = a^*(c_x) + a(s_x)
      \]
      with
      \begin{alignat*}{2}
        c & = \delta + p, & \qquad p & = \sum_{n=1}^\infty \frac{1}{(2n)!}
        \big( k \overline{k} \big)^n, \\
        s & = k + r, & \qquad r & = \sum_{n=1}^\infty \frac{1}{(2n+1)!} k
        \big( \overline{k} k \big)^n,
      \end{alignat*}
      where $\delta$ is the Dirac delta, and the product of functions in the
      power series is the convolution of integral kernels, that is, $fg(x,y) =
      \int dz \, f(x,z) g(z,y)$. Furthermore,
      \[
        \| p \|_{L^2} \le e^{\| k \|_{L^2}} \qquad \text{and} \qquad \| s
        \|_{L^2} \le e^{\| k \|_{L^2}}.
      \]
    \item \label{l:bt3} It follows from \ref{l:bt2} that, for $f \in
      L^2(\R^3)$,
      \[
        T(k)^* a(f) T(k) = a(Cf) + a^*(S\overline{f}) \qquad \text{and} \qquad
        T(k)^* a^*(f) T(k) = a^*(Cf) + a(S\overline{f}),
      \]
      where $C$ and $S$ are integral operators on $L^2(\R^3)$, with integral
      kernels $c$ and $s$, respectively. Moreover, if we write $C = I + P$,
      where $I$ is the identity operator and $P$ has integral kernel $p$, then
      \[
        \| P \|_{HS} \le e^{\| k \|_{L^2}} \qquad \text{and} \qquad \| S
        \|_{HS} \le e^{\| k \|_{L^2}},
      \]
      where $\| \, \cdot \, \|_{HS}$ is the Hilbert-Schmidt norm for operators
      on $L^2(\R^3)$.
  \end{enumerate}
\end{lem}


\begin{lem}[Estimates for integral kernels] \label{l:kernels}
  Let $\varphi \in H^1(\R^3)$, and let $f_N=1-w_N$ be the solution of the
  zero-energy scattering equation as in Lemma \ref{l:w}. Let
  \[
    k(x,y) = - N w_N(x-y) \varphi(x) \varphi(y),
  \]
  and consider the integral kernels $p$, $r$ and $s$ as in Lemma \ref{l:bt}.
  Then, there exists a constant $C$, uniform in $N$, that depends only on the
  unscaled potential $V$, such that, for almost all $x,y \in \R^3$,
    \begin{alignat}{2}
      \| k \|_{L^2} & \le C \| \varphi \|_{H^1}^2, \qquad & 
      \| \nabla_1 k \|_{L^2} & \le C \| \varphi \|_{H^1}^2 \sqrt{N}, \tag{i}
      \label{k} \\
      |k(x,y)| & < N |\varphi(x)| \, |\varphi(y)|, \qquad & \| \nabla_1 k
      \overline{k} \|_{L^2} & \le C \| \varphi \|_{H^1}^2, \notag \\
      \| p \|_{L^2} & \le e^{C \| \varphi \|_{H^1}^2}, \qquad & 
      \| \nabla_1 p \|_{L^2} & \le e^{C \| \varphi \|_{H^1}^2}, 
      \tag{ii} \label{p} \\
      \| r \|_{L^2} & \le e^{C \| \varphi \|_{H^1}^2}, \qquad & 
      \| \nabla_1 r \|_{L^2} & \le e^{C \| \varphi \|_{H^1}^2}, 
      \tag{iii} \label{r} \\
      |r(x,y)| & \le e^{C \| \varphi \|_{H^1}} |\varphi(x)| \, |\varphi(y)|,
      \qquad & \notag \\
      \| s \|_{L^2} & \le e^{C \| \varphi \|_{H^1}^2}, \qquad & 
      \| \nabla_1 s \|_{L^2} & \le 2 e^{C \| \varphi \|_{H^1}^2} \sqrt{N}.
      \tag{iv} \label{s}
    \end{alignat}
  Furthermore, if $\varphi \in H^2(\R^3)$, then
  \begin{equation} \label{sup}
    \sup_{x \in \R^3} \, \norm{p_x}_{L^2} \apprle e^{C \| \varphi \|_{H^1}^2}
    \norm{\varphi}_{H^2} \qquad \text{and} \qquad \sup_{x \in \R^3}
    \norm{s_x}_{L^2} \apprle e^{C \| \varphi \|_{H^1}^2}
    \norm{\varphi}_{H^2}. \tag{v}
  \end{equation}
\end{lem}


\begin{prp} \label{p:wphi}
  Let $\varphi \in H^1(\R^3)$, and let $f_N=1-w_N$ be the solution of the
  zero-energy scattering equation as in Lemma \ref{l:w}. Then, for almost all
  $x,y \in \R^3$,
  \begin{align*}
    \int dz \, N^2 w_N(x-z)^2 |\varphi(z)|^2 & \le C^2 \| \nabla \varphi
    \|_{L^2}^2, \\
    \int dz \, N^2 w_N(x-z) w_N(z-y) |\varphi(z)|^2 & \le C^2 \| \nabla \varphi
    \|_{L^2}^2, \\
    \int dx dy \, N^2 |\nabla w_N(x-y)|^2 |\varphi(x)|^2 |\varphi(y)|^2 & \le
    C^2 N \| \varphi \|_{H^1}^2,
  \end{align*}
  where $C$ is a constant, uniform in $N$, that depends only on the unscaled
  potential $V$.
\end{prp}


\begin{proof}[Proof of Proposition \ref{p:wphi}]
  Using Lemma \ref{l:w}\ref{lw3}, and the operator inequality $-4 \Delta \ge
  |x|^{-2}$ on $L^2(\R^3)$, we find that
  \begin{align*}
    \int dz \, N^2 w_N(x-z)^2 |\varphi(z)|^2 & \le C_1^2 \int dz \, |x-z|^{-2}
    |\varphi(z)|^2 \\
    & = C_1^2 \langle \varphi(\,\cdot\,+x), |\,\cdot\,|^{-2}
    \varphi(\,\cdot\,+x) \rangle_{L^2} \\
    & \le -4 C_1^2 \langle \varphi(\,\cdot\,+x), \Delta \varphi(\,\cdot\,+x)
    \rangle_{L^2} = 4 C_1^2 \| \nabla \varphi \|_{L^2}^2,
  \end{align*}
  where $C_1 = \max\{a_0, R\}$ is a constant from Lemma \ref{l:w}. This proves
  the first estimate.


  Now, by H\"older's inequality, and the above estimate,
  \begin{align*}
    & \int dz \, N^2 w_N(x-z) w_N(z-y) |\varphi(z)|^2 \\
    & \le \left( \int dz \, N^2 w_N(x-z)^2 |\varphi(z)|^2 \right)^{1/2} \left(
    \int dz \, N^2 w_N(z-y)^2 |\varphi(z)|^2 \right)^{1/2} \le 4C_1^2 \|
    \nabla \varphi \|_{L^2}^2.
  \end{align*}
  This gives the second estimate in the proposition.


  Finally, by Young's inequality, Sobolev's inequality, and Lemma
  \ref{l:w}\ref{lw5},
  \begin{align*}
    \int dx dy \, N^2 |\nabla w_N(x-y)|^2 |\varphi(x)|^2 |\varphi(y)|^2 & \le
    N^2 \| | \nabla w_N|^2 \|_{L^1} \| \varphi \|_{L^4}^4 \\
    & \apprle C_2 N \| \varphi \|_{H^1}^4 \int dx \, \frac{N^3}{(|Nx|^2 +
    1)^2} \apprle C_2 N \| \varphi \|_{H^1}^4,
  \end{align*}
  where $C_2$ is a constant from Lemma \ref{l:w}, that depends only on $V$ and
  $a_0$. This proves the third estimate and concludes the proof of the
  proposition.
\end{proof}


\begin{proof}[Proof of Lemma \ref{l:kernels}]
  \eqref{k} We will repeteadly apply Proposition \ref{p:wphi}, Sobolev's
  inequality, Leibinz' rule, and triangle inequality. We begin with
  \[
    \| k \|_{L^2}^2 = \int dx \, |\varphi(x)|^2 \int dy \, N^2 w_N(x-y)^2
    |\varphi(y)|^2 \le C^2 \| \varphi \|_{L^2}^2 \| \nabla \varphi \|_{L^2}^2
    \apprle C^2 \| \varphi \|_{H^1}^4
  \]
  and
  \begin{align*}
    \| \nabla_1 k \|_{L^2} & \le \| N w_N(x-y) \nabla \varphi(x) \varphi(y)
    \|_{L^2} + \| N \nabla w_N(x-y) \varphi(x) \varphi(y) \|_{L^2} \\
    & \le C \| \nabla \varphi \|_{L^2}^2 + C \sqrt{N} \| \varphi \|_{H^1}^2
    \apprle C \sqrt{N} \| \varphi \|_{H^1}^2.
  \end{align*}
  This proves the first two estimates of part \eqref{k}. Now,
  \[
    \| \nabla_1 k \overline{k} \|_{L^2} \le \| I \|_{L^2} + \| J \|_{L^2}
  \]
  with
  \[
    \| I \|_{L^2}^2 = \int dx \, |\nabla \varphi(x)|^2 \int dy \,
    |\varphi(y)|^2 \left| \int dz \, N^2 w_N(x-z) w_N(z-y) |\varphi(z)|^2
    \right|^2 \apprle C^4 \| \varphi \|_{H^1}^8
  \]
  and
  \[
    \| J \|_{L^2}^2 = \int dx \, |\varphi(x)|^2 \int dy \, |\varphi(y)|^2
    \left| \int dz \, N^2 \nabla w_N(x-z) w_N(z-y) |\varphi(z)|^2 \right|^2
    \apprle C^4 \| \varphi \|_{H^1}^8.
  \]
  Therefore,
  \[
    \| \nabla_1 k \overline{k} \|_{L^2} \le \| I \|_{L^2} + \| J \|_{L^2}
    \apprle C^2 \| \varphi \|_{H^1}^4.
  \]
  This gives the third estimate. To prove the forth estimate, observe that
  Lemma \ref{l:w}\ref{lw1} says that $w_N < 1$. Hence,
  \[
    |k(x,y)| = N w_N(x-y) |\varphi(x)| \, |\varphi(y)| < |\varphi(x)| \,
    |\varphi(y)|.
  \]
  This completes the proof of part \eqref{k}.


  \eqref{p} The first bound follows directly from Lemma \ref{l:bt}\ref{l:bt2}
  and part \eqref{k}:
  \[
    \| p \|_{L^2} \le e^{\| k \|_{L^2}} \le e^{C \| \varphi \|_{H^1}^2}.
  \]
  Now, by H\"older's inequality, and part \eqref{k},
  \[
    \| \nabla_1 (k \overline{k})^n \|_{L^2} = \| (\nabla_1 k \overline{k}) (k
    \overline{k})^{n-1} \|_{L^2} \le \| \nabla_1 k \overline{k} \|_{L^2} \| k
    \|_{L^2}^{2(n-1)} \le C^{2n} \| \varphi \|_{H_1}^{4n}.
  \]
  \marginpar{Exchanging limits here\dots}
  Consequently, 
  \[
    \| \nabla_1 p \|_{L^2} \le \sum_{n=1}^\infty \frac{1}{(2n)!} \| \nabla_1
    (k \overline{k})^n \|_{L^2} \le e^{C \| \varphi \|_{H^1}^2}.
  \]


  \eqref{r} The proof of the estimates for $\| r \|_{L^2}$ and $\| \nabla_1 r
  \|_{L^2}$ is essentially the same as the proof of part \eqref{p}. We omit
  the details. To bound $|r(x,y)|$, first observe that, by H\"older's
  inequality, Proposition \ref{p:wphi}, and part \eqref{k},
  \begin{align*}
    & |k( \overline{k} k)^n (x,y)| = | k ( \overline{k} k)^{n-1} \overline{k}
    k(x,y)| \\
    & = |\varphi(x)| \, |\varphi(y)| \, \left| \int dz_1 dz_2 \, N^2
    w_N(x-z_1) w_N(z_2-y) \varphi(z_1) \varphi(z_2) (\overline{k}
    k)^{n-1}k(z_1,z_2) \right| \\
    & \le |\varphi(x)| \, |\varphi(y)| \, \| (\overline{k} k)^{n-1} k \|_{L^2}
    \\
    & \quad \times \left( \int dz_1 N^2 w_N(x-z_1)^2 |\varphi(z_1)|^2 \int
    dz_2 \, N^2 w_N(z_2-y) |\varphi(z_2)|^2 \right)^{1/2} \\
    & \le |\varphi(x)| \, |\varphi(y)| \, C^2 \| \nabla \varphi \|_{L^2}^2 \|
    k \|_{L^2}^{2n-1} \le |\varphi(x)| \, |\varphi(y)| \, C^{2n} \| \varphi
    \|_{H^1}^{4n}.
  \end{align*}
  Thus,
  \[
    |r(x,y)| \le \sum_{n=1}^\infty \frac{|k( \overline{k} k)^n
    (x,y)|}{(2n+1)!} \le |\varphi(x)| \, |\varphi(y)| e^{C \| \varphi
    \|_{H^1}^2}.
  \]
  This completes the proof of part \eqref{r}.


  \eqref{s} Recall that $s = k + r$. Then the estimates for $\| s \|_{L^2}$
  and $\| \nabla_1 s \|_{L^2}$ follow by applying triangle inequality and
  parts \eqref{k} and \eqref{r}.


  \eqref{sup} We give the proof of the estimate for $\sup_{x \in \R^3} \| s_x
  \|_{L^2}$. The proof for $\sup_{x \in \R^3} \| p_x \|_{L^2}$ is similar.
  First, by H\"older's inequality, and Proposition \ref{p:wphi},
  \begin{align*}
    |k(\overline{k} k)^n(x,y)| & = \left| \varphi(x) \int dz \, N w_N(x-z)
    \varphi(z) (\overline{k} k)^n(z,y) \right| \\
    & \le |\varphi(x)| \left( \int dz \, N^2 w_N(x-z)^2 |\varphi(z)|^2
    \right)^{1/2} \| (\overline{k} k)^n(\, \cdot\,, y) \|_{L^2} \\
    & \le C \| \nabla \varphi \|_{L^2} | \varphi(x)| \, \| (\overline{k}
    k)^n(\, \cdot\,, y) \|_{L^2}.
  \end{align*}
  Hence, using the series representation for $s = k+r$, and part \eqref{k},
  \[
    \| s_x \|_{L^2} \le \sum_{n=0}^\infty \frac{\| k (\overline{k} k)^n(x,\,
    \cdot\,) \|_{L^2}}{(2n+1)!} \le |\varphi(x)| \sum_{n=0}^\infty
    \frac{C^{2n+1} \| \varphi \|_{H^1}^{2n+1}}{(2n+1)!} \le e^{C \|
    \varphi\|_{H^1}^2} |\varphi(x)|.
  \]
  Therefore, by Sobolev's inequality,
  \[
    \sup_{x \in \R^3} \| s_x \|_{L^2} \le e^{C \| \varphi \|_{H^1}^2} \sup_{x
    \in \R^3} |\varphi(x)| \apprle e^{C \| \varphi \|_{H^1}^2} \| \varphi
    \|_{H^2}.
  \]
  This proves part \eqref{sup} and completes the proof of the lemma.
\end{proof}


\begin{proof}[Proof of Lemma \ref{l:bt}]
  \ref{l:bt1} The operator $B(k)$ is unbounded, densely defined, and
  anti-self-adjoint. Hence, by the spectral theorem, the unitary operator
  $T(k) = e^{B(k)}$ is well-defined and $T(k)^* = T(k)^{-1} = T(-k)$.


  \ref{l:bt2} Let $\chi$ be the characteristic function, and consider a real
  number $M > 0$. We give a proof of the formula on the subspace $\chi(\N \le
  M) \mathcal{F}$, and then we indicate how to extend it to the whole
  $\mathcal{F}$. First, note that $B(k)$ and $a(f)$ are bounded operators on
  $\chi(\N \le M) \mathcal{F}$. Recall the formula
  \[
    e^X Y e^{-X} = Y + [X,Y] + \frac{1}{2!} [X,[X,Y]] + \cdots
  \]
  Then, observing that $[a_x a_y, a_z] = [a_x^* a_y^*, a_z^*] = 0$ and
  \[
    [a_x a_y, a_z^*] = \delta(x-z) a_y + \delta(y-z) a_x \qquad \text{and}
    \qquad [a_x^* a_y^*, a_z] = -\delta(z-x) a_y^* + \delta(z-y) a_x^*,
  \]
  one easily finds the expression
  \begin{align*}
    T(x)^* a_x T(k) & = e^{-B(k)} a_x e^{B(k)} = a_x - [B(k), a_x] +
    \frac{1}{2!} [B(k), [B(k), a_x]] + \cdots \\
    & = a_x + \int dy \, \sum_{n=1}^\infty \frac{1}{(2n)!} \big( k
    \overline{k} \big)^n(x,y) a_y + \int dy \, \sum_{n=0}^\infty
    \frac{1}{(2n+1)!} k \big( \overline{k} k \big)^n(x,y) a_y^* \\
    & = a_x + a(p_x) + a(s_x) = a(c_x) + a(s_x).
  \end{align*}
  Furthermore, applying H\"older's inequality, we get
  \[
    \| p \|_{L^2} \le \sum_{n=1}^\infty \frac{1}{(2n)!} \| k \|_{L^2}^{2n} \le
    e^{\| k \|_{L^2}},
  \]
  and similarly $\| s \|_{L^2} \le e^{\| k \|_{L^2}}$.


  \ref{l:bt3} Recall that the Hilbert-Schmidt norm of an operator is equal to
  the $L^2$-norm of its integral kernel. Then part \ref{l:bt3} is just a
  restatement of part \ref{l:bt2}.
\end{proof}


\section{The generator}
Define
\bd
U_N(t) := T^\ast_t W^\ast_t e^{-it \Hcal_N} W_0 T_0.
\ed
The generator $\Lcal_N(t)$ is defined by
\bd
\Lcal_N(t) U_N(t) = i \partial_t U_N(t).
\ed
We calculate that
\begin{align*}
\Lcal_N(t) 	& = (i \partial_t T^*_t) T_t + T^*_t \left( (i \partial_t W^*_t) W_t + W^*_t \Hcal_N W_t \right) T_t \\
		& =: (i \partial_t T^*_t) T_t + T^*_t \Lcal^{(0)}_N(t) T_t
\end{align*}
where
\begin{align*}
& \Lcal^{(0)}_N(t) = \Kcal + \Vcal_N \\
		& + N^{1/2} \left(  a^*\left( (w_N N V_N \ast \lvert \ph \rvert^2)\ph \right) + \hc  \right) \\
		& + N^0	    \left(  \frac{1}{2}\int \di x \di y\, NV_N(x-y)\left( \cc{\ph(x)} \cc{\ph(y)} a_y a_x + \hc \right) \right) \\
		& + N^0	    \left(  \int \di x \di y\, NV_N(x-y)\left( \lvert \ph(x) \rvert^2 a^*_y a_y + \cc{\ph(x)} \ph(y) a^*_y a_x \right) \right) \\
		& + N^{-1/2}\left(  \int \di x \di y\, NV_N(x-y) \left( \cc{\ph(x)} a^*_y a_y a_x + \hc \right)  \right) \\
		& + N b(N,t),
\end{align*}
with a phase (which, like all phases, will be dropped from now on without any further comment)
\bd
b(N,t) = \norm{\nabla \ph}_{L^2}^2 - \Im \scal{\ph}{\phdot} + \frac{1}{2}\int \di x \di y\, NV_N(x-y) \lvert \ph(x)\rvert^2 \lvert \ph(y) \rvert^2.
\ed
Here we have made use of a cancellation in the term $\Ocal({N^{1/2}})$ (term which is linear in creation/annihilation operators) due to $\ph$ satisfying the modified Hartree equation
\bd
i\partial_t \ph = -\Delta \ph + \left(f_N N V_N \ast \lvert \ph \rvert^2 \right) \ph.
\ed
Compare \cite{RS2009}, but notice that due to the factor $f_N$ in the Hartree equation, in our case the cancellation is incomplete. This is essential for ensuring the correct coupling constant $8\pi a_0$ and ensures further cancelation with the cubic terms. This cancelation is revealed through the Bogoliubov transformation, to be calculated now.

We now identify cancellations between linear terms stemming from the transformation of linear terms and linear terms stemming from normal-ordering transformed cubic operators:
\begin{align*}
& T^*_t \Lcal^{(0)}_N(t) T_t = \int \di x\, T^*_t a^*_x T_t \bigg[   \frac{1}{2}(-\Delta_x T^*_t a_x T_t) \\
& + \frac{1}{4}\int \di y V_N(x-y) T^*_t a^*_y a_y a_x T_t \\
& + N^{1/2} \ph(x) \int \di y\, w_N N V_N(x-y) \lvert \ph(y) \rvert^2 \tagg{linearterm} \\
& + \frac{1}{2} \int \di y\, NV_N(x-y)  \ph(x) \ph(y)  T^*_t a^*_y T_t \\
& + \frac{1}{2} \int \di y\, NV_N(x-y) \left(  \lvert \ph(y)\rvert^2 T^*_t a_x T_t + \ph(x) \cc{\ph(y)} T^*_t a_y T_t  \right) \\
& + N^{-1/2} \int \di y\, NV_N(x-y) \cc{\ph(y)} T^*_t a_x a_y T_t  \bigg]\tagg{cubicterm} \\
& + \hc
\end{align*}
Between the term \eqr{linearterm} and \eqr{cubicterm}, after normal ordering and plugging in $c = \delta + p$ and $s = k + r$, we find that the linear term is cancelled up to two remainder terms (the $N^{1/2}$-term is completely cancelled, i.\,e.\ line \eqr{linearterm}). Result:
\begin{align*}
& T^*_t \Lcal^{(0)}_N(t) T_t = \int \di x\, T^*_t a^*_x T_t \bigg[   \frac{1}{2}(-\Delta_x T^*_t a_x T_t) \\
& + \frac{1}{4}\int \di y V_N(x-y) T^*_t a^*_y a_y a_x T_t \\
& + \frac{1}{2} \int \di y\, NV_N(x-y)  \ph(x) \ph(y)  T^*_t a^*_y T_t \\
& + \frac{1}{2} \int \di y\, NV_N(x-y) \left(  \lvert \ph(y)\rvert^2 T^*_t a_x T_t + \ph(x) \cc{\ph(y)} T^*_t a_y T_t  \right) \\
& + N^{-1/2} \int \di y\, NV_N(x-y) \cc{\ph(y)} \bigg( r(x,y) + \scal{p_x}{s_y} + \\
& \qquad \qquad  a^*(s_x) a^*(s_y) + a^*(s_x) a(c_y)  + a^*(s_y) a(c_x) + a(c_x) a(c_y)  \bigg)  \bigg] \\
& + \hc
\end{align*}
Now we apply Lemma \ref{l:bt} everywhere and expand all the terms, normal order all terms, and observe that many terms appear twice (maybe in the hermitian conjugate or with $x$ and $y$ interchanged):
\begin{align*}
& T^*_t \Lcal_N^{(0)}(t) T_t = \\ 
& \frac{1}{2} \int \di x\, \left[ a^*(c_x) a(-\Delta_x c_x) + \boxed{2 a^*(c_x) a^*(-\Delta_x s_x)} + a^*(-\Delta_x s_x) a(s_x) \right] \tagg{cancellation_kinetic} \\
& + \frac{1}{2}\int \di x \di y\, NV_N(x-y) \times \\
& \times \Big[   \frac{1}{2N}\bigg( a^*(c_x) a^*(c_y) a(c_y) a(c_x) + 4 a^*(c_x) a^*(c_y) a^*(s_x) a(c_y) \\
				      & \qquad\qquad + 2 a^*(c_x) a^*(c_y) a^*(s_y) a^*(s_x) + 2 a^*(c_x) a^*(s_x) a(s_y) a(c_y) \\
				      & \qquad\qquad + 2 a^*(c_x) a^*(s_y) a(s_y) a(c_x) + 4 a^*(c_x) a^*(s_y) a^*(s_x) a(s_y) \\
				      & \qquad\qquad + a^*(s_y) a^*(s_x) a(s_x) a(s_y) \bigg) \\
& + \frac{1}{N}\bigg(   \boxed{a^*(c_x)a^*(c_y) \scal{c_y}{s_x}} + a^*(c_x) a(c_y) \scal{s_y}{s_x} \tagg{cancellation_normalorder} \\
			& \qquad\quad + a^*(c_x) a(s_y) \scal{c_y}{s_x} + a^*(c_x) a(c_x) \scal{s_y}{s_y} \\
			& \qquad\quad + 2 a^*(c_x) a^*(s_x) \scal{s_y}{s_y} + 2a^*(c_x)a^*(s_y) \scal{s_y}{s_x} \\
			& \qquad\quad + a^*(c_y) a(s_x) \scal{c_y}{s_x} +  a^*(s_y) a(s_y) \scal{s_x}{s_x}\\
			& \qquad\quad + a^*(s_y) a^*(s_x) \scal{s_x}{c_y} + a^*(s_y) a(s_x) \scal{s_y}{s_x}   \bigg) \\
& + \ph(x)\ph(y) \Big( \boxed{a^*(c_x) a^*(c_y)} + 2 a^*(c_x) a(s_y) +a(s_x) a(s_y) \Big) \tagg{cancellation_standard} \\
& + \ph(x) \cc{\ph(y)} \Big( a^*(c_x) a(c_y) + 2 a^*(c_x) a^*(s_y) + a^*(s_y) a(s_x) \Big) \\
& + \lvert \ph(y) \rvert^2 \Big( a^*(c_x) a(c_x) + 2 a^*(c_x) a^*(s_x) + a^*(s_x) a(s_x) \Big) \\
& + \frac{2}{\sqrt{N}}\cc{\ph(y)} \bigg(    a^*(c_x) a^*(s_x) a^*(s_y) + a^*(c_x) a^*(s_x) a(c_y) + a^*(s_x) a^*(s_y) a(s_x)\\
					    & \qquad\qquad\qquad + a^*(c_x) a^*(s_y) a(c_x) + a^*(c_x) a(c_x) a(c_y)+ a^*(s_x) a(s_x) a(c_y) \\
					    & \qquad\qquad\qquad + a^*(s_y) a(s_x) a(c_x) + a(s_x) a(c_x) a(c_y)    \bigg) \\
& + \frac{2}{\sqrt{N}}\cc{\ph(y)} \bigg(    a^*(s_x) \scal{s_x}{s_y} + a^*(s_y) \scal{s_x}{s_x}  + a(c_y) \scal{s_x}{s_x} + a(c_x) \scal{s_x}{s_y} \\
					    & \qquad\qquad\qquad + a^*(c_x)r(x,y) + a^*(c_x)\scal{p_x}{s_y} + a(s_x)r(x,y) + a(s_x)\scal{p_x}{s_y}		\bigg)    \Big] + \hc
\end{align*}
We now proceed to identify a cancellation between the second summand in line \eqr{cancellation_kinetic}, the first summand in line \eqr{cancellation_normalorder} and the first summand in line \eqr{cancellation_standard}. To see the cancellation, we expand $c = \delta + p$ and $s = k + r$ and use the product rule for the Laplacian $-\Delta_x k(y,x)$, then notice that the lhs of the zero-energy scattering equation \eqr{scatteringequation} appears. There are however some (more regular) remainder terms left, see \eqref{l7}, \eqref{l8}, \eqref{l14} and \eqref{l18} in the final result for the generator.

Then the final result for the generator is:
\begin{align}
& T^*_t \Lcal_N^{(0)}(t) T_t = \nonumber \\ 
& \frac{1}{2} \int \di x\, \bigg[ a^*(c_x) \int \di y\, a^*_y \Big( N \nabla w_N(x-y) \nabla_x \ph(x) 2 \ph(y) \label{l7}\\
& \qquad\qquad \qquad\qquad \qquad	+ Nw_N(x-y) \Delta_x \ph(x) \ph(y) - \Delta_x r(y,x) \Big) \label{l8}\\
& \qquad\qquad 			+ a^*(c_x) a(-\Delta_x c_x) + a^*(-\Delta_x s_x) a(s_x) \bigg] \label{l9} \\
& + \frac{1}{2}\int \di x \di y\, NV_N(x-y) \times \nonumber \\
& \times \Big[   \frac{1}{2N}\bigg( a^*(c_x) a^*(c_y) a(c_y) a(c_x) + 4 a^*(c_x) a^*(c_y) a^*(s_x) a(c_y) \label{l10}\\
				      & \qquad\qquad + 2 a^*(c_x) a^*(c_y) a^*(s_y) a^*(s_x) + 2 a^*(c_x) a^*(s_x) a(s_y) a(c_y) \label{l11}\\
				      & \qquad\qquad + 2 a^*(c_x) a^*(s_y) a(s_y) a(c_x) + 4 a^*(c_x) a^*(s_y) a^*(s_x) a(s_y) \label{l12}\\
				      & \qquad\qquad + a^*(s_y) a^*(s_x) a(s_x) a(s_y) \bigg) \label{l13}\\
& + \frac{1}{N}\bigg(   a^*(c_x) a^*(c_y) \Big( r(y,x) + \scal{p_y}{s_x} \Big) + a^*(c_x) a^*(p_y) k(y,x) \label{l14} \\
      & \qquad\quad + a^*(c_x) a(c_y) \scal{s_y}{s_x} + a^*(s_y) a(s_y) \scal{s_x}{s_x} + a^*(s_y) a(s_x) \scal{s_y}{s_x} \label{l15}\\
      & \qquad\quad + a^*(c_x) a(s_y) \scal{c_y}{s_x} + a^*(c_x) a(c_x) \scal{s_y}{s_y} + a^*(s_y) a^*(s_x) \scal{s_x}{c_y} \label{l16}\\
      & \qquad\quad + 2a^*(c_x) a^*(s_x) \scal{s_y}{s_y} + 2a^*(c_x)a^*(s_y) \scal{s_y}{s_x} + a^*(c_y) a(s_x) \scal{c_y}{s_x}    \bigg) \label{l17}\\
& + \ph(x)\ph(y) \Big( a^*(c_x) a^*(p_y) + 2 a^*(c_x) a(s_y) +a(s_x) a(s_y) \Big) \label{l18}\\
& + \ph(x) \cc{\ph(y)} \Big( a^*(c_x) a(c_y) + 2 a^*(c_x) a^*(s_y) + a^*(s_y) a(s_x) \Big) \label{l19}\\
& + \lvert \ph(y) \rvert^2 \Big( a^*(c_x) a(c_x) + 2 a^*(c_x) a^*(s_x) + a^*(s_x) a(s_x) \Big) \label{l20}\\
& + \frac{2}{\sqrt{N}}\cc{\ph(y)} \bigg(    a^*(c_x) a^*(s_x) a^*(s_y) + a^*(c_x) a^*(s_x) a(c_y) \label{l21}\\
					    & \qquad\qquad\qquad + a^*(c_x) a^*(s_y) a(c_x) + a^*(c_x) a(c_x) a(c_y) + a^*(s_y) a(s_x) a(c_x) \label{l22}\\
					    & \qquad\qquad\qquad + a^*(s_x) a^*(s_y) a(s_x) + a^*(s_x) a(s_x) a(c_y) + a(s_x) a(c_x) a(c_y)  \bigg) \label{l23}\\
& + \frac{2}{\sqrt{N}}\cc{\ph(y)} \bigg(    a^*(s_x) \scal{s_x}{s_y} + a^*(s_y) \scal{s_x}{s_x}  + a(c_y) \scal{s_x}{s_x} + 							a(c_x) \scal{s_x}{s_y} \label{l24}\\
					    & \qquad\qquad\qquad + a^*(c_x)r(x,y) + a^*(c_x)\scal{p_x}{s_y} + a(s_x)r(x,y) + 			a(s_x)\scal{p_x}{s_y}		\bigg)    \Big] \label{l25}\\
&+ \hc
\end{align}

\begin{rem}[This is the punch line of the paper!]The cancellations enable us to give the necessary estimates for all terms, see chapter \ref{ch:generatorestimates}. If we had not identified the cancellation between the boxed terms, the best estimates with which the authors were able to come up with were (focusing on the most singular contributions):\\
\underline{for \eqref{eq:cancellation_kinetic}}
\bd
\scal{\psi}{\int \di x\, a^\ast(c_x) a^\ast(-\Delta_x s_x) \psi}
= \int \di x\, \scal{\nabla_x a_x \psi}{a^\ast(\nabla_x s_x)} + \mbox{reg.} 
\leq \ev{\Kcal} + \norm{\nabla_x s}^2 \ev{\Ncal+1} + \mbox{reg.}, 
\ed
where $\norm{\nabla_x s}^2 = \mathcal{O}(N^1)$.\\
\underline{for \eqref{eq:cancellation_normalorder}}
\begin{align*}
& \scal{\psi}{\frac{1}{2}\int \di x\di y\, V_N(x-y) a^\ast(c_x) a^\ast(c_y) \scal{c_y}{s_x}\psi} \\
& \leq \frac{1}{2} \int \di x\di y\, V_N(x-y) \lvert k(y,x)\rvert \lvert \scal{a(c_y)a(c_x)\psi}{\psi} \rvert + \mbox{reg.} \\
& \leq \frac{1}{2} \int \di x\di y\, V_N(x-y) N\lvert \ph(x)\rvert \lvert \ph(x)\rvert \norm{a(c_y)a(c_x)\psi} \norm{\psi} + \mbox{reg.} \\
& \leq 2 \varepsilon \ev{\tilV} + N \frac{1}{2}\int \di x\di y\, NV_N(x-y) \lvert\ph(x)\rvert^2 \lvert \ph(y)\rvert^2 \norm{\psi}^2 \frac{1}{\varepsilon}  + \mbox{reg.},
\end{align*}
where the second term is just a number, but of order $N^1$. (One could add a constant to $\Lcal_N(t)$ which compensates the term here, but this won't help for $[\Lcal_N(t),\Ncal]$, because constant drop from the commutator.)\\
\underline{for \eqref{eq:cancellation_standard}}
\begin{align*}
& \scal{\psi}{\frac{1}{2}\int \di x\di y\, NV_N(x-y) \ph(x)\ph(y) a^\ast(c_x) a^\ast(c_y) \psi} \\
& \leq 2 \varepsilon \ev{\tilV} + N \frac{1}{2}\int \di x\di y\, NV_N(x-y) \lvert\ph(x)\rvert^2 \lvert \ph(y)\rvert^2 \norm{\psi}^2 \frac{1}{\varepsilon},
\end{align*}
same problem as the previous estimate.

In lemma \ref{lem:lncommutatorbound}, an upper bound for $\ev{[\Lcal_N(t),\Ncal]}$ of order $N^1$ would ruin the application of Gronwall's lemma. \marginpar{even though we add $\Lcal$ before applying Gronwall?}
\end{rem}

\section{Estimates for the terms of the generator}
\label{ch:generatorestimates}
In this chapter we denote the expectation value $\scal{\psi}{A\psi}$ by $\ev{A}$. In this chapter, $C$ is a constant which can change from line to line in inequalities, but which is always independent of $N$ and $t$. The constant $C$ can depend on $\norm{\ph}_{H^1}$.

All bounds proven in this chapter hold as operator inequalities, independent of the choice of $\psi$. In particular, the structure of the time evolution $\psi_N(t) = U_N(t)\psi_0$ is not used in this chapter.

For convenience, we remind the reader of the simple but useful estimate
\bd
\int \di x\, \norm{a(c_x)\psi}^2 \leq 2(1+\norm{p}^2)\scal{\psi}{\Ncal\psi} \leq C\ev{\Ncal}.
\ed
We now have the following list of estimates for all terms of the generator.\newline
\underline{We start with estimates for the kinetic terms:}\newline
\estlist{7}{1} (here, the cancellation is important!)
\begin{align*}
& \lvert \scal{\psi}{\frac{1}{2}\int \di x\,a^\ast(c_x) \int \di y\,a^\ast_y N\nabla_x w_N(x-y)\nabla_x \ph(x) 2\ph(y) \psi}\rvert \\
\leq & \lvert \int \di x \di y\, Nw_N(x-y) \nabla_x \ph(x)  \nabla_y \ph(y) \scal{a_y\psi}{a^\ast_x\psi} \rvert \\
& + \lvert \int \di x \di y\,  Nw_N(x-y) \nabla_x \ph(x) \ph(y)  \scal{(\nabla_y a_y)\psi}{a^\ast_x \psi} \rvert \\
& + \lvert \int \di x \di y\, Nw_N(x-y) \nabla_x  \ph(x) \nabla_y \ph(y) \scal{a_y \psi}{a^\ast(p_x)\psi} \rvert \\
& + \lvert \int \di x \di y\, Nw_N(x-y) \nabla_x \ph(x) \ph(y) \scal{(\nabla_y a_y)\psi}{a^\ast(p_x)\psi} \rvert \\
\leq & \int \di y \lvert \nabla_y \ph(y)\rvert \norm{a_y \psi} \norm{a^\ast(Nw_N(\cdot -y)\nabla\ph(\cdot))\psi} \\
& + \int \di y \lvert \ph(y)\rvert \norm{(\nabla_y a_y)\psi} \norm{a^\ast(Nw_N(\cdot-y)\nabla\varphi(\cdot))\psi} \\
& + \int \di x\di y\, \frac{C}{\lvert x-y\rvert} \lvert \nabla_x\ph(x)\rvert \lvert \nabla_y \ph(y)\rvert \norm{a_y \psi} \norm{p_x} \norm{(\Ncal+1)^{1/2}\psi} \\
& + \int \di x\di y\, \frac{C}{\lvert x-y\rvert} \lvert \nabla_x\ph(x)\rvert \lvert \ph(y)\rvert \norm{(\nabla_y a_y)\psi} \norm{p_x} \norm{(\Ncal+1)^{1/2}\psi} \\
\leq & C \norm{\Delta\ph}_{L^2} \norm{(\Ncal+1)^{1/2}\psi} \int \di y \lvert \nabla_y \ph(y) \rvert \norm{a_y \psi} \\
& + C \norm{\Delta \ph}_{L^2} \norm{(\Ncal+1)^{1/2}\psi} \int \di y \frac{1}{\sqrt{\varepsilon}} \lvert \ph(y)\rvert \sqrt{\varepsilon} \norm{(\nabla_y a_y)\psi} \\
& + \norm{(\Ncal+1)^{1/2}\psi} C \left(\int \di x\di y\, \lvert \nabla_x\ph(x)\rvert^2 \frac{\lvert\nabla_y \ph(y)\rvert^2}{\lvert x-y \rvert^2} \right)^{1/2} \left(\int \di x\di y\, \norm{a_y \psi}^2 \norm{p_x}^2 \right)^{1/2} \\
& + \norm{(\Ncal+1)^{1/2}\psi} C \left(\int \di x\di y\, \lvert \nabla_x\ph(x)\rvert^2 \frac{\lvert \ph(y)\rvert^2}{\lvert x-y\rvert^2} \right)^{1/2} \left(\int \di x\di y\, \norm{(\nabla_y a_y)\psi}^2 \norm{p_x}^2 \right)^{1/2} \\
\leq & C \norm{\Delta \ph}_{L^2} \norm{(\Ncal+1)^{1/2}\psi} \norm{\nabla \ph}_{L^2} \norm{\Ncal^{1/2}\psi} \\
& + C\norm{\Delta \ph}_{L^2} \frac{1}{\sqrt{\varepsilon}} \norm{(\Ncal+1)^{1/2}\psi}  \sqrt{\varepsilon}\norm{\Kcal^{1/2}\psi} \norm{\ph}_{L^2} \\
& + C \norm{(\Ncal+1)^{1/2}\psi} \norm{\nabla \ph}_{L^2} \norm{\Delta \ph}_{L^2} \norm{p} \norm{\Ncal^{1/2}\psi} \\
& + C \frac{1}{\sqrt{\varepsilon}}\norm{(\Ncal+1)^{1/2}\psi} \sqrt{\varepsilon}\norm{\Kcal^{1/2}\psi} \norm{\nabla \ph}_{L^2}^2 \norm{p} \\
\leq & C \norm{\ph}_{H^2} \left( (1+\frac{1}{\varepsilon})\ev{\Ncal+1} + \varepsilon\ev{\Kcal} \right) + C \left( \frac{1}{\varepsilon}\ev{\Ncal+1} + \varepsilon \ev{\Kcal} \right).
\end{align*}
\estlist{8}{1} (here, the cancellation is important!)
\begin{align*}
&\lvert \scal{\psi}{\frac{1}{2}\int \di x\di y\,a^\ast(c_x) a^\ast_y N w_N(x-y) \Delta_x \ph(x) \ph(y) \psi}\rvert\\
\leq & \frac{1}{2} \int \di x\, \norm{a(c_x)\psi} \lvert \Delta_x \ph(x)\rvert \norm{a^\ast(Nw_N(x-\cdot)\ph(\cdot))\psi} \\
\leq & \frac{1}{2} \int \di x\, \norm{a(c_x)\psi} \lvert \Delta_x \ph(x)\rvert \norm{N w_N(x-\cdot)\ph(\cdot)}_{L^2} \norm{(\Ncal+1)^{1/2}\psi} \\
\leq & \frac{1}{2} C \norm{\nabla\ph}_{L^2} \norm{(\Ncal+1)^{1/2} \psi} \left( \int \di x\, \lvert \Delta_x \ph(x)\rvert^2 \right)^{1/2} \left( \int \di x\, \norm{a(c_x)\psi}^2 \right)^{1/2} \\
\leq & \frac{1}{2} C \norm{\ph}_{H^1} \norm{\ph}_{H^2} \ev{\Ncal+1} \\
\leq & C \norm{\ph}_{H^2} \ev{\Ncal+1}
\end{align*}
where $\norm{Nw_N(x-\cdot)\ph(\cdot)}_{L^2} \leq C\norm{\nabla \ph}_{L^2}$ (rhs independent of $x$) was used, which is a consequence of lemma \ref{l:w} and Hardy's inequality.\\
\estlist{8}{2}
For this estimate, it is useful to know that
\begin{align*}
 & \int \di x\, \norm{\nabla_x a(c_x)\psi}^2 \leq 2 \int \di x\, \norm{\nabla_x a_x \psi}^2 + 2\int \di x\, \norm{a(\nabla_x p_x)\psi}^2 \\
 \leq & 2\ev{\Kcal} + 2\int \di x\, \norm{\nabla_x p_x}^2 \norm{\Ncal^{1/2}\psi}^2 = 2\ev{\Kcal}+ 2\norm{\nabla_x p}^2 \ev{\Ncal}.
\end{align*}
Using this (and integration by parts), we get
\begin{align*}
 & \lvert \scal{\psi}{\frac{1}{2}\int \di x\, a^\ast(c_x) \int \di y\, (-\Delta_x r(y,x))\psi} \rvert \\
= & \lvert \scal{\psi}{\frac{1}{2}\int \di x\, \nabla_x a^\ast(c_x) \int \di y\, a^\ast_y \nabla_x r(y,x) \psi}\rvert \\
\leq & \frac{1}{2}\int \di x\, \norm{\nabla_x a(c_x)\psi} \norm{a^\ast(\nabla_x r_x)\psi} \\
\leq & \frac{1}{2}\left( \varepsilon \int \di x\, \norm{\nabla_x a(c_x) \psi}^2 \right)^{1/2} \left( \int \di x\, \norm{\nabla_x r_x}^2 \right)^{1/2} \frac{1}{\sqrt{\varepsilon}} \norm{(\Ncal+1)^{1/2}\psi} \\
\leq & \frac{1}{2}\norm{\nabla_x r} \left( \varepsilon 2\ev{\Kcal} + \varepsilon 2 \norm{\nabla_x p}^2 \ev{\Ncal} + \frac{1}{\varepsilon}\ev{\Ncal+1} \right) \\
\leq & C \left( \varepsilon \ev{\Kcal} + (\varepsilon+\frac{1}{\varepsilon})\ev{\Ncal+1} \right).
\end{align*}
\estlist{9}{1}
For this term we prove to different estimates. First notive that
\begin{align*}
 & \scal{\psi}{\frac{1}{2}\int \di x\, a^\ast(c_x) a(-\Delta_x c_x)\psi} = \scal{\psi}{\frac{1}{2}\int \di x\, a^\ast(\nabla_x c_x)a(\nabla_x c_x)\psi} \\
= & \scal{\psi}{\frac{1}{2}\int \di x \left( \nabla_x a^\ast_x + a^\ast(\nabla_x p_x) \right)\left(  \nabla_x a_x + a(\nabla_x p_x) \right) \psi} \\
= & \scal{\psi}{\frac{1}{2}\Kcal \psi} + \frac{1}{2}\int \di x\scal{\nabla_x a_x \psi}{a(\nabla_x p_x)\psi} + \frac{1}{2}\int \di x\, \scal{a(\nabla_x p_x)\psi}{\nabla_x a_x \psi} + \frac{1}{2}\int \di x\, \norm{a(\nabla_x p_x)\psi}^2.
\end{align*}
The first estimate we get from this (leaving away the last summand, which is positive) is
\begin{equation}
\tag{A}
\label{eq:bound1A}
\begin{split}
 \ev{\Kcal} & \leq \scal{\psi}{\int \di x\, a^\ast(c_x) a(-\Delta_x c_x)\psi} + 2\int \di x\, \sqrt{\varepsilon} \norm{\nabla_x a_x \psi} \frac{1}{\sqrt{\varepsilon}} \norm{a(\nabla_x p_x)\psi} \\
& \leq \scal{\psi}{\int \di x\, a^\ast(c_x) a(-\Delta_x c_x)\psi} + 2\left( \varepsilon \ev{\Kcal} + \frac{1}{\varepsilon}  \norm{\nabla_x p}^2 \ev{\Ncal}\right) \\
& \leq \scal{\psi}{\int \di x\,a^\ast(c_x) a(-\Delta_x c_x)\psi} + C\left( \varepsilon \ev{\Kcal} + \frac{1}{\varepsilon}\ev{\Ncal} \right),
\end{split}
\end{equation}
where the term of the generator appears on the rhs.\\
For the second estimate, we proceed as follows:
\begin{equation}
\tag{B}
\label{eq:bound1B}
\begin{split}
 & \scal{\psi}{\frac{1}{2} \int \di x\, a^\ast(c_x) a(-\Delta_x c_x)\psi} \\
\leq & \frac{1}{2}\ev{\Kcal} + \left( \ev{\Kcal} + \norm{\nabla_x p}^2 \ev{\Ncal}\right) + \frac{1}{2}\norm{\nabla_x p}^2 \ev{\Ncal} \\
\leq & C\left( \ev{\Kcal} + \ev{\Ncal} \right).
\end{split}
\end{equation}
\estlist{9}{2}
First, notice that, using the product rule, integration by parts and the fact that $-\nabla_x w_N(x-y) = \nabla_y w_N(x-y)$, we have
\begin{align*}
 a(\nabla_x s_x) & = \cc{\ph(x)} a(-Nw_N(x-\cdot)\nabla\ph(\cdot)) + \cc{\ph(x)}\,a\left(\nabla(-N w_N(x-\cdot)\ph(\cdot))\right) \\
 & + \cc{\nabla_x \ph(x)} a(-Nw_N(x-\cdot)\ph(\cdot)) + a(\nabla_x r_x).
\end{align*}
(the $\nabla$ in the second summand has to be understood with respect to the 'dot'-variable of $a$)
Using $(a+b+c+d)^2 \leq 4(a^2+b^2+c^2+d^2)$, we now conclude
\begin{equation}
%\tag{B}
\label{eq:bound2B}
\begin{split}
 & \scal{\psi}{\frac{1}{2}\int \di x\, a^\ast(-\Delta s_x) a(s_x)\psi} = \frac{1}{2}\int \di x\, \norm{a(\nabla_x s_x)\psi}^2 \\
\leq & \frac{1}{2} \int \di x \bigg( \lvert \ph(x)\rvert \norm{a(-Nw_N(x-\cdot)\nabla\ph(\cdot))\psi} \\
& \qquad\quad + \lvert \ph(x)\rvert \norm{a\left(\nabla(-N w_N(x-\cdot)\ph(\cdot))\right)\psi} \\
& \qquad\quad + \lvert \nabla_x \ph(x)\rvert \norm{a(-Nw_N(x-\cdot)\ph(\cdot))\psi} + \norm{a(\nabla_x r_x)\psi}  \bigg)^2 \\
\stackrel{\eqr{agradnorm}}{\leq} & 2\int \di x\, \bigg( \lvert \ph(x)\rvert^2 \norm{-N w_N(x-\cdot)\nabla\ph(\cdot)}_{L^2}^2 \norm{\Ncal^{1/2}\psi}^2  \\
& \qquad\quad+ \lvert \ph(x)\rvert^2 \norm{-Nw_N(x-\cdot) \ph(\cdot)}_{L^2}^2 \norm{\Kcal^{1/2}\psi}^2  \\
& \qquad\quad + \lvert \nabla_x \ph(x)\rvert^2 \norm{-N w_N(x-\cdot) \ph(\cdot)}_{L^2}^2 \norm{\Ncal^{1/2}\psi}^2 + \norm{\nabla_x r_x}^2 \norm{\Ncal^{1/2}\psi}^2 \bigg) \\
\leq & 2 \int \di x\, \bigg( \lvert \ph(x)\rvert^2 \norm{\Delta \ph}_{L^2}^2 C \norm{\Ncal^{1/2}\psi}^2 + \lvert \ph(x)\rvert^2 \norm{\nabla \ph}_{L^2}^2 C \norm{\Kcal^{1/2}\psi}^2 \\
& \qquad\qquad + \lvert \nabla_x \ph(x)\rvert^2 \norm{\nabla \ph}_{L^2}^2 C \norm{\Ncal^{1/2}\psi}^2 + \norm{\nabla_x r_x}^2 \norm{\Ncal^{1/2}\psi}^2 \bigg)\\
\leq & C \norm{\ph}_{H^2}^2 \ev{\Ncal} + C\ev{\Kcal} + C\ev{\Ncal}.
\end{split} 
\end{equation}
Here we made use of the following estimates: $\norm{-N w_N(x-\cdot) \nabla\ph(\cdot)}_{L^2}^2 \leq C\norm{\Delta \ph}_{L^2}$, $\norm{-N w_N(x-\cdot)\ph(\cdot)}^2 \leq C \norm{\nabla \ph}^2$  and $\int \norm{\nabla_x r_x}^2 \di x= \norm{\nabla_x r}^2 \leq C$. (The first two are consequences of lemma \ref{l:w} and Hardy's inequality.)\newline

\underline{Estimates for the non-kinetic terms:}\newline
In proving the following estimates, the important observations are
\begin{itemize}
 \item $a^*(c_x)\psi$ is ill-defined because $c_x = \delta_x + p_x$, thus each $a^\ast(c_x)$ has to be moved to the other argument of the scalar product as $a(c_x)$.
 \item for summands containing $a(c_x) a(c_y) \psi$, insert $\sqrt{\varepsilon} \frac{1}{\sqrt{\varepsilon}}$ and then use Hoelder, so that $\varepsilon\ev{\tilV}$ is obtained
 \item $a^\#(s_x)$ is always ok, and can easily be estimated using the $L^2$-Norm of $s_x$. with $p$ instead of $s$, this is okay, too.
 \item use $\sup_x \norm{s_x}^2$, $\sup_x \norm{p_x}^2$ and $\norm{\ph}_\infty$ to simplify integrals such that $\int \di y\, NV_N(x-y) = b_0$ (independent of $x$) can be used
\item always use Cauchy-Schwarz, in such away that at most two operators $a^\#$ act on each $\psi$. Usually Hoelder (with $p=q=2$) is applied to all integrals, so that we get an upper bound by a sum of two terms. In using Hoelder, we always split $V_N$ into two square roots, so that we get $V_N$ in each integral, not $V_N^2$. ($\int N V_N$ scales like $N^0$, but for other powers of $N V_N$ integrals do depend on $N$ non-trivially.)
\item for $a^\# a^\# a^\#$ and $a^\# a^\# a^\# a^\#$, we need $\Ncal^2$, so a prefactor $1/N$ is necessary. 
\end{itemize}
We define
\bd
\tilV := \frac{1}{4} \int \di x\di y\, V_N(x-y) a^\ast(c_x) a^\ast(c_y)a(c_y)a(c_x).
\ed
We now give the estimates for all the terms explicitly.
\newline\estlist{10}{1}
By definition of $\tilV$, we have
\bd
\scal{\psi}{\frac{1}{2}\int \di x\di y\, NV_N(x-y)\frac{1}{2N} a^\ast(c_x) a^\ast(c_y) a(c_y) a(c_x)\psi} = \ev{\tilV}.
\ed
\estlist{10}{2}
\begin{align*}
 & \lvert \scal{\psi}{\dxyNV \frac{1}{2N} 4 a^\ast(c_x)a^\ast(c_y) a^\ast(s_x)a(c_y)\psi}\rvert \\
\leq & \int \di x\di y\, \sqrt{V_N(x-y)}^2 \frac{\sqrt{\varepsilon}}{2} \norm{a(c_y) a(c_x)\psi} \frac{2}{\sqrt{\varepsilon}} \norm{a^\ast(s_x)a(c_y)\psi} \\
\leq & \varepsilon \frac{1}{4} \int \di x\di y\, V_N(x-y) \scal{\psi}{a^\ast(c_x)a^\ast(c_y)a(c_y)a(c_x)\psi} + \frac{4}{\varepsilon} \int \di x\di y\, V_N(x-y) \norm{a^\ast(s_x)a(c_y)\psi}^2 \\
\leq & \varepsilon \ev{\tilV} + \frac{4}{\varepsilon}\int \di x\di y\, V_N(x-y) \norm{s_x}^2 \norm{a(c_y)\Ncal^{1/2}\psi}^2 \\
\leq & \varepsilon \ev{\tilV} + \frac{4}{\varepsilon} \sup_x \norm{s_x}^2 \int \di y\, \norm{a(c_y)\Ncal^{1/2}\psi}^2 \underbrace{\int \di x\, V_N(x-y)}_{=\, b_0/N} \\
\leq & \varepsilon \ev{\tilV} + \frac{4}{\varepsilon} \sup_x \norm{s_x}^2 \frac{b_0}{N} 2 (1+\norm{p}^2)\ev{\Ncal^2} \\
\leq & \varepsilon \ev{\tilV} + C\norm{\ph}_{H^2}^2 \frac{1}{\varepsilon} \frac{1}{N}\ev{\Ncal^2}.
\end{align*}
\estlist{11}{1}
\begin{align*}
& \lvert \scal{\psi}{\dxyNV \frac{1}{2N}2a^\ast(c_x)a^\ast(c_y) a^\ast(s_y) a^\ast(s_x) \psi}\rvert \\
\leq & 2\frac{1}{4}\int \di x\di y\, V_N(x-y)  \norm{a(c_y)a(c_x)\psi} \norm{a^\ast(s_y)a^\ast(s_x)\psi} \\
\leq & \varepsilon 2 \ev{\tilV} + \frac{1}{2\varepsilon} \int \di x\di y\, V_N(x-y) \norm{a^\ast(s_y)a^\ast(s_x)\psi}^2 \\
\leq & \varepsilon 2 \ev{\tilV} + \frac{1}{2\varepsilon} \int \di x\di y\, V_N(x-y) \norm{s_y}^2 \norm{s_x}^2 \ev{(\Ncal+2)^2} \\
\leq & \varepsilon 2 \ev{\tilV} + \frac{1}{2\varepsilon} \sup_x \norm{s_x}^2 \underbrace{\int \di y\, \norm{s_y}^2 \int \di x\, V_N(x-y)}_{=\, \norm{s}^2 b_0/N} \ev{(\Ncal+2)^2} \\
\leq & \varepsilon 2\ev{\tilV} + C\norm{\ph}_{H^2}^2 \frac{1}{\varepsilon} \frac{1}{N}\ev{(\Ncal+2)^2}.
\end{align*}
\estlist{11}{2}
\begin{align*}
& \lvert \scal{\psi}{\dxyNV \frac{1}{2N} 2 a^\ast(c_x) a^\ast(s_x) a(s_y) a(c_y)\psi}\rvert \\
\leq & \frac{1}{2}\int \di x\di y\, V_N(x-y) \norm{a(s_x) a(c_x) \psi} \norm{a(s_y) a(c_y)\psi} \\
\leq & \int \di x\di y\, V_N(x-y) \norm{s_y}^2 \norm{a(c_y) \Ncal^{1/2}\psi}^2\\
\leq & \sup_y \norm{s_y}^2 \int \di y\, \norm{a(c_y)\Ncal^{1/2}\psi}^2 \int \di x\, V_N(x-y)\\
\leq & C \norm{\ph}_{H^2}^2 \frac{1}{N}\ev{\Ncal^2}.
\end{align*}
Here we made use of the fact that both summand from the Hoelder inequality are equal (just rename the integration variables $x$ as $y$ and $y$ as $x$).\newline
\estlist{12}{1}
\begin{align*}
& \lvert \scal{\psi}{\dxyNV \frac{1}{2N}2a^\ast(c_x)a^\ast(s_y)a(s_y)a(c_x)\psi}\rvert \\
\leq & \frac{1}{2}\int \di x\di y\, V_N(x-y) \norm{a(s_y)a(c_x)\psi}^2 \\
\leq & \frac{1}{2}\int \di x\di y\, V_N(x-y) \norm{s_y}^2 \norm{a(c_x)\Ncal^{1/2}\psi}^2 \\
\leq & \frac{1}{2}\sup_y \norm{s_y}^2 \int \di x\, \norm{a(c_x)\Ncal^{1/2}\psi}^2 \int \di y\,V_N(x-y) \\
\leq & \frac{1}{2}C\norm{\ph}_{H^2}^2 2(1+\norm{p}^2)\ev{\Ncal^2} \frac{b_0}{N} \\
\leq & C \norm{\ph}_{H^2}^2 \frac{1}{N} \ev{\Ncal^2}.
\end{align*}
\estlist{12}{2}
\begin{align*}
 & \lvert \scal{\psi}{\dxyNV \frac{1}{2N} 4 a^\ast(c_x) a^\ast(s_y) a^\ast(s_x) a(s_y) \psi}\rvert \\
\leq & \int \di x\di y\, V_N(x-y) \norm{a(s_y)a(c_x)\psi}\norm{a^\ast(s_x)a(s_y)\psi} \\
\leq & \int \di x\di y\, V_N(x-y) \left( \norm{s_y}^2 \norm{a(c_x)\Ncal^{1/2}\psi}^2 + \norm{s_x}^2 \norm{a(s_y)\Ncal^{1/2}\psi}^{1/2}\right) \\
\leq & C \norm{\ph}_{H^2}^2 2(1+\norm{p}^2)\frac{b_0}{N}\ev{\Ncal^2} + C\norm{\ph}_{H^2}^2 \norm{s}^2 \frac{b_0}{N} \ev{\Ncal^2} \\
\leq & C\norm{\ph}_{H^2}^2 \frac{1}{N}\ev{\Ncal^2}.
\end{align*}
\estlist{13}{1}
\begin{align*}
& \lvert \scal{\psi}{\dxyNV \frac{1}{2N} a^\ast(s_y) a^\ast(s_x) a(s_x)a(s_y)\psi} \rvert \\
= & \frac{1}{4} \int \di x\di y\, V_N(x-y) \norm{a(s_x)a(s_y)\psi}^2 \\
\leq & \frac{1}{4} \int \di x\di y\, V_N(x-y) \norm{s_x}^2 \norm{s_y}^2 \norm{\Ncal\psi}^2 \\
\leq & \frac{1}{4} \sup_x \norm{s_x}^2 \int \di y\, \norm{s_y}^2 \int \di x\, V_N(x-y) \ev{\Ncal^2}\\
\leq & C\norm{\ph}_{H^2}^2 \frac{1}{N}\ev{\Ncal^2}.
\end{align*}
\estlist{14}{1} (here, cancellation is important!)
\begin{align*}
& \lvert \scal{\psi}{\dxyNV \frac{1}{N}a^\ast(c_x) a^\ast(c_y) \left( r(y,x)+\scal{p_y}{s_x} \right)\psi} \rvert \\
\leq & \varepsilon \dxyV \norm{a(c_y)a(c_x)\psi}^2 + \frac{1}{\varepsilon} \dxyV \lvert r(y,x)+\scal{p_y}{s_x} \rvert^2 \\
\leq & \varepsilon 2\ev{\tilV} + \frac{1}{\varepsilon} \int \di x\di y\, V_N(x-y) \left( C\lvert\ph(x)\rvert^2\lvert\ph(y)\rvert^2 +\norm{p_y}^2\norm{s_x}^2 \right) \\
\leq & \varepsilon 2\ev{\tilV} + \frac{C}{\varepsilon} \norm{\ph}_\infty^2 \int \di x \di y\, \lvert \ph(y)\rvert^2 V_N(x-y) + \frac{1}{\varepsilon} \sup_x\norm{s_x}^2 \int \di x\di y\, \norm{p_y}^2 V_N(x-y) \\
\leq & \varepsilon 2\ev{\tilV} + C\norm{\ph}_{H^2}^2 \frac{1}{\varepsilon N}.
\end{align*}
\estlist{14}{2}
\begin{align*}
& \lvert \scal{\psi}{\dxyNV \frac{1}{N} a^\ast(c_x) a^\ast(p_y) k(y,x)\psi} \rvert \\
\leq & \dxyV \norm{a(c_x)\psi} \norm{a^\ast(p_y)\psi} \underbrace{\lvert k(y,x)\rvert}_{\leq\, N\lvert \ph(x)\rvert\lvert \ph(y)\rvert} \\
\leq & \dxyNV \norm{a(c_x)\psi}^2 \lvert\ph(x)\rvert^2 + \dxyNV \norm{a^\ast(p_y)\psi}^2 \lvert \ph(y)\rvert^2\\
\leq & \frac{1}{2} \norm{\ph}_\infty^2 \left( b_0 \int \di x\,\norm{a(c_x)\psi}^2 + b_0 \int \di y\,\norm{a^\ast(p_y)\psi}^2 \right) \\
\leq & C \norm{\ph}_{H^2}^2 \ev{\Ncal+1}.
\end{align*}
\estlist{15}{1}
\begin{align*}
& \lvert \scal{\psi}{\dxyNV \frac{1}{N}a^\ast(c_x)a(c_y) \scal{s_y}{s_x}\psi} \rvert \\
\leq & \dxyV \lvert \scal{s_y}{s_x}\rvert \norm{a(c_x)\psi} \norm{a(c_y)\psi} \\
\leq & \dxyV \norm{s_y} \norm{s_x} \norm{a(c_x)\psi} \norm{a(c_y)\psi} \\
\leq & \int \di x \di y\, V_N(x-y) \norm{s_y}^2 \norm{a(c_y)\psi}^2 \\
%\leq & \frac{b_0}{N} \sup_y \norm{s_y}^2 \int \di x \norm{a(c_x)\psi}^2 \\
\leq & C \norm{\ph}_{H^2}^2 \frac{1}{N}\ev{\Ncal}. 
\end{align*}
\estlist{15}{2}
\begin{align*}
& \lvert \scal{\psi}{\dxyNV \frac{1}{N}a^\ast(s_y)a(s_y)\scal{s_x}{s_x}\psi} \rvert \\
\leq & \dxyV \norm{s_x}^2 \norm{a(s_y)\psi}^2 \\
%\leq & \sup_x \norm{s_x}^2 \frac{1}{2}\int \di y\, \norm{a(s_y)\psi}^2 \int \di x\, V_N(x-y) \\
%\leq & C \norm{\ph}_{H^2}^2 \frac{b_0}{2} \norm{s}^2 \frac{1}{N}\ev{\Ncal}\\
\leq & C \norm{\ph}_{H^2}^2 \frac{1}{N}\ev{\Ncal}.
\end{align*}
\estlist{15}{3}
\begin{align*}
& \lvert \scal{\psi}{\dxyNV \frac{1}{N} a^\ast(s_y) a(s_x) \scal{s_y}{s_x}\psi} \rvert \\
\leq & \dxyV \norm{s_x} \norm{s_y} \norm{a(s_y)\psi} \norm{a(s_x)\psi} \\
\leq & \dxyV \norm{s_x}^2 \norm{s_y}^2 \norm{\Ncal^{1/2}\psi}^2 \\
\leq & C\norm{\ph}_{H^2}^2 \frac{1}{N}\ev{\Ncal}.
\end{align*}
\estlist{16}{1}
\begin{align*}
& \lvert \scal{\psi}{\dxyNV \frac{1}{N}a^\ast(c_x) a(s_y) \scal{c_y}{s_x}\psi} \rvert \\
\leq & \dxyV \norm{a(c_x)\psi} \norm{a(s_y)\psi} \lvert s(y,x)+\scal{p_y}{s_x} \rvert \\
\leq & \frac{C}{2}\int \di x\di y\, N V_N(x-y) \norm{a(c_x)\psi}^2 \lvert \ph(x)\rvert^2 + \frac{C}{2}\int \di x\di y\, N V_N(x-y) \norm{a(s_y)\psi}^2 \lvert \ph(y)\rvert^2\\
& \quad + \dxyV \norm{a(c_x)\psi}^2 \norm{s_x}^2 + \dxyV \norm{a(s_y)\psi}^2 \norm{p_y}^2 \\
\leq & \frac{C}{2} \norm{\ph}_\infty^2 \int \di x\, \norm{a(c_x)\psi}^2 \int \di y\, NV_N(x-y) + \frac{C}{2} \norm{\ph}_\infty^2 \int \di y\, \norm{a(s_y)\psi}^2 b_0 \\
& \quad + \frac{1}{2}\sup_x\norm{s_x}^2 \int \di x\, \norm{a(c_x)\psi}^2 \frac{b_0}{N} + \frac{1}{2} \sup_y \norm{p_y}^2 \int \di y\, \norm{a(s_y)\psi}^2 b_0/N \\
\leq & \frac{C b_0}{2} \norm{\ph}_{H^2}^2(1+\frac{1}{N})\left( \int \di x\,\norm{a(c_x)\psi}^2 + \int \di y\,\norm{a(s_y)\psi}^2 \right) \\
\leq & C\norm{\ph}_{H^2}^2 (1+\frac{1}{N})\ev{\Ncal}.
\end{align*}
\estlist{16}{2}
\begin{align*}
& \lvert \scal{\psi}{\dxyNV \frac{1}{N} a^\ast(c_x) a(c_x)\scal{s_y}{s_y}\psi} \rvert \\
\leq & \dxyV \norm{s_y}^2 \norm{a(c_x)\psi}^2 \\
\leq & \frac{1}{2} \sup_y \norm{s_y}^2 \int \di x\, \norm{a(c_x)\psi}^2 \int \di y\, V_N(x-y) \\
\leq & C\norm{\ph}_{H^2}^2 \frac{1}{N}\ev{\Ncal}.
\end{align*}
\estlist{16}{3}
\begin{align*}
 & \lvert \scal{\psi}{\dxyNV \frac{1}{N}a^\ast(s_y)a^\ast(s_x)\scal{s_x}{c_y}\psi} \rvert \\
\leq & \dxyV \norm{a(s_y)\psi} \norm{a^\ast(s_x)\psi} \lvert\scal{s_x}{c_y}\rvert \\
\leq & \frac{C}{2}\int \di x\di y\, N V_N(x-y) \norm{a(s_y)\psi}^2 \lvert\ph(y)\rvert^2 + \frac{C}{2}\int \di x\di y\, N V_N(x-y) \norm{a^\ast(s_x)\psi}^2 \lvert \ph(x)\rvert^2 \\
& \quad + \dxyV \norm{a(s_y)\psi}^2 \norm{p_y}^2 + \dxyV \norm{a^\ast(s_x)\psi}^2 \norm{s_x}^2 \\
\leq & C\norm{\ph}_\infty^2\frac{b_0}{2} \int \di y\, \norm{s_y}^2 \ev{\Ncal} + C \norm{\ph}_\infty^2 \frac{b_0}{2} \int \di x\, \norm{s_x}^2 \ev{\Ncal+1} \\
& \quad + \frac{1}{2} \sup_y \norm{p_y}^2 \frac{b_0}{N} \int \di y\, \norm{s_y}^2 \ev{\Ncal} + \frac{1}{2} \sup_x \norm{s_x}^2 \frac{b_0}{N} \int \di x\, \norm{s_x}^2 \ev{\Ncal+1}\\
\leq & C \norm{\ph}_{H^2}^2 (1+\frac{1}{N})\ev{\Ncal+1}.
\end{align*}
\estlist{17}{1}
\begin{align*}
 & \lvert \scal{\psi}{\dxyNV \frac{1}{N}2 a^\ast(c_x)a^\ast(s_x) \scal{s_y}{s_y}\psi} \rvert \\
\leq & \int \di x\di y\, V_N(x-y) \norm{s_y}^2 \lvert \scal{a(c_x)\psi}{a^\ast(s_x)\psi}\rvert \\
\leq & \sup_y \norm{s_y}^2 \int \di x\, \frac{b_0}{N} \norm{a(c_x)\psi}^2 + \sup_y \norm{s_y}^2 \int \di x\, \frac{b_0}{N} \norm{a^\ast(s_x)\psi}^2 \\
\leq & C \norm{\ph}_{H^2}^2 \frac{1}{N}\ev{\Ncal+1}.
\end{align*}
\estlist{17}{2}
\begin{align*}
& \lvert \scal{\psi}{\dxyNV \frac{1}{N} 2 a^\ast(c_x) a^\ast(s_y) \scal{s_y}{s_x} \psi} \rvert \\
\leq & \int \di x\di y\, V_N(x-y) \norm{s_x}^2 \norm{a(c_x)\psi}^2 + 
\int \di x\di y\, V_N(x-y) \norm{s_y}^2 \norm{a^\ast(s_y)\psi}^2 \\
\leq & C \norm{\ph}_{H^2}^2 \frac{1}{N}\ev{\Ncal+1}.
\end{align*}
\estlist{17}{3}
\begin{align*}
& \lvert \scal{\psi}{\dxyNV \frac{1}{N} a^\ast(c_y) a(s_x) \scal{c_y}{s_x} \psi} \rvert \\
\leq & \dxyV \lvert \scal{c_y}{s_x} \rvert \norm{a(c_y)\psi} \norm{a(s_x)\psi} \\
\leq & \frac{C}{2}\int \di x\di y\, N V_N(x-y) \norm{a(c_y)\psi}^2 \lvert\ph(y)\rvert^2 + \frac{C}{2}\int \di x\di y\, N V_N(x-y) \norm{a(s_x)\psi}^2 \lvert \ph(x)\rvert^2 \\
& \quad + \dxyV \norm{a(c_y)\psi}^2 \norm{p_y}^2 + \dxyV \norm{a(s_x)\psi}^2 \norm{s_x}^2 \\
\leq & C \norm{\ph}_{H^2}^2 (1+\frac{1}{N})\ev{\Ncal}.
\end{align*}
\estlist{18}{1} (here, cancellation is important!)
\begin{align*}
& \lvert \scal{\psi}{\dxyNV \ph(x)\ph(y) a^\ast(c_x) a^\ast(p_y)\psi} \rvert \\
\leq & \dxyNV \lvert\ph(x)\rvert \lvert\ph(y)\rvert \norm{a(c_x)\psi} \norm{a^\ast(p_y)\psi}\\
\leq & \dxyNV \lvert \ph(x)\rvert^2 \norm{a(c_x)\psi}^2 + \dxyNV \lvert\ph(y)\rvert^2 \norm{a^\ast(p_y)\psi}^2 \\
\leq & C\norm{\ph}_{H^2}^2 \ev{\Ncal+1}. 
\end{align*}
\estlist{18}{2} like the previous estimate
\begin{align*}
& \lvert \scal{\psi}{\dxyNV \ph(x)\ph(y) 2a^\ast(c_x)a(s_y)\psi} \rvert \\
\leq & \dxyNV \lvert\ph(x)\rvert^2 \norm{a(c_x)\psi}^2 + \dxyNV \lvert \ph(y)\rvert^2 \norm{a(s_y)\psi}^2 \\
\leq & C\norm{\ph}_{H^2}^2 \ev{\Ncal}.
\end{align*}
\estlist{18}{3}
\begin{align*}
 & \lvert \scal{\psi}{\dxyNV \ph(x)\ph(y) a(s_x) a(s_y)\psi} \rvert \\
\leq & \dxyNV \lvert \ph(x)\rvert^2 \norm{a^\ast(s_x)\psi}^2 + \dxyNV \lvert \ph(y)\rvert^2 \norm{a(s_y)\psi}^2 \\
\leq & C\norm{\ph}_{H^2}^2 \ev{\Ncal+1}
\end{align*}
\estlist{19}{1}
\begin{align*}
 & \lvert \scal{\psi}{\dxyNV \ph(x) \cc{\ph(y)} a^\ast(c_x)a(c_y)\psi} \rvert \\
\leq & \dxyNV \lvert \ph(x)\rvert \lvert\ph(y)\rvert \norm{a(c_x)\psi} \norm{a(c_y)\psi} \\
\leq & \int \di x \di y\, NV_N(x-y) \lvert \ph(x)\rvert^2 \norm{a(c_x)\psi}^2 \\
\leq & C \norm{\ph}_{H^2}^2 \ev{\Ncal}.
\end{align*}
\estlist{19}{2}
\begin{align*}
 & \lvert \scal{\psi}{\dxyNV \ph(x)\cc{\ph(y)} 2a^\ast(c_x) a^\ast(s_y)\psi} \rvert \\
\leq & \int \di x\di y\, NV_N(x-y) \lvert \ph(x)\rvert \lvert \ph(y)\rvert \norm{a(c_x)\psi} \norm{a^\ast(s_y)\psi}\\
\leq & \int \di x\di y\, NV_N(x-y) \lvert \ph(x)\rvert^2 \norm{a(c_x)\psi}^2 + \int \di x\di y\, NV_N(x-y) \lvert\ph(y)\rvert^2 \norm{a^\ast(s_y)\psi}^2 \\
\leq & \norm{\ph}_\infty^2 b_0 C \ev{\Ncal+1} \\
\leq & C\norm{\ph}_{H^2}^2 \ev{\Ncal+1}.
\end{align*}
\estlist{19}{3}
\begin{align*}
& \lvert \scal{\psi}{\dxyNV \ph(x)\cc{\ph(y)}a^\ast(s_y)a(s_x)\psi}\rvert \\
\leq & \int \di x \di y\, N V_N(x-y) \lvert \ph(x)\rvert^2 \norm{a(s_x)\psi}^2 \\
% \leq & b_0 \norm{\ph}_\infty^2 \norm{s}^2 \ev{\Ncal}\\
\leq & C\norm{\ph}_{H^2}^2 \ev{\Ncal}.
\end{align*}
\estlist{20}{1}
\begin{align*}
 & \lvert \scal{\psi}{\dxyNV \lvert \ph(y)\rvert^2 a^\ast(c_x)a(c_x)\psi} \rvert \\
\leq & \dxyNV \lvert\ph(y)\rvert^2 \norm{a(c_x)\psi}^2 \\
\leq & C \norm{\ph}_{H^2}^2 \ev{\Ncal}.
\end{align*}
\estlist{20}{2}
\begin{align*}
& \lvert \scal{\psi}{\dxyNV \lvert\ph(y)\rvert^2 2 a^\ast(c_x) a^\ast(s_x)\psi} \rvert \\
\leq & \int \di x\di y\, NV_N(x-y) \lvert\ph(y)\rvert^2 \norm{a(c_x)\psi} \norm{a^\ast(s_x)\psi} \\
\leq & \norm{\ph}_\infty^2 \left( \int \di x\, \norm{a(c_x)\psi}^2 + \int \di x \norm{a^\ast(s_x)\psi}^2\right) b_0 \\
\leq & C\norm{\ph}_{H^2}^2 \ev{\Ncal+1}.
\end{align*}
\estlist{20}{3}
\begin{align*}
 & \lvert \scal{\psi}{\dxyNV \lvert \ph(y)\rvert^2 a^\ast(s_x) a(s_x)\psi} \rvert \\
\leq & \dxyNV \lvert\ph(y)\rvert^2 \norm{a(s_x)\psi}^2\\
%\leq & \frac{1}{2} \norm{\ph}_\infty^2 b_0 \norm{s}^2 \ev{\Ncal} \\
\leq & C\norm{\ph}_{H^2}^2 \ev{\Ncal}.
\end{align*}
\estlist{21}{1}
\begin{align*}
 & \lvert \scal{\psi}{\dxyNV \frac{2}{\sqrt{N}}\cc{\ph(y)}a^\ast(c_x)a^\ast(s_x)a^\ast(s_y)\psi} \rvert \\
\leq & \int \di x\di y\, NV_N(x-y) \lvert\ph(y)\rvert \norm{a(c_x)\psi}\frac{1}{\sqrt{N}}\norm{a^\ast(s_x)a^\ast(s_y)\psi} \\
\leq & \norm{\ph}_\infty^2 \int \di x\di y\, NV_N(x-y) \norm{a(c_x)\psi}^2 + \int \di x\di y\, NV_N(x-y) \frac{1}{N} \norm{a^\ast(s_x)a^\ast(s_y)\psi}^2 \\
\leq & \norm{\ph}_{H^2}^2 C \ev{\Ncal} + \frac{1}{N}\ev{(\Ncal+2)^2} \int \di x\di y\, NV_N(x-y) \norm{s_x}^2 \norm{s_y}^2 \\
\leq & C\norm{\ph}_{H^2}^2 \left( \ev{\Ncal} + \frac{1}{N}\ev{(\Ncal+2)^2} \right).
\end{align*}
\estlist{21}{2} like previous term
\begin{align*}
 & \lvert \scal{\psi}{\dxyNV \frac{2}{\sqrt{N}}\cc{\ph(y)} a^\ast(c_x) a^\ast(s_x) a(c_y)\psi } \rvert \\
\leq & \norm{\ph}_\infty^2 \int \di x\di y\, N V_N(x-y) \norm{a(c_x)\psi}^2 + \int \di x\di y\, NV_N(x-y) \frac{1}{N} \norm{a^\ast(s_x)a(c_y)\psi}^2 \\
\leq & \norm{\ph}_{H^2}^2 C \ev{\Ncal} + \frac{1}{N} \sup_x \norm{s_x}^2 \int \di x\di y\, NV_N(x-y) \norm{a(c_y)\Ncal^{1/2}\psi}^2 \\
\leq & C\norm{\ph}_{H^2}^2 \left( \ev{\Ncal} + \frac{1}{N}\ev{\Ncal^2} \right).
\end{align*}
\estlist{22}{1} like previous term
\begin{align*}
& \lvert \scal{\psi}{\dxyNV \frac{2}{\sqrt{N}}\cc{\ph(y)} a^\ast(c_x) a^\ast(s_y) a(c_x)\psi } \rvert \\
\leq & \norm{\ph}_\infty^2 C \ev{\Ncal} + \frac{1}{N} \int \di x\di y\, NV_N(x-y) \norm{a^\ast(s_y)a(c_x)\psi}^2 \\
\leq & \norm{\ph}_\infty^2 C \ev{\Ncal} + \frac{1}{N}\sup_y \norm{s_y}^2 b_0 2(1+\norm{p}^2) \ev{\Ncal^2} \\
\leq & C\norm{\ph}_{H^2}^2 \left( \ev{\Ncal} + \frac{1}{N}\ev{\Ncal^2} \right).
\end{align*}
\estlist{22}{2} insert $\sqrt{\varepsilon} \frac{1}{\sqrt{\varepsilon}}$, use Hoelder and $\tilV$:
\begin{align*}
 & \lvert \scal{\psi}{\dxyNV \frac{2}{\sqrt{N}}\cc{\ph(y)} a^\ast(c_x)a(c_x)a(c_y)\psi} \rvert \\
\leq & \int \di x\di y\, NV_N(x-y) \lvert \ph(y)\rvert \frac{1}{\sqrt{\varepsilon}} \norm{a(c_x)\psi} \norm{a(c_x)a(c_y)\psi} \frac{1}{\sqrt{N}}\sqrt{\varepsilon} \\
\leq & \frac{1}{\varepsilon}\int \di x\di y\, NV_N(x-y) \lvert \ph(y)\rvert^2 \norm{a(c_x)\psi}^2 + \varepsilon \int \di x\di y\, V_N(x-y) \scal{\psi}{a^\ast(c_y)a^\ast(c_x)a(c_x)a(c_y) \psi} \\
\leq & C\norm{\ph}_{H^2}^2 \frac{1}{\varepsilon}\ev{\Ncal} + \varepsilon 4\ev{\tilV}.
\end{align*}
\estlist{22}{3}
\begin{align*}
& \lvert \scal{\psi}{\dxyNV \frac{2}{\sqrt{N}} \cc{\ph(y)}a^\ast(s_y)a(s_x)a(c_x)\psi} \rvert \\
\leq & \int \di x\di y\, NV_N(x-y) \lvert \ph(y)\rvert \norm{a(s_y)\psi} \norm{a(s_x)a(c_x)\psi}\frac{1}{\sqrt{N}} \\
\leq & \int \di x \di y\, NV_N(x-y) \lvert\ph(y)\rvert^2 \norm{a(s_y)\psi}^2 + \int \di x\di y\, NV_N(x-y) \norm{a(s_x)a(c_x)\psi}^2 \frac{1}{N} \\
\leq & \norm{\ph}_\infty^2 b_0 \norm{s}^2 \ev{\Ncal} + \frac{1}{N}b_0 \sup_x\norm{s_x}^2 \int \di x\, \norm{a(c_x)\Ncal^{1/2}\psi}^2 \\
\leq & C \norm{\ph}_{H^2}^2 \left( \ev{\Ncal} + \frac{1}{N}\ev{\Ncal^2}\right). 
\end{align*}
\estlist{23}{1}
\begin{align*}
& \lvert \scal{\psi}{\dxyNV \frac{2}{\sqrt{N}}\cc{\ph(y)} a^\ast(s_x) a^\ast(s_y)a(s_x)\psi} \rvert \\
\leq & \int \di x\di y NV_N(x-y) \lvert \ph(y)\rvert \norm{a(s_x)\psi} \frac{1}{\sqrt{N}}\norm{a^\ast(s_y)a(s_x)\psi} \\
\leq & C \norm{\ph}_{H^2}^2 \left( \ev{\Ncal} + \frac{1}{N}\ev{\Ncal^2} \right).
\end{align*}
\estlist{23}{2}
\begin{align*}
 & \lvert \scal{\psi}{\dxyNV \frac{2}{\sqrt{N}} \cc{\ph(y)} a^\ast(s_x) a(s_x) a(c_y) \psi}\rvert \\
\leq & \norm{\ph}_\infty^2 \int \di x\di y\, NV_N(x-y) \norm{a(s_x)\psi}^2 + \int \di x \di y\, V_N(x-y) \norm{a(s_x)a(c_y)\psi}^2 \\
\leq & C \norm{\ph}_{H^2}^2 \ev{\Ncal} + \sup_x \norm{s_x}^2 \frac{b_0}{N} \int \di y\, \norm{a(c_y)\Ncal^{1/2}\psi}^2\\
\leq & C \norm{\ph}_{H^2}^2 \left( \ev{\Ncal} + \frac{1}{N}\ev{\Ncal^2} \right). 
\end{align*}
\estlist{23}{3} insert $\sqrt{\varepsilon} \frac{1}{\sqrt{\varepsilon}}$, use Hoelder and $\tilV$:
\begin{align*}
& \lvert \scal{\psi}{\dxyNV \frac{2}{\sqrt{N}} \cc{\ph(y)} a(s_x) a(c_x) a(c_y)\psi} \rvert \\
\leq & \int \di x\di y\, NV_N(x-y) \lvert \ph(y)\rvert \frac{1}{\sqrt{\varepsilon}} \norm{a^\ast(s_x)\psi} \norm{a(c_x)a(c_y)\psi}\frac{1}{\sqrt{N}}\sqrt{\varepsilon} \\
\leq & \int \di x\di y\, NV_N(x-y) \lvert \ph(y)\rvert^2 \norm{a^\ast(s_x)\psi}^2\frac{1}{\varepsilon} + \int \di x\di y\, NV_N(x-y) \frac{1}{N} \scal{\psi}{a^\ast(c_y)a^\ast(c_x)a(c_x)a(c_y)\psi}\varepsilon \\
\leq & \frac{1}{\varepsilon} \norm{\ph}_\infty^2 b_0 \int \di x\, \norm{s_x}^2 \ev{\Ncal+1} + \varepsilon 4 \ev{\tilV}\\
\leq & \varepsilon 4 \ev{\tilV} + C\norm{\ph}_{H^2}^2 \frac{1}{\varepsilon}\ev{\Ncal+1}.
\end{align*}
\estlist{24}{1}
\begin{align*}
 & \lvert \scal{\psi}{\dxyNV \frac{2}{\sqrt{N}}\cc{\ph(y)}a^\ast(s_x)\scal{s_x}{s_y}\psi} \rvert \\
\leq & \int \di x\di y\, NV_N(x-y) \lvert \ph(y)\rvert \frac{1}{\sqrt{N}} \norm{s_x} \norm{s_y} \norm{\psi} \norm{a(s_x)\psi} \\
\leq & \int \di x\di y\, NV_N(x-y) \lvert \ph(y)\rvert \frac{1}{\sqrt{N}} \norm{s_x}^2 \norm{s_y} \norm{\Ncal^{1/2}\psi} \norm{\psi} \\
\leq & \sup_x \norm{s_x}^2 \frac{1}{\sqrt{N}} \int \di x\di y\, NV_N(x-y) \lvert \ph(y)\rvert \norm{s_y} \ev{\Ncal+1} \\
\leq & \sup_x \norm{s_x}^2 \frac{1}{\sqrt{N}} b_0 \ev{\Ncal+1} \left( \int \di y\, \lvert \ph(y)\rvert^2 \right)^{1/2} \left( \int \di y\, \norm{s_y}^2 \right)^{1/2} \\
\leq & C\norm{\ph}_{H^2}^2 \frac{1}{\sqrt{N}}\ev{\Ncal+1}.
\end{align*}
\estlist{24}{2} like previous estimate
\begin{align*}
& \lvert \scal{\psi}{\dxyNV \frac{2}{\sqrt{N}}\cc{\ph(y)} a^\ast(s_y) \scal{s_x}{s_x} \psi}\rvert \\
\leq &\int \di x\di y\, N V_N(x-y) \lvert \ph(y)\rvert \norm{s_x}^2 \norm{s_y} \frac{1}{\sqrt{N}}\ev{\Ncal+1} \\
\leq & \sup_x \norm{s_x}^2 \frac{1}{\sqrt{N}} b_0 \norm{\ph}_{L^2} \norm{s} \ev{\Ncal+1} \\
\leq & C\norm{\ph}_{H^2}^2 \frac{1}{\sqrt{N}}\ev{\Ncal+1}.
\end{align*}
\estlist{24}{3}
\begin{align*}
& \lvert \scal{\psi}{\dxyNV \frac{2}{\sqrt{N}}\cc{\ph(y)} a(c_y) \scal{s_x}{s_x} \psi}\rvert \\
\leq & \int \di x\di y\, NV_N(x-y) \lvert\ph(y)\rvert \norm{s_x}^2 \frac{1}{\sqrt{N}} \norm{\psi} \norm{a(c_y)\psi} \\
\leq & \sup_x \norm{s_x}^2 b_0 \frac{1}{\sqrt{N}} \norm{\psi} \norm{\ph}_{L^2} \left( \int \di y \norm{a(c_y)\psi}^2 \right)^{1/2}\\
\leq & C\norm{\ph}_{H^2}^2 \frac{1}{\sqrt{N}} \ev{\Ncal+1}.  
\end{align*}
\estlist{24}{4} like previous estimate
\begin{align*}
& \lvert \scal{\psi}{\dxyNV \frac{2}{\sqrt{N}} \cc{\ph(y)} a(c_x)\scal{s_x}{s_y}\psi} \rvert \\
\leq & \int \di x\di y\, NV_N(x-y) \lvert \ph(y)\rvert \norm{s_x} \norm{s_y} \frac{1}{\sqrt{N}} \norm{\psi} \norm{a(c_x)\psi} \\
\leq & \left( \int \di x\di y\, NV_N(x-y) \norm{a(c_x)\psi}^2 \norm{s_x}^2 \right)^{1/2} \left( \int \di x\di y\, N V_N(x-y) \norm{s_y}^2 \lvert \ph(y)\rvert^2 \right)^{1/2} \frac{\norm{\psi}}{\sqrt{N}} \\
\leq & C\norm{\ph}_{H^2}^2 \frac{1}{\sqrt{N}}\ev{\Ncal+1}. 
\end{align*}
\estlist{25}{1}
\begin{align*}
 & \lvert \scal{\psi}{\dxyNV \frac{2}{\sqrt{N}} \cc{\ph(y)} a^\ast(c_x) r(x,y) \psi} \rvert \\
\leq & \int \di x \di y\,N V_N(x-y) \frac{1}{\sqrt{N}} \lvert\ph(y)\rvert \norm{a(c_x)\psi} C\lvert\ph(x)\rvert\lvert\ph(y)\rvert \norm{\psi} \\
\leq & C \frac{1}{\sqrt{N}} \sup_y \lvert \ph(y)\rvert^2 \norm{\psi} \int \di x\di y\, NV_N(x-y) \lvert \ph(x)\rvert \norm{a(c_x)\psi} \\
\leq & C \frac{1}{\sqrt{N}} \norm{\ph}_\infty^2 \norm{\psi} b_0 \left( \int \di x\, \lvert \ph(x)\rvert^2 \right)^{1/2} \left( \int \di x\, \norm{a(c_x)\psi}^2 \right)^{1/2} \\
\leq & C \norm{\ph}_{H^2}^2 \frac{1}{\sqrt{N}}\ev{\Ncal+1}.
\end{align*}
\estlist{25}{2} similar to the previous estimate, just use $\scal{p_x}{s_y} \leq \norm{p_x}\norm{s_y}$ instead of $\lvert r(x,y)\rvert \leq C\lvert\ph(x)\rvert\lvert \ph(y)\rvert$ to get
\begin{align*}
 & \lvert \scal{\psi}{\dxyNV \frac{2}{\sqrt{N}}\cc{\ph(y)} a^\ast(c_x) \scal{p_x}{s_y}\psi} \rvert \\
\leq & C \norm{\ph}_{H^2}^2 \frac{1}{\sqrt{N}} \ev{\Ncal+1}.
\end{align*}
\estlist{25}{3} similar to the previous two estimates:
\begin{align*}
 & \lvert \scal{\psi}{\dxyNV \frac{2}{\sqrt{N}}\cc{\ph(y)}a(s_x)r(x,y)\psi} \rvert \\
\leq & C\norm{\ph}_{H^2}^2 \frac{1}{\sqrt{N}} \ev{\Ncal+1}.
\end{align*}
\estlist{25}{4} similar to the previous three estimates:
\begin{align*}
& \lvert \scal{\psi}{\dxyNV \frac{2}{\sqrt{N}}\cc{\ph(y)} a(s_x) \scal{p_x}{s_y} \psi} \rvert \\
\leq & C\norm{\ph}_{H^2}^2 \frac{1}{\sqrt{N}}\ev{\Ncal+1}. 
\end{align*}

%%%%%%%%%%%%%%%%%%%%%%%%%%%%%%%%%%%%%%%%%

We still have to give a bound for the $\ev{(\partial_t T^*_t)T_t}$ part of the generator. We start with a number of simple lemmata.
\begin{lem}
\label{lm:Bbound}
Let $f_1, f_2 \in L^2(\Rbb^3\times \Rbb^3)$. Then
 \bd
\lvert \scal{\psi}{\frac{1}{2} \int \di x\di y \left( f_1(x,y) a^\ast_x a^\ast_y + f_2(x,y) a_x a_y \right) \psi} \rvert \leq \ev{\Ncal+1} \frac{\norm{f_1}+\norm{f_2}}{2}.
\ed
and
 \bd
\lvert \scal{\psi}{\frac{1}{2} \int \di x\di y \left( f_1(x,y) a^\ast_x a_y + f_2(x,y) a_x a^\ast_y \right)  \psi} \rvert \leq \ev{\Ncal} \frac{\norm{f_1}+\norm{f_2}}{2} + \frac{1}{2} \left\lvert \int f_2(x,x) \di x \right\rvert \scal{\psi}{\psi}.
\ed
\end{lem}
\begin{proof} Proof of the first estimate, using Hoelder's inequality:
 \begin{align*}
  & \lvert \scal{\psi}{\frac{1}{2} \int \di x\di y \left( f_1(x,y) a^\ast_x a^\ast_y + f_2(x,y) a_x a_y \right) \psi} \rvert \\
& \leq \frac{1}{2} \sum_{i=1}^2 \lvert \scal{\psi}{\int \di x\di y\, f_i(x,y) a^\ast_x a^\ast_y \psi }\rvert \\
& \leq \frac{1}{2} \sum_{i=1}^2 \int \di y \norm{a_y \psi} \norm{a^\ast(f_i(\cdot,y))\psi} \\
& \leq \frac{1}{2} \sum_{i=1}^2 \left( \int \di y_1\, \norm{a_{y_1}\psi}^2 \int \di y_2\, \norm{f_i(\cdot,y_2)}^2 \norm{(\Ncal+1)^{1/2} \psi}^2 \right)^{1/2} \\
& \leq \ev{\Ncal+1} \frac{\norm{f_1}+\norm{f_2}}{2}.
 \end{align*}
Proof of the second estimate, using the CCR and Hoelder's inequality:
 \begin{align*}
& \lvert \scal{\psi}{\frac{1}{2} \int \di x\di y \left( f_1(x,y) a^\ast_x a_y + f_2(x,y) a_x a^\ast_y \right)  \psi} \rvert\\
& \leq \frac{1}{2} \lvert \scal{\psi}{\int \di x\di y\, f_1(x,y) a^\ast_x a_y \psi} \rvert + \frac{1}{2}\lvert \scal{\psi}{\int \di x \di y\, f_2(x,y) a^\ast_y a_x \psi}\rvert \\
&\quad + \frac{1}{2} \lvert \scal{\psi}{\int \di x \di y\, f_2(x,y) \delta(x-y) \psi}\rvert \\
& \leq \frac{1}{2} \int \di x\, \norm{a_x \psi} \norm{a(f_1(x,\cdot))\psi} + \frac{1}{2} \int \di y\, \norm{a_y\psi} \norm{a(f_2(\cdot,y))\psi} + \frac{1}{2} \lvert \int f_2(x,x) \di x \rvert \scal{\psi}{\psi} \\
& \leq \frac{1}{2} \left( \int \di x_1 \norm{a_{x_1}\psi}^2 \int \di x_2 \norm{f_1(x_2,\cdot)}^2 \norm{\Ncal^{1/2}\psi}^2 \right)^{1/2}\\
& \quad + \frac{1}{2} \left( \int \di x_1 \norm{a_{x_1}\psi}^2 \int \di x_2 \norm{f_2(\cdot,x_2)}^2 \norm{\Ncal^{1/2}\psi}^2 \right)^{1/2} + \frac{1}{2}\lvert \int f_2(x,x) \di x\rvert \scal{\psi}{\psi} \\
& \leq \frac{1}{2}\ev{\Ncal} \left(\left(\int \di x \norm{f_1(x,\cdot)}^2 \right)^{1/2} + \left(\int \di x \norm{f_2(\cdot,x)}^2 \right)^{1/2} \right) + \frac{1}{2}\lvert \int f_2(x,x) \di x\rvert \scal{\psi}{\psi}. 	\qedhere
 \end{align*}
\end{proof}

\begin{lem}
\label{lm:highercommutators}
 For each $n \in \Nbb$ and each $i \in \{1,2\}$, there exists $f_{n,i} \in L^2(\Rbb^3 \times \Rbb^3)$ such that
\begin{itemize}
 \item \underline{for $n$ even we have}
\bd
\ad^n_B(\dot B) = \frac{1}{2} \int \di x\di y\left( f_{n,1}(x,y) a^\ast_y a^\ast_x + f_{n,2}(x,y) a_x a_y \right)
\ed
and $\norm{f_{n,i}} \leq \norm{\dot k} (2\norm{k})^n$ for all $n \in \Nbb$ and $i \in \{1,2\}$. 
 \item \underline{for $n$ odd we have}
\bd
\ad^n_B(\dot B) = \frac{1}{2} \int \di x\di y\left( f_{n,1}(x,y) a^\ast_x a_y + f_{n,2}(x,y) a_x a^\ast_y \right)
\ed
and the following bounds hold: $\norm{f_{n,i}} \leq \norm{\dot k} (2\norm{k})^n$ and $\int \lvert f_{n,i}(x,x)\rvert \di x \leq \norm{\dot k} (2\norm{k})^n$ for all $n \in \Nbb$ and $i \in \{1,2\}$. 
\end{itemize}
\end{lem}
\begin{proof} The proof is by induction in $n$. In the inductive step, we have to treat the cases of even $n$ and odd $n$ separately (but the calculations are similar).\\
\underline{basis:}\\
We take the case of $n=0$ as the basis. For
\bd
\ad^0_B(\dot B) = \dot B = \frac{1}{2}\int \di x\di y\left( \dot k(x,y) a^\ast_x a^\ast_y - \cc{\dot k(x,y)} a_x a_y \right)
\ed
the estimate stated in the even case is clearly fulfilled.\\
\underline{inductive step, starting from even $n$:}\\
We calculate (by using the CCR and appropriately renaming integration variables)
\begin{align*}
& \ad^{n+1}_B(\dot B) = [B,\ad^n_B(\dot B)] \\
& = \left[\frac{1}{2} \int \di x\di y\left( k(x,y)a^\ast_x a^\ast_y - \cc{k(x,y)}a_x a_y \right), \frac{1}{2}\int \di x\di y\left( f_{n,1}(x,y) a^\ast_x a^\ast_y + f_{n,2}(x,y) a_x a_y \right)\right] \\
& = \frac{1}{2} \int \di x\di z \left(f_{n+1,1}(x,z) a^\ast_x a_z + f_{n+1,2}(x,z) a_x a^\ast_z \right),
\end{align*}
where
\begin{equation}
\label{eq:even}
\begin{split}
f_{n+1,1}(x,z) & = -\frac{1}{2} \int \di y \left( k(x,y) \left( f_{n,2}(z,y) + f_{n,2}(y,z) \right) + \cc{k(y,z)}\left( f_{n,2}(x,y) + f_{n,2}(y,x)\right) \right)\\
f_{n+1,2}(x,z) & = -\frac{1}{2} \int \di y \left( k(y,z) \left( f_{n,1}(x,y) + f_{n,1}(y,x) \right) + \cc{k(x,y)}\left( f_{n,1}(z,y) + f_{n,1}(y,z)\right) \right), 
\end{split}
\end{equation}
so, as $n+1$ is odd, we have the correct expression for $\ad^{n+1}_B(\dot B)$. We still have to check for validity of the estimates.
Clearly
\be{normnorm}
\begin{split}
\norm{f_{n+1,1}}_{L^2} & \leq \frac{1}{2} \bigg( \norm{\int \di y\, k(x,y) f_{n,2}(z,y)}_{L^2} + \norm{\int \di y\, k(x,y) f_{n,2}(y,z)}_{L^2} \\
& \quad + \norm{\int \di y\, \cc{k(y,z)}f_{n,2}(x,y)}_{L^2} + \norm{\int \di y\, \cc{k(y,z)} f_{n,2}(y,x)}_{L^2} \bigg),
\end{split}
\ee
where by abuse of notation, $x$ and $z$ are actually the variables which are integrated over in calculating the $L^2(\Rbb^3\times \Rbb^3)$-norm (cf.\ next equation).
Using Hoelder's inequality we get
\begin{align*}
\norm{\int \di y\,k(x,y) f_{n,2}(z,y)}^2 &= \int \di x\di z \left\lvert \int \di y\,k(x,y) f_{n,2}(z,y) \right\rvert^2 \\
& \leq \int \di x \di z \int \di y_1\,\lvert k(x,y_1) \rvert^2 \int \di y_2\,\lvert f_{n,2}(z,y_2) \rvert^2 = \norm{k}^2 \norm{f_{n,2}}^2. 
\end{align*}
The other three summands in \eqr{normnorm} obey the same bound, so by making use of the inductive hypothesis, we obtain
\bd
\norm{f_{n+1,1}}_{L^2} \leq \norm{\dot k} (2\norm{k})^{n+1},
\ed
which was to be proven. The calculation for $\norm{f_{n+1,2}}_{L^2}$ works the same way. We still have to proof the second estimate of the lemma, so we calculate using Hoelder's inequality that
\bd
\int \lvert f_{n+1,1}(x,x) \rvert \di x \leq 2\norm{k} \norm{f_{n,2}} \leq \norm{\dot k} (2\norm{k})^{n+1}
\ed
and in the same way $\int \lvert f_{n+1,2}(x,x) \rvert \di x$.\\
\underline{inductive step, starting from odd $n$:}\\
We calculate (by using the CCR and appropriately renaming integration variables)
\begin{align*}
& \ad^{n+1}_B(\dot B)\\
& = \left[ \frac{1}{2}\int \di x \di y \left( k(x,y)a^\ast_x a^\ast_y - \cc{k(x,y)} a_x a_y \right) , \frac{1}{2}\int \di x\di y \left( f_{n,1}(x,y) a^\ast_x a_y + f_{n,2}(x,y) a_x a^\ast_y \right) \right] \\
& = \frac{1}{2} \int \di x\di z \left( a^\ast_x a^\ast_z f_{n+1,1}(x,z) + a_x a_z f_{n+1,2}(x,z) \right)
\end{align*}
where
\begin{equation}
\label{eq:odd}
\begin{split}
f_{n+1,1}(x,z) & = - \int \di y\, k(x,y)\left( f_{n,1}(z,y) + f_{n,2}(y,z) \right) \\
f_{n+1,2}(x,z) & = - \int \di y\, \cc{k(x,y)}\left( f_{n,1}(y,z) + f_{n,2}(z,y) \right),
\end{split}
\end{equation}
so, as $n+1$ is even, we have the correct expression for $\ad^{n+1}_B(\dot B)$. Validity of the estimate for $\norm{f_{n+1,1}}$ and $\norm{f_{n+1,2}}$ again follows by invoking Hoelder's inequality and the inductive hypothesis.
\end{proof}

The second estimate in the following lemma is only needed for odd $n$.
\begin{lem}
\label{lm:deldelt}
For all $n \in \Nbb$ and all $i \in \{1,2\}$, we have the estimates
\bd
\norm{\dot f_{n,i}} \leq 2n \norm{\dot k}^2 (2\norm{k})^{n-1} + \norm{\ddot k}(2\norm{k})^n.
\ed
For odd $n$, we have the estimate
\bd
\int \lvert \dot f_{n,i}(x,x) \rvert \di x \leq 2n \norm{\dot k}^2 (2\norm{k})^{n-1} + \norm{\ddot k}(2\norm{k})^n .
\ed
\end{lem}
\begin{proof}
The first estimate is proved by induction. The basis is trivial. For the inductive step, we again have to treat the cases of even and odd $n$ separately: We employ \eqref{eq:even} respectively \eqref{eq:odd}, use Hoelder's inequality, then make use of $\norm{f_{n,i}} \leq \norm{\dot k} (2\norm{k})^n$ and finally apply the inductive hypothesis.

For the second estimate, we use \eqref{eq:even} to express $f_{n,i}$ in terms of the $f_{n-1,i}$, use Hoelder's inequality and apply the first estimate.
\end{proof}

\begin{lem}
\label{lm:timederivative}
There exists a constant $C$ independent of $N$ and $t$ such that the following estimate holds for all $t \in \Rbb$ and all $\psi \in \fock$ (where $\ev{\cdot}= \scal{\psi}{\cdot\psi}$):
 \bd
  \lvert \scal{\psi}{(\partial_t T^*_t)T_t \psi} \rvert \leq \norm{\dot k} e^{2\norm{k}} \ev{\Ncal+1} \leq C \norm{\ph}_{H^2}^2 \ev{\Ncal+1}.
 \ed
Furthermore
 \bd
    \lvert \scal{\psi}{\partial_t \left[ \left(\partial_t T^\ast_t\right) T_t \right] \psi} \rvert \leq \left( 2\norm{\dot k}^2 + \norm{\ddot k} \right) e^{2\norm{k}} \ev{\Ncal+1}.
 \ed
\end{lem}
\begin{proof}
The first estimate is proved by expanding the time derivative as a series,
\bd
\left(\partial_t e^{-B(t)} \right) e^{B(t)} = - \dot B(t) +
\frac{1}{2!}[B(t),\dot B(t)] - \frac{1}{3!}[B(t),[B(t),\dot B(t)]] + \dots = \sum_{n=0}^\infty \frac{(-1)^{n+1}}{(n+1)!} \ad^n_B(\dot B),
\ed
and then using the lemmata \ref{lm:highercommutators}, \ref{lm:Bbound} and \ref{lm:phdotregularity}.\\
For the second estimate, differentiate the summands individually (i.\,e. place a dot on the respective $f_{n,i}$) and then estimate all summands like before, now using lemma \ref{lm:deldelt}.
\end{proof}

\section{Estimates for the terms of $\partial_t \Lcal_N(t)$}
In this section, we collect the estimates for all the terms of $\dot \Lcal_N(t)$.

What is already done:\marginpar{overview of current status}\\
The term $\partial_t ((\partial _t T^\ast)T)$ is bounded by $\ev{\Ncal+1}$ (see lemma in previous chapter).\\
The time derivatives of the terms in lines \eqref{l7}, and \eqref{l10} through \eqref{l15} are bounded by $\Ncal^2/N$, $\Ncal$, $\Kcal$ and $\tilV$ (and as a consequence, by $\Ncal$ and $\Lcal$). (I'll type these estimates when I'm back. The method of proof is the same as in chapter \ref{ch:generatorestimates}, but easier because $a^\ast(\dot p_x) = \partial_t a^\ast(c_x)$ is regular while $a^\ast(c_x)$ is singular.)

However, lots of regularity estimates are needed, and so far not proven (I think $\ph \in H^4$ shoudl be sufficient).

\section{Estimating the number of fluctuations}
\begin{lem}
\label{lem:kvbounds}
There exists a constant $C$ independent of $N$ and $t$ (but possibly depending on $\norm{\ph}_{H^1}$) such that for all $\psi \in \fock$, the expectation value $\ev{\cdot} = \scal{\psi}{\cdot\psi}$ satisfies
\bd
\ev{\Kcal} + \ev{\tilV} \leq 2 \ev{\Lcal_N(t)} + C \norm{\ph}_{H^2}^4 \ev{\frac{1}{N}\Ncal^2 + \Ncal + 1}.
\ed
As $\Kcal$ and $\tilV$ are non-negative operators, the same upper bound holds for each of $\Kcal$ and $\tilV$ individually.
\end{lem}
\begin{proof}
We know that $\Lcal_N(t) = (i\partial_t T^\ast_t)T_t + T^\ast_t \Lcal_N^{(0)}(t)T_t$ and we have explictly calculated and bounded $T^\ast_t \Lcal_N^{(0)}(t)T_t$. We now write only line \eqref{l9} and the first summand in line \eqref{l10} explicitly, all the other terms we collect in $\Lcal_{\textrm{rest}}$:
\begin{align*}
\Lcal_N(t) & = \frac{1}{2} \int \di x\, a^\ast(c_x) a(-\Delta_x c_x) + \frac{1}{2}\int \di x\, a^\ast(-\Delta_x s_x)a(s_x) \\
& \quad + \frac{1}{2}\int \di x\di y\, NV_N(x-y) \frac{1}{2N} a^\ast(c_x) a^\ast(c_y) a(c_y) a(c_x) + \Lcal_{\textrm{rest}}.
\end{align*}
We define $\tilK := \frac{1}{2} \int \di x\, a^\ast(c_x) a(-\Delta_x c_x)$ and observe that
\bd
\ev{\frac{1}{2}\int \di x\, a^\ast(-\Delta_x s_x)a(s_x)} = \frac{1}{2}\int \di x\, \norm{a(\nabla_x s_x)\psi}^2 \geq 0,
\ed
so by leaving away this non-negative term we obtain
\be{Lcalbound}
\ev{\Lcal_N(t)} \geq \ev{\tilK + \tilV} - \lvert\ev{\Lcal_{\textrm{rest}}}\rvert.
\ee
We now collect the estimates for lines \eqref{l7}, \eqref{l8} and \eqref{l10} through \eqref{l25} from chapter \ref{ch:generatorestimates} to obtain an upper bound for $\lvert\ev{\Lcal_{\textrm{rest}}}\rvert$, i.\,e.\ a lower bound for $-\lvert\ev{\Lcal_{\textrm{rest}}}\rvert$.
(To simplify the notation we insert factors of $\norm{\ph}_{H^2} \geq \norm{\ph}_{L^2} =1$ in some places, leave away prefactors $\frac{1}{N}$ and $\frac{1}{\sqrt{N}}$ where there does not follow a $\Ncal^2$, and use that $1 \leq 1/\varepsilon$. Numerical prefactors are absorbed into $C$.)
\bd
\lvert\ev{\Lcal_{\textrm{rest}}}\rvert \leq C \norm{\ph}_{H^2}^2 \left( \frac{1}{\varepsilon}\ev{\frac{1}{N}\Ncal^2 + \Ncal + 1} + \varepsilon\ev{\Kcal+\tilV} \right).
\ed
So using \eqr{Lcalbound} we conclude that
\bd
\ev{\Lcal_N(t)} \geq \ev{\tilK+\tilV} - C\norm{\ph}_{H^2}^2 \left( \frac{1}{\varepsilon}\ev{\frac{1}{N}\Ncal^2 +\Ncal+1} + \varepsilon \ev{\Kcal+\tilV} \right).
\ed
Combining this with $\ev{\tilK} \geq \ev{\Kcal} - C\left(\varepsilon\ev{\Kcal} + \frac{1}{\varepsilon}\ev{\Ncal} \right)$ from \eqref{eq:bound1A}, we arrive at (where $C$ is being redefined)
\bd
\ev{\Lcal_N(t)} \geq  \ev{\Kcal+\tilV} - C\norm{\ph}_{H^2}^2 \left( \frac{1}{\varepsilon}\ev{\frac{1}{N}\Ncal^2 +\Ncal+1} + \varepsilon \ev{\Kcal+\tilV} \right).
\ed
Now choose $\varepsilon = 1/(2C\norm{\ph}_{H^2}^2)$ (with $C$ having the same value as in the previous equation) to conclude
\bd
\ev{\Lcal_N(t)} + 2\left( C \norm{\ph}_{H^2}^2 \right)^2 \ev{\frac{1}{N}\Ncal^2 + \Ncal + 1} \geq \frac{1}{2}\ev{\Kcal+\tilV},
\ed
thus the proof is finished.
\end{proof}

\begin{lem}
There exists a constant $C$ independent of $N$ and $t$ (but possibly depending on $\norm{\ph}_{H^1}$) such that for all $\psi \in \fock$, the expectation value $\ev{\cdot} = \scal{\psi}{\cdot\psi}$ satisfies
\bd
\ev{\Lcal_N(t)} \geq -C \norm{\ph}_{H^2}^4 \ev{\frac{1}{N}\Ncal^2 + \Ncal + 1}.
\ed 
\end{lem}
\begin{proof}
As $\ev{\Kcal} \geq 0$ and $\ev{\tilV} \geq 0$, this lemma is an easy corollary of lemma \ref{lem:kvbounds}.
\end{proof}

\begin{lem}
\label{lem:ldotbounds}
 \bd
  \frac{\di}{\di t} \ev{\Lcal} = \ev{\dot \Lcal} \leq C \ev{\Lcal + \Ncal}
 \ed
\end{lem}
\begin{proof}
We can almost control $\lvert \scal{\psi}{\partial_t \left[ \left(\partial_t T^\ast_t\right) T_t \right] \psi} \rvert$ (regularity of $\phddot$ ist still missing).
 TODO. \marginpar{TODO.}
\end{proof}


\begin{lem}
\label{lem:lncommutatorbound}
There exists a constant $C$ independent of $N$ and $t$ (but possibly depending on $\norm{\ph}_{H^1}$) such that for all $\psi \in \fock$, the expectation value $\ev{\cdot} = \scal{\psi}{\cdot\psi}$ satisfies
 \bd
  \lvert \ev{[\Lcal_N(t),\Ncal]} \rvert \leq C \norm{\ph}_{H^2}^2 \left( \ev{N+1} + \frac{1}{N}\ev{\Ncal^2} + \ev{\tilV} + \ev{\Kcal} \right).
 \ed
\end{lem}
\begin{proof}
We start by calculating $[T^\ast_t \Lcal_N^{(0)}(t)T_t,\Ncal]$. Summands which contain an equal number of annihilation and creation operators commute with $\Ncal$ and thus drop. The other summands get numerical prefactors, but except of that stay unchanged. Because of these prefactors, the sign of the $+\hc$-part changes. So we get
\begin{align*}
& [T^\ast_t \Lcal_N^{(0)}(t) T_t,\Ncal] = \\ 
& \frac{1}{2} \int \di x\, \bigg[(-2) a^*(c_x) \int \di y\, a^*_y \Big( N \nabla w_N(x-y) \nabla_x \ph(x) 2 \ph(y) \\
& \qquad\qquad \qquad\qquad \qquad	+ Nw_N(x-y) \Delta_x \ph(x) \ph(y) - \Delta_x r(y,x) \Big)\bigg] \\
& + \frac{1}{2}\int \di x \di y\, NV_N(x-y) \times \\
& \times \Big[   \frac{1}{2N}\bigg( 4(-2) a^*(c_x) a^*(c_y) a^*(s_x) a(c_y) + 2(-4) a^*(c_x) a^*(c_y) a^*(s_y) a^*(s_x)\\
				      & \qquad\qquad + 4(-2) a^*(c_x) a^*(s_y) a^*(s_x) a(s_y)\bigg) \\
& + \frac{1}{N}\bigg(  (-2) a^*(c_x) a^*(c_y) \Big( r(y,x) + \scal{p_y}{s_x} \Big) + (-2)a^*(c_x) a^*(p_y) k(y,x) \\
      & \qquad\quad + (-2)a^*(s_y) a^*(s_x) \scal{s_x}{c_y} + 2(-2)a^*(c_x) a^*(s_x) \scal{s_y}{s_y} + 2(-2)a^*(c_x)a^*(s_y) \scal{s_y}{s_x} \bigg)\\
& + \ph(x)\ph(y) \Big( (-2) a^*(c_x) a^*(p_y) +(+2)a(s_x) a(s_y) \Big)\\
& + \ph(x) \cc{\ph(y)} 2(-2) a^*(c_x) a^*(s_y) + \lvert \ph(y) \rvert^2 2(-2) a^*(c_x) a^*(s_x)\\
& + \frac{2}{\sqrt{N}}\cc{\ph(y)} \bigg(    (-3)a^*(c_x) a^*(s_x) a^*(s_y) + (-1)a^*(c_x) a^*(s_x) a(c_y)\\
					    & \qquad\qquad\qquad + (-1)a^*(c_x) a^*(s_y) a(c_x) + (+1)a^*(c_x) a(c_x) a(c_y) + (+1)a^*(s_y) a(s_x) a(c_x)\\
					    & \qquad\qquad\qquad + (-1)a^*(s_x) a^*(s_y) a(s_x) + (+1)a^*(s_x) a(s_x) a(c_y) + (+3)a(s_x) a(c_x) a(c_y)  \bigg)\\
& + \frac{2}{\sqrt{N}}\cc{\ph(y)} \bigg(    (-1)a^*(s_x) \scal{s_x}{s_y} + (-1)a^*(s_y) \scal{s_x}{s_x}  + (+1)a(c_y) \scal{s_x}{s_x} + (+1)a(c_x) \scal{s_x}{s_y}\\
					    & \qquad\qquad\qquad + (-1)a^*(c_x)r(x,y) + (-1)a^*(c_x)\scal{p_x}{s_y} + (+1)a(s_x)r(x,y) + (+1)a(s_x)\scal{p_x}{s_y}	\bigg)    \Big]\\
&- \hc
\end{align*}
Now choose e.\,g.\ $\varepsilon = 1$ and use the estimates from chapter \ref{ch:generatorestimates} to bound all summands of the commutator. Together with lemma \ref{lm:timederivative} for bounding $\ev{(\partial_t T^\ast_t) T_t}$, the theorem is now proven.
\end{proof}


\begin{prp}
 $N$-independent bound on $\ev{\Ncal}$ by employing all the above and Gronwall. \marginpar{TODO}
\end{prp}
\begin{proof}
 By lemma \ref{lem:lncommutatorbound}, lemma \ref{l:N2} and lemma \ref{lem:kvbounds}, $\ev{[\Lcal,\Ncal]}$ is bounded by $\ev{\Lcal+\Ncal}$. So by lemma \ref{lem:ldotbounds}, the sum $\Lcal+C \Ncal$ is bounded above, and for $C$ large enough, it is positive, so we get a bound for $\Ncal$.
\end{proof}


\section{Main result}
\begin{thm}
 Convergence of reduced density matrix for Bogoliubov state.
\end{thm}

\begin{thm}
 Convergence of reduced density matrix for factorized initial state.
\end{thm}

\begin{lem}
\marginpar{might fit somewhere else better, if it's worth a lemma}
 Convergence of $\Tr \lvert \project{\ph} - \project{\varphi_t} \rvert$.
\end{lem}


\section{A-priori estimates}


Let $f(x,y)$ be a function of two variables. To simplify the notation we write
\[
  f_y(x) = f(x,y).
\]
Consider two operators $A$ and $B$ on the Fock space $\mathcal{F}$. The
notation
\[
  A \le B
\]
means that $\langle \psi, A \psi \rangle \le \langle \psi, B \psi \rangle$ for
all $\psi \in \mathcal{F}$.


\begin{lem}
  \label{l:N2}
  Let $U_t$ be the unitary evolution defined in (X). Then, \marginpar{could we have an explicit sentence here which says $\frac{1}{N}\ev{(\Ncal+2)^2} \leq C \ev{\Ncal+1}$, please? (where $C$ does not depend on $t$)}
  \begin{equation}
    U_t^* \N^2 U_t \le C_1 \big[ N U_t^* (\N+1) U_t + N (\N+1) + (\N+1)^2
    \big], \tag{i}
  \end{equation}
  where $C_1$ is a constant that depends only on $\| p \|_{L^2}$ and $\| s
  \|_{L^2}$. Furthermore,
  \begin{equation}
    \begin{split}
      U_t^* \K \N U_t & \le C_2 \big[ N U_t^* \K U_t + (1 + \| \nabla_1 s
      \|_{L^2}^2) U_t^* (\N+1) U_t \\
      & \quad + ( 1 + N^{-1} \| \nabla_1 s \|_{L^2}^2) (\N^2 + N(\N+1)) \\
      & \quad + T^* \phi(\varphi) W^*( \mathcal{H}_N + \N ) W \psi(\varphi) T
      \\
      & \quad + N^{-1} T^* N W^*( \mathcal{H}_N + \N ) W \N T \big]
    \end{split}
    \tag{ii}
  \end{equation}
  where $C_2$ is a constant that depends only on $\| p \|_{L^2}$, $\| s
  \|_{L^2}$, $\| \varphi_t \|_{H^2}$ and $\| \nabla_1 p \|_{L^2}$.
\end{lem}


The proof of Lemma \ref{l:N2} is based on two propositions, which we prove
first.


\begin{prp}
  \label{p:TNT}
  Let $p, s \in L^2(\R^3 \times \R^3)$ and $f \in H^1(\R^3)$. Then,
  \begin{align}
    T^* \N T & \apprle C (\N+1), \label{TNT} \tag{i} \\
    T^* \K T & \apprle C \K + \| \nabla_1 s \|_{L^2}^2, \label{TKT} \tag{ii}
    \\
    T^* \N^2 T & \apprle C^2 (\N+1)^2, \label{TN2T} \tag{iii} \\
    \langle \psi, T^* \K \N T \psi \rangle & \apprle C^2 \big( |\langle \psi,
    T^* \K T \N \psi \rangle| + \langle \psi, \K \psi \rangle + \| \nabla_1 s
    \|_{L^2}^2 \big), \label{TKNT} \tag{iv} \\
    \phi(f) \K \phi(f) & \apprle \| f \|_{L^2}^2 \K (\N+1) + \| \nabla f
    \|_{L^2}^2 (\N + 1), \label{fkf} \tag{v}
  \end{align}
  with $C = 1 + \| p \|_{L^2}^2 + \| s \|_{L^2}^2$. Moreover, the same
  inequalities hold true with $T$ replaced by~$T^*$.
\end{prp}


\begin{prp}
  \label{p:formulae}
  Let $\varphi \in L^2(\R^3)$ with $\| \varphi \|_{L^2} = 1$. The following
  pull-through formulae hold true:
  \begin{align}
    \N W(\sqrt{N} \varphi)^* & = W(\sqrt{N} \varphi)^* (\N - \sqrt{N}
    \phi(\varphi) + N), \tag{i} \\
    \N W(\sqrt{N} \varphi) & = W(\sqrt{N} \varphi) (\N + \sqrt{N}
    \phi(\varphi) + N), \tag{ii} \\
    W(\sqrt{N} \varphi)^* \phi(\varphi) & = (\phi(\varphi) + 2 \sqrt{N})
    W(\sqrt{N} \varphi)^*, \tag{iii} \\
    \phi(\varphi) T & = T \phi(C \varphi + S \overline{\varphi}). \tag{iv}
  \end{align}
\end{prp}


\begin{proof}[Proof of Proposition \ref{p:formulae}]
  We give only an outline of the proof. Recall that $\phi(\varphi) =
  a(\varphi) + a^*(\varphi)$. Then, parts (i), (ii) and (iii) follow easily by
  a short calculation using parts (ii) and (iii) of Lemma \ref{l:W}.
  Similarly, part (iv) follows from Lemma \ref{l:bt}.
\end{proof}


\begin{proof}[Proof of Proposition \ref{p:TNT}]
  (i) Write
  \[
    \langle \psi, T^* \N T \psi \rangle = \int dx \, \langle T^* a_x T \psi,
    T^* a_x T \psi \rangle = \int dx \, \| (a_x + a(p_x) + a^*(s_x)) \psi
    \|^2.
  \]
  Then, by Cauchy-Schwarz inequality, and Lemma \ref{l:a},
  \begin{align*}
    \langle \psi, T^* \N T \psi \rangle & \apprle \int dx \, \| a_x \psi \|^2
    + \int dx \, \| a(p_x) \psi \|^2 + \int dx \, \| a^*(s_x) \psi \|^2 \\
    & \apprle \langle \psi, \N \psi \rangle + \int dx \, \| p_x \|_{L^2}^2
    \langle \psi, \N \psi \rangle + \int dx \, \| s_x \|_{L^2}^2 \langle
    \psi, (\N+1) \psi \rangle \\
    & \apprle (1 + \| p \|_{L^2}^2 + \| s \|_{L^2}^2) \langle \psi, (\N+1) \psi
    \rangle.
  \end{align*}


  (ii) Similarly, first observe that
  \[
    \langle \psi, T^* \K T \psi \rangle = \int dx \, \langle T^* \nabla_x a_x
    T \psi, T^* \nabla_x a_x T \psi \rangle = \int dx \, \| (\nabla a_x +
    a(\nabla_x p_x) + a^*(\nabla_x s_x)) \psi \|^2.
  \]
  Thus, by Cauchy-Schwarz inequality, and Lemma \ref{l:a},
  \begin{align*}
    \langle \psi, T^* \K T \psi \rangle & \apprle \int dx \, \| \nabla_x a_x
    \psi \|^2 + \int dx \, \| a(\nabla_x p_x) \psi \|^2 + \int dx \, \|
    a^*(\nabla_x s_x) \psi \|^2 \\
    & = \langle \psi, \K \psi \rangle + \int dx \, \| a(\nabla_x p_x) \psi
    \|^2 + \int dx \, \| a(\nabla_x s_x) \psi \|^2 + \| \nabla_1 s \|_{L^2}^2
    \\
    & \apprle (1 + \| p \|_{L^2}^2 + \| s \|_{L^2}^2) \langle \psi, \K \psi
    \rangle + \| \nabla_1 s \|_{L^2}^2.
  \end{align*}


  (iii) Write
  \begin{align*}
    & \langle \psi, T^* \N^2 T \psi \rangle \\
    & = \int dxdy \, \langle \psi, T^* a_x^* a_x a_y^* a_y T \psi \rangle \\
    & = \int dx \, \langle \psi, T^* a_x^* \N a_x T \psi \rangle + \langle
    \psi, T^* \N T \psi \rangle \\
    & = \int dx \, \langle (a_x + a(p_x) + a^*(s_x)) \psi, T^* \N T (a_x +
    a(p_x) + a^*(s_x)) \psi \rangle + \langle \psi, T^* \N T \psi \rangle.
  \end{align*}
  Then, applying part (i) and Cauchy-Schwarz inequality, and using the
  pull-through formula $a_x \N^{1/2} = (\N+1)^{1/2} a_x$ and Lemma \ref{l:a},
  we obtain
  \begin{align*}
    & \langle \psi, T^* \N^2 T \psi \rangle \\
    & \apprle C \int dx \, \| (\N+1)^{1/2} (a_x + a(p_x) + a^*(s_x)) \psi \|^2
    + C \langle \psi, (\N+1) \psi \rangle \\
    & \apprle C \int dx \, (\| a_x \N \psi \|^2 + \| a(p_x) \N^{1/2} \psi \|^2
    + \| a^*(s_x) (\N+2)^{1/2} \psi \|^2 ) + C \langle \psi, (\N+1) \psi
    \rangle \\
    & \apprle C^2 \langle \psi, (\N+1)^2 \psi \rangle,
  \end{align*}
  where $C = 1 + \| p \|_{L^2}^2 + \| s \|_{L^2}^2$.


  (iv) Write
  \begin{align*}
    \K \N & = \int dx dy \, \nabla_x a_x^* \nabla_x a_x a_y^* a_y \\
    & = \int dx dy \, \nabla_x a_x^* a_y^* \nabla_x a_x a_y + \int dx dy \,
    \nabla_x a_x^* a_y \nabla_x \delta(x-y) \\
    & = \int dx \, \nabla_x a_x^* \N \nabla_x a_x + \K.
  \end{align*}
  Then, by part (i),
  \begin{align*}
    \left \langle \psi, T^* \int dx \, \nabla_x a_x^* \N \nabla_x a_x T \psi
    \right \rangle & = \int dx \, \langle T^* \nabla_x a_x T \psi, T^* \N T
    T^* \nabla_x a_x T \psi \rangle, \\
    & \le C \int dx \, \langle T^* \nabla_x a_x T \psi, \N T^* \nabla_x a_x T
    \psi \rangle + C \langle \psi, T^* \K T \psi \rangle.
  \end{align*}
  Hence, combining these two expressions,
  \begin{equation}
    \label{ep5}
    \langle \psi, T^* \K \N T \psi \rangle \le C \int dx \, \langle T^*
    \nabla_x a_x T \psi, \N T^* \nabla_x a_x T \psi \rangle + 2C \langle \psi,
    T^* \K T \psi \rangle.
  \end{equation}
  Next, in order to estimate the first term in this inequality, we first write
  \begin{align*}
    & \int dx \, \langle T^* \nabla_x a_x T \psi, \N T^* \nabla_x a_x T \psi
    \rangle \\
    & = \int dx \, \langle T^* \nabla_x a_x T \psi, \N ( a(\nabla_x c_x) +
    a^*(\nabla_x s_x) ) \psi \rangle \\
    & = \int dx \, \langle T^* \nabla_x a_x T \psi, ( a(\nabla_x c_x) (\N-1) +
    a^*(\nabla_x s_x) (\N+1) ) \psi \rangle \\
    & = \int dx \, \langle T^* \nabla_x a_x T \psi, \big( (a(\nabla_x c_x) +
    a^*(\nabla s_x)) \N - (a(\nabla_x c_x) - a^*(\nabla_x s_x)) \big) \psi
    \rangle \\
    & = \int dx \, \langle T^* \nabla_x a_x T \psi, ( T^* \nabla_x a_x T \N -
    T \nabla_x a_x T^* ) \psi \rangle \\
    & = \langle \psi, T^* \K T \N \psi \rangle - \int dx \, \langle T^*
    \nabla_x a_x T \psi, T \nabla_x a_x T^* \psi \rangle.
  \end{align*}
  Thus, by triangle inequality,
  \begin{align*}
    \int dx \, \langle T^* \nabla_x a_x T \psi, \N T^* \nabla_x a_x T \psi
    \rangle & \le |\langle \psi, T^* \K T \N \psi \rangle| + \int dx \,
    |\langle T^* \nabla_x a_x T \psi, T \nabla_x a_x T^* \psi \rangle| \\
    & \le |\langle \psi, T^* \K T \N \psi \rangle| + \langle \psi, T^* \K T
    \psi \rangle + \langle \psi, T \K T^* \psi \rangle.
  \end{align*}
  Therefore, combining this with \eqref{ep5}, and using part (iii), we find
  that
  \begin{align*}
    \langle \psi, T^* \K \N T \psi \rangle & \apprle C |\langle \psi, T^* \K T
    \N \psi \rangle| + C \langle \psi, T^* \K T \psi \rangle + C \langle \psi,
    T \K T^* \psi \rangle \\
    & \apprle C^2 \big( |\langle \psi, T^* \K T \N \psi \rangle| + \langle
    \psi, \K \psi \rangle + \| \nabla_1 s \|_{L^2}^2 \big).
  \end{align*}


  (v) By the Cauchy-Schwarz inequality, a brief calculation, and Lemma
  \ref{l:a},
  \begin{align*}
    \langle \psi, \phi(\varphi) \K \phi(\varphi) \psi \rangle & = \| \K^{1/2}
    (a^*(\varphi) + a(\varphi) ) \psi \|^2 \\
    & \apprle \langle \psi, a(\varphi) \K a^*(\varphi) \psi \rangle + \langle
    \psi, a^*(\varphi) \K a(\varphi) \psi \rangle \\
    & = \int dx \, \| a^*(\varphi) \nabla_x a_x \psi \|^2 + \int dx \, \|
    a(\varphi) \nabla_x a_x \psi \|^2 + \int dx \, |\nabla \varphi(x)|^2 \\
    & \quad - 2 \Re \int dx dy \, \langle \nabla \varphi(x) a_y \psi, f(y)
    \nabla_x a_x \psi \rangle \\
    & \apprle \| \varphi \|_{L^2}^2 \int dx \, \| (\N + 1)^{1/2} \nabla_x a_x
    \psi \|^2 + \int dx \, | \nabla \varphi(x) |^2 \\
    & \quad + \int dx dy \, | \nabla \varphi(x)|^2 \| a_y \psi \|^2 + \int dx
    dy \, |f(y)|^2 \| \nabla_x a_x \psi \|^2 \\
    & = \| \varphi \|_{L^2}^2 \langle \psi, \K (\N + 1) \psi \rangle + \|
    \nabla \varphi \|_{L^2}^2 \langle \psi, (\N + 1) \psi \rangle.
  \end{align*}
  This completes the proof of the proposition.
\end{proof}


\begin{proof}[Proof of Lemma \ref{l:N2}]
  Let $C = 1 + \| p \|_{L^2}^2 + \| s \|_{L^2}^2$. To simplify the notation we
  write
  \[
    U_t = T_t^* W_t^* e^{-it \mathcal{H}_N} WT
  \]
  with $W_t = W(\sqrt{N} \varphi_t)$, $W = W(\sqrt{N} \varphi)$ and $T = T_0$.


  (i) By Proposition \ref{p:TNT},
  \begin{equation}
    \label{ep4}
    \begin{split}
      \langle U_t \psi, \N^2 U_t \psi \rangle & = \langle T_t U_t \psi, T_t
      \N^2 T_t^* T_t U_t \psi \rangle \\
      & \apprle C^2 \langle T_t U_t \psi, (\N+1)^2 T_t U_t \psi \rangle \\
      & \apprle C^2 \langle T_t U_t \psi, \N^2 T_t U_t \psi \rangle + C^3
      \langle U_t \psi, (\N+1) U_t \psi \rangle. \\
    \end{split}
  \end{equation} 
  Now, using the formulae in Proposition \ref{p:formulae}, and noting that
  $\N$ commutes with $\mathcal{H}_N$,
  \begin{align*}
    & \langle T_t U_t \psi, \N^2 T_t U_t \psi \rangle \\
    & = \langle T_t U_t \psi, \N^2 W_t^* e^{-it \mathcal{H}_N} WT \psi \rangle
    \\
    & = \langle T_t U_t \psi, \N W_t^* (\N - \sqrt{N} \phi(\varphi_t) + N)
    e^{-it \mathcal{H}_N} WT \psi \rangle \\
    & = N \langle T_t U_t \psi, \N T_t U_t \psi \rangle - \sqrt{N} \langle T_t
    U_t \psi, \N ( \phi(\varphi_t) + 2\sqrt{N}) W_t^* e^{-it\mathcal{H}_N} WT
    \psi \rangle \\
    & \quad + \langle T_t U_t \psi, \N W_t^* e^{-it\mathcal{H}_N} W(\N +
    \sqrt{N} \phi(\varphi) + N) T \psi \rangle \\
    & = - \sqrt{N} \langle T_t U_t \psi, \N \phi(\varphi_t) W_t^*
    e^{-it\mathcal{H}_N} WT \psi \rangle + \sqrt{N} \langle T_t U_t \psi, \N
    W_t^* e^{-it\mathcal{H}_N} W T \phi(C \varphi + S \overline{\varphi}) \psi
    \rangle \\
    & \quad + \langle T_t U_t \psi, \N W_t^* e^{-it\mathcal{H}_N} W \N T \psi
    \rangle \\
    & = - \sqrt{N} \langle \N T_t U_t \psi, \phi(\varphi_t) T_t U_t \psi
    \rangle + \sqrt{N} \langle \N T_t U_t \psi, T_t U_t \phi(C \varphi + S
    \overline{\varphi}) \psi \rangle \\
    & \quad + \langle \N T_t U_t \psi, T_t U_t T^* \N T \psi \rangle.
  \end{align*}
  Hence, applying Cauchy-Schwarz inequality and Lemma \ref{l:a}, and observing
  that the operator norm is controlled by the Hilbert-Schmidt norm, we get
  \begin{align*}
    & \langle T_t U_t \psi, \N^2 T_t U_t \psi \rangle \\
    & \le \| \N T_t U_t \psi \| ( \sqrt{N} \| \phi(\varphi_t) T_t U_t \psi \|
    + \sqrt{N} \| \phi(C \varphi + S \overline{\varphi}) \psi \| + \| \N T
    \psi \| ) \\
    & \le \| \N T_t U_t \psi \| ( 2 \sqrt{N} \| \varphi_t \|_{L^2} \|
    (\N+1)^{1/2} T_t U_t \psi \| \\
    & \quad + 2 \sqrt{N} ( 1 + \| p \|_{L^2} + \| s \|_{L^2}) \| \varphi
    \|_{L^2} \| (\N+1)^{1/2} \psi \| + \| \N T \psi \| ) \\
    & \le \frac{3}{4} \| \N T_t U_t \psi \|^2 + 16N \| (\N + 1)^{1/2} T_t U_t
    \psi \|^2 \\
    & \quad + 16N (1 + \| p \|_{L^2} + \| s \|_{L^2})^2 \| (\N + 1)^{1/2} \psi
    \|^2 + 4\| \N T \psi \|^2.
  \end{align*}
  Thus, using again Proposition \ref{p:TNT},
  \begin{align*}
    & \langle T_t U_t \psi, \N^2 T_t U_t \psi \rangle \\
    & \apprle N \langle U_t \psi, T_t^* (\N+1) T_t U_t \psi \rangle \\
    & \quad + N (1 + \| p \|_{L^2}^2 + \| s \|_{L^2}^2) \langle \psi, (\N+1)
    \psi \rangle + \langle \psi, T^* \N^2 T \psi \rangle \\
    & \apprle N C \langle U_t \psi, (\N+1) U_t \psi \rangle + N C \langle
    \psi, (\N+1) \psi \rangle + C^2 \langle \psi, (\N+1)^2 \psi \rangle.
  \end{align*}
  Therefore, recalling \eqref{ep4},
  \begin{align*}
    \langle U_t \psi, \N^2 U_t \psi \rangle 
    & \apprle C^2 \langle T_t U_t \psi, \N^2 T_t U_t \psi \rangle + C^3
    \langle U_t \psi, (\N+1) U_t \psi \rangle. \\
    & \apprle N C^3 \langle \psi, U_t^* (\N+1) U_t \psi \rangle + N C^3
    \langle \psi, (\N+1) \psi \rangle + C^4 \langle \psi, (\N+1)^2 \psi
    \rangle.
  \end{align*}
  This completes the proof of part (i).


  (ii) Applying parts (iv) and (iii) of Proposition \ref{p:TNT},
  \begin{equation}
    \label{nk1}
    \begin{split}
      \langle U_t \psi, \K \N U_t \psi \rangle & = \langle T_t U_t \psi, T_t
      \K \N T_t^* T_t U_t \psi \rangle \\
      & \apprle C | \langle T_t U_t \psi, T_t \K T_t^* \N T_t U_t \psi \rangle
      | + C^2 \langle T_t U_t \psi, \K T_t U_t \psi \rangle + C \| \nabla_1 s
      \|_{L^2}^2 \\
      & \apprle C | \langle T_t U_t \psi, T_t \K T_t^* \N T_t U_t \psi \rangle
      | + C^3 \langle U_t \psi, \K U_t \psi \rangle + C^2 \| \nabla_1 s
      \|_{L^2}^2.
    \end{split}
  \end{equation}
  Hence, we are left to estimating $| \langle T_t U_t \psi, T_t \K T_t^* \N
  T_t U_t \psi \rangle |$.


  Using the pull-through formulae from Proposition \ref{p:formulae}, and
  recalling that $\N$ commutes with $\mathcal{H}_N$, we find that
  \begin{equation}
    \label{nk2}
    \begin{split}
      & | \langle T_t U_t \psi, T_t \K T_t^* \N T_t U_t \psi \rangle | \\
      & = | \langle T_t U_t \psi, T_t \K T_t^* W_t^* (\N - \sqrt{N}
      \phi(\varphi_t) + N) e^{-it \mathcal{H}_N} WT \psi \rangle | \\
      & \le N \langle U_t \psi, \K U_t \psi \rangle + \sqrt{N} | \langle T_t
      U_t \psi, T_t \K T_t^* ( \phi(\varphi_t) + 2 \sqrt{N} ) T_t U_t \psi
      \rangle | \\
      & \quad + | \langle T_t U_t \psi, T_t \K T_t^* W_t^* e^{-it
      \mathcal{H}_N} W (\N + \sqrt{N} \phi(\varphi) + N) T \psi \rangle | \\
      & \le 4 N \langle U_t \psi, \K U_t \psi \rangle + \sqrt{N} | \langle T_t
      U_t \psi, T_t \K T_t^* \phi(\varphi_t) T_t U_t \psi \rangle | \\
      & \quad + | \langle T_t U_t \psi, T_t \K T_t^* T_t U_t T^* (\N +
      \sqrt{N} \phi(\varphi) ) T \psi \rangle|.
    \end{split}
  \end{equation}
  We next estimate the last two terms in this inequality.


  By Cauchy-Schwarz inequality and Proposition \ref{p:TNT}(v), for any
  $\varepsilon > 0$,
  \begin{align*}
    & \sqrt{N} | \langle T_t U_t \psi, T_t \K T_t^* \phi(\varphi_t) T_t U_t
    \psi \rangle | \\
    & = | \langle \sqrt{N} (T_t \K T_t^*)^{1/2} T_t U_t \psi, (T_t \K
    T_t^*)^{1/2} \phi(\varphi_t) T_t U_t \psi \rangle | \\
    & \le \varepsilon^{-1} N \langle U_t \psi, \K U_t \psi \rangle +
    \varepsilon \langle U_t \psi, T_t^* \phi(\varphi_t) T_t \K T_t^*
    \phi(\varphi_t) T_t U_t \psi \rangle \\
    & = \varepsilon^{-1} N \langle U_t \psi, \K U_t \psi \rangle +
    \varepsilon \langle U_t \psi, \phi(C \varphi_t + S \overline{\varphi_t})
    \K \phi(C \varphi_t + S \overline{\varphi_t}) U_t \psi \rangle \\
    & \le ( \varepsilon^{-1} N + \varepsilon C) \langle U_t \psi, \K U_t \psi
    \rangle + \varepsilon C \langle U_t \psi, \K \N U_t \psi \rangle \\
    & \quad + ( \| \nabla \varphi \|_{L^2}^2 + \| \nabla_1 p \|_{L^2}^2 + \|
    \nabla_1 s \|_{L^2}^2) \langle U_t \psi, (\N + 1) U_t \psi \rangle
  \end{align*}
  and
  \begin{align*}
    & | \langle T_t U_t \psi, T_t \K T_t^* T_t U_t T^* (\N + \sqrt{N}
    \phi(\varphi) ) T \psi \rangle| \\
    & = | \langle (T_t \K T_t^*)^{1/2} T_t U_t \psi, (T_t \K T_t^*)^{1/2} T_t
    U_t T^* (\N + \sqrt{N} \phi(\varphi) ) T \psi \rangle| \\
    & \le N \langle U_t \psi, \K U_t \psi \rangle + \frac{1}{N} \| (T_t \K
    T_t^*)^{1/2} T_t U_t T^* ( \N + \sqrt{N} \phi(\varphi) ) T \psi \|^2.
  \end{align*}
  Hence, combining this with \eqref{nk2} and \eqref{nk1}, and choosing
  $\varepsilon$ so that $\varepsilon C < 1/2$, we obtain
  \begin{equation}
    \label{nk3}
    \begin{split}
      \langle U_t \psi, \K \N U_t \psi \rangle & \le C_1 N \langle U_t \psi,
      \K U_t \psi \rangle + C_1 (1 + \| \nabla_1 s \|_{L^2}^2) \langle U_t
      \psi, (\N+1) U_t \psi \rangle \\
      & \qquad + \frac{C_1}{N} \| (T_t \K T_t^*)^{1/2} T_t U_t T^* ( \N +
      \sqrt{N} \phi(\varphi) ) T \psi \|^2,
    \end{split}
  \end{equation}
  where $C_1$ is a constant that depends only on $\| p \|_{L^2}$, $\| s
  \|_{L^2}$, $\| \varphi \|_{H^1}$ and $\| \nabla_1 p \|_{L^2}$. In order to
  conclude the proof, we need a few more estimates.

  
  By Proposition \ref{p:TNT}(ii), Cauchy-Schwarz inequality, and Lemma
  \ref{l:a},
  \begin{align*}
    & \| (T_t \K T_t^*)^{1/2} T_t U_t T^* ( \N + \sqrt{N} \phi(\varphi) ) T
    \psi \|^2 \\
    & \apprle C \| \K^{1/2} T_t U_t T^* ( \N + \sqrt{N} \phi(\varphi) ) T \psi
    \|^2 + \| \nabla_1 s \|_{L^2}^2 ( \| \N T \psi \|^2 + N \| \phi(\varphi) T
    \psi \|^2) \\
    & \apprle C \| \K^{1/2} T_t U_t T^* ( \N + \sqrt{N} \phi(\varphi) ) T \psi
    \|^2 + C^2 \| \nabla_1 s \|_{L^2}^2 \langle \psi, (\N^2 + N(\N+1)) \psi
    \rangle.
  \end{align*}
  Now, since $V \ge 0$, observe that $\K \le \mathcal{H}_N$. Hence, using
  again the formulae in Proposition \ref{p:formulae}, and Proposition
  \ref{p:TNT},
  \begin{align*}
    & \| \K^{1/2} T_t U_t T^* ( \N + \sqrt{N} \phi(\varphi) ) T \psi \|^2 \\
    & = \langle e^{-it \mathcal{H}_N} W (\N + \sqrt{N} \phi(\varphi)) T \psi,
    W_t \K W_t^* e^{-it \mathcal{H}_N} W (\N + \sqrt{N} \psi(\varphi)) T \psi
    \rangle \\
    & = N \| \nabla \varphi_t \|_{L^2}^2 \| (\N + \sqrt{N} \phi(\varphi)) T
    \psi \|^2 + \| \K^{1/2} e^{-it \mathcal{H}_N} W (\N + \sqrt{N}
    \psi(\varphi) ) T \psi \|^2 \\
    & \quad + \sqrt{N} \langle e^{-it \mathcal{H}_N} W (\N + \sqrt{N}
    \phi(\varphi)) T \psi, \phi(\Delta \varphi_t) e^{-it \mathcal{H}_N} W (\N
    + \sqrt{N} \psi(\varphi)) T \psi \rangle \\
    & \apprle N C^2 \| \nabla \varphi_t \|_{L^2}^2 \langle \psi, (\N^2 + N (\N
    + 1)) \psi \rangle + \| \mathcal{H}_{N}^{1/2} W (\N + \sqrt{N} \phi
    (\varphi) ) T \psi \|^2 \\
    & \quad + \sqrt{N} \| \Delta \varphi_t \|_{L^2}^2 \| (\N + \sqrt{N}
    \psi(\varphi) ) T \psi \| \, \| (\N+1)^{1/2} W (\N + \sqrt{N}
    \phi(\varphi) ) T \psi \| \\
    & \apprle N C^2 \| \varphi_t \|_{H^1}^2 \langle \psi, (\N^2 + N (\N + 1))
    \psi \rangle + N \langle W \phi(\varphi) T \psi, (\mathcal{H}_N + \N) W
    \phi(\varphi) T \psi \rangle \\
    & \quad + \langle W \N T \psi, (\mathcal{H}_N + \N) W \N T \psi \rangle.
  \end{align*}
  Therefore, combining the last two inequalities with \eqref{nk3}, we obtain
  \begin{align*}
    \langle U_t \psi, \K \N U_t \psi \rangle & \le C_2 \Big[ N \langle U_t
    \psi, \K U_t \psi \rangle + (1 + \| \nabla_1 s \|_{L^2}^2) \langle U_t
    \psi, (\N+1) U_t \psi \rangle \\
    & \quad + \big( 1 + \frac{1}{N} \| \nabla_1 s \|_{L^2}^2 \big) \langle
    \psi, (\N^2 + N (\N + 1)) \psi \rangle \\
    & \quad + \langle W \phi(\varphi) T \psi, (\mathcal{H}_N + \N) W
    \phi(\varphi) T \psi \rangle \\
    & \quad + \frac{1}{N} \langle W \N T \psi, (\mathcal{H}_N + \N) W \N T
    \psi \rangle \Big],
  \end{align*}
  where $C_2$ is a constant that depends only on $\| p \|_{L^2}$, $\| s
  \|_{L^2}$, $\| \varphi_t \|_{H^2}$ and $\| \nabla_1 p \|_{L^2}$.
\end{proof}


\section{Nonlinear Hartree and Gross-Pitaevskii equations}


\begin{lem}
  Let $\varphi \in H^2(\R^3)$ with $\| \varphi \|_{L^2} = 1$. Suppose that $f
  \in L^1(\R^3)$ and $V \in C_c^\infty(\R^3)$ with $fV \ge 0$. For $N > 0$,
  consider a solution $\varphi_t^{(N)} \in H^1(\R^3)$ of the nonlinear Hartree
  equation
  \begin{displaymath}
    i \partial_t \varphi_t^{(N)} = - \Delta \varphi_t^{(N)} + (N f_N V_N *
    |\varphi_t^{(N)}|^2) \varphi_t^{(N)}
  \end{displaymath}
  with initial data $\varphi^{(N)}_0 = \varphi$, where $f_N V_N(x) = N^2
  f(Nx)V(Nx)$. Consider also a solution $\varphi_t \in H^1(\R^3)$ of the
  nonlinear Gross-Pitaevskii equation
  \begin{displaymath}
    i \partial_t \varphi_t = - \Delta \varphi_t + 8 \pi a_0 |\varphi_t|^2
    \varphi_t
  \end{displaymath}
  with initial data $\varphi_0 = \varphi$, where $8 \pi a_0 = \int f V$. Then,
  for all $N > 0$ and $t \ge 0$,
  \begin{alignat}{2}
    \| \varphi_t^{(N)} \|_{H^1} & \le C, & \qquad \| \varphi_t \|_{H^1} &
    \le C, \tag{i} \\
    \| \varphi_t^{(N)} \|_{H^2} & \le \| \varphi \|_{H^2} e^{K t}, & \qquad
    \| \varphi_t \|_{H^2} & \le \| \varphi \|_{H^2} e^{K t}, \tag{ii}
  \end{alignat}
  where $C$ and $K$ are constants that depend only on $\| fV \|_{L^1}$,
  $\text{supp }V$ and $\| \varphi \|_{H^1}$. Furthermore,
  \begin{equation}
    \| \varphi_t^{(N)} - \varphi_t \|_{L^2} \le \frac{C}{N} e^{e^{K t}},
    \tag{iii}
  \end{equation}
  where $C$ and $K$ are constants that depend only on $\| fV \|_{L^1}$,
  $\text{supp }V$ and $\| \varphi \|_{H^2}$.
\end{lem}


\subsection{Introduction/conservation of energy}


Nonlinear Hartree equation
\begin{displaymath}
  i \partial_t \varphi_t = - \Delta \varphi_t + (V * |\varphi_t|^2) \varphi_t
\end{displaymath}
with initial data $\varphi_0 = \varphi$, and energy
\begin{displaymath}
  \mathcal{E}(\varphi) = \int dx \, |\nabla \varphi(x)|^2 + \frac{1}{2} \int
  dx \, (V * |\varphi|^2)(x) |\varphi(x)|^2.
\end{displaymath}


Nonlinear Gross-Pitaevskii equation
\begin{displaymath}
  i \partial_t \varphi_t = - \Delta \varphi_t + 8 \pi a_0 |\varphi_t|^2
  \varphi_t
\end{displaymath}
with initial data $\varphi_0 = \varphi$, and energy
\begin{displaymath}
  \mathcal{E}_{GP}(\varphi) = \int dx \, |\nabla \varphi(x)|^2 + 4 \pi a_0
  \int dx \, |\varphi(x)|^4,
\end{displaymath}
where $8 \pi a_0 = \int f V$.


\begin{prp}
  \label{p:energy}
  Let $\varphi \in H^1(\R^3)$ and $V \in L^1(\R^3)$. Then,
  \begin{displaymath}
    \mathcal{E}(\varphi) \apprle \| \varphi \|_{H^1}^2 + \| V \|_{L^1} \|
    \varphi \|_{H^1}^4 \qquad \text{and} \qquad \mathcal{E}_{GP}(\varphi)
    \apprle \| \varphi \|_{H^1}^2 + |a_0| \| \varphi \|_{L^2} \| \varphi
    \|_{H^1}^3.
  \end{displaymath}
\end{prp}


\begin{proof}
  By Young's inequality, Sobolev's inequality, and H\"older's inequality, 
  \begin{displaymath}
    \int dx \, (V * |\varphi|^2)(x) |\varphi(x)|^2 \le \| V \|_{L^1} \|
    \varphi^2 \|_{L^2}^2 = \| V \|_{L^1} \| \varphi \|_{L^4}^4 \apprle \| V
    \|_{L^1} \| \varphi \|_{H^1}^4
  \end{displaymath}
  and
  \begin{displaymath}
    \int dx \, |\varphi(x)|^4 \le \| \varphi \|_{L^2} \| \varphi^3 \|_{L^2} =
    \| \varphi \|_{L^2} \| \varphi \|_{L^6}^3 \apprle \| \varphi \|_{L^2} \|
    \varphi \|_{H^1}^3.
  \end{displaymath}
  Observing that $\| \varphi \|_{H^1}^2 = \| \varphi \|_{L^2}^2 + \| \nabla
  \varphi \|_{L^2}^2$, we obtain the desired estimates.
\end{proof}


\subsection{$H^2$-regularity}


\begin{prp}
  \label{p:reg1}
  Let $\varphi \in H^2(\R^3)$ with $\| \varphi \|_{L^2} = 1$ and $V \in
  L^1(\R^3)$ with $V \ge 0$. Consider a solution $\varphi_t \in H^1(\R^3)$ of
  the nonlinear Hartree equation
  \begin{displaymath}
    i \partial_t \varphi_t = - \Delta \varphi_t + (V * |\varphi_t|^2)
    \varphi_t
  \end{displaymath}
  with initial data $\varphi_0 = \varphi$. Then, there exists a real number $T
  > 0$ such that
  \begin{displaymath}
    \| \varphi_t \|_{H^2} \le C \| \varphi \|_{H^2}
  \end{displaymath}
  for all $t \in [0,T]$, where the constants $C$ and $T$ depend only on $\| V
  \|_{L^1}$ and $\| \varphi \|_{H^1}$.
\end{prp}


\begin{cor}
  \label{c:reg1}
  Under the hypothesis of Proposition \ref{p:reg1}, for all $t \ge 0$,
  \begin{displaymath}
    \| \varphi_t \|_{H^2} \le \| \varphi \|_{H^2} e^{Kt},
  \end{displaymath}
  where $K$ is a constant that depends only on $\| V \|_{L^1}$ and $\| \varphi
  \|_{H^1}$.
\end{cor}


\begin{proof}
  Recall that $V \ge 0$. By conservation of mass and energy, and by
  Proposition \ref{p:energy},
  \begin{displaymath}
    \| \varphi_t \|_{H^1}^2 \le \| \varphi_t \|_{L^2}^2 +
    \mathcal{E}(\varphi_t) = 1 + \mathcal{E}(\varphi) \apprle \| \varphi
    \|_{H^1}^2 + \| V \|_{L^1} \| \varphi \|_{H^1}^4.
  \end{displaymath}
  Hence, by Proposition \ref{p:reg1}, given $t > 0$, there are constants $C$
  and $T$, depending only on $\| V \|_{L^1}$ and $\| \varphi \|_{H^1}$, such
  that
  \begin{displaymath}
    \| \varphi_t \|_{H^2} \le C \| \varphi_{nT} \|_{H^2}
  \end{displaymath}
  for some $n \ge 0$ with $nT < t$. Iterating this argument backwards in time
  (and again noticing that $C$ and $T$ can be chosen uniformly in time) one
  easily finds that
  \begin{displaymath}
    \| \varphi_t \|_{H^2} \le C \| \varphi_{nT} \|_{H^2} \le C^{n+1} \|
    \varphi \|_{H^2} = (C^{1/T})^{(n+1)T}
    \| \varphi \|_{H^2} \le e^{Kt} \| \varphi \|_{H^2},
  \end{displaymath}
  for some constant $K$ depending only on $\| V \|_{L^1}$ and $\| \varphi
  \|_{H^1}$.
\end{proof}


\begin{prp}
  \label{p:reg2}
  Let $\varphi \in H^2(\R^3)$ with $\| \varphi \|_{L^2} = 1$ and $a_0 \in \R$
  with $a_0 \ge 0$. Consider a solution $\varphi_t \in H^1(\R^3)$ of the
  nonlinear Gross-Pitaevskii equation
  \begin{displaymath}
    i \partial_t \varphi_t = - \Delta \varphi_t + 8 \pi a_0 |\varphi_t|^2
    \varphi_t
  \end{displaymath}
  with initial data $\varphi_0 = \varphi$. Then, there exists a real number $T
  > 0$ such that
  \begin{displaymath}
    \| \varphi_t \|_{H^2} \le C \| \varphi \|_{H^2}
  \end{displaymath}
  for all $t \in [0,T]$, where the constants $C$ and $T$ depend only on $a_0$
  and $\| \varphi \|_{H^1}$.
\end{prp}


\begin{cor}
  \label{c:reg2}
  Under the hypothesis of Proposition \ref{p:reg2}, for all $t \ge 0$,
  \begin{displaymath}
    \| \varphi_t \|_{H^2} \le \| \varphi \|_{H^2} e^{Kt},
  \end{displaymath}
  where $K$ is a constant that depends only on $a_0$ and $\| \varphi
  \|_{H^1}$.
\end{cor}


The proofs of Propositions \ref{p:reg1} and \ref{p:reg2} are based on
well-known Strichartz estimates for the free Schr\"odinger evolution and the
following lemma.


For $T > 0$ set
\begin{displaymath}
  L_t^q L_x^r = L^q([0,T], L^r(\R^3)).
\end{displaymath}


\begin{lem}
  \label{l:interp}
  Let $V \in L^1(\R^3)$, $f \in L_t^{q_1} L_x^{r_1}$ and $g \in L_t^{q_2}
  L_x^{r_2}$ with $q_j, r_j \in [1,\infty]$ for $j \in \{1,2\}$. Then $(V * f)
  g \in L_t^q L_x^r$ with $q^{-1} = s^{-1} + q_1^{-1} + q_2^{-1}$ and $r^{-1}
  = r_1^{-1} + r_2^{-1}$ for $s \in [1, \infty]$. Furthermore,
  \begin{displaymath}
    \| (V * f)g \|_{L_t^q L_x^r} \le \| V \|_{L^1} T^{1/s} \| f \|_{L_t^{q_1}
    L_x^{r_1}} \| g \|_{L_t^{q_2} L_x^{r_2}}.
  \end{displaymath}
\end{lem}


\begin{proof}
  Applying H\"older's inequality in space, and then in time, we find that
  \begin{displaymath}
    \| (V * f) g \|_{L_t^q L_x^r} \le \| V * f \|_{L_t^u L_x^{r_1}} \| g
    \|_{L_t^{q_2} L_x^{r_2}}
  \end{displaymath}
  with $q^{-1} = u^{-1} + q_2^{-1}$ and $r^{-1} = r_1^{-1} + r_2^{-1}$. Now,
  applying Young's inequality in space, H\"older's inequality in time, and
  observing that $V$ is time-independent, we get
  \begin{displaymath}
    \| V * f \|_{L_t^u L_x^{r_1}} \le \| V \|_{L^1} T^{1/s} \| f \|_{L_t^{q_1}
    L_x^{r_1}}
  \end{displaymath}
  with $u^{-1} = s^{-1} + q_1^{-1}$. Combining all this we obtain the desired
  result.
\end{proof}


\begin{proof}[Proof of Proposition \ref{p:reg1}]
  Write
  \begin{displaymath}
    \varphi_t = e^{it\Delta} \varphi - i \int_0^t ds \, e^{i(t-s)\Delta} (V *
    |\varphi_s|^2) \varphi_s.
  \end{displaymath}
  Differentiating this equation and using integration by parts we find that
  \begin{displaymath}
    \Delta \varphi_t = e^{it \Delta} \Delta \varphi - i \int_0^t ds \,
    e^{i(t-s) \Delta} \big[ (V * |\varphi_s|^2) \Delta \varphi_s + 2(V *
    \nabla |\varphi_s|^2) \nabla \varphi_s + (V * \Delta |\varphi_s|^2)
    \varphi_s \big].
  \end{displaymath}


  The $L_t^\infty L_x^2$-norm of the above expression can be controlled using
  Strichartz estimates for the free Schr\"odinger evolution $e^{it\Delta}$
  (see \cite[Theorem 1.2]{KT}). By these estimates, and Lemma~\ref{l:interp}, \marginpar{in the following equations, is it really $\varphi^2$ or maybe $\lvert \varphi\rvert^2$?}
  \begin{align}
    & \| \Delta \varphi_{(\cdot)} \|_{L_t^\infty L_x^2} \notag \\
    & \apprle \| \Delta \varphi \|_{L^2} + \| (V * |\varphi_{(\cdot)}|^2)
    \Delta \varphi_{(\cdot)} + 2(V * \nabla |\varphi_{(\cdot)}|^2) \nabla
    \varphi_{(\cdot)} + (V * \Delta |\varphi_{(\cdot)}|^2) \varphi_{(\cdot)}
    \|_{L_t^2 L_x^{6/5}} \notag \\
    & \apprle \| \Delta \varphi \|_{L^2} + \| V \|_{L^1} T^{1/2} \Big[ \sup_{t
    \in [0,T]} \| \varphi_t^2 \|_{L^3} \sup_{t \in [0,T]} \| \Delta \varphi_t
    \|_{L^2} \notag \\
    & \quad + \sup_{t \in [0,T]} \| \nabla |\varphi_t|^2 \|_{L^{3/2}} \sup_{t
    \in [0,T]} \| \nabla \varphi_t \|_{L^6} + \sup_{t \in [0,T]} \| \Delta
    |\varphi_t|^2 \|_{L^{3/2}} \sup_{t \in [0,T]} \| \varphi_t \|_{L^6} \Big].
    \label{Lap}
  \end{align}
  We next estimate each term in this inequality.
  
  
  Applying Sobolev's inequality, Leibniz' rule, and H\"older's inequality,
  we find that
  \begin{displaymath}
    \| \varphi_t^2 \|_{L^3} = \| \varphi_t \|_{L^6}^2 \apprle \| \varphi_t
    \|_{H^1}^2
  \end{displaymath}
  and
  \begin{displaymath}
    \| \nabla |\varphi_t|^2 \|_{L^{3/2}} \le 2 \| \overline{\varphi_t} \nabla
    \varphi_t \|_{L^{3/2}} \le 2 \| \varphi_t \|_{L^6} \| \nabla \varphi_t
    \|_{L^2} \apprle \| \varphi_t \|_{H^1}^2
  \end{displaymath}
  and
  \begin{align*}
    \| \Delta |\varphi_t|^2 \|_{L^{3/2}} & \le 2 \| \overline{\varphi_t}
    \Delta \varphi_t \|_{L^{3/2}} + 2 \| \nabla \overline{\varphi_t} \nabla
    \varphi_t \|_{L^{3/2}} \\
    & \le 2 \| \varphi_t \|_{L^6} \| \Delta \varphi_t \|_{L^2} + 2\| \nabla
    \varphi_t \|_{L^2} \| \nabla \varphi_t \|_{L^6} \apprle \| \varphi_t
    \|_{H^1} \| \varphi_t \|_{H^2}.
  \end{align*}
  Recall that $V \ge 0$. By conservation of mass and energy, and by
  Proposition \ref{p:energy},
  \begin{displaymath}
    \| \varphi_t \|_{H^1}^2 \le \| \varphi_t \|_{L^2}^2 +
    \mathcal{E}(\varphi_t) = 1 + \mathcal{E}(\varphi) \le C,
  \end{displaymath}
  where $C$ is a constant that depends only on $\| V \|_{L^1}$ and $\| \varphi
  \|_{H^1}$. Thus, substituting all this into \eqref{Lap}, and using Sobolev's
  inequality again, we get
  \begin{align*}
    \sup_{t \in [0,T]} \| \Delta \varphi_t \|_{L^2} & \apprle \| \Delta
    \varphi \|_{L^2} + \| V \|_{L^1} T^{1/2} \sup_{t \in [0,T]} \| \varphi_t
    \|_{H^1}^2 \sup_{t \in [0,T]} \| \varphi_t \|_{H^2} \\ & \apprle \|
    \Delta \varphi \|_{L^2} + \| V \|_{L^1} T^{1/2} C \sup_{t \in [0,T]} \|
    \varphi_t \|_{H^2}.
  \end{align*}
  Therefore,
  \begin{align*}
    \sup_{t \in [0,T]} \| \varphi_t \|_{H^2} & \apprle \sup_{t \in [0,T]}
    \big( \| \varphi \|_{L^2} + \| \nabla \varphi_t \|_{L^2} + \| \Delta
    \varphi_t \|_{L^2} \big) \\
    & \apprle 1 + C^{1/2} + \| \Delta \varphi \|_{L^2} + \| V \|_{L^1} T^{1/2}
    C \sup_{t \in [0,T]} \| \varphi_t \|_{H^2}.
  \end{align*}
  Hence, by choosing $T > 0$ sufficiently small, but depending only on $\| V
  \|_{L^1}$ and $\| \varphi \|_{H^1}$, we conclude that
  \begin{displaymath}
    \sup_{t \in [0,T]} \| \varphi_t \|_{H^2} \le C (1 + \| \Delta \varphi
    \|_{L^2}) \le 2C \| \varphi \|_{H^2},
  \end{displaymath}
  where $C$ is a constant that depends only on $\| V \|_{L^1}$ and $\| \varphi
  \|_{H^1}$.
\end{proof}


\subsection{$\varphi_t^{(N)}$ converges to $\varphi_t$ in $L^2$}


\begin{lem}
  Let $\varphi \in H^2(\R^3)$ with $\| \varphi \|_{L^2} = 1$. Suppose that $f
  \in L^1(\R^3)$ and $V \in C_c^\infty(\R^3)$ with $fV \ge 0$. For $N > 0$,
  consider a solution $\varphi_t^{(N)} \in H^1(\R^3)$ of the nonlinear Hartree
  equation
  \begin{displaymath}
    i \partial_t \varphi_t^{(N)} = - \Delta \varphi_t^{(N)} + (N f_N V_N *
    |\varphi_t^{(N)}|^2) \varphi_t^{(N)}
  \end{displaymath}
  with initial data $\varphi^{(N)}_0 = \varphi$, where $f_N V_N(x) = N^2
  f(Nx)V(Nx)$. Consider also a solution $\varphi_t \in H^1(\R^3)$ of the
  nonlinear Gross-Pitaevskii equation
  \begin{displaymath}
    i \partial_t \varphi_t = - \Delta \varphi_t + 8 \pi a_0 |\varphi_t|^2
    \varphi_t
  \end{displaymath}
  with initial data $\varphi_0 = \varphi$, where $8 \pi a_0 = \int f V$. Then,
  for all $N > 0$ and $t \ge 0$,
  \begin{displaymath}
    \| \varphi_t^{(N)} - \varphi_t \|_{L^2} \le \frac{C}{N} e^{e^{K t}},
  \end{displaymath}
  where $C$ and $K$ are constants that depend only on $\| fV \|_{L^1}$,
  $\text{supp }V$ and $\| \varphi \|_{H^2}$. 
\end{lem}

\begin{proof}
  Observing that $\langle \Delta \varphi_t, \varphi_t^{(N)} \rangle = \langle
  \varphi_t, \Delta \varphi_t^{(N)} \rangle$, one easily finds
  \begin{equation}
    \label{ddt}
    \begin{aligned}
      \partial_t \| \varphi_t - \varphi_t^{(N)} \|_{L^2}^2 & = -2 \Im \langle
      \varphi_t, (N f_N V_N * |\varphi_t^{(N)}|^2 - 8 \pi a_0 |\varphi_t|^2)
      \varphi_t^{(N)} \rangle \\
      & = 2 \Im \langle \varphi_t, (N f_N V_N * |\varphi_t^{(N)}|^2 - 8 \pi
      a_0 |\varphi_t|^2) (\varphi_t - \varphi_t^{(N)}) \rangle \\
      & = 2 \Im \langle \varphi_t, (N f_N V_N * |\varphi_t|^2 - 8 \pi a_0
      |\varphi_t|^2) (\varphi_t - \varphi_t^{(N)}) \rangle \\
      & \quad + 2 \Im \langle \varphi_t, N f_N V_N * (|\varphi_t^{(N)}|^2 -
      |\varphi_t|^2) (\varphi_t - \varphi_t^{(N)}) \rangle.
    \end{aligned}
  \end{equation}
  In order to apply Gronwall's inequality, we next estimate each term in this
  expression.


  By H\"older's inequality, and Sobolev's inequality,
  \begin{align*}
    & |\langle \varphi_t, (N f_N V_N * |\varphi_t|^2 - 8 \pi a_0
    |\varphi_t|^2) (\varphi_t - \varphi_t^{(N)}) \rangle| \\
    & \qquad \le \| \varphi_t \|_{L^6} \| (N f_N V_N * |\varphi_t|^2 - 8 \pi
    a_0 |\varphi_t|^2) (\varphi_t - \varphi_t^{(N)}) \|_{L^{6/5}} \\
    & \qquad \apprle \| \varphi_t \|_{H^1} (\| \varphi_t \|_{L^2} + \|
    \varphi_t^{(N)} \|_{L^2}) \| N f_N V_N * |\varphi_t|^2 - 8 \pi a_0
    |\varphi_t|^2 \|_{L^3}.
  \end{align*}
  By triangle inequality, Sobolev's inequality, and Young's inequality,
  \begin{align*}
    & |\langle \varphi_t, N f_N V_N * (|\varphi_t^{(N)}|^2 - |\varphi_t|^2)
    (\varphi_t - \varphi_t^{(N)}) \rangle| \\
    & \le \int dx dy \, |\varphi_t(x)| |\varphi_t(x) - \varphi_t^{(N)}(x)| N
    f_N V_N(x-y) (|\varphi_t^{(N)}(y)| + |\varphi_t(y)|) |\varphi_t^{(N)}(y) -
    \varphi_t(y)| \\
    & \le \| \varphi_t \|_{L^\infty} ( \| \varphi_t^{(N)} \|_{L^\infty} + \|
    \varphi_t \|_{L^\infty} ) \int dx dy \, |\varphi_t(x) -
    \varphi_t^{(N)}(x)| N f_N V_N(x-y) |\varphi_t^{(N)}(y) - \varphi_t(y)| \\
    & \apprle \| \varphi_t \|_{H^2} ( \| \varphi_t^{(N)} \|_{H^2} + \|
    \varphi_t \|_{H^2} ) \| fV \|_{L^1} \| \varphi_t^{(N)} - \varphi_t
    \|_{L^2}^2.
  \end{align*}
  Recall that $V \ge 0$. By conservation of mass and energy, and by
  Proposition \ref{p:energy},
  \begin{align*}
    \| \varphi_t^{(N)} \|_{H^1}^2 & \le \| \varphi_t^{(N)} \|_{L^2}^2 +
    \mathcal{E}_N(\varphi_t^{(N)}) = 1 + \mathcal{E}_N(\varphi) \apprle \|
    \varphi \|_{H^1}^2 + \| fV \|_{L^1} \| \varphi \|_{H^1}^4, \\
    \| \varphi_t \|_{H^1}^2 & \le \| \varphi_t \|_{L^2}^2 +
    \mathcal{E}_{GP}(\varphi_t) = 1 + \mathcal{E}_{GP}(\varphi) \apprle \|
    \varphi \|_{H^1}^2 + \| fV \|_{L^1} \| \varphi \|_{H^1}^4.
  \end{align*}
  Thus, substituting all this into \eqref{ddt}, and using Corollaries
  \ref{c:reg1} and \ref{c:reg2} to bound $\| \varphi_t^{(N)} \|_{H^2}$ and $\|
  \varphi_t \|_{H^2}$, we obtain
  \begin{equation}
    \label{endproof2}
    \partial_t \| \varphi_t^{(N)} - \varphi_t \|_{L^2}^2 \le C \| \varphi
    \|_{H^2}^2 e^{Kt} \| \varphi_t^{(N)} - \varphi_t \|_{L^2}^2 + C \| N f_N
    V_N * |\varphi_t|^2 - 8 \pi a_0 |\varphi_t|^2 \|_{L^3},
  \end{equation}
  where $C$ and $K$ are constants that depend only on $\| fV \|_{L^1}$ and $\|
  \varphi \|_{H^1}$. We are left to estimating the second term in
  \eqref{endproof2}.


  Write
  \begin{align*}
    N f_N V_N * |\varphi_t|^2(x) - 8 \pi a_0 |\varphi_t|^2(x) & = \int dy
    \big( |\varphi_t(x-y)|^2 - |\varphi_t(x)|^2 \big) N^3 fV(Ny) \\
    & = \int dz \big( |\varphi_t(x-z/N)|^2 - |\varphi_t(x)|^2 \big) fV(z).
  \end{align*}
  Let $R$ be such that $\text{supp }V \subset \{ x \in \R^3 \; | \;\; |x| \le
  R \}$. By Minkowski's, H\"older's, and Sobolev's inequalities,
  \begin{align*}
    \| N f_N V_N * |\varphi_t|^2 - 8 \pi a_0 |\varphi_t|^2 \|_{L^3} & \le \int
    dz \, \| |\varphi_t(\, \cdot \, -z/N)|^2 - |\varphi_t|^2 \|_{L^3} |fV(z)|
    \\
    & \le \| fV \|_{L^1} \sup_{|z| \le R} \| |\varphi_t(\, \cdot \, - z/N)|^2
    - |\varphi_t|^2 \|_{L^3}.
  \end{align*}
  Given $\varepsilon = 1/N$, there exists $\psi_t \in C^\infty(\R^3)$ such
  that $\| \varphi_t - \psi_t \|_{H^2} < 1/N$. Hence, by H\"older's
  inequality, Sobolev's inequality and an $\varepsilon/3$-argument, the mean
  value theorem (with some constant $0 \le c \le 1$), and Sobolev's inequality
  again,
  \begin{align*}
    \| |\varphi_t(\, \cdot \, - z/N)|^2 - |\varphi_t|^2 \|_{L^3} & \le 2 \|
    \varphi_t \|_{L^6} \| |\varphi_t(\, \cdot \, - z/N)| - |\varphi_t|
    \|_{L^6} \\
    & \apprle \| \varphi_t \|_{H^1} \big( 1/N + \| |\psi_t(\, \cdot \, - z/N)|
    - |\psi_t| \|_{L^6} \big) \\
    & \apprle \| \varphi_t \|_{H^1} \big( 1/N + |z|/N \| \nabla |\psi_t(\,
    \cdot \, - c z/N)| \|_{L^6} \big) \\
    & \apprle \| \varphi_t \|_{H^1} \big( 1/N + |z|/N \| \psi_t \|_{H^2}
    \big) \\
    & \apprle \| \varphi_t \|_{H^1} \big( 1/N + |z|/N^2 + |z|/N \| \varphi_t
    \|_{H^2} \big).
  \end{align*}
  Therefore, using again Corollary \ref{c:reg2},
  \begin{align*}
    & \| N f_N V_N * |\varphi_t|^2 - 8 \pi a_0 |\varphi_t|^2 \|_{L^3} \\
    & \apprle \| fV \|_{L^1} \| \varphi_t \|_{H^1} \Big( \frac{1}{N} +
    \frac{R}{N^2} + \frac{R}{N} \| \varphi_t \|_{H^2} \Big) \le
    \frac{C}{N}(1 + \| \varphi_t \|_{H^2}) \le \frac{2C}{N} \| \varphi
    \|_{H^2} e^{Kt},
  \end{align*}
  where $C$ is a constant that depends only on $\| fV \|_{L^1}$, $\text{supp
  }V$ and $\| \varphi \|_{H^1}$. Substituting this into \eqref{endproof2}, we
  obtain
  \begin{displaymath}
    \partial_t \| \varphi_t^{(N)} - \varphi_t \|_{L^2}^2 \le C e^{Kt} \|
    \varphi_t^{(N)} - \varphi_t \|_{L^2}^2 + \frac{C}{N} e^{Kt}.
  \end{displaymath}
  By Gronwall's inequality,
  \begin{displaymath}
    \| \varphi_t^{(N)} - \varphi_t \|_{L^2} \le e^{K^{-1} C e^{Kt}} \Big( \|
    \varphi_0^{(N)} - \varphi_0 \|_{L^2} + \frac{C}{KN} e^{Kt} \Big) \le
    \frac{C_2}{N} e^{e^{K_2 t}},
  \end{displaymath}
  for some constants $C_2$ and $K_2$ depending only on $\| fV \|_{L^1}$,
  $\text{supp }V$ and $\| \varphi \|_{H^2}$.
\end{proof}

\subsection{Regularity of $\phdot$}
\begin{lem}[$L^2$-bound for $\phdot$]
\label{lm:phdotregularity}
Let $\ph$ be a solution of the modified Hartree equation. Then
 \bd
\norm{\phdot}_{L^2} \leq \norm{\ph}_{H^2} + 8\pi a_0 \norm{\ph}_{H^2}^2.
\ed
For $k(x,y) = -N w_N(x-y) \ph(x) \ph(y)$, we have the bound
\bd
\norm{\dot k} \leq 4 \max\{R,a_0\} \norm{\ph}_{H^1} \left( \norm{\ph}_{H^2} + 8\pi a_0 \norm{\ph}_{H^2}^2\right) \leq C \norm{\ph}_{H^2}^2,
\ed
where $C$ is a constant independent of $N$ and $t$.
\end{lem}
\begin{proof} The modified Hartree equation reads
\bd
i \phdot = -\Delta \ph + \left(N f_N V_N \ast \lvert \ph \rvert^2 \right) \ph.
\ed
It follows that
\bd
\norm{\phdot}_{L^2} \leq \norm{\ph}_{H^2} + \norm{\left(N f_N V_N \ast \lvert \ph\rvert^2 \right)\ph}_{L^2}.
\ed
Now we calculate
\begin{align*}
& \norm{\left(N f_N V_N \ast \lvert \ph\rvert^2 \right)\ph}_{L^2} \\
& = \int \di x\, \lvert \ph(x)\rvert^2 \left\lvert \int \di y\, N f_NV_N(x-y) \lvert \ph(y)\rvert^2 \right\rvert^2 \\
& \leq \int \di x\, \lvert \ph(x)\rvert^2 \left\lvert \norm{\ph}_\infty^2 \int \di y\, N f_N V_N(x-y) \right\rvert^2 \\
& \leq \norm{\ph}_{L^2}^2 \norm{\ph}_\infty^4 (8\pi a_0)^2.
\end{align*}
We now prove the bound for $\norm{\dot k}$ using Hardy's inequality.
\begin{align*}
\norm{\dot k}^2 & = \int \di x\di y\, \left\lvert \frac{\di}{\di t} \left( -N w_N(x-y) \ph(x)\ph(y) \right) \right\rvert^2 \\
& \leq 2 \int \di x\di y\, \lvert N w_N(x-y) \phdot(x) \ph(y) \rvert^2 + 2 \int \di x \di y\, \lvert N w_N(x-y) \ph(x) \phdot(y) \rvert^2 \\
& \leq 4 \left( \max\{R,a_0\} \right)^2 \int \di x\di y\, \lvert \phdot(x)\rvert^2 \frac{\lvert\ph(y)\rvert^2}{\lvert x-y\rvert^2} \\
& \leq 16 \left( \max\{R,a_0\} \right)^2 \int \di x\di y\, \lvert \phdot(x)\rvert^2 \lvert \nabla_y \ph(y)\rvert^2 \\
& \leq \left( 4 \max\{R,a_0\} \right)^2 \norm{\phdot}_{L^2}^2 \norm{\ph}_{H^1}^2.
\end{align*}
Now use the estimate from the first part of the lemma to estimate $\norm{\phdot}_{L^2}^2$ and the proof is complete.
\end{proof}

\begin{lem}
We need estimates for $\norm{\ddot k}$.\marginpar{TODO}
\end{lem}
\begin{proof}
To prove this lemma, we need higher regularity (probably $H^4$) of $\ph$, because bounding $\ddot k$ needs $\norm{\phddot}_{L^2}$.  
\end{proof}

\bibliographystyle{plain}
\bibliography{gross-pitaevskii}


\end{document}
