%\documentclass[11pt,a4paper,twoside,headsepline]{scrartcl}
\documentclass[11pt,a4paper]{scrartcl} % Added by GBdO
\usepackage[a4paper,hmargin=2.8cm,vmargin=2.8cm]{geometry} % Added by GBdO
\usepackage[utf8x]{inputenc}
%\usepackage{showkeys}
\usepackage[bookmarksnumbered=true]{hyperref} % Added by GBdO
\usepackage[color,notref,notcite]{showkeys} % Added by GBdO
\usepackage{scrtime}
\usepackage{amsmath, amsthm, amssymb}
\usepackage{dsfont,wasysym} % added by GBdO

% Original header
%\usepackage{scrpage2}
%\ihead{Gross-Pitaevskii Equation}
%\chead{Page: \thepage}
%\ohead{Version: \today, \thistime}
%\pagestyle{scrheadings}

% How about this footer?
\usepackage{fancyhdr}
\pagestyle{fancy}
\fancyhf{}
\renewcommand{\headrulewidth}{0pt}
\lfoot{{\footnotesize GP equation}}
\cfoot{\thepage}
\rfoot{{\footnotesize \today, \thistime}}

% Nicer marginpar with smaller font
\let\oldmarginpar\marginpar
\renewcommand\marginpar[1]{\-\oldmarginpar[\raggedleft\footnotesize #1]%
  {\raggedright\footnotesize #1}}

%%%%%%%%%%%%%%%%%%%%%%%%% Theorems etc %%%%%%%%%%%%%%%%%%%%%%%%%%%%%%%%%%%%%%%%%

\newtheorem{thm}{Theorem}[section]
\newtheorem{cor}[thm]{Corollary}
\newtheorem{prp}[thm]{Proposition}
\newtheorem{lem}[thm]{Lemma}
\newtheorem{dfn}[thm]{Definition}
\newtheorem{exm}[thm]{Example}

\newtheorem*{rem}{Remark}
\newtheorem*{hyp}{Hypothesis}

%%%%%%%%%%%%%%%%%%%%%%%% Gustavo's aliases %%%%%%%%%%%%%%%%%%%%%%%%%%%%%%%%%%%%%

\newcommand{\R}{\mathds{R}}
\newcommand{\N}{\mathcal{N}}
\newcommand{\K}{\mathcal{K}}

%%%%%%%%%%%%%%%%%%%%%%%% Notation %%%%%%%%%%%%%%%%%%%%%%%%%%%%%%%%%%%%%%%%%%%%%%

\newcommand{\di}{\textrm{d}}		% differential (for integrals)
\newcommand{\Lcal}{\mathcal{L}}		% calligraphic L
\newcommand{\Ncal}{\mathcal{N}}		% calligraphic N
\newcommand{\Kcal}{\mathcal{K}}		% calligraphic N
\newcommand{\Vcal}{\mathcal{V}}		% calligraphic V
\newcommand{\Hcal}{\mathcal{H}}		% calligraphic H
\newcommand{\Ocal}{\mathcal{O}}		% big-O, order-of
\newcommand{\tilV}{\tilde{\mathcal{V}}_N}		% symbol for the smeared potential energy part of the hamiltonian
\newcommand{\estlist}[2]{\underline{Line \ref{l#1}, summand #2:}}
\newcommand{\hc}{\mbox{h.c.}}		%hermitian conjugate
\newcommand{\scal}[2]{\big<#1,#2\big>} % scalaer product
\newcommand{\cc}[1]{\overline{#1}}	% complex conjugate
\newcommand{\Rbb}{\mathbb{R}}		% real numbers
\newcommand{\Cbb}{\mathbb{C}}		% complex numbers
\newcommand{\Nbb}{\mathbb{N}}		% natural numbers
\renewcommand{\Re}{\operatorname{Re}\,} 	%RealPart
\renewcommand{\Im}{\operatorname{Im}\,} 	%ImaginaryPart
\newcommand{\norm}[1]{\lVert#1\rVert}	%Norm
\newcommand{\ev}[1]{\big<#1\big>}	%expectation value
\newcommand{\ph}{\varphi_t^{(N)}}	% solution of N-dependent Hartree equation
\newcommand{\phdot}{\dot{\varphi}_t^{(N)}}	% time derivative of solution of N-dependent Hartree equation
\newcommand{\sqn}{\sqrt{N}}		% square root of N
\newcommand{\project}[1]{\lvert #1 \big>\big< #1\rvert}	% orthogonal projection operator
\newcommand{\Tr}{\operatorname{Tr}}	% Trace
\newcommand{\dxyNV}{\frac{1}{2}\int \di x\di y N V_N(x-y)} % abbreviation for the one-half integral dx dy NV_N(x-y) which appears everywhere
\newcommand{\dxyV}{\frac{1}{2}\int \di x\di y V_N(x-y)} % abbreviation for the one-half integral dx dy V_N(x-y) which appears everywhere

\newcommand{\be}[1]{\begin{equation}\label{eq:#1}}	%begin equation with label
\newcommand{\ee}{\end{equation}}
\newcommand{\bd}{\begin{displaymath}}			% abbreviation begin displaymath
\newcommand{\ed}{\end{displaymath}}

\newcommand{\tagg}[1]{ \stepcounter{equation} \tag{\theequation} \label{eq:#1} } % add tag and label in align*-environments

\newcommand{\eqr}[1]{\eqref{eq:#1}}			%eqref with prefix :eq

%%%%%%%%%%%%%%%%%%%%%%%%% main content %%%%%%%%%%%%%%%%%%%%%%%%%%%%%%%


%\allowdisplaybreaks    % Let's try to avoid this. GBdO.


\begin{document}


\tableofcontents


\newpage


\section{Fock space representation}


We define the bosonic Fock space $\mathcal{F}$ over $L^2(\R^3)$ as the Hilbert
space
\[
  \mathcal{F} = \bigoplus_{n \ge 0} L^2(\R^3)^{\otimes_s n} = \mathds{C}
  \oplus \bigoplus_{n \ge 1} L^2_s(\R^{3n}),
\]
with the convention that $L^2(\R^3)^{\otimes_s 0} = \mathds{C}$. Here
$L^2_s(\R^{3n})$ is the subspace of $L^2(\R^{3n})$ consisting of all functions
that are symmetric with respect to arbitrary permutations of the $n$
variables. Vectors in $\mathcal{F}$ are sequences $\psi = \{ \psi^{(n)} \}_{n
\ge 0}$ of $n$-particle wave functions $\psi^{(n)} \in L^2_s(\R^{3n})$. The
scalar product on $\mathcal{F}$ is defined as
\begin{align*}
  \langle \psi_1, \psi_2 \rangle & = \sum_{n \ge 0} \langle \psi_1^{(n)},
  \psi_2^{(n)} \rangle_{L^2(\R^{3n})} \\
  & = \overline{\psi_1^{(0)}} \psi_2^{(0)} + \sum_{n \ge 1} \int dx_1 \cdots
  dx_n \overline{\psi_1^{(n)}}(x_1, \dots, x_n) \psi_2^{(n)}(x_1, \dots, x_n).
\end{align*}
An $N$ particle state with wave function $\psi_N$ is described on
$\mathcal{F}$ by the sequence $\{ \psi^{(n)} \}_{n \ge 0}$ where $\psi^{(n)}
= 0$ for all $n \neq N$ and $\psi^{(N)} = \psi_N$. The vector $\{ 1, 0, 0,
\dots \} \in \mathcal{F}$ is called the vacuum, and will be denote by
$\Omega$.


We will make use of operator valued distributions $a_x^*$ and $a_x$, with $x
\in \R^3$, defined so that
\begin{equation}
  \label{a}
  \begin{aligned}
    a^*(f) & = \int \di x\, f(x) a^*_x, \\
      a(f) & = \int \di x\, \cc{f(x)} a_x.
  \end{aligned}
\end{equation}
It is also convenient to introduce the notation
\marginpar{Either this or $a(\nabla f)$. I'll think about it.}
\bd
\mbox{For } (\nabla a)(f) := \int \di x\, \cc{f(x)}\nabla_x a_x, \mbox{ we
have } \norm{(\nabla a)(f)\psi} \leq \norm{f}_{L^2} \norm{\Kcal^{1/2}\psi}.
\ed


\begin{lem}
  \label{l:a}
  Let $f \in H^1(\R^3)$. Then, for any $\psi \in \mathcal{F}$,
  \begin{equation}
    \label{aNorm}
    \begin{aligned}
      \norm{a(f)\psi} & \leq \norm{f}_{L^2} \norm{\Ncal^{1/2}\psi}, \\
      \norm{a(\nabla f)\psi} & \leq \norm{f}_{L^2} \norm{\Kcal^{1/2}\psi}, \\
      \norm{a^*(f)\psi} & \leq \norm{f}_{L^2} \norm{(\Ncal+1)^{1/2}\psi}, \\
      \norm{\phi(f) \psi} & \leq 2 \norm{f}_{L^2} \norm{(\N+1)^{1/2} \psi}.
    \end{aligned}
  \end{equation}
\end{lem}


Hamiltonian etc.
\bd
\Kcal := \int \di x\, \nabla_x a^*_x \nabla_x a_x,
\quad
\Vcal_N := \frac{1}{2}\int\di x \di y\, V_N(x-y) a^*_x a^*_y a_y a_x
\ed
\bd
V_N(x) := N^2V(Nx), \quad \Hcal := \Kcal + \Vcal_N, \quad \Ncal := \int \di
x\, a^*_x a_x
\ed


For $f \in L^2(\R^3)$, we define the Weyl operator on $\mathcal{F}$ as
\bd
%A(f) := a^*(f) - a(f), \quad W(f) := \exp(A(f))
W(f) = \exp(a^*(f) - a(f)).
\ed


\begin{lem}
  \label{l:W}
  Let $f, g \in L^2(\R^3)$.
  \begin{itemize}
    \item[(i)] The Weyl operator satisfies the relations
      \[
        W(f) W(g) = W(g) W(f) e^{-2i \Im \langle f, g \rangle_{L^2}} = W(f+g)
        e^{-i \Im \langle f, g \rangle_{L^2}}.
      \]
    \item[(ii)] The operator $W(f)$ is unitary on $\mathcal{F}$ and
      \[
        W(f)^* = W(f)^{-1} = W(-f).
      \]
    \item[(iii)] We have
      \[
        W(f)^* a_x W(f) = a_x + f(x) \qquad \text{and} \qquad W(f)^* a_x^* W(f)
        = a_x^* + \overline{f(x)}.
      \]
    \item[(iv)] It follows from (iii) that coherent states are eigenvectors of
      anihilation operators:
      \[
        a_x \psi(f) = f(x) \psi(f) \qquad \text{so that} \qquad a(g) \psi(f)
        = \langle g, f \rangle_{L^2} \psi.
      \]
    \item[(v)] The expectation of the number of particles in the coherent
      state $\psi(f)$ is given by $\| f \|_{L^2}^2$, that is,
      \[
        \langle \psi(f), \N^2 \psi(f) \rangle = \| f \|_{L^2}^2.
      \]
      Also the variance of the number of particles in $\psi(f)$ is given by
      $\| f \|_{L^2}$ (the distribution of $\N$ is poisson), that is,
      \[
        \langle \psi(f), \N^2 \psi(f) \rangle - \langle \psi(f), \N \psi(f)
        \rangle^2 = \| f \|_{L^2}^2.
      \]
    \item[(vi)] Coherent states are normalized but not orthogonal to each
      other. In fact,
      \[
        \langle \psi(f), \psi(g) \rangle = e^{-\frac{1}{2} (\| f \|_{L^2}^2 +
        \| g \|_{L^2}^2 - 2 \langle f, g \rangle_{L^2} )} \qquad \text{so
        that} \qquad |\langle \psi(f), \psi(g) \rangle| = e^{-\frac{1}{2} \| f
        - g \|_{L^2}^2}.
      \]
    \end{itemize}
\end{lem}


\section{Zero-energy scattering equation}


Consider $V \in C_c^\infty(\R^3)$, and let $f$ be the solution of the
zero-energy scattering equation
\[
  \left( -\Delta + \frac{1}{2} V \right) f = 0
\]
with normalization $\lim_{|x|\to\infty} f(x) = 1$. We will write
\[
  f = 1 - w
\]
with $\lim_{|x|\to\infty} w(x) = 0$. The scattering length $a_0$ of $V$ is
defined as
\[
  a_0 = \lim_{|x| \to \infty} w(x)|x|.
\]
Since $V$ has compact support, we have
\[
  f(x) = 1 - \frac{a_0}{|x|} \qquad \text{for } |x| \ge R,
\]
where $R$ is such that $\text{supp }V \subset \{ x \in \R^3 \; | \;\; |x| \le
R \}$. From the zero-energy scattering equation, we also have the identity
\[
  \int dx \, V(x) f(x) = 8 \pi a_0.
\]
By scaling, the scattering length of the potential $V_N(x) = N^2 V(Nx)$ is
$a_N = a_0/N$, and the zero-energy scattering equation for the potential $V_N$
is
\begin{equation}
  \label{eq:scatteringequation}
  \left( -\Delta + \frac{1}{2} V_N \right) f_N = 0,
\end{equation}
where $f_N(x) = 1 - w_N(x)$ with $w_N(x) = w(Nx)$. Note that $w_N(x) =
a_N/|x|$ for $|x| \ge R/N$.


\begin{lem}[\cite{ESY2010}]
\marginpar{We might clean up this lemma if we don't use everything}
  \label{l:w}
  Let $V \in C_c^\infty(\R^3)$ with $V \ge 0$, and suppose that $V$ is
  spherical symmetric with scattering length $a_0$. Let
  \[
    \rho = \sup_{r \ge 0} r^2 V(r) + \int_0^\infty dr \, r V(r),
  \]
  and consider the solution $f_N = 1-w_N$ of the zero-energy scattering
  equation with scaled potential $V_N$. Then, the following hold with
  constants uniform in $N$.
  \begin{itemize}
    \item[(i)] There exists a constant $C_0 > 0$, which depends on the
      unscaled potential $V$, such that
      \[
        C_0 \le 1 - w_N(x) \le 1 \qquad \text{for all } x \in \R^3.
      \]
    \item[(ii)] Let $R$ be such that $\text{supp }V \subset \{ x \in \R^3 \; |
      \;\; |x| \le R \}$. Then,
      \[
        w_N(x) = \frac{a_0}{N|x|} \qquad \text{for all } x \in \R^3 \text{
        with } |x| > R/N.
      \]
    \item[(iii)] From (i) and (ii) it follows that
    \[
      |w_N(x)| \le \max\{a_0, R\} \frac{1}{N|x|} \qquad \text{for all } x \in
      \R^3.
    \]
    \item[(iv)] There exist constants $C_1$ and $C_2$, depending only on $V$,
      such that
      \[
        |\nabla w_N(x)| \le C_1 N \qquad \text{and} \qquad | \nabla^2 w_N(x)|
        \le C_2 N^2
      \]
      for all $x \in \R^3$. Moreover, there exists a universal constant
      $C$, such that
      \[
        |\nabla w_N(x)| \le \frac{C a_0}{N |x|^2}, \qquad |\nabla w_N(x)| \le
        \frac{C \rho}{|x|}, \qquad |\nabla^2 w_N(x)| \le \frac{C \rho}{N^2
        |x|^2}
      \]
      for all $x \in \R^3$.
    \item[(v)] From (iv) it follows that
      \[
        |\nabla w_N(x)| \le \max\{ C_1, C a_0 \} \frac{2N}{N^2|x|^2 + 1}
        \qquad \text{for all } x \in \R^3.
      \]
  \end{itemize}
\end{lem}


\section{Bogoliubov transformation}


\bd
a^\ast(q_y) := \int \di z\, q(z,y) a^*_z = a^*(q(\cdot,y)) , \quad a(q_y) :=
\int \di z\, \cc{q(z,y)} a_z = a(q(\cdot,y))
\ed


\begin{lem}[Bogoliubov transformation \cite{GMM2010}]
\label{l:bt}
 \bd
 T_t^* a^*_x T_t = \int \left( \cc{s(y,x)} a_y + c(y,x) a^*_y \right) \di y =
 a^*(c_x) + a(s_x).
 \ed
\bd
 T_t^* a_x T_t = a(c_x) + a^*(s_x).
\ed
\end{lem}


\begin{lem}[Formula in Fock space \cite{RS2009}]
\bd
\left(\partial_t e^{C(t)} \right) e^{-C(t)} = \dot C(t) +
\frac{1}{2!}[C(t),\dot C(t)]+ \frac{1}{3!}[C(t),[C(t),\dot C(t)]] + \dots
\ed
%\bd
%e^{C(t)} H e^{-C(t)} = H + [C(t),H] + \frac{1}{2!}[C(t),[C(t),H]] + \dots
%\ed
\bd
W^*(f) a_x W(f) = a_x + f(x), \quad W^*(f)a^*_x W(f) = a^*_x + \cc{f(x)}
\ed
\bd
[a_x,a^*_y] = \delta(x-y), \quad [a(f),a^*(g)] = \scal{f}{g}
\ed
\end{lem}


Let
\[
  T = e^{B(k)}
\]
with
\[
  B(k) = \frac{1}{2} \int dxdy \, (k(x,y) a_x^* a_y^* - \overline{k(x,y)} a_x
  a_y).
\]
Then,
\begin{align*}
  T^* a_x T & = a(c(\,\cdot\,,x)) + a^*(s(\,\cdot\,,x)) \\
  & = a_x + a(p(\,\cdot\,,x)) + a^*(s(\,\cdot\,,x))
\end{align*}
with
\begin{alignat*}{2}
  c & = \delta + p, & \qquad p & = \sum_{n=1}^\infty \frac{1}{(2n)!} \big( k
  \overline{k} \big)^n, \\
  s & = k + r, & \qquad r & = \sum_{n=1}^\infty \frac{1}{(2n+1)!} \big( k
  \overline{k} \big)^n k,
\end{alignat*}
where $\delta$ is the Dirac delta, and the product of functions in the power
series is the convolution of integral kernels, that is, $fg(x,y) = \int dz \,
f(x,z) g(z,y)$. Consequently, for $f \in L^2(\R^3)$,
\[
  T^* a(f) T = a(Cf) + a^*(S\overline{f}),
\]
where $C$ and $S$ are integral operators on $L^2(\R^3)$ with integral kernels
$c$ and $s$, respectively.


\begin{prp}
  \label{p:psr}
  Let $\varphi \in H^1(\R^3)$, and let $f_N=1-w_N$ be the solution of the
  zero-energy scattering equation as in Lemma \ref{l:w}. Let
  \[
    k(x,y) = - N w_N(x-y) \varphi(x) \varphi(y).
  \]
  Then, for $N \ge 1$ and $n \ge 1$,
  \begin{align}
    \| k \|_{L^2} & \apprle \| \varphi \|_{H^1}^2, \label{k} \tag{i} \\
    \| \nabla_1 k \|_{L^2} & \apprle \| \varphi \|_{H^1}^2 \sqrt{N},
    \label{gradk} \tag{ii} \\
    \| (k \overline{k})^n \|_{L^2} & \le C^n \| \varphi \|_{H^1}^{4n},
    \tag{iii} \\
    \| (\nabla_1 k \overline{k})^n \|_{L^2} & \le C^n \| \varphi
    \|_{H^1}^{4n}, \tag{iv}
  \end{align}
  where $C$ is a constant that depends only on $a_0$ and $\rho$ from Lemma
  \ref{l:w}. Consequently,
  \begin{equation}
    \begin{alignedat}{2}
      \| p \|_{L^2} & \le C_1, \qquad & \| \nabla_1 p \|_{L^2} & \le C_1, \\
      \| r \|_{L^2} & \le C_1, \qquad & \| \nabla_1 r \|_{L^2} & \le C_1,
    \end{alignedat}
    \tag{v}
  \end{equation}
  and
  \begin{equation}
    \| s \|_{L^2} \le C_1, \qquad \| \nabla_1 s \|_{L^2} \apprle \| \varphi
    \|_{H^1}^2 \sqrt{N}, \tag{vi}
  \end{equation}
  where $C_1$ is a constant that depends only on $\| \varphi \|_{H^1}$, and
  $a_0$ and $\rho$ from Lemma \ref{l:w}.
\end{prp}


\begin{proof}
  (i) Using Lemma \ref{l:w}(iii), and the operator inequality $-4 \Delta \ge
  |x|^{-2}$ on $L^2(\R^3)$, we find that
  \begin{equation}
    \begin{aligned}
      \int dy \, N^2 w_N(x-\,\cdot\,)^2 |\varphi(y)|^2 & \apprle \int dy \,
      |x-y|^{-2} |\varphi(y)|^2 \\
      & = \langle \varphi(\,\cdot\,+x), |\,\cdot\,|^{-2} \varphi(\,\cdot\,+x)
      \rangle_{L^2} \\
      & \le -4 \langle \varphi(\,\cdot\,+x), \Delta \varphi(\,\cdot\,+x)
      \rangle_{L^2} = 4 \| \nabla \varphi \|_{L^2}^2.
    \end{aligned}
    \label{w}
  \end{equation}
  Thus,
  \[
    \| k \|_{L^2}^2 = \int dx \, |\varphi(x)|^2 \int dy \, N^2
    w_N(x-\,\cdot\,)^2 |\varphi(y)|^2 \apprle \| \varphi \|_{L^2}^2 \| \nabla
    \varphi \|_{L^2}^2 \apprle \| \varphi \|_{H^1}^4.
  \]


  (ii) By the Leibniz rule and triangle inequality, there are two terms to be
  estimated:
  \[
    \| \nabla_1 k \|_{L^2} \le \| N w_N(x-y) \nabla \varphi(x) \varphi(y)
    \|_{L^2} + \| N \nabla w_N(x-y) \varphi(x) \varphi(y) \|_{L^2}.
  \]
  For the first term, similarly as in part (i), we have
  \[
    \| N w_N(x-y) \nabla \varphi(x) \varphi(y) \|_{L^2} \apprle \| \nabla
    \varphi \|_{L^2}^2 \apprle \| \varphi \|_{H^1}^2.
  \]
  To bound the second term, we apply Young's inequality, Sobolev's inequality,
  and Lemma \ref{l:w}(v), to find
  \begin{align*}
    \| N \nabla w_N(x-y) \varphi(x) \varphi(y) \|_{L^2}^2 & = N^2 \int dx dy
    \, |\nabla w_N(x-y)|^2 |\varphi(x)|^2 |\varphi(y)|^2 \\
    & \le N^2 \| | \nabla w_N|^2 \|_{L^1} \| \varphi \|_{L^4}^4 \\
    & \apprle N \| \varphi \|_{H^1}^4 \int dx \, \frac{N^3}{(|Nx|^2 + 1)^2}
    \apprle N \| \varphi \|_{H^1}^4.
  \end{align*}
  Therefore,
  \[
    \| \nabla_1 k \|_{L^2} \apprle \| \varphi \|_{H^1}^2 + \sqrt{N} \| \varphi
    \|_{H^1}^2 \apprle \sqrt{N} \| \varphi \|_{H^1}^2.
  \]


  (iii) The proof is by induction on $n$. We first prove the estimate for
  $n=1$. We begin observing that, by H\"older's inequality and \eqref{w},
  \begin{equation}
    \begin{aligned}
      & \int dz \, N^2 w_N(x-z) w_N(z-y) |\varphi(z)|^2 \\
      & \le \left( \int dz \, N^2 w_N(x-z)^2 |\varphi(z)|^2 \right)^{1/2} \left(
      \int dz \, N^2 w_N(z-y)^2 |\varphi(z)|^2 \right)^{1/2} \apprle \| \nabla
      \varphi \|_{L^2}^2.
    \end{aligned}
    \label{ww}
  \end{equation}
  Hence,
  \begin{align*}
    \| k \overline{k} \|_{L^2}^2 & = \int dx \, |\varphi(x)|^2 \int dy \,
    |\varphi(y)|^2 \left| \int dz \, N^2 w_N(x-z) w_N(z-y) |\varphi(z)|^2
    \right|^2 \le C^2 \| \varphi \|_{H^1}^8.
  \end{align*}
  Now, suppose that $\| (k \overline{k})^n \|_{L^2} \le C^{2n} \| \varphi
  \|_{H^1}^{4n}$. Then, by interchanging the integration order, and applying
  H\"older's inequality twice,
  \begin{align*}
    & \| (k \overline{k})^{n+1} \|_{L^2}^2 = \int dz_1 dz_2 \int dx \,
    k\overline{k}(x,z_1) \overline{ k \overline{k}}(x,z_2) \int dy \,
    (k\overline{k})^n(z_1,y) \overline{ (k \overline{k})^n}(z_2,y) \\
    & \le \int dz_1 \, \left( \int dx \, | k\overline{k}(x,z_1)|^2
    \right)^{1/2} \left( \int dy \, | (k\overline{k})^n(z_1,y)|^2
    \right)^{1/2} \\
    & \qquad \times \int dz_2 \left( \int dx \, | k\overline{k}(x,z_2)|^2
    \right)^{1/2} \left( \int dy \, | (k\overline{k})^n(z_2,y)|^2
    \right)^{1/2} \\
    & \le \| k \overline{k} \|_{L^2}^2 \| (k \overline{k})^n \|_{L_2}^2 \le
    C^{2(n+1)} \| \varphi \|_{H^1}^{8(n+1)}.
  \end{align*}
  This completes the induction step and proves the desired estimate.


  (iv) We first prove the estimate for $n=1$. By the Leibniz rule and triangle
  inequality,
  \[
    \| \nabla_1 k \overline{k} \|_{L^2} \le \| I \|_{L^2} + \| J \|_{L^2},
  \]
  where, using \eqref{ww},
  \begin{align*}
    \| I \|_{L^2}^2 & = \int dx \, |\nabla \varphi(x)|^2 \int dy \,
    |\varphi(y)|^2 \left| \int dz \, N^2 w_N(x-z) w_N(z-y) |\varphi(z)|^2
    \right|^2 \apprle \| \varphi \|_{H^1}^8,
  \end{align*}
  and by parts (iii) and (iv) of Lemma \ref{l:w}, and estimates \eqref{w} and
  \eqref{ww}, \begin{align*}
    \| J \|_{L^2}^2 & = \int dx \, |\varphi(x)|^2 \int dy \, |\varphi(y)|^2
    \left| \int dz \, N^2 \nabla w_N(x-z) w_N(z-y) |\varphi(z)|^2 \right|^2 \\
    & \apprle \int dx \, |\varphi(x)|^2 \int dz_1
    \frac{|\varphi(z_1)|^2}{|x-z_1|^2} \int dz_2
    \frac{|\varphi(z_2)|^2}{|x-z_2|^2} \int dy \,
    \frac{|\varphi(y)|^2}{|z_1-y| \, |z_2-y|} \\
    & \apprle \| \varphi \|_{L^2}^2 \| \nabla \varphi \|_{L^2}^6 \apprle \|
    \varphi \|_{H^1}^8.
  \end{align*}
  Thus,
  \[
    \| \nabla_1 k \overline{k} \|_{L^2} \le \| I \|_{L^2} + \| J \|_{L^2}
    \le C^2 \| \varphi \|_{H^1}^4.
  \]
  Consequently, by H\"older's inequality (similarly as in the proof of part
  (iii)), and then applying part (iii),
  \[
    \| \nabla_1 (k \overline{k})^n \|_{L^2} = \| (\nabla_1 k \overline{k}) (k
    \overline{k})^{n-1} \|_{L^2} \le \| \nabla_1 k \overline{k} \|_{L^2} \| (k
    \overline{k})^{n-1} \|_{L_2} \le C^{2n} \| \varphi \|_{H_1}^{8n}.
  \]


  \marginpar{Unfinished}
  (v) By H\"older's inequality (as in the proof of (iii)) and then by (i) and
  (iii) we have that
  \[
    \sum_{n=1}^\infty \left \| \frac{(k \overline{k})^n k}{(1+2n)!} \right
    \|_{L^2}^2 \le \sum_{n=1}^\infty \frac{C^n \| \varphi
    \|_{H^1}^{4n+2}}{(1+2n)!} \le e^{C \| \varphi \|_{H^1}^6}.
  \]
  Hence, by the dominated convergence theorem (or something easier ?) we find
  that $r \in L^2$ and $\| r \|_{L^2} \le e^{C \| \varphi \|_{H^1}^6}$.
  Similarly for $p$ \dots. Furthermore, since \dots \dots, we can interchange
  differentiation with summation to obtain $\nabla_1 p \in L^2$, \dots \dots,
  with $\| \nabla_1 p \|_{L^2} \le \dots$ \dots


  (vi) Recall that $s = k + r$. Observe that part (vi) follows from (i), (ii)
  and (v).
\end{proof}


\section{The generator}
The generator $\Lcal_N(t)$ is defined by
\bd
\Lcal_N(t) U_N(t) = i \partial_t U_N(t).
\ed
We calculate that
\begin{align*}
\Lcal_N(t) 	& = (i \partial_t T^*_t) T_t + T^*_t \left( (i \partial_t W^*_t) W_t + W^*_t \Hcal_N W_t \right) T_t \\
		& =: (i \partial_t T^*_t) T_t + T^*_t \Lcal^{(0)}_N(t) T_t
\end{align*}
where
\begin{align*}
& \Lcal^{(0)}_N(t) = \Kcal + \Vcal_N \\
		& + N^{1/2} \left(  a^*\left( (w_N N V_N \ast \lvert \ph \rvert^2)\ph \right) + \hc  \right) \\
		& + N^0	    \left(  \frac{1}{2}\int \di x \di y\, NV_N(x-y)\left( \cc{\ph(x)} \cc{\ph(y)} a_y a_x + \hc \right) \right) \\
		& + N^0	    \left(  \int \di x \di y\, NV_N(x-y)\left( \lvert \ph(x) \rvert^2 a^*_y a_y + \cc{\ph(x)} \ph(y) a^*_y a_x \right) \right) \\
		& + N^{-1/2}\left(  \int \di x \di y\, NV_N(x-y) \left( \cc{\ph(x)} a^*_y a_y a_x + \hc \right)  \right) \\
		& + N b(N,t),
\end{align*}
with a phase (which, like all phases, will be dropped from now on without any further comment)
\bd
b(N,t) = \norm{\nabla \ph}_{L^2}^2 - \Im \scal{\ph}{\phdot} + \frac{1}{2}\int \di x \di y\, NV_N(x-y) \lvert \ph(x)\rvert^2 \lvert \ph(y) \rvert^2.
\ed
Here we have made use of a cancellation in the term $\Ocal({N^{1/2}})$ (term which is linear in creation/annihilation operators) due to $\ph$ satisfying the modified Hartree equation
\bd
i\partial_t \ph = -\Delta \ph + \left(f_N N V_N \ast \lvert \ph \rvert^2 \right) \ph.
\ed
Compare \cite{RS2009}, but notice that due to the factor $f_N$ in the Hartree equation, in our case the cancellation is incomplete. This is essential for ensuring the correct coupling constant $8\pi a_0$ and ensures further cancelation with the cubic terms. This cancelation is revealed through the Bogoliubov transformation, to be calculated now.

We now identify cancellations between linear terms stemming from the transformation of linear terms and linear terms stemming from normal-ordering transformed cubic operators:
\begin{align*}
& T^*_t \Lcal^{(0)}_N(t) T_t = \int \di x\, T^*_t a^*_x T_t \bigg[   \frac{1}{2}(-\Delta_x T^*_t a_x T_t) \\
& + \frac{1}{4}\int \di y V_N(x-y) T^*_t a^*_y a_y a_x T_t \\
& + N^{1/2} \ph(x) \int \di y\, w_N N V_N(x-y) \lvert \ph(y) \rvert^2 \tagg{linearterm} \\
& + \frac{1}{2} \int \di y\, NV_N(x-y)  \ph(x) \ph(y)  T^*_t a^*_y T_t \\
& + \frac{1}{2} \int \di y\, NV_N(x-y) \left(  \lvert \ph(y)\rvert^2 T^*_t a_x T_t + \ph(x) \cc{\ph(y)} T^*_t a_y T_t  \right) \\
& + N^{-1/2} \int \di y\, NV_N(x-y) \cc{\ph(y)} T^*_t a_x a_y T_t  \bigg]\tagg{cubicterm} \\
& + \hc
\end{align*}
Between the term \eqr{linearterm} and \eqr{cubicterm}, after normal ordering and plugging in $c = \delta + p$ and $s = k + r$, we find that the linear term is cancelled up to two remainder terms (the $N^{1/2}$-term is completely cancelled, i.\,e.\ line \eqr{linearterm}). Result:
\begin{align*}
& T^*_t \Lcal^{(0)}_N(t) T_t = \int \di x\, T^*_t a^*_x T_t \bigg[   \frac{1}{2}(-\Delta_x T^*_t a_x T_t) \\
& + \frac{1}{4}\int \di y V_N(x-y) T^*_t a^*_y a_y a_x T_t \\
& + \frac{1}{2} \int \di y\, NV_N(x-y)  \ph(x) \ph(y)  T^*_t a^*_y T_t \\
& + \frac{1}{2} \int \di y\, NV_N(x-y) \left(  \lvert \ph(y)\rvert^2 T^*_t a_x T_t + \ph(x) \cc{\ph(y)} T^*_t a_y T_t  \right) \\
& + N^{-1/2} \int \di y\, NV_N(x-y) \cc{\ph(y)} \bigg( r(x,y) + \scal{p_x}{s_y} + \\
& \qquad \qquad  a^*(s_x) a^*(s_y) + a^*(s_x) a(c_y)  + a^*(s_y) a(c_x) + a(c_x) a(c_y)  \bigg)  \bigg] \\
& + \hc
\end{align*}
Now we apply Lemma \ref{l:bt} everywhere and expand all the terms, normal order all terms, and observe that many terms appear twice (maybe in the hermitian conjugate or with $x$ and $y$ interchanged):
\begin{align*}
& T^*_t \Lcal_N^{(0)}(t) T_t = \\ 
& \frac{1}{2} \int \di x\, \left[ a^*(c_x) a(-\Delta_x c_x) + \boxed{2 a^*(c_x) a^*(-\Delta_x s_x)} + a^*(-\Delta_x s_x) a(s_x) \right] \tagg{cancellation_kinetic} \\
& + \frac{1}{2}\int \di x \di y\, NV_N(x-y) \times \\
& \times \Big[   \frac{1}{2N}\bigg( a^*(c_x) a^*(c_y) a(c_y) a(c_x) + 4 a^*(c_x) a^*(c_y) a^*(s_x) a(c_y) \\
				      & \qquad\qquad + 2 a^*(c_x) a^*(c_y) a^*(s_y) a^*(s_x) + 2 a^*(c_x) a^*(s_x) a(s_y) a(c_y) \\
				      & \qquad\qquad + 2 a^*(c_x) a^*(s_y) a(s_y) a(c_x) + 4 a^*(c_x) a^*(s_y) a^*(s_x) a(s_y) \\
				      & \qquad\qquad + a^*(s_y) a^*(s_x) a(s_x) a(s_y) \bigg) \\
& + \frac{1}{N}\bigg(   \boxed{a^*(c_x)a^*(c_y) \scal{c_y}{s_x}} + a^*(c_x) a(c_y) \scal{s_y}{s_x} \tagg{cancellation_normalorder} \\
			& \qquad\qquad + a^*(c_x) a(s_y) \scal{c_y}{s_x} + a^*(c_x) a(c_x) \scal{s_y}{s_y} \\
			& \qquad\qquad + 2 a^*(c_x) a^*(s_x) \scal{s_y}{s_y} + 2a^*(c_x)a^*(s_y) \scal{s_y}{s_x} \\
			& \qquad\qquad + a^*(c_y) a(s_x) \scal{c_y}{s_x} +  a^*(s_y) a(s_y) \scal{s_x}{s_x}\\
			& \qquad\qquad + a^*(s_y) a^*(s_x) \scal{s_x}{c_y} + a^*(s_y) a(s_x) \scal{s_y}{s_x}   \bigg) \\
& + \ph(x)\ph(y) \Big( \boxed{a^*(c_x) a^*(c_y)} + 2 a^*(c_x) a(s_y) +a(s_x) a(s_y) \Big) \tagg{cancellation_standard} \\
& + \ph(x) \cc{\ph(y)} \Big( a^*(c_x) a(c_y) + 2 a^*(c_x) a^*(s_y) + a^*(s_y) a(s_x) \Big) \\
& + \lvert \ph(y) \rvert^2 \Big( a^*(c_x) a(c_x) + 2 a^*(c_x) a^*(s_x) + a^*(s_x) a(s_x) \Big) \\
& + \frac{2}{\sqrt{N}}\cc{\ph(y)} \bigg(    a^*(c_x) a^*(s_x) a^*(s_y) + a^*(c_x) a^*(s_x) a(c_y) + a^*(s_x) a^*(s_y) a(s_x)\\
					    & \qquad\qquad + a^*(c_x) a^*(s_y) a(c_x) + a^*(c_x) a(c_x) a(c_y)+ a^*(s_x) a(s_x) a(c_y) \\
					    & \qquad\qquad + a^*(s_y) a(s_x) a(c_x) + a(s_x) a(c_x) a(c_y)    \bigg) \\
& + \frac{2}{\sqrt{N}}\cc{\ph(y)} \bigg(    a^*(s_x) \scal{s_x}{s_y} + a^*(s_y) \scal{s_x}{s_x}  + a(c_y) \scal{s_x}{s_x} + a(c_x) \scal{s_x}{s_y} \\
					    & \qquad\qquad + a^*(c_x)r(x,y) + a^*(c_x)\scal{p_x}{s_y} + a(s_x)r(x,y) + a(s_x)\scal{p_x}{s_y}		\bigg)    \Big] + \hc
\end{align*}
We now proceed to identify a cancellation between the second summand in line \eqr{cancellation_kinetic}, the first summand in line \eqr{cancellation_normalorder} and the first summand in line \eqr{cancellation_standard}. To see the cancellation, we expand $c = \delta + p$ and $s = k + r$ and use the product rule for the Laplacian $-\Delta_x k(y,x)$, then notice that the lhs of the zero-energy scattering equation \eqr{scatteringequation} appears. There are however 6 remainder terms left (see final result for the generator below).

Then the final result for the generator is:
\begin{align}
& T^*_t \Lcal_N^{(0)}(t) T_t = \nonumber \\ 
& \frac{1}{2} \int \di x\, \bigg[ a^*(c_x) \int \di y\, a^*_y \Big( N \nabla w_N(x-y) \nabla_x \ph(x) 2 \ph(y) \label{l7}\\
& \qquad\qquad \qquad\qquad						+ Nw_N(x-y) \Delta_x \ph(x) \ph(y) - \Delta_x r(y,x) \Big) \label{l8}\\
& \qquad\qquad 			+ a^*(c_x) a(-\Delta_x c_x) + a^*(-\Delta_x s_x) a(s_x) \bigg] \label{l9} \\
& + \frac{1}{2}\int \di x \di y\, NV_N(x-y) \times \nonumber \\
& \times \Big[   \frac{1}{2N}\bigg( a^*(c_x) a^*(c_y) a(c_y) a(c_x) + 4 a^*(c_x) a^*(c_y) a^*(s_x) a(c_y) \label{l10}\\
				      & \qquad\qquad + 2 a^*(c_x) a^*(c_y) a^*(s_y) a^*(s_x) + 2 a^*(c_x) a^*(s_x) a(s_y) a(c_y) \label{l11}\\
				      & \qquad\qquad + 2 a^*(c_x) a^*(s_y) a(s_y) a(c_x) + 4 a^*(c_x) a^*(s_y) a^*(s_x) a(s_y) \label{l12}\\
				      & \qquad\qquad + a^*(s_y) a^*(s_x) a(s_x) a(s_y) \bigg) \label{l13}\\
& + \frac{1}{N}\bigg(   a^*(c_x) a^*(c_y) \Big( r(y,x) + \scal{p_y}{s_x} \Big) + a^*(c_x) a^*(p_y) k(y,x) \label{l14} \\
      & \qquad\qquad + a^*(c_x) a(c_y) \scal{s_y}{s_x} + a^*(s_y) a(s_y) \scal{s_x}{s_x} + a^*(s_y) a(s_x) \scal{s_y}{s_x} \label{l15}\\
      & \qquad\qquad + a^*(c_x) a(s_y) \scal{c_y}{s_x} + a^*(c_x) a(c_x) \scal{s_y}{s_y} + a^*(s_y) a^*(s_x) \scal{s_x}{c_y} \label{l16}\\
      & \qquad\qquad + 2a^*(c_x) a^*(s_x) \scal{s_y}{s_y} + 2a^*(c_x)a^*(s_y) \scal{s_y}{s_x} + a^*(c_y) a(s_x) \scal{c_y}{s_x}    \bigg) \label{l17}\\
& + \ph(x)\ph(y) \Big( a^*(c_x) a^*(p_y) + 2 a^*(c_x) a(s_y) +a(s_x) a(s_y) \Big) \label{l18}\\
& + \ph(x) \cc{\ph(y)} \Big( a^*(c_x) a(c_y) + 2 a^*(c_x) a^*(s_y) + a^*(s_y) a(s_x) \Big) \label{l19}\\
& + \lvert \ph(y) \rvert^2 \Big( a^*(c_x) a(c_x) + 2 a^*(c_x) a^*(s_x) + a^*(s_x) a(s_x) \Big) \label{l20}\\
& + \frac{2}{\sqrt{N}}\cc{\ph(y)} \bigg(    a^*(c_x) a^*(s_x) a^*(s_y) + a^*(c_x) a^*(s_x) a(c_y) \label{l21}\\
					    & \qquad\qquad + a^*(c_x) a^*(s_y) a(c_x) + a^*(c_x) a(c_x) a(c_y) + a^*(s_y) a(s_x) a(c_x) \label{l22}\\
					    & \qquad\qquad + a^*(s_x) a^*(s_y) a(s_x) + a^*(s_x) a(s_x) a(c_y) + a(s_x) a(c_x) a(c_y)  \bigg) \label{l23}\\
& + \frac{2}{\sqrt{N}}\cc{\ph(y)} \bigg(    a^*(s_x) \scal{s_x}{s_y} + a^*(s_y) \scal{s_x}{s_x}  + a(c_y) \scal{s_x}{s_x} + 							a(c_x) \scal{s_x}{s_y} \label{l24}\\
					    & \qquad\qquad + a^*(c_x)r(x,y) + a^*(c_x)\scal{p_x}{s_y} + a(s_x)r(x,y) + 			a(s_x)\scal{p_x}{s_y}		\bigg)    \Big] + \hc \label{l25}
\end{align}
The cancellations enable us to give estimates for all terms, only using $\Ncal$, $\Ncal^2$, never $\Kcal^2$, which we expect to be too large due to the singular short scale structure of the two-particle correlations.


\section{Estimates for the terms of the generator}
\begin{lem}
The following basic estimates are often used (where $a^\#$ stands for $a^\ast$ or $a$):
\bd
\int \di x\, \norm{a(c_x)\psi}^2 \leq 2(1+\norm{p}^2)\scal{\psi}{\Ncal\psi}
\ed
\bd
\int \di x\di y\, NV_N(x-y) \norm{a(c_x)\psi}^2 \leq b_0 2(1+\norm{p}^2) \scal{\psi}{\Ncal\psi}
\ed
\bd
\int \di x\di y\, NV_N(x-y) \norm{a^\#(s_x)\psi}^2 \leq b_0 \norm{s}^2 \scal{\psi}{(\Ncal+1)\psi}
\ed
We also make use of the following estimates (with $C$ independent of $N$ and $t$): \marginpar{this ests should be put in the right place in the document}
\bd
\sup_{x} \norm{s_x}_2^2 \leq C \norm{\ph}_{H^2}^2, \quad \sup_{x} \norm{p_x}_2^2 \leq C \norm{\ph}_{H^2}^2,
\ed
\bd
\lvert r(x,y)\rvert \leq C \lvert\ph(x)\rvert \lvert\ph(y)\rvert, \quad \lvert k(x,y)\rvert \leq N \lvert \ph(x)\rvert \lvert \ph(y)\rvert.
\ed
\end{lem}
\begin{proof}
Trivial.
\end{proof}

We define
\bd
\tilV := \frac{1}{4} \int \di x\di y\, V_N(x-y) a^\ast(c_x) a^\ast(c_y)a(c_y)a(c_x).
\ed

We now have the following list of estimates for all terms of the generator (here, $C$ is always a constant independent of $N$ and $t$; the expectation value is $\ev{A} = \scal{\psi}{A\psi}$):\newline
\estlist{7}{1}
\begin{align*}
& \lvert \scal{\psi}{\frac{1}{2}\int \di x a^\ast(c_x) \int \di y a^\ast_y \left( N\nabla_x w_n(x_y)\nabla_x \ph(x) 2\ph(y)\right) \psi}\rvert \\
\leq & \lvert \int \di x \di y\, Nw_N(x-y) \nabla_x \ph(x)  \nabla_y \ph(y) \scal{a_y\psi}{a^\ast_x\psi} \rvert \\
& + \lvert \int \di x \di y\,  Nw_N(x-y) \nabla_x \ph(x) \ph(y)  \scal{(\nabla_y a_y)\psi}{a^\ast \psi} \rvert \\
& + \lvert \int \di x \di y\, Nw_N(x-y) \nabla_x  \ph(x) \nabla_y \ph(y) \scal{a_y \psi}{a^\ast(p_x)\psi} \rvert \\
& + \lvert \int \di x \di y\, Nw_N(x-y) \nabla_x \ph(x) \ph(y) \scal{(\nabla_y a_y)\psi}{a^\ast(p_x)\psi} \rvert \\
\leq & \int \di y \lvert \nabla_y \ph(y)\rvert \norm{a_y \psi} \norm{a^\ast(Nw_N(\cdot -y)\nabla\ph(\cdot))\psi} \\
& + \int \di y \lvert \ph(y)\rvert \norm{(\nabla_y a_y)\psi} \norm{a^\ast(Nw_N(\cdot-y)\nabla\varphi(\cdot))\psi} \\
& + \int \di x\di y\, \frac{C}{\lvert x-y\rvert} \lvert \nabla_x\ph(x)\rvert \lvert \nabla_y \ph(y)\rvert \norm{a_y \psi} \norm{p_x} \norm{(\Ncal+1)^{1/2}\psi} \\
& + \int \di x\di y\, \frac{C}{{x-y}} \lvert \nabla_x\ph(x)\rvert \lvert \ph(y)\rvert \norm{(\nabla_y a_y)\psi} \norm{p_x} \norm{(\Ncal+1)^{1/2}\psi} \\
\leq & C \norm{\Delta\ph}_{L^2} \norm{(\Ncal+1)^{1/2}\psi} \int \di y \lvert \nabla_y \ph(y) \rvert \norm{a_y \psi} \\
& + C \norm{\Delta \ph}_{L^2} \norm{(\Ncal+1)^{1/2}\psi} \int \di y \frac{1}{\sqrt{\varepsilon}} \lvert \ph(y)\rvert \sqrt{\varepsilon} \norm{(\nabla_y a_y)\psi} \\
& + \norm{(\Ncal+1)^{1/2}\psi} C \left( \di x\di y\, \lvert \nabla_x\ph(x)\rvert^2 \frac{\nabla_y \ph(y)}{\lvert x-y \rvert^2} \right)^{1/2} \left( \di x\di y\, \norm{a_y \psi}^2 \norm{p_x}^2 \right)^{1/2} \\
& + \norm{(\Ncal+1)^{1/2}\psi} C \left( \di x\di y\, \lvert \nabla_x\ph(x)\rvert^2 \frac{\lvert \ph(y)\rvert^2}{\lvert x-y\rvert^2} \right)^{1/2} \left( \di x\di y\, \norm{(\nabla_y a_y)\psi}^2 \norm{p_x}^2 \right)^{1/2} \\
\leq & C \norm{\Delta \ph}_{L^2} \norm{(\Ncal+1)^{1/2}\psi} \norm{\nabla \ph}_{L^2} \norm{\Ncal^{1/2}\psi} \\
& + C\norm{\Delta \ph}_{L^2} \norm{(\Ncal+1)^{1/2}\psi} \frac{1}{\sqrt{\varepsilon}}\norm{\ph}_{L^2} \sqrt{\varepsilon}\norm{\Kcal^{1/2}\psi} \\
& + C \norm{(\Ncal+1)^{1/2}\psi} \norm{\nabla \ph}_{L^2} \norm{\Delta \ph}_{L^2} \norm{p} \norm{\Ncal^{1/2}\psi} \\
& + C \frac{1}{\sqrt{\varepsilon}}\norm{(\Ncal+1)^{1/2}\psi} \norm{\nabla \ph}_{L^2}^2 \norm{p} \sqrt{\varepsilon}\norm{\Kcal^{1/2}\psi} \\
\leq & C \norm{\ph}_{H^2} \left( \ev{\Ncal+1}(1+\frac{1}{\varepsilon}) + \ev{\Kcal}\varepsilon \right) + C \left( \frac{1}{\varepsilon}\ev{\Ncal+1} + \varepsilon \ev{\Kcal} \right).
\end{align*}
\estlist{8}{1}
\begin{align*}
&\lvert \scal{\psi}{\frac{1}{2}\int \di x\di y\, N w_N(x-y) a^\ast(c_x) a^\ast_y \Delta_x \ph(x) \ph(y) \psi}\rvert\\
\leq & \frac{1}{2} \int \di x\, \norm{a(c_x)\psi} \lvert \Delta_x \ph(x)\rvert \norm{a^\ast(Nw_N(x-\cdot)\ph(\cdot))\psi} \\
\leq & \frac{1}{2} \int \di x\, \norm{a(c_x)\psi} \lvert \Delta_x \ph(x)\rvert \norm{N w_N(x-\cdot)\ph(\cdot)}_{L^2} \norm{(\Ncal+1)^{1/2}\psi} \\
\leq & \frac{1}{2} C \norm{\nabla\ph}_{L^2} \norm{(\Ncal+1)^{1/2} \psi} \left( \int \di x\, \lvert \Delta_x \ph(x)\rvert^2 \right)^{1/2} \left( \int \di x\, \norm{a(c_x)\psi}^2 \right)^{1/2} \\
\leq & \frac{1}{2} C \norm{\ph}_{H^1} \norm{\ph}_{H^2} \ev{\Ncal+1} \\
\leq & C \norm{\ph}_{H^2} \ev{\Ncal+1}
\end{align*}
\estlist{8}{2}
For this estimate, it is useful to know that
\begin{align*}
 & \int \di x\, \norm{\nabla_x a(c_x)\psi}^2 \leq 2 \int \di x\, \norm{\nabla_x a_x \psi}^2 + 2\int \di x\, \norm{a(\nabla_x p_x)\psi}^2 \\
 \leq & 2\ev{\Kcal} + 2\int \di x\, \norm{\nabla_x p_x}^2 \norm{\Ncal^{1/1}\psi}^{1/2} = 2\ev{\Kcal}+ 2\norm{\nabla_x p}^2 \ev{\Ncal}.
\end{align*}
Using this, we get
\begin{align*}
 & \lvert \scal{\psi}{\frac{1}{2}\int \di x\, a^\ast(c_x) \int \di y\, (-\Delta_x r(y,x))\psi} \rvert \\
= & \lvert \scal{\psi}{\frac{1}{2}\int \di x\, \nabla_x a^\ast(c_x) \int \di y\, a^\ast_y (\nabla_x r(y,x))\psi}\rvert \\
\leq & \frac{1}{2}\int \di x\, \norm{\nabla_x a(c_x)} \norm{a^\ast(\nabla_xr_x)\psi} \\
\leq & \frac{1}{2}\left( \varepsilon \int \di x\, \norm{\nabla_x a(c_x) \psi}^2 \right)^{1/2} \left( \int \di x\, \norm{\nabla_x r_x}^2 \right)^{1/2} \frac{1}{\sqrt{\varepsilon}} \norm{(\Ncal+1)^{1/2}\psi} \\
\leq & \frac{1}{2}\norm{\nabla_x r} \left( \varepsilon 2\ev{\Kcal} + \varepsilon 2 \norm{\nabla_x p}^2 \ev{\Ncal} + \frac{1}{\varepsilon}\ev{\Ncal+1} \right) \\
\leq & C \left( \varepsilon \ev{\Kcal} + (\varepsilon+\frac{1}{\varepsilon})\ev{\Ncal+1} \right).
\end{align*}
\estlist{9}{1}
This term we bound in two different ways. We start with an equality:
\begin{align*}
 & \scal{\psi}{\frac{1}{2}\int \di x\, a^\ast(c_x) a(-\Delta_x c_x)\psi} = \scal{\psi}{\frac{1}{2}\int \di x\, a^\ast(\nabla_x c_x)a(\nabla_x c_x)\psi} \\
= & \scal{\psi}{\frac{1}{2}\int \di x \left( \nabla_x a^\ast_x a^\ast(\nabla_x p_x) \right)\left(  \nabla_x a_x + a(\nabla_x p_x) \right) \psi} \\
= & \scal{\psi}{\frac{1}{2}\Kcal \psi} + \frac{1}{2}\int \di x\scal{\nabla_x a_x \psi}{a(\nabla_x p_x)\psi} + \frac{1}{2}\int \di x\, \scal{a(\nabla_x p_x)\psi}{\nabla_x a_x \psi} + \frac{1}{2}\int \di x\, \norm{a(\nabla_x p_x)\psi}^2.
\end{align*}
The first estimate we get from this (leaving away the last summand, which is positive) is
\begin{align*}
 \ev{\Kcal} & \leq \scal{\psi}{\int \di x\, a^\ast(c_x) a(-\Delta_x c_x)\psi} + 2\int \di x\, \sqrt{\varepsilon} \norm{\nabla_x a_x \psi} \frac{1}{\sqrt{\varepsilon}} \norm{a(\nabla_x p_x)\psi} \\
& \leq \scal{\psi}{\int \di x\, a^\ast(c_x) a(-\Delta_x c_x)\psi} + 2\left( \varepsilon \ev{\Kcal} + \frac{1}{\varepsilon}  \norm{\nabla_x p}^2 \ev{\Ncal}\right) \\
& \leq \scal{\psi}{\int \di x a^\ast(c_x) a(-\Delta_x c_x)\psi} + C\left( \varepsilon \ev{\Kcal} + \frac{1}{\varepsilon}\ev{\Ncal} \right),
\end{align*}
where the term of the generator appears on the rhs.\\
For the second estimate, we proceed as follows:
\begin{align*}
 & \scal{\psi}{\frac{1}{2} \int \di x\, a^\ast(c_x) a(-\Delta_x c_x)\psi} \\
\leq & \frac{1}{2}\ev{\Kcal} + \left( \ev{\Kcal} + \norm{\nabla_x p}^2 \ev{\Ncal}\right) + \frac{1}{2}\norm{\nabla_x p}^2 \ev{\Ncal} \\
\leq & C\left( \ev{\Kcal} + \ev{\Ncal} \right).
\end{align*}
\estlist{9}{2} For this term we need two estimates, too. The first one is just that the term is non-negative. For the second estimate, notice that, using the product rule, integration by parts and the fact that $-\nabla_x w_N(x-y) = \nabla_y w_N(x-y)$, we have
\begin{align*}
 a(\nabla_x s_x) & = \cc{\ph(x)} a(-Nw_N(x-\cdot)\nabla\ph(\cdot)) + \cc{\ph{x}} (\nabla a)(- N w_N(x-\cdot)\ph(\cdot)) \\
 & + \cc{\nabla_x \ph(x)} a(-Nw_N(x-\cdot)\ph(\cdot)) + a(\nabla_x r_x).
\end{align*}
Using $(a+b+c+d)^2 \leq 4(a^2+b^2+c^2+d^2)$, we now conclude
\begin{align*}
 & \scal{\psi}{\frac{1}{2}\int \di x\, a^\ast(-\Delta s_x) a(s_x)\psi} = \frac{1}{2}\int \di x\, \norm{a(\nabla_x s_x)\psi}^2 \\
\leq & \frac{1}{2} \int \di x \bigg( \lvert \ph(x)\rvert \norm{a(-Nw_N(x-\cdot)\nabla\ph(\cdot))\psi} + \lvert \ph(x)\rvert \norm{(\nabla a)(-N w_N(x-\cdot)\ph(\cdot))\psi} \\
& \qquad + \lvert \nabla_x \ph(x)\rvert \norm{a(-Nw_N(x-\cdot)\ph(\cdot))\psi} + \norm{a(\nabla_x r_x)\psi}  \bigg)^2 \\
\leq & 2\int \di x\, \bigg( \lvert \ph(x)\rvert^2 \norm{-N w_N(x-\cdot)\nabla\ph(\cdot)}_{L^2}^2 \norm{\Ncal^{1/2}\psi}^2 + \lvert \ph(x)\rvert^2 \norm{-Nw_N(x-\cdot) \ph(\cdot)}_{L^2}^2 \norm{\Kcal^{1/2}\psi}^2  \\
& \qquad + \lvert \nabla_x \ph(x)\rvert^2 \norm{-N w_N(x-\cdot) \ph(\cdot)}_{L^2}^2 \norm{\Ncal^{1/2}\psi}^2 + \norm{\nabla_x r_x}^2 \norm{\Ncal^{1/2}\psi}^2 \bigg) \\
& \leq 2 \int \di x\, \left( \lvert \ph(x)\rvert^2 \norm{\Delta \ph}_{L^2}^2 C \norm{\Ncal^{1/2}\psi}^2 \right) + 2 \int \di x \left( \lvert \ph(x)\rvert^2 \norm{\nabla \ph}_{L^2}^2 C \norm{\Kcal^{1/2}\psi}^2 \right) \\
& \qquad + 2 \int \di x\, \left( \lvert \nabla_x \ph(x)\rvert^2 \norm{\nabla \ph}_{L^2}^2 C \norm{\Ncal^{1/2}\psi}^2 \right) + 2 \int \di x\, \norm{\nabla_x r_x}^2 \norm{\Ncal^{1/2}\psi}^2\\
\leq & C \norm{\ph}_{H^2}^2 \ev{\Ncal} + C\ev{\Kcal} + C\ev{\Ncal}. 
\end{align*}
Here we made use of the following estimates: $\norm{-N w_N(x-\cdot) \nabla\ph(\cdot)}_{L^2}^2 \leq C\norm{\Delta \ph}_{L^2}$, $\norm{-N w_N(x-\cdot)\ph(\cdot)}^2 \leq C \norm{\nabla \ph}^2$  and $\int \norm{\nabla_x r_x}^2 \di x= \norm{\nabla_x r}^2 \leq C$.

In proving the following estimates, the important observations are
\begin{itemize}
 \item $a^*(c_x)\psi$ is ill-defined because $c_x = \delta_x + p_x$, thus has to be transferred to the other argument of the scalar product as $a(c_x)$
 \item for summands containing $a(c_x) a(c_y) \psi$, insert $\sqrt{\varepsilon} \frac{1}{\sqrt{\varepsilon}}$ and then use Hoelder, so that $\varepsilon\ev{\tilV}$ is obtained
 \item $a^\#(s_x)$ is always ok, and can easily be estimated using the $L^2$-Norm of $s_x$. with $p$ instead of $s$, this is okay, too.
 \item use $\sup_x \norm{s_x}^2$, $\sup_x \norm{p_x}^2$ and $\norm{\ph}_\infty$ to simplify integrals such that $\int \di y\, NV_N(x-y) = b_0$ (independent of $x$) can be used
\item always use Cauchy-Schwarz, in such away that at most two operators $a^\#$ act on each $\psi$. Usually Hoelder (with $p=q=2$) is applied to all integrals, so that we get an upper bound by a sum of two terms. In using Hoelder, we always split $V_N$ into two square roots, so that we get $V_N$ in each integral, not $V_N^2$.
\item for $a^\# a^\# a^\#$ and $a^\# a^\# a^\# a^\#$, we need $\Ncal^2$, so a prefactor $1/N$ is necessary. 
\end{itemize}
We now give the estimates for all the terms explicitly.
\newline\estlist{10}{1}
By definition of $\tilV$, we have
\bd
\scal{\psi}{\frac{1}{2}\int \di x\di y\, NV_N(x-y)\frac{1}{2N} a^\ast(c_x) a^\ast(c_y) a(c_y) a(c_x)\psi} = \ev{\tilV}.
\ed
\estlist{10}{2}
\begin{align*}
 & \lvert \scal{\psi}{\frac{1}{4}\int \di x\di y\, V_N(x-y) 4 a^\ast(c_x)a^\ast(c_y) a^\ast(s_x)a(c_y)\psi}\rvert \\
\leq & \int \di x\di y\, \sqrt{V_N(x-y)}^2 \frac{\sqrt{\varepsilon}}{2} \norm{a(c_y) a(c_x)\psi} \frac{2}{\sqrt{\varepsilon}} \norm{a^\ast(s_x)a(c_y)\psi} \\
\leq & \varepsilon \frac{1}{4} \int \di x\di y\, V_N(x-y) \scal{\psi}{a^\ast(c_x)a^\ast(c_y)a(c_y)a(c_x)\psi} + \frac{4}{\varepsilon} \int \di x\di y\, V_N(x-y) \norm{a^\ast(s_x)a(c_y)\psi}^2 \\
\leq & \varepsilon \ev{\tilV} + \frac{4}{\varepsilon}\int \di x\di y\, V_N(x-y) \norm{s_x}^2 \norm{a(c_y)\Ncal^{1/2}\psi}^2 \\
\leq & \varepsilon \ev{\tilV} + \frac{4}{\varepsilon} \sup_x \norm{s_x}^2 \int \di y\, \norm{a(c_y)\Ncal^{1/2}\psi}^2 \int \di x\, V_N(x-y) \\
\leq & \varepsilon \ev{\tilV} + \frac{4}{\varepsilon} \sup_x \norm{s_x}^2 \frac{b_0}{N} 2 (1+\norm{p}^2)\ev{\Ncal^2} \\
\leq & \varepsilon \ev{\tilV} + \norm{\ph}_{H^2}^2 \frac{C}{\varepsilon} \frac{1}{N}\ev{\Ncal^2}.
\end{align*}
\estlist{11}{1}
\begin{align*}
& \lvert \scal{\psi}{\frac{1}{2}\int \di x\di y\, NV_N(x-y) \frac{1}{2N}2a^\ast(c_x)a^\ast(c_y) a^\ast(s_y) a^\ast(s_x) \psi}\rvert \\
\leq & 2\frac{1}{4}\int \di x\di y\, \sqrt{V_N(x-y)}^2  \norm{a(c_y)a(c_x)\psi} \norm{a^\ast(s_y)a^\ast(s_x)\psi} \\
\leq & \varepsilon 2 \ev{\tilV} + \frac{1}{2\varepsilon} \int \di x\di y\, V_N(x-y) \norm{a^\ast(s_y)a^\ast(s_x)\psi}^2 \\
\leq & \varepsilon 2 \ev{\tilV} + \frac{1}{2\varepsilon} \int \di x\di y\, V_N(x-y) \norm{s_y}^2 \norm{s_x}^2 \ev{(\Ncal+2)^2} \\
\leq & \varepsilon 2 \ev{\tilV} + \frac{1}{2\varepsilon} \sup_x \norm{s_x}^2 \underbrace{\int \di y\, \norm{s_y}^2 \int \di x\, V_N(x-y)}_{=\, \norm{s}^2 b_0/N} \ev{(\Ncal+2)^2} \\
\leq & \varepsilon 2\ev{\tilV} + \norm{\ph}_{H^2}^2 \frac{C}{\varepsilon} \frac{1}{N}\ev{(\Ncal+2)^2}.
\end{align*}
\estlist{11}{2}
\begin{align*}
& \lvert \scal{\psi}{\frac{1}{2}\int \di x\di y\, N V_N(x-y) \frac{1}{2N} 2 a^\ast(c_x) a^\ast(s_x) a(s_y) a(c_y)\psi}\rvert \\
\leq & \frac{1}{2}\int \di x\di y\, \sqrt{V_N(x-y)}^2 \norm{a(s_x) a(c_x) \psi} \norm{a(s_y) a(c_y)\psi} \\
\leq & \int \di x\di y\, V_N(x-y) \norm{s_y}^2 \norm{a(c_y) \Ncal^{1/2}\psi}^2\\
\leq & \sup_y \norm{s_y}^2 \int \di y\, \norm{a(c_y)\Ncal^{1/2}\psi}^2 \int \di x\, V_N(x-y)\\
\leq & \norm{\ph}_{H^2}^2 C \frac{1}{N}\ev{\Ncal^2}.
\end{align*}
Here we made use of the fact that both summand from the Hoelder inequality are equal (just rename the integration variables $x$ as $y$ and $y$ as $x$).\newline
\estlist{12}{1}
\begin{align*}
& \lvert \scal{\psi}{\frac{1}{2}\int \di x\di y\, N V_N(x-y) \frac{1}{2N}2a^\ast(c_x)a^\ast(s_y)a(s_y)a(c_x)\psi}\rvert \\
\leq & \frac{1}{2}\int \di x\di y\, V_N(x-y) \norm{a(s_y)a(c_x)\psi}^2 \\
\leq & \frac{1}{2}\int \di x\di y\, V_N(x-y) \norm{s_y}^2 \norm{a(c_x)\Ncal^{1/2}\psi}^2 \\
\leq & \frac{1}{2}\sup_y \norm{s_y}^2 \int \di x\, \norm{a(c_x)\Ncal^{1/2}\psi}^2 \int \di y\,V_N(x-y) \\
\leq & \frac{1}{2}C\norm{\ph}_{H^2}^2 \frac{b_0}{N} 2(1+\norm{p}^2)\ev{\Ncal^2} \\
\leq & \norm{\ph}_{H^2}^2 C \frac{1}{N} \ev{\Ncal^2}.
\end{align*}
\estlist{12}{2}
\begin{align*}
 & \lvert \scal{\psi}{\frac{1}{2}\int \di x\di y\, NV_N(x-y) \frac{1}{2N} 4 a^\ast(c_x) a^\ast(s_y) a^\ast(s_x) a(s_y) \psi}\rvert \\
\leq & \int \di x\di y\, V_N(x-y) \norm{a(s_y)a(c_x)\psi}{a^\ast(s_x)a(s_y)\psi} \\
\leq & \int \di x\di y\, V_N(x-y) \left( \norm{s_y}^2 \norm{a(c_x)\Ncal^{1/2}\psi}^2 + \norm{s_x}^2 \norm{a(s_y)\Ncal{1/2}\psi}^{1/2}\right) \\
\leq & C \norm{\ph}_{H^2}^2 2(1+\norm{p}^2)\frac{b_0}{N}\ev{\Ncal^2} + C\norm{\ph}_{H^2}^2 \norm{s}^2 \frac{b_0}{N} \ev{\Ncal^2} \\
\leq & \norm{\ph}_{H^2}C \frac{1}{N}\ev{\Ncal^2}.
\end{align*}
\estlist{13}{1}
\begin{align*}
& \lvert \scal{\psi}{\frac{1}{2}\int \di x\di y\, NV_N(x-y) \frac{1}{2N} a^\ast(s_y) a^\ast(s_x) a(s_x)a(s_y)\psi} \rvert \\
= & \frac{1}{4} \int \di x\di y\, V_N(x-y) \norm{a(s_x)a(s_y)\psi}^2 \\
\leq & \frac{1}{4} \int \di x\di y\, V_N(x-y) \norm{s_x}^2 \norm{s_y}^2 \norm{\Ncal\psi}^2 \\
\leq & \frac{1}{4} \sup_x \norm{s_x}^2 \underbrace{\int \di y \norm{s_y}^2 \int \di x V_N(x-y)}_{=\, \norm{s}^2 b_0/N} \ev{\Ncal^2}\\
\leq & \norm{\ph}_{H^2}^2 C \frac{1}{N}\ev{\Ncal^2}.
\end{align*}
We now have estimated all summands which are quartic in $a^\#$.\\
\estlist{14}{1}
\begin{align*}
& \lvert \scal{\psi}{\dxyNV \frac{1}{N}a^\ast(c_x) a^\ast(c_y) \left( r(y,x)+\scal{p_y}{s_x} \right)\psi} \rvert \\
\leq & \varepsilon \dxyV \norm{a(c_y)a(c_x)\psi}^2 + \frac{1}{\varepsilon} \dxyV \lvert r(y,x)+\scal{p_y}{s_x} \rvert^2 \\
\leq & \varepsilon 2\ev{\tilV} + \frac{1}{\varepsilon} \int \di x\di y\, V_N(x-y) \left( C\lvert\ph(x)\rvert^2\lvert\ph(y)\rvert^2 +\norm{p_y}^2\norm{s_x}^2 \right) \\
\leq & \varepsilon 2\ev{\tilV} + \frac{C}{\varepsilon} \norm{\ph}_\infty^2 \int \di y\int \di x\, \lvert \ph(y)\rvert^2 V_N(x-y) + \frac{1}{\varepsilon} \sup_x\norm{s_x}^2 \int \di x\int \di y\, \norm{p_y}^2 V_N(x-y) \\
\leq & \varepsilon 2\ev{tilV} + \norm{\ph}_{H^2}^2 \frac{C}{\varepsilon N}.
\end{align*}
\estlist{14}{2}
\begin{align*}
& \lvert \scal{\psi}{\dxyNV \frac{1}{N} a^\ast(c_x) a^\ast(p_y) k(y,x)\psi} \rvert \\
\leq & \dxyV \norm{a(c_x)\psi} \norm{a^\ast(p_y)\psi} \underbrace{\lvert k(y,x)\rvert}_{\leq\, V\lvert \ph(x)\rvert\lvert \ph(y)\rvert} \\
\leq & \dxyNV \norm{a(c_x)\psi}^2 \lvert\ph(x)\rvert^2 + \dxyNV \norm{a^\ast(p_y)\psi}^2 \lvert \ph(y)\rvert^2\\
\leq & \frac{1}{2} \norm{\ph}_\infty^2 \left( b_0 \int \di x\,\norm{a(c_x)\psi}^2 + b_0 \int \di y\,\norm{a^\ast(p_y)\psi}^2 \right) \\
\leq & C \norm{\ph}_{H^2}^2 \ev{\Ncal+1}.
\end{align*}
\estlist{15}{1}
\begin{align*}
& \lvert \scal{\psi}{\dxyNV \frac{1}{N}a^\ast(c_x)a(c_y) \scal{s_y}{s_x}\psi} \rvert \\
\leq & \dxyV \lvert \scal{s_y}{s_x}\rvert \norm{a(c_x)\psi} \norm{a(c_y)\psi} \\
\leq & \dxyV \norm{s_y} \norm{s_x} \norm{a(c_x)\psi} \norm{a(c_y)\psi} \\
\leq & \int \di x \di y\, V_N(x-y) \norm{s_y}^2 \norm{a(c_y)\psi}^2 \\
\leq & \frac{b_0}{N} \sup_y \norm{s_y}^2 \int \di x \norm{a(c_x)\psi}^2 \\
\leq & \norm{\ph}_{H^2}^2 C\frac{1}{N}\ev{\Ncal}. 
\end{align*}
\estlist{15}{2}
\begin{align*}
& \lvert \scal{\psi}{\dxyNV \frac{1}{N}a^\ast(s_y)a(s_y)\scal{s_x}{s_x}\psi} \rvert \\
\leq & \dxyV \norm{s_x}^2 \norm{a(s_y)\psi}^2 \\
\leq & \sup_x \norm{s_x}^2 \frac{1}{2}\int \di y\, \norm{a(s_y)\psi}^2 \int \di x\, V_N(x-y) \\
\leq & C \norm{\ph}_{H^2}^2 \frac{b_0}{2} \norm{s}^2 \frac{1}{N}\ev{\Ncal}\\
\leq & C \norm{\ph}_{H^2}^2 \frac{1}{N}\ev{\Ncal}.
\end{align*}
\estlist{15}{3}
\begin{align*}
& \lvert \scal{\psi}{\dxyNV \frac{1}{N} a^\ast(s_y) a(s_x) \scal{s_y}{s_x}\psi} \rvert \\
\leq & \dxyV \norm{s_x} \norm{s_y} \norm{a(s_y)\psi} \norm{a(s_x)\psi} \\
\leq & \dxyV \norm{s_x}^2 \norm{s_y}^2 \ev{\Ncal} \\
\leq & \norm{\ph}_{H^2}^2 C \frac{1}{N}\ev{\Ncal}.
\end{align*}
\estlist{16}{1}
\begin{align*}
& \lvert \scal{\psi}{\dxyNV \frac{1}{N}a^\ast(c_x) a(s_y) \scal{c_y}{s_x}\psi} \rvert \\
\leq & \dxyV \norm{a(c_x)\psi} \norm{a(s_y)\psi} \left( s(y,x)+\scal{p_y}{s_x} \right)\\
\leq & C\dxyNV \norm{a(c_x)\psi}^2 \lvert \ph(x)\rvert^2 + C \dxyNV \norm{a(s_y)\psi}^2 \lvert \ph(y)\rvert^2\\
& \qquad + \dxyV \norm{a(c_x)\psi}^2 \norm{s_x}^2 + \dxyV \norm{a(s_y)\psi}^2 \norm{p_y}^2 \\
\leq & \frac{C}{2} \norm{\ph}_\infty^2 \int \di x\, \norm{a(c_x)\psi}^2 \int \di y\, NV_N(x-y) + \frac{C}{2} \norm{\ph}_\infty^2 \int \di y\, \norm{a(s_y)\psi}^2 b_0 \\
\leq & \frac{1}{2}\sup_x\norm{s_x}^2 \int \di x\, \norm{a(c_x)\psi}^2 \frac{b_0}{N} + \frac{1}{2} \sup_y \norm{p_y}^2 \int \di y\, \norm{a(s_y)\psi}^2 b_0/N \\
\leq & \frac{C b_0}{2} \norm{\ph}_{H^2}^2(1+\frac{1}{N})\left( \int \di x\,\norm{a(c_x)\psi}^2 + \int \di y\,\norm{a(s_y)\psi}^2 \right) \\
\leq & \norm{\ph}_{H^2}^2 C(1+\frac{1}{N})\ev{\Ncal}.
\end{align*}
\estlist{16}{2}
\begin{align*}
& \lvert \scal{\psi}{\dxyNV \frac{1}{N} a^\ast(c_x) a(c_x)\scal{s_y}{s_y}\psi} \rvert \\
\leq & \dxyV \norm{s_y}^2 \norm{a(c_x)\psi}^2 \\
\leq & \frac{1}{2} \sup_y \norm{s_y}^2 \int \di x\, \norm{a(c_x)\psi}^2 \int \di y\, V_N(x-y) \\
\leq & \norm{\ph}_{H^2}^2 C \frac{1}{N}\ev{\Ncal}.
\end{align*}
\estlist{16}{3}
\begin{align*}
 & \lvert \scal{\psi}{\dxyNV \frac{1}{N}a^\ast(s_y)a^\ast(s_x)\scal{s_x}{c_y}\psi} \rvert \\
\leq & \dxyV \norm{a(s_y)\psi} \norm{a^\ast(s_x)\psi} \lvert\scal{s_x}{c_y}\rvert \\
\leq & C \dxyNV \norm{a(s_y)\psi}^2 \lvert\ph(y)\rvert^2 + C \dxyNV \norm{a^\ast(s_x)\psi}^2 \lvert \ph(x)\rvert^2 \\
& \qquad + \dxyV \norm{a(s_y)\psi}^2 \norm{p_y}^2 + \dxyV \norm{a^\ast(s_x)\psi}^2 \norm{s_x}^2 \\
\leq & C\norm{\ph}_\infty^2\frac{b_0}{2} \int \di y\, \norm{s_y}^2 \ev{\Ncal} + C \norm{\ph}_\infty^2 \frac{b_0}{2} \int \di x\, \norm{s_x}^2 \ev{(\Ncal+1)} \\
& \qquad + \frac{1}{2} \sup_y \norm{p_y}^2 \frac{b_0}{N} \int \di y\, \norm{s_y}^2 \ev{\Ncal} + \frac{1}{2} \sup_x \norm{s_x}^2 \frac{b_0}{N} \int \di x\, \norm{s_x}^2 \ev{(\Ncal+1)}\\
\leq & \norm{\ph}_{H^2}^2 C(1+\frac{1}{N})\ev{\Ncal+1}.
\end{align*}
\estlist{17}{1}
\begin{align*}
 & \lvert \scal{\psi}{\dxyNV \frac{1}{N}2 a^\ast(c_x)a^\ast(s_x) \scal{s_y}{s_y}\psi} \rvert \\
\leq & \int \di x\di y\, V_N(x-y) \norm{s_y}^2 \lvert \scal{a(c_x)\psi}{a^\ast(s_x)\psi}\rvert \\
\leq & \sup_y \norm{s_y}^2 \int \di x\, \frac{b_0}{N} \norm{a(c_x)\psi}^2 + \sup_y \norm{s_y}^2 \int \di x\, \frac{b_0}{N} \norm{a^\ast(s_x)\psi}^2 \\
\leq & \norm{\ph}_{H^2}^2 C \frac{1}{N}\ev{\Ncal+1}.
\end{align*}
\estlist{17}{2}
\begin{align*}
& \lvert \scal{\psi}{\dxyNV \frac{1}{N} 2 a^\ast(c_x) a^\ast(s_y) \scal{s_y}{s_x} \psi} \rvert \\
\leq & \int \di x\di y\, V_N(x-y) \norm{s_x}^2 \norm{a(c_x)\psi}^2 + 
\int \di x\di y\, V_N(x-y) \norm{s_y}^2 \norm{a^\ast(s_y)\psi}^2 \\
\leq & \norm{\ph}_{H^2}^2 C \frac{1}{N}\ev{\Ncal+1}.
\end{align*}
\estlist{17}{3}
To be continued.



\section{A-priori estimates}
\begin{lem}
 \bd
  \norm{k} \leq C, \quad \norm{\nabla k} \leq \sqrt{N}
 \ed
 \bd
  \lvert p(x,y) \rvert \leq C \lvert \ph(x) \rvert\cdot \lvert \ph(y) \rvert, \quad \norm{p} \leq C, \quad \norm{\nabla p} \leq C
 \ed
\bd
\mbox{maybe } \sup_x \norm{s_x}_2^2 \leq C \mbox{ (can be used ?)}
\ed
\end{lem}

\begin{lem}
\bd
\Ncal \leq C T^*_t (\Ncal + 1) T_t, \quad \Ncal^2 \leq C T^*_t (\Ncal + 1)^2 T_t, \quad T^*_t (\Ncal + 1) T_t \leq \Ncal
\ed
\end{lem}


\begin{lem}
\label{lem:nsquaredbound}
For $\ev{\cdot} = \scal{U_N(t)\Omega}{\cdot U_N(t)\Omega}$ or maybe easier $\ev{\cdot} = \scal{U_N(t)T^*\Omega}{\cdot U_N(t)T^*\Omega}$, we have \marginpar{$\Ncal\Kcal$ not needed anymore?}
 \bd
 \frac{1}{N}\ev{\Ncal^2} \leq C \ev{\Ncal}, \quad \frac{1}{N}\ev{\Ncal \Kcal} \leq \ev{\Kcal}
 \ed
\end{lem}

\begin{lem}
 \bd
  \ev{(\partial_t T^*_t)T_t} \leq C \ev{\Ncal}
 \ed
\end{lem}
\begin{proof}
remark: contains only quadratic terms $a^* a$ and $a a$. coefficients are complicated. TODO.
\end{proof}


\section{Estimating the number of fluctuations}
\begin{lem}
\bd
-c\ev{\Ncal} \leq \ev{\Lcal}
\ed 
\end{lem}
\begin{proof}
 done
\end{proof}


\begin{lem}
\label{lem:kvbounds}
 \bd
\ev{\Kcal} \leq  C \ev{\Lcal + \Ncal}
\ed
and for terms in $T^*_t \Vcal T_t$ of the form $A^* A$, where $A$ is quadratic in annihilation and creation operators, we also need bounds by $\Lcal$.
\end{lem}
\begin{proof}
 done
\end{proof}


\begin{lem}
\label{lem:ldotbounds}
 \bd
  \frac{\di}{\di t} \ev{\Lcal} = \ev{\dot \Lcal} \leq C \ev{\Lcal + \Ncal}
 \ed
\end{lem}
\begin{proof}
 TODO.
\end{proof}


\begin{lem}
\label{lem:lncommutatorbound}
 \bd
  \frac{\di}{\di t}\ev{\Ncal} = \ev{[\Lcal,\Ncal]} \leq C \left( \ev{N} + \frac{1}{N}\ev{\Ncal^2} + \frac{1}{N} \ev{\Ncal \Kcal} \right)
 \ed
(rhs can have also other terms for which lemma \ref{lem:kvbounds} holds)
\end{lem}
\begin{proof}
We have \marginpar{I have to retype this, but in priciple done}
\begin{align*}
& [T^*_t \Lcal_N^{(0)}(t) T_t,\Ncal] = \\ 
& \int \di x\, \bigg[ a(c_x) \int \di y\, a_y \Big( N \nabla w_N(x-y) \cc{\nabla_x \ph(x)} 2 \cc{\ph(y)} \\
& \qquad\qquad \qquad\qquad						+ Nw_N(x-y) \cc{\Delta_x \ph(x) \ph(y)} - \cc{\Delta_x r(y,x)} \Big) \bigg]\\
& + \frac{1}{2}\int \di x \di y\, NV_N(x-y) \times \\
& \times \Big[   \frac{4}{N}\bigg( a^*(c_y)a(s_x)a(c_y)a(c_x) + a(s_x)a(s_y)a(c_y)a(c_x) + a^*(s_y)a(s_x)a(s_y)a(c_x) \bigg) \\
& + \frac{2}{N}\bigg(  a(c_x) a_y \Big( \cc{r(y,x)} + \scal{s_x}{p_y} \Big) + a(c_x) a(p_y) \Big( \cc{s(y,x)} + \scal{s_x}{p_y} \Big) \\
			& \qquad\qquad + 2 a(c_x) a(s_x) \scal{s_y}{s_y} + 2 a(c_x)a(s_y) \scal{s_x}{s_y} + a(s_y) a(s_x) \scal{c_y}{s_x} \bigg)\\
& + \ph(x)\ph(y) 2\Big(  a(c_x) a(p_y) + a(s_x) a(s_y) \Big) \\
& + \ph(x) \cc{\ph(y)} 4 a(c_x) a(s_y) \qquad \qquad + \lvert \ph(y) \rvert^2 4 a(c_x) a(s_x) \\
& + \frac{2}{\sqrt{N}}\cc{\ph(y)} \bigg(   3 a(s_y)a(s_x)a(c_x) + a^*(c_y)a(s_x)a(c_x) \\
					    & \qquad\qquad + a^*(c_x)a(s_y)a(c_x) + a^*(c_x) a(c_x) a(c_y) + a^*(s_y) a(s_x) a(c_x)\\
					    & \qquad\qquad + a^*(s_x)a(s_y)a(s_x) + a^*(s_x) a(s_x) a(c_y) + 3 a(s_x) a(c_x) a(c_y) \bigg) \\
& + \frac{2}{\sqrt{N}}\cc{\ph(y)} \bigg(    a(s_x) \scal{s_y}{s_x} + a(s_y) \scal{s_x}{s_x}  + a(c_y) \scal{s_x}{s_x} + a(c_x) \scal{s_x}{s_y} \\
					    & \qquad\qquad + a(c_x)\cc{r(x,y)} + a(c_x)\scal{s_y}{p_x} + a(s_x)r(x,y) + a(s_x)\scal{p_x}{s_y}  \bigg)    \Big] - \hc
\end{align*}
\end{proof}


\begin{prp}
 $N$-independent bound on $\ev{\Ncal}$ by employing all the above and Gronwall.
\end{prp}
\begin{proof}
 By lemma \ref{lem:lncommutatorbound}, lemma \ref{lem:nsquaredbound} and lemma \ref{lem:kvbounds}, $\ev{[\Lcal,\Ncal]}$ is bounded by $\ev{\Lcal+\Ncal}$. So by lemma \ref{lem:ldotbounds}, the sum $\Lcal+C \Ncal$ is bounded above, and for $C$ large enough, it is positive, so we get a bound for $\Ncal$.
\end{proof}


\section{Main result}
\begin{thm}
 Convergence of reduced density matrix for Bogoliubov state.
\end{thm}

\begin{thm}
 Convergence of reduced density matrix for factorized initial state.
\end{thm}


\section{Uniform regularity for the Hartree equation}
\begin{lem}[Uniform $H^2$-regularity]
 XYZ
\end{lem}
\begin{lem}[Convergence of $\ph$ to $\varphi_t$ in $L^2$]
 XZY
\end{lem}

\begin{lem}
 Convergence of $\Tr \lvert \project{\ph} - \project{\varphi_t} \rvert$.
\end{lem}


\section{Operator inequalities (or a priori estimates)}


Let $f(x,y)$ be a function of two variables. To simplify the notation we write
\[
  f_y(x) = f(x,y).
\]
Consider two operators $A$ and $B$ on the Fock space $\mathcal{F}$. The
notation
\[
  A \le B
\]
means that $\langle \psi, A \psi \rangle \le \langle \psi, B \psi \rangle$ for
all $\psi \in \mathcal{F}$.


\begin{lem}
  \label{l:N2}
  Let $U_t$ be the unitary evolution defined in (X). Then,
%  \begin{align}
%    \langle U_t \psi, \N^2 U_t \psi \rangle & \apprle N \langle U_t \psi,
%    (\N+1) U_t \psi \rangle + N \langle \psi, (\N+1) \psi \rangle + \langle
%    \psi, (\N+1)^2 \psi \rangle, \tag{i} \\
%    \langle U_t \psi, \N \K U_t \psi \rangle & \apprle \tag{ii}
%  \end{align}
%  (Or we can state as an operator inequality on $\mathcal{F}$)
  \begin{align}
    U_t^* \N^2 U_t & \apprle N U_t^* (\N+1) U_t + N (\N+1) + (\N+1)^2, \tag{i}
    \\
    U_t^* \N \K U_t & \apprle \tag{ii}
  \end{align}
\end{lem}


The proof of Lemma \ref{l:N2} is based on the next two propositions, which we
prove first.


\begin{prp}
  \label{p:TNT}
  Let $p, s \in L^2(\R^3 \times \R^3)$. Then,
  \begin{align}
    T^* \N T & \apprle (1 + \| p \|_{L^2}^2 + \| s \|_{L^2}^2) (\N+1),
    \label{TNT} \tag{i} \\
    T^* \N^2 T & \apprle (1 + \| p \|_{L^2}^2 + \| s \|_{L^2}^2)^2 (\N+1)^2,
    \label{TN2T} \tag{ii} \\
    T^* \K T & \apprle \K + (\| \nabla_1 p \|_{L^2}^2 + \| \nabla_1 s
    \|_{L^2}^2) (\N+1), \label{TKT} \tag{iii} \\
    T^* \K \N T & \apprle \label{TKNT} \tag{iv}
  \end{align}
\end{prp}


\begin{prp}
  \label{p:formulae}
  Let $\varphi \in L^2(\R^3)$ with $\| \varphi \|_{L^2} = 1$. The following
  pull-through formulae hold true:
  \begin{align}
    \N W(\sqrt{N} \varphi)^* & = W(\sqrt{N} \varphi)^* (\N - \sqrt{N}
    \psi(\varphi) + N), \tag{i} \\
    \N W(\sqrt{N} \varphi) & = W(\sqrt{N} \varphi) (\N + \sqrt{N}
    \psi(\varphi) + N), \tag{ii} \\
    W(\sqrt{N} \varphi)^* \phi(\varphi) & = (\phi(\varphi) + 2 \sqrt{N})
    W(\sqrt{N} \varphi)^*, \tag{iii} \\
    \phi(\varphi) T & = T \phi(C \varphi + S \overline{\varphi}). \tag{iv}
  \end{align}
\end{prp}


\begin{proof}[Proof of Proposition \ref{p:formulae}]
  \marginpar{This seems clear; we might be brief here.}
  Recall that $\phi(\varphi) = a(\varphi) + a^*(\varphi)$. Parts (i), (ii) and
  (iii) follow by a short calculation using parts (ii) and (iii) of Lemma
  \ref{l:W}. Similarly, part (iv) follows by Lemma \ref{l:bt}.
\end{proof}


\begin{proof}[Proof of Proposition \ref{p:TNT}]
  (i) Write
  \[
    \langle \psi, T^* \N T \psi \rangle = \int dx \, \langle T^* a_x T \psi,
    T^* a_x T \psi \rangle = \int dx \, \| (a_x + a(p_x) + a^*(s_x)) \psi
    \|^2.
  \]
  Then, by Cauchy-Schwarz inequality, and Lemma \ref{l:a},
  \begin{align*}
    \langle \psi, T^* \N T \psi \rangle & \le 5 \int dx \, \| a_x \psi \|^2 +
    5 \int dx \, \| a(p_x) \psi \|^2 + 5 \int dx \, \| a^*(s_x) \psi \|^2 \\
    & \le 5 \langle \psi, \N \psi \rangle + 5 \int dx \, \| p_x \|_{L^2}^2
    \langle \psi, \N \psi \rangle + 5 \int dx \, \| s_x \|_{L^2}^2 \langle
    \psi, (\N+1) \psi \rangle \\
    & \le 5 (1 + \| p \|_{L^2}^2 + \| s \|_{L^2}^2) \langle \psi, (\N+1) \psi
    \rangle.
  \end{align*}


  (ii) Write
  \begin{align*}
    & \langle \psi, T^* \N^2 T \psi \rangle \\
    & = \int dxdy \, \langle \psi, T^* a_x^* a_x a_y^* a_y T \psi \rangle \\
    & = \int dx \, \langle \psi, T^* a_x^* \N a_x T \psi \rangle + \langle
    \psi, T^* \N T \psi \rangle \\
    & = \int dx \, \langle \psi, (a_x^* + a^*(p_x) + a(s_x)) T^* \N T (a_x +
    a(p_x) + a^*(s_x)) \psi \rangle + \langle \psi, T^* \N T \psi \rangle.
  \end{align*}
  Then, applying part (i) and Cauchy-Schwarz inequality, using that $a_x
  \N^{1/2} = (\N+1)^{1/2} a_x$, and using Lemma \ref{l:a}, we obtain
  \begin{align*}
    & \langle \psi, T^* \N^2 T \psi \rangle \\
    & \le 5C \int dx \, \| (\N+1)^{1/2} (a_x + a(p_x) + a^*(s_x)) \psi \|^2 + C
    \langle \psi, (\N+1) \psi \rangle \\
    & \le 25C \int dx \, ( \langle a_x^* a_x \psi, \N \psi \rangle + \| a(p_x)
    \N^{1/2} \psi \|^2 + \| a^*(s_x) (\N+2)^{1/2} \psi \|^2 ) + C \langle
    \psi, (\N+1) \psi \rangle \\
    & \le (5C)^2 \langle \psi, (\N+2)^2 \psi \rangle \le 4(5C)^2 \langle \psi,
    (\N+1)^2 \psi \rangle,
  \end{align*}
  where $C = 1 + \| p \|_{L^2}^2 + \| s \|_{L^2}^2$.


  \marginpar{Proof of (iii) is similar to proof of (i); we could omit.} (iii)
  Similarly as in the proof of (i), write
  \[
    \langle \psi, T^* \K T \psi \rangle = \int dx \, \langle T^* \nabla_x a_x
    T \psi, T^* \nabla_x a_x T \psi \rangle = \int dx \, \| (\nabla a_x +
    a(\nabla_x p_x) + a^*(\nabla_x s_x)) \psi \|^2.
  \]
  Then, by Cauchy-Schwarz inequality, and Lemma \ref{l:a},
  \begin{align*}
    \langle \psi, T^* \K T \psi \rangle & \le 5 \int dx \, \| \nabla_x a_x
    \psi \|^2 + 5 \int dx \, \| a(\nabla_x p_x) \psi \|^2 + 5 \int dx \, \|
    a^*(\nabla_x s_x) \psi \|^2 \\
    & \le 5 \langle \psi, \K \psi \rangle + 5 \int dx \, \| \nabla_x p_x
    \|_{L^2}^2 \langle \psi, \N \psi \rangle + 5 \int dx \, \| \nabla_x s_x
    \|_{L^2}^2 \langle \psi, (\N+1) \psi \rangle \\
    & \le 5 \langle \psi, \K \psi \rangle + 5 (\| \nabla_1 p \|_{L^2}^2 + \|
    \nabla_1 s \|_{L^2}^2) \langle \psi, (\N+1) \psi \rangle.
  \end{align*}
\end{proof}


\begin{proof}[Proof of Lemma \ref{l:N2}]
  Let $C = 1 + \| p \|_{L^2}^2 + \| s \|_{L^2}^2$. By Proposition
  \ref{p:TNT},
  \begin{equation}
    \label{ep4}
    \begin{aligned}
      \langle U_t \psi, \N^2 U_t \psi \rangle & = \langle T_t U_t \psi, T_t
      \N^2 T_t^* T_t U_t \psi \rangle \\
      & \apprle C^2 \langle T_t U_t \psi, \N^2 T_t U_t \psi \rangle + C^2
      \langle T_t U_t \psi, \N T_t U_t \psi \rangle + C^2 \\
      & \apprle C^2 \langle T_t U_t \psi, \N^2 T_t U_t \psi \rangle + C^3
      \langle U_t \psi, (\N+1) U_t \psi \rangle. \\
    \end{aligned}
    \end{equation} 
  Now, by Proposition \ref{p:formulae}, and observing that $\N$ commutes with
  $\mathcal{H}_N$,
  \begin{align*}
    & \langle T_t U_t \psi, \N^2 T_t U_t \psi \rangle \\
    & = \langle T_t U_t \psi, \N^2 W_t^* e^{-it \mathcal{H}_N} WT \psi \rangle
    \\
    & = \langle T_t U_t \psi, \N W_t^* (\N - \sqrt{N} \phi(\varphi_t) + N)
    e^{-it \mathcal{H}_N} WT \psi \rangle \\
    & = N \langle T_t U_t \psi, \N T_t U_t \psi \rangle - \sqrt{N} \langle T_t
    U_t \psi, \N ( \phi(\varphi_t) + 2\sqrt{N}) W_t^* e^{-it\mathcal{H}_N} WT
    \psi \rangle \\
    & \quad + \langle T_t U_t \psi, \N W_t^* e^{-it\mathcal{H}_N} W(\N +
    \sqrt{N} \phi(\varphi) + N) T \psi \rangle \\
    & = - \sqrt{N} \langle T_t U_t \psi, \N \phi(\varphi_t) W_t^*
    e^{-it\mathcal{H}_N} WT \psi \rangle + \sqrt{N} \langle T_t U_t \psi, \N
    W_t^* e^{-it\mathcal{H}_N} W T \phi(C \varphi + S \overline{\varphi}) \psi
    \rangle \\
    & \quad + \langle T_t U_t \psi, \N W_t^* e^{-it\mathcal{H}_N} W \N T \psi
    \rangle \\
    & = - \sqrt{N} \langle \N T_t U_t \psi, \phi(\varphi_t) T_t U_t \psi
    \rangle + \sqrt{N} \langle \N T_t U_t \psi, T_t U_t \phi(C \varphi + S
    \overline{\varphi}) \psi \rangle \\
    & \quad + \langle \N T_t U_t \psi, T_t U_t T^* \N T \psi \rangle.
  \end{align*}
  Hence, applying Cauchy-Schwarz inequality and Lemma \ref{l:a}, and observing
  that the operator norm is controlled by the Hilbert-Schmidt norm,
  \begin{align*}
    & \langle T_t U_t \psi, \N^2 T_t U_t \psi \rangle \\
    & \le \| \N T_t U_t \psi \| ( \sqrt{N} \| \phi(\varphi_t) T_t U_t \psi \|
    + \sqrt{N} \| \phi(C \varphi + S \overline{\varphi}) \psi \| + \| \N T
    \psi \| ) \\
    & \le \| \N T_t U_t \psi \| ( 2 \sqrt{N} \| \varphi_t \|_{L^2} \|
    (\N+1)^{1/2} T_t U_t \psi \| \\
    & \quad + 2 \sqrt{N} ( 1 + \| c \|_{L^2} + \| s \|_{L^2}) \| \varphi
    \|_{L^2} \| (\N+1)^{1/2} \psi \| + \| \N T \psi \| ) \\
    & \le \frac{3}{4} \| \N T_t U_t \psi \|^2 + 16N \| \varphi_t \|_{L^2}^2
    \| (\N + 1)^{1/2} T_t U_t \psi \|^2 \\
    & \quad + 16N (1 + \| c \|_{L^2} + \| s \|_{L^2})^2 \| \varphi \|_{L^2}^2
    \| (\N + 1)^{1/2} \psi \|^2 + 4\| \N T \psi \|^2.
  \end{align*}
  Thus, using again Proposition \ref{p:TNT},
  \begin{align*}
    & \langle T_t U_t \psi, \N^2 T_t U_t \psi \rangle \\
    & \apprle N \| \varphi_t \|_{L^2}^2 \langle U_t \psi, T_t^* (\N+1) T_t U_t
    \psi \rangle \\
    & \quad + N (1 + \| c \|_{L^2}^2 + \| s \|_{L^2}^2) \| \varphi
    \|_{L^2}^2 \langle \psi, (\N+1) \psi \rangle + \langle \psi, T^* \N^2 T
    \psi \rangle \\
    & \apprle N C \| \varphi_t \|_{L^2}^2 \langle U_t \psi, (\N+1) U_t \psi
    \rangle + N C \| \varphi \|_{L^2}^2 \langle \psi, (\N+1) \psi \rangle +
    C^2 \langle \psi, (\N+1)^2 \psi \rangle.
  \end{align*}
  Therefore, recalling \eqref{ep4},
  \begin{align*}
    \langle U_t \psi, \N^2 U_t \psi \rangle 
    & \apprle C^2 \langle T_t U_t \psi, \N^2 T_t U_t \psi \rangle + C^3
    \langle U_t \psi, (\N+1) U_t \psi \rangle. \\
    & \apprle N C^3 (\| \varphi_t \|_{L^2}^2+1) \langle U_t \psi, (\N+1) U_t
    \psi \rangle \\
    & \quad + N C^3 \| \varphi \|_{L^2}^2 \langle \psi, (\N+1) \psi \rangle +
    C^4 \langle \psi, (\N+1)^2 \psi \rangle.
  \end{align*}
\end{proof}


\section{Nonlinear Hartree and Gross-Pitaevskii equations}


\begin{lem}
  Let $\varphi \in H^2(\R^3)$ with $\| \varphi \|_{L^2} = 1$. Suppose that $f
  \in L^1(\R^3)$ and $V \in C_c^\infty(\R^3)$ with $fV \ge 0$. For $N > 0$,
  consider a solution $\varphi_t^{(N)} \in H^1(\R^3)$ of the nonlinear Hartree
  equation
  \begin{displaymath}
    i \partial_t \varphi_t^{(N)} = - \Delta \varphi_t^{(N)} + (N f_N V_N *
    |\varphi_t^{(N)}|^2) \varphi_t^{(N)}
  \end{displaymath}
  with initial data $\varphi^{(N)}_0 = \varphi$, where $f_N V_N(x) = N^2
  f(Nx)V(Nx)$. Consider also a solution $\varphi_t \in H^1(\R^3)$ of the
  nonlinear Gross-Pitaevskii equation
  \begin{displaymath}
    i \partial_t \varphi_t = - \Delta \varphi_t + 8 \pi a_0 |\varphi_t|^2
    \varphi_t
  \end{displaymath}
  with initial data $\varphi_0 = \varphi$, where $8 \pi a_0 = \int f V$. Then,
  for all $N > 0$ and $t \ge 0$,
  \begin{alignat}{2}
    \| \varphi_t^{(N)} \|_{H^1} & \le C, & \qquad \| \varphi_t \|_{H^1} &
    \le C, \tag{i} \\
    \| \varphi_t^{(N)} \|_{H^2} & \le \| \varphi \|_{H^2} e^{K t}, & \qquad
    \| \varphi_t \|_{H^2} & \le \| \varphi \|_{H^2} e^{K t}, \tag{ii}
  \end{alignat}
  where $C$ and $K$ are constants that depend only on $\| fV \|_{L^1}$,
  $\text{supp }V$ and $\| \varphi \|_{H^1}$. Furthermore,
  \begin{equation}
    \| \varphi_t^{(N)} - \varphi_t \|_{L^2} \le \frac{C}{N} e^{e^{K t}},
    \tag{iii}
  \end{equation}
  where $C$ and $K$ are constants that depend only on $\| fV \|_{L^1}$,
  $\text{supp }V$ and $\| \varphi \|_{H^2}$.
\end{lem}


\subsection{Introduction/conservation of energy}


Nonlinear Hartree equation
\begin{displaymath}
  i \partial_t \varphi_t = - \Delta \varphi_t + (V * |\varphi_t|^2) \varphi_t
\end{displaymath}
with initial data $\varphi_0 = \varphi$, and energy
\begin{displaymath}
  \mathcal{E}(\varphi) = \int dx \, |\nabla \varphi(x)|^2 + \frac{1}{2} \int
  dx \, (V * |\varphi|^2)(x) |\varphi(x)|^2.
\end{displaymath}


Nonlinear Gross-Pitaevskii equation
\begin{displaymath}
  i \partial_t \varphi_t = - \Delta \varphi_t + 8 \pi a_0 |\varphi_t|^2
  \varphi_t
\end{displaymath}
with initial data $\varphi_0 = \varphi$, and energy
\begin{displaymath}
  \mathcal{E}_{GP}(\varphi) = \int dx \, |\nabla \varphi(x)|^2 + 4 \pi a_0
  \int dx \, |\varphi(x)|^4,
\end{displaymath}
where $8 \pi a_0 = \int f V$.


\begin{prp}
  \label{p:energy}
  Let $\varphi \in H^1(\R^3)$ and $V \in L^1(\R^3)$. Then,
  \begin{displaymath}
    \mathcal{E}(\varphi) \apprle \| \varphi \|_{H^1}^2 + \| V \|_{L^1} \|
    \varphi \|_{H^1}^4 \qquad \text{and} \qquad \mathcal{E}_{GP}(\varphi)
    \apprle \| \varphi \|_{H^1}^2 + |a_0| \| \varphi \|_{L^2} \| \varphi
    \|_{H^1}^3.
  \end{displaymath}
\end{prp}


\begin{proof}
  By Young's inequality, Sobolev's inequality, and H\"older's inequality, 
  \begin{displaymath}
    \int dx \, (V * |\varphi|^2)(x) |\varphi(x)|^2 \le \| V \|_{L^1} \|
    \varphi^2 \|_{L^2}^2 = \| V \|_{L^1} \| \varphi \|_{L^4}^4 \apprle \| V
    \|_{L^1} \| \varphi \|_{H^1}^4
  \end{displaymath}
  and
  \begin{displaymath}
    \int dx \, |\varphi(x)|^4 \le \| \varphi \|_{L^2} \| \varphi^3 \|_{L^2} =
    \| \varphi \|_{L^2} \| \varphi \|_{L^6}^3 \apprle \| \varphi \|_{L^2} \|
    \varphi \|_{H^1}^3.
  \end{displaymath}
  Observing that $\| \varphi \|_{H^1}^2 = \| \varphi \|_{L^2}^2 + \| \nabla
  \varphi \|_{L^2}^2$, we obtain the desired estimates.
\end{proof}


\subsection{$H^2$-regularity}


\begin{prp}
  \label{p:reg1}
  Let $\varphi \in H^2(\R^3)$ with $\| \varphi \|_{L^2} = 1$ and $V \in
  L^1(\R^3)$ with $V \ge 0$. Consider a solution $\varphi_t \in H^1(\R^3)$ of
  the nonlinear Hartree equation
  \begin{displaymath}
    i \partial_t \varphi_t = - \Delta \varphi_t + (V * |\varphi_t|^2)
    \varphi_t
  \end{displaymath}
  with initial data $\varphi_0 = \varphi$. Then, there exists a real number $T
  > 0$ such that
  \begin{displaymath}
    \| \varphi_t \|_{H^2} \le C \| \varphi \|_{H^2}
  \end{displaymath}
  for all $t \in [0,T]$, where the constants $C$ and $T$ depend only on $\| V
  \|_{L^1}$ and $\| \varphi \|_{H^1}$.
\end{prp}


\begin{cor}
  \label{c:reg1}
  Under the hypothesis of Proposition \ref{p:reg1}, for all $t \ge 0$,
  \begin{displaymath}
    \| \varphi_t \|_{H^2} \le \| \varphi \|_{H^2} e^{Kt},
  \end{displaymath}
  where $K$ is a constant that depends only on $\| V \|_{L^1}$ and $\| \varphi
  \|_{H^1}$.
\end{cor}


\begin{proof}
  Recall that $V \ge 0$. By conservation of mass and energy, and by
  Proposition \ref{p:energy},
  \begin{displaymath}
    \| \varphi_t \|_{H^1}^2 \le \| \varphi_t \|_{L^2}^2 +
    \mathcal{E}(\varphi_t) = 1 + \mathcal{E}(\varphi) \apprle \| \varphi
    \|_{H^1}^2 + \| V \|_{L^1} \| \varphi \|_{H^1}^4.
  \end{displaymath}
  Hence, by Proposition \ref{p:reg1}, given $t > 0$, there are constants $C$
  and $T$, depending only on $\| V \|_{L^1}$ and $\| \varphi \|_{H^1}$, such
  that
  \begin{displaymath}
    \| \varphi_t \|_{H^2} \le C \| \varphi_{nT} \|_{H^2}
  \end{displaymath}
  for some $n \ge 0$ with $nT < t$. Iterating this argument backwards in time
  (and again noticing that $C$ and $T$ can be chosen uniformly in time) one
  easily finds that
  \begin{displaymath}
    \| \varphi_t \|_{H^2} \le C \| \varphi_{nT} \|_{H^2} \le C^{n+1} \|
    \varphi \|_{H^2} = (C^{1/T})^{(n+1)T}
    \| \varphi \|_{H^2} \le e^{Kt} \| \varphi \|_{H^2},
  \end{displaymath}
  for some constant $K$ depending only on $\| V \|_{L^1}$ and $\| \varphi
  \|_{H^1}$.
\end{proof}


\begin{prp}
  \label{p:reg2}
  Let $\varphi \in H^2(\R^3)$ with $\| \varphi \|_{L^2} = 1$ and $a_0 \in \R$
  with $a_0 \ge 0$. Consider a solution $\varphi_t \in H^1(\R^3)$ of the
  nonlinear Gross-Pitaevskii equation
  \begin{displaymath}
    i \partial_t \varphi_t = - \Delta \varphi_t + 8 \pi a_0 |\varphi_t|^2
    \varphi_t
  \end{displaymath}
  with initial data $\varphi_0 = \varphi$. Then, there exists a real number $T
  > 0$ such that
  \begin{displaymath}
    \| \varphi_t \|_{H^2} \le C \| \varphi \|_{H^2}
  \end{displaymath}
  for all $t \in [0,T]$, where the constants $C$ and $T$ depend only on $a_0$
  and $\| \varphi \|_{H^1}$.
\end{prp}


\begin{cor}
  \label{c:reg2}
  Under the hypothesis of Proposition \ref{p:reg2}, for all $t \ge 0$,
  \begin{displaymath}
    \| \varphi_t \|_{H^2} \le \| \varphi \|_{H^2} e^{Kt},
  \end{displaymath}
  where $K$ is a constant that depends only on $a_0$ and $\| \varphi
  \|_{H^1}$.
\end{cor}


The proofs of Propositions \ref{p:reg1} and \ref{p:reg2} are based on
well-known Strichartz estimates for the free Schr\"odinger evolution and the
following lemma.


For $T > 0$ set
\begin{displaymath}
  L_t^q L_x^r = L^q([0,T], L^r(\R^3)).
\end{displaymath}


\begin{lem}
  \label{l:interp}
  Let $V \in L^1(\R^3)$, $f \in L_t^{q_1} L_x^{r_1}$ and $g \in L_t^{q_2}
  L_x^{r_2}$ with $q_j, r_j \in [1,\infty]$ for $j \in \{1,2\}$. Then $(V * f)
  g \in L_t^q L_x^r$ with $q^{-1} = s^{-1} + q_1^{-1} + q_2^{-1}$ and $r^{-1}
  = r_1^{-1} + r_2^{-1}$ for $s \in [1, \infty]$. Furthermore,
  \begin{displaymath}
    \| (V * f)g \|_{L_t^q L_x^r} \le \| V \|_{L^1} T^{1/s} \| f \|_{L_t^{q_1}
    L_x^{r_1}} \| g \|_{L_t^{q_2} L_x^{r_2}}.
  \end{displaymath}
\end{lem}


\begin{proof}
  Applying H\"older's inequality in space, and then in time, we find that
  \begin{displaymath}
    \| (V * f) g \|_{L_t^q L_x^r} \le \| V * f \|_{L_t^u L_x^{r_1}} \| g
    \|_{L_t^{q_2} L_x^{r_2}}
  \end{displaymath}
  with $q^{-1} = u^{-1} + q_2^{-1}$ and $r^{-1} = r_1^{-1} + r_2^{-1}$. Now,
  applying Young's inequality in space, H\"older's inequality in time, and
  observing that $V$ is time-independent, we get
  \begin{displaymath}
    \| V * f \|_{L_t^u L_x^{r_1}} \le \| V \|_{L^1} T^{1/s} \| f \|_{L_t^{q_1}
    L_x^{r_1}}
  \end{displaymath}
  with $u^{-1} = s^{-1} + q_1^{-1}$. Combining all this we obtain the desired
  result.
\end{proof}


\begin{proof}[Proof of Proposition \ref{p:reg1}]
  Write
  \begin{displaymath}
    \varphi_t = e^{it\Delta} \varphi - i \int_0^t ds \, e^{i(t-s)\Delta} (V *
    |\varphi_s|^2) \varphi_s.
  \end{displaymath}
  Differentiating this equation and using integration by parts we find that
  \begin{displaymath}
    \Delta \varphi_t = e^{it \Delta} \Delta \varphi - i \int_0^t ds \,
    e^{i(t-s) \Delta} \big[ (V * |\varphi_s|^2) \Delta \varphi_s + 2(V *
    \nabla |\varphi_s|^2) \nabla \varphi_s + (V * \Delta |\varphi_s|^2)
    \varphi_s \big].
  \end{displaymath}


  The $L_t^\infty L_x^2$-norm of the above expression can be controlled using
  Strichartz estimates for the free Schr\"odinger evolution $e^{it\Delta}$
  (see \cite[Theorem 1.2]{KT}). By these estimates, and Lemma~\ref{l:interp},
  \begin{align}
    & \| \Delta \varphi_{(\cdot)} \|_{L_t^\infty L_x^2} \notag \\
    & \apprle \| \Delta \varphi \|_{L^2} + \| (V * |\varphi_{(\cdot)}|^2)
    \Delta \varphi_{(\cdot)} + 2(V * \nabla |\varphi_{(\cdot)}|^2) \nabla
    \varphi_{(\cdot)} + (V * \Delta |\varphi_{(\cdot)}|^2) \varphi_{(\cdot)}
    \|_{L_t^2 L_x^{6/5}} \notag \\
    & \apprle \| \Delta \varphi \|_{L^2} + \| V \|_{L^1} T^{1/2} \Big[ \sup_{t
    \in [0,T]} \| \varphi_t^2 \|_{L^3} \sup_{t \in [0,T]} \| \Delta \varphi_t
    \|_{L^2} \notag \\
    & \quad + \sup_{t \in [0,T]} \| \nabla |\varphi_t|^2 \|_{L^{3/2}} \sup_{t
    \in [0,T]} \| \nabla \varphi_t \|_{L^6} + \sup_{t \in [0,T]} \| \Delta
    |\varphi_t|^2 \|_{L^{3/2}} \sup_{t \in [0,T]} \| \varphi_t \|_{L^6} \Big].
    \label{Lap}
  \end{align}
  We next estimate each term in this inequality.
  
  
  Applying Sobolev's inequality, the Leibniz rule, and H\"older's inequality,
  we find that
  \begin{displaymath}
    \| \varphi_t^2 \|_{L^3} = \| \varphi_t \|_{L^6}^2 \apprle \| \varphi_t
    \|_{H^1}^2
  \end{displaymath}
  and
  \begin{displaymath}
    \| \nabla |\varphi_t|^2 \|_{L^{3/2}} \le 2 \| \overline{\varphi_t} \nabla
    \varphi_t \|_{L^{3/2}} \le 2 \| \varphi_t \|_{L^6} \| \nabla \varphi_t
    \|_{L^2} \apprle \| \varphi_t \|_{H^1}^2
  \end{displaymath}
  and
  \begin{align*}
    \| \Delta |\varphi_t|^2 \|_{L^{3/2}} & \le 2 \| \overline{\varphi_t}
    \Delta \varphi_t \|_{L^{3/2}} + 2 \| \nabla \overline{\varphi_t} \nabla
    \varphi_t \|_{L^{3/2}} \\
    & \le 2 \| \varphi_t \|_{L^6} \| \Delta \varphi_t \|_{L^2} + 2\| \nabla
    \varphi_t \|_{L^2} \| \nabla \varphi_t \|_{L^6} \apprle \| \varphi_t
    \|_{H^1} \| \varphi_t \|_{H^2}.
  \end{align*}
  Recall that $V \ge 0$. By conservation of mass and energy, and by
  Proposition \ref{p:energy},
  \begin{displaymath}
    \| \varphi_t \|_{H^1}^2 \le \| \varphi_t \|_{L^2}^2 +
    \mathcal{E}(\varphi_t) = 1 + \mathcal{E}(\varphi) \le C,
  \end{displaymath}
  where $C$ is a constant that depends only on $\| V \|_{L^1}$ and $\| \varphi
  \|_{H^1}$. Thus, substituting all this into \eqref{Lap}, and using Sobolev's
  inequality again, we get
  \begin{align*}
    \sup_{t \in [0,T]} \| \Delta \varphi_t \|_{L^2} & \apprle \| \Delta
    \varphi \|_{L^2} + \| V \|_{L^1} T^{1/2} \sup_{t \in [0,T]} \| \varphi_t
    \|_{H^1}^2 \sup_{t \in [0,T]} \| \varphi_t \|_{H^2} \\ & \apprle \|
    \Delta \varphi \|_{L^2} + \| V \|_{L^1} T^{1/2} C \sup_{t \in [0,T]} \|
    \varphi_t \|_{H^2}.
  \end{align*}
  Therefore,
  \begin{align*}
    \sup_{t \in [0,T]} \| \varphi_t \|_{H^2} & \apprle \sup_{t \in [0,T]}
    \big( \| \varphi \|_{L^2} + \| \nabla \varphi_t \|_{L^2} + \| \Delta
    \varphi_t \|_{L^2} \big) \\
    & \apprle 1 + C^{1/2} + \| \Delta \varphi \|_{L^2} + \| V \|_{L^1} T^{1/2}
    C \sup_{t \in [0,T]} \| \varphi_t \|_{H^2}.
  \end{align*}
  Hence, by choosing $T > 0$ sufficiently small, but depending only on $\| V
  \|_{L^1}$ and $\| \varphi \|_{H^1}$, we conclude that
  \begin{displaymath}
    \sup_{t \in [0,T]} \| \varphi_t \|_{H^2} \le C (1 + \| \Delta \varphi
    \|_{L^2}) \le 2C \| \varphi \|_{H^2},
  \end{displaymath}
  where $C$ is a constant that depends only on $\| V \|_{L^1}$ and $\| \varphi
  \|_{H^1}$.
\end{proof}


\subsection{$\varphi_t^{(N)}$ converges to $\varphi_t$ in $L^2$ (using
Gronwall)}


\begin{lem}
  Let $\varphi \in H^2(\R^3)$ with $\| \varphi \|_{L^2} = 1$. Suppose that $f
  \in L^1(\R^3)$ and $V \in C_c^\infty(\R^3)$ with $fV \ge 0$. For $N > 0$,
  consider a solution $\varphi_t^{(N)} \in H^1(\R^3)$ of the nonlinear Hartree
  equation
  \begin{displaymath}
    i \partial_t \varphi_t^{(N)} = - \Delta \varphi_t^{(N)} + (N f_N V_N *
    |\varphi_t^{(N)}|^2) \varphi_t^{(N)}
  \end{displaymath}
  with initial data $\varphi^{(N)}_0 = \varphi$, where $f_N V_N(x) = N^2
  f(Nx)V(Nx)$. Consider also a solution $\varphi_t \in H^1(\R^3)$ of the
  nonlinear Gross-Pitaevskii equation
  \begin{displaymath}
    i \partial_t \varphi_t = - \Delta \varphi_t + 8 \pi a_0 |\varphi_t|^2
    \varphi_t
  \end{displaymath}
  with initial data $\varphi_0 = \varphi$, where $8 \pi a_0 = \int f V$. Then,
  for all $N > 0$ and $t \ge 0$,
  \begin{displaymath}
    \| \varphi_t^{(N)} - \varphi_t \|_{L^2} \le \frac{C}{N} e^{e^{K t}},
  \end{displaymath}
  where $C$ and $K$ are constants that depend only on $\| fV \|_{L^1}$,
  $\text{supp }V$ and $\| \varphi \|_{H^2}$. 
\end{lem}

\begin{proof}
  Observing that $\langle \Delta \varphi_t, \varphi_t^{(N)} \rangle = \langle
  \varphi_t, \Delta \varphi_t^{(N)} \rangle$, one easily finds that
  \begin{equation}
    \label{ddt}
    \begin{aligned}
      \partial_t \| \varphi_t - \varphi_t^{(N)} \|_{L^2}^2 & = -2 \Im \langle
      \varphi_t, (N f_N V_N * |\varphi_t^{(N)}|^2 - 8 \pi a_0 |\varphi_t|^2)
      \varphi_t^{(N)} \rangle \\
      & = 2 \Im \langle \varphi_t, (N f_N V_N * |\varphi_t^{(N)}|^2 - 8 \pi
      a_0 |\varphi_t|^2) (\varphi_t - \varphi_t^{(N)}) \rangle \\
      & = 2 \Im \langle \varphi_t, (N f_N V_N * |\varphi_t|^2 - 8 \pi a_0
      |\varphi_t|^2) (\varphi_t - \varphi_t^{(N)}) \rangle \\
      & \quad + 2 \Im \langle \varphi_t, N f_N V_N * (|\varphi_t^{(N)}|^2 -
      |\varphi_t|^2) (\varphi_t - \varphi_t^{(N)}) \rangle.
    \end{aligned}
  \end{equation}
  We next estimate each term in this expression in order to apply Gronwall's
  inequality.


  By H\"older's inequality, and Sobolev's inequality,
  \begin{align*}
    & |\langle \varphi_t, (N f_N V_N * |\varphi_t|^2 - 8 \pi a_0
    |\varphi_t|^2) (\varphi_t - \varphi_t^{(N)}) \rangle| \\
    & \qquad \le \| \varphi_t \|_{L^6} \| (N f_N V_N * |\varphi_t|^2 - 8 \pi
    a_0 |\varphi_t|^2) (\varphi_t - \varphi_t^{(N)}) \|_{L^{6/5}} \\
    & \qquad \apprle \| \varphi_t \|_{H^1} (\| \varphi_t \|_{L^2} + \|
    \varphi_t^{(N)} \|_{L^2}) \| N f_N V_N * |\varphi_t|^2 - 8 \pi a_0
    |\varphi_t|^2 \|_{L^3}.
  \end{align*}
  By triangle inequality, Sobolev's inequality, and Young's inequality,
  \begin{align*}
    & |\langle \varphi_t, N f_N V_N * (|\varphi_t^{(N)}|^2 - |\varphi_t|^2)
    (\varphi_t - \varphi_t^{(N)}) \rangle| \\
    & \le \int dx dy \, |\varphi_t(x)| |\varphi_t(x) - \varphi_t^{(N)}(x)| N
    f_N V_N(x-y) (|\varphi_t^{(N)}(y)| + |\varphi_t(y)|) |\varphi_t^{(N)}(y) -
    \varphi_t(y)| \\
    & \le \| \varphi_t \|_{L^\infty} ( \| \varphi_t^{(N)} \|_{L^\infty} + \|
    \varphi_t \|_{L^\infty} ) \int dx dy \, |\varphi_t(x) -
    \varphi_t^{(N)}(x)| N f_N V_N(x-y) |\varphi_t^{(N)}(y) - \varphi_t(y)| \\
    & \apprle \| \varphi_t \|_{H^2} ( \| \varphi_t^{(N)} \|_{H^2} + \|
    \varphi_t \|_{H^2} ) \| fV \|_{L^1} \| \varphi_t^{(N)} - \varphi_t
    \|_{L^2}^2.
  \end{align*}
  Recall that $V \ge 0$. By conservation of mass and energy, and by
  Proposition \ref{p:energy},
  \begin{align*}
    \| \varphi_t^{(N)} \|_{H^1}^2 & \le \| \varphi_t^{(N)} \|_{L^2}^2 +
    \mathcal{E}_N(\varphi_t^{(N)}) = 1 + \mathcal{E}_N(\varphi) \apprle \|
    \varphi \|_{H^1}^2 + \| fV \|_{L^1} \| \varphi \|_{H^1}^4, \\
    \| \varphi_t \|_{H^1}^2 & \le \| \varphi_t \|_{L^2}^2 +
    \mathcal{E}_{GP}(\varphi_t) = 1 + \mathcal{E}_{GP}(\varphi) \apprle \|
    \varphi \|_{H^1}^2 + \| fV \|_{L^1} \| \varphi \|_{H^1}^4.
  \end{align*}
  Thus, substituting all this into \eqref{ddt}, and using Corollaries
  \ref{c:reg1} and \ref{c:reg2} to bound $\| \varphi_t^{(N)} \|_{H^2}$ and $\|
  \varphi_t \|_{H^2}$, we obtain
  \begin{equation}
    \label{endproof2}
    \partial_t \| \varphi_t^{(N)} - \varphi_t \|_{L^2}^2 \le C \| \varphi
    \|_{H^2}^2 e^{Kt} \| \varphi_t^{(N)} - \varphi_t \|_{L^2}^2 + C \| N f_N
    V_N * |\varphi_t|^2 - 8 \pi a_0 |\varphi_t|^2 \|_{L^3},
  \end{equation}
  where $C$ and $K$ are constants that depend only on $\| fV \|_{L^1}$ and $\|
  \varphi \|_{H^1}$. We are left to estimating the second term in
  \eqref{endproof2}.


  Write
  \begin{align*}
    N f_N V_N * |\varphi_t|^2(x) - 8 \pi a_0 |\varphi_t|^2(x) & = \int dy
    \big( |\varphi_t(x-y)|^2 - |\varphi_t(x)|^2 \big) N^3 fV(Ny) \\
    & = \int dz \big( |\varphi_t(x-z/N)|^2 - |\varphi_t(x)|^2 \big) fV(z).
  \end{align*}
  Let $R$ be such that $\text{supp }V \subset \{ x \in \R^3 \; | \;\; |x| \le
  R \}$. By Minkowski's, H\"older's, and Sobolev's inequalities,
  \begin{align*}
    \| N f_N V_N * |\varphi_t|^2 - 8 \pi a_0 |\varphi_t|^2 \|_{L^3} & \le \int
    dz \, \| |\varphi_t(\, \cdot \, -z/N)|^2 - |\varphi_t|^2 \|_{L^3} |fV(z)|
    \\
    & \le \| fV \|_{L^1} \sup_{|z| \le R} \| |\varphi_t(\, \cdot \, - z/N)|^2
    - |\varphi_t|^2 \|_{L^3}.
  \end{align*}
  Given $\varepsilon = 1/N$, there exists $\psi_t \in C^\infty(\R^3)$ such
  that $\| \varphi_t - \psi_t \|_{H^2} < 1/N$. Hence, by H\"older's
  inequality, Sobolev's inequality and an $\varepsilon/3$-argument, the mean
  value theorem (with some constant $0 \le c \le 1$), and Sobolev's inequality
  again,
  \begin{align*}
    \| |\varphi_t(\, \cdot \, - z/N)|^2 - |\varphi_t|^2 \|_{L^3} & \le 2 \|
    \varphi_t \|_{L^6} \| |\varphi_t(\, \cdot \, - z/N)| - |\varphi_t|
    \|_{L^6} \\
    & \apprle \| \varphi_t \|_{H^1} \big( 1/N + \| |\psi_t(\, \cdot \, - z/N)|
    - |\psi_t| \|_{L^6} \big) \\
    & \apprle \| \varphi_t \|_{H^1} \big( 1/N + |z|/N \| \nabla |\psi_t(\,
    \cdot \, - c z/N)| \|_{L^6} \big) \\
    & \apprle \| \varphi_t \|_{H^1} \big( 1/N + |z|/N \| \psi_t \|_{H^2}
    \big) \\
    & \apprle \| \varphi_t \|_{H^1} \big( 1/N + |z|/N^2 + |z|/N \| \varphi_t
    \|_{H^2} \big).
  \end{align*}
  Therefore, using again Corollary \ref{c:reg2},
  \begin{align*}
    & \| N f_N V_N * |\varphi_t|^2 - 8 \pi a_0 |\varphi_t|^2 \|_{L^3} \\
    & \apprle \| fV \|_{L^1} \| \varphi_t \|_{H^1} \Big( \frac{1}{N} +
    \frac{R}{N^2} + \frac{R}{N} \| \varphi_t \|_{H^2} \Big) \le
    \frac{C}{N}(1 + \| \varphi_t \|_{H^2}) \le \frac{2C}{N} \| \varphi
    \|_{H^2} e^{Kt},
  \end{align*}
  where $C$ is a constant that depends only on $\| fV \|_{L^1}$, $\text{supp
  }V$ and $\| \varphi \|_{H^1}$. Substituting this into \eqref{endproof2}, we
  find that
  \begin{displaymath}
    \partial_t \| \varphi_t^{(N)} - \varphi_t \|_{L^2}^2 \le C e^{Kt} \|
    \varphi_t^{(N)} - \varphi_t \|_{L^2}^2 + \frac{C}{N} e^{Kt}.
  \end{displaymath}
  By Gronwall's inequality,
  \begin{displaymath}
    \| \varphi_t^{(N)} - \varphi_t \|_{L^2} \le e^{K^{-1} C e^{Kt}} \Big( \|
    \varphi_0^{(N)} - \varphi_0 \|_{L^2} + \frac{C}{KN} e^{Kt} \Big) \le
    \frac{C_2}{N} e^{e^{K_2 t}},
  \end{displaymath}
  for some constants $C_2$ and $K_2$ depending only on $\| fV \|_{L^1}$,
  $\text{supp }V$ and $\| \varphi \|_{H^2}$.
\end{proof}


\subsection{$\varphi_t^{(N)}$ converges to $\varphi_t$ in $L^2$ (using
Strichartz) (to be deleted)}


\begin{prp}
  \label{p:httogp}
  For $N > 0$, let $\varphi^{(N)}, \varphi \in H^2(\R^3)$ with $\|
  \varphi^{(N)} \|_{L^2} = 1$ and $\| \varphi \|_{L^2} = 1$. Suppose that $f
  \in L^1(\R^3)$ and $V \in C_c^\infty(\R^3)$ with $fV \ge 0$. Consider a
  solution $\varphi_t^{(N)} \in H^1(\R^3)$ of the nonlinear Hartree equation
  \begin{displaymath}
    i \partial_t \varphi_t^{(N)} = - \Delta \varphi_t^{(N)} + (N f_N V_N *
    |\varphi_t^{(N)}|^2) \varphi_t^{(N)}
  \end{displaymath}
  with initial data $\varphi^{(N)}_0 = \varphi^{(N)}$, where $f_N V_N(x) = N^2
  f(Nx)V(Nx)$. Consider also a solution $\varphi_t \in H^1(\R^3)$ of the
  nonlinear Gross-Pitaevskii equation
  \begin{displaymath}
    i \partial_t \varphi_t = - \Delta \varphi_t + 8 \pi a_0 |\varphi_t|^2
    \varphi_t
  \end{displaymath}
  with initial data $\varphi_0 = \varphi$, where $8 \pi a_0 = \int f V$. Then,
  there exist a real number $T > 0$ such that
  \begin{displaymath}
    \| \varphi_t - \varphi_t^{(N)} \|_{L^2} \le \| \varphi - \varphi^{(N)}
    \|_{L^2} + \frac{C}{N}\| \varphi^{(N)} \|_{H^2}
  \end{displaymath}
  for all $t \in [0,T]$, where $C$ is a constant that depends only on $\| fV
  \|_{L^1}$, $\text{supp }V$, $\| \varphi^{(N)} \|_{H^1}$ and $\| \varphi
  \|_{H^1}$, and $T$ depends only on $\| fV \|_{L^1}$, $\| \varphi^{(N)}
  \|_{H^1}$ and $\| \varphi \|_{H^1}$.
\end{prp}


\begin{proof}
  Write
  \begin{align*}
    \varphi_t - \varphi_t^{(N)} & = e^{it\Delta}(\varphi - \varphi^{(N)}) + i
    \int_0^t ds \, e^{i(t-s) \Delta} [ (N f_N V_N * |\varphi_s^{(N)}|^2)
    \varphi_s^{(N)} - 8 \pi a_0 |\varphi_s|^2 \varphi_s ] \\
    & = e^{it\Delta}(\varphi - \varphi^{(N)}) + i \int_0^t ds \, e^{i(t-s)
    \Delta} [ (N f_N V_N * |\varphi_s|^2 - 8 \pi a_0 |\varphi_s|^2) \varphi_s
    \\
    & \quad + (N f_N V_N * |\varphi_s^{(N)}|^2)(\varphi_s^{(N)} - \varphi_s) +
    (N f_N V_N * (|\varphi_s^{(N)}|^2 - |\varphi|^2)) \varphi_s].
  \end{align*}
  As in the proof of Proposition \ref{p:reg1}, by Strichartz estimates,
  \begin{equation}
    \begin{aligned}
      & \| \varphi_{(\cdot)} - \varphi_{(\cdot)}^{(N)} \|_{L_t^\infty L_x^2}
      \apprle \| \varphi - \varphi^{(N)} \|_{L^2} + \| (N f_N V_N *
      |\varphi_{(\cdot)}|^2 - 8 \pi a_0 |\varphi_{(\cdot)}|^2)
      \varphi_{(\cdot)} \\
      & \qquad \qquad + (N f_N V_N *
      |\varphi_{(\cdot)}^{(N)}|^2)(\varphi_{(\cdot)}^{(N)} -
      \varphi_{(\cdot)})+ (N f_N V_N * (|\varphi_{(\cdot)}^{(N)}|^2 -
      |\varphi_{(\cdot)}|^2) \varphi_{(\cdot)} \|_{L_t^2 L_x^{6/5}}.
    \end{aligned}
    \label{diff}
  \end{equation}
  We next estimate each term in this inequality.


  By H\"older's inequality and conservation of mass,
  \begin{displaymath}
    \| (N f_N V_N * |\varphi_{(\cdot)}|^2 - 8 \pi a_0 |\varphi_{(\cdot)}|^2)
    \varphi_{(\cdot)} \|_{L_t^2 L_x^{6/5}} \le T^{1/2} \sup_{t \in [0,T]} \| N
    f_N V_N * |\varphi_t|^2 - 8 \pi a_0 |\varphi_t|^2 \|_{L^3}.
  \end{displaymath}
  By Lemma \ref{l:interp} and Sobolev's inequality,
  \begin{align*}
    & \| (N f_N V_N * |\varphi_{(\cdot)}^{(N)}|^2)(\varphi_{(\cdot)}^{(N)} -
    \varphi_{(\cdot)}) \|_{L_t^2 L_x^{6/5}} \\
    & \qquad \le \| fV \|_{L^1} T^{1/2} \sup_{t \in [0,T]} \|
    |\varphi_t^{(N)}|^2 \|_{L^3} \sup_{t \in [0,T]} \| \varphi_t^{(N)} -
    \varphi_t \|_{L^2} \\
    & \qquad \apprle \| fV \|_{L^1} T^{1/2} \sup_{t \in [0,T]} \|
    \varphi_t^{(N)} \|_{H^1}^2 \sup_{t \in [0,T]} \| \varphi_t^{(N)} -
    \varphi_t \|_{L^2},
  \end{align*}
  and also by H\"older's inequality and triangle inequality,
  \begin{align*}
    & \| (N f_N V_N * (|\varphi_{(\cdot)}^{(N)}|^2 - |\varphi_{(\cdot)}|^2))
    \varphi_{(\cdot)}) \|_{L_t^2 L_x^{6/5}} \\
    & \qquad \le \| fV \|_{L^1} T^{1/2} \sup_{t \in [0,T]} \|
    |\varphi_t^{(N)}|^2 - |\varphi_t|^2 \|_{L^{3/2}} \sup_{t \in [0,T]} \|
    \varphi_t \|_{L^6} \\
    & \qquad \le \| fV \|_{L^1} T^{1/2} \sup_{t \in [0,T]} \|
    \varphi_t^{(N)} - \varphi_t \|_{L^2} \sup_{t \in [0,T]} (\|
    \varphi_t^{(N)} \|_{H^1} + \| \varphi_t \|_{H^1}) \| \varphi_t \|_{H^1}.
  \end{align*}
  Recall that $V \ge 0$. By conservation of mass and energy, and by
  Proposition \ref{p:energy},
  \begin{align*}
    \| \varphi_t^{(N)} \|_{H^1}^2 & \le \| \varphi_t^{(N)} \|_{L^2}^2 +
    \mathcal{E}_N(\varphi_t^{(N)}) = 1 + \mathcal{E}_N(\varphi^{(N)}) \le C_1,
    \\
    \| \varphi_t \|_{H^1}^2 & \le \| \varphi_t \|_{L^2}^2 +
    \mathcal{E}_{GP}(\varphi_t) = 1 + \mathcal{E}_{GP}(\varphi) \le C_2,
  \end{align*}
  where $C_1$ and $C_2$ are constants that depend only on $\| fV \|_{L^1}$,
  and $\| \varphi^{(N)} \|_{H^1}$ and $\| \varphi \|_{H^1}$, respectively.
  Thus, substituting all this into \eqref{diff}, and choosing $T > 0$
  sufficiently small, but depending only on $\| fV \|_{L^1}$, $\|
  \varphi^{(N)} \|_{H^1}$ and $\| \varphi \|_{H^1}$, we obtain
  \begin{equation}
    \sup_{t \in [0,T]} \| \varphi_t - \varphi_t^{(N)} \|_{L^2} \le \| \varphi
    - \varphi^{(N)} \|_{L^2} + C \sup_{t \in [0,T]} \| N f_N V_N *
    |\varphi_t|^2 - 8 \pi a_0 |\varphi_t|^2 \|_{L^3},
    \label{endproof}
  \end{equation}
  where $C$ is a constant that depends only on $\| fV \|_{L^1}$, $\|
  \varphi^{(N)} \|_{H^1}$ and $\| \varphi \|_{H^1}$. We are left to estimating
  the right hand side of \eqref{endproof}.


  Write
  \begin{align*}
    N f_N V_N * |\varphi_t|^2(x) - 8 \pi a_0 |\varphi_t|^2(x) & = \int dy
    \big( |\varphi_t(x-y)|^2 - |\varphi_t(x)|^2 \big) N^3 fV(Ny) \\
    & = \int dz \big( |\varphi_t(x-z/N)|^2 - |\varphi_t(x)|^2 \big) fV(z).
  \end{align*}
  Let $R$ be such that $\text{supp }V \subset \{ x \in \R^3 \; | \;\; |x| \le
  R \}$. By Minkowski's, H\"older's, and Sobolev's inequalities,
  \begin{align*}
    \| N f_N V_N * |\varphi_t|^2 - 8 \pi a_0 |\varphi_t|^2 \|_{L^3} & \le \int
    dz \, \| |\varphi_t(\, \cdot \, -z/N)|^2 - |\varphi_t|^2 \|_{L^3} |fV(z)|
    \\
    & \le \| fV \|_{L^1} \sup_{|z| \le R} \| |\varphi_t(\, \cdot \, - z/N)|^2
    - |\varphi_t|^2 \|_{L^3}.
  \end{align*}
  Given $\varepsilon = 1/N$, there exists $\psi_t \in C^\infty(\R^3)$ such
  that $\| \varphi_t - \psi_t \|_{H^2} < 1/N$. Hence, by H\"older's
  inequality, Sobolev's inequality and an $\varepsilon/3$-argument, the mean
  value theorem (with some constant $0 \le c \le 1$), and Sobolev's inequality
  again,
  \begin{align*}
    \| |\varphi_t(\, \cdot \, - z/N)|^2 - |\varphi_t|^2 \|_{L^3} & \le 2 \|
    \varphi_t \|_{L^6} \| |\varphi_t(\, \cdot \, - z/N)| - |\varphi_t|
    \|_{L^6} \\
    & \apprle \| \varphi_t \|_{H^1} \big( 1/N + \| |\psi_t(\, \cdot \, - z/N)|
    - |\psi_t| \|_{L^6} \big) \\
    & \apprle \| \varphi_t \|_{H^1} \big( 1/N + |z|/N \| \nabla |\psi_t(\,
    \cdot \, - c z/N)| \|_{L^6} \big) \\
    & \apprle \| \varphi_t \|_{H^1} \big( 1/N + |z|/N \| \psi_t \|_{H^2} \big)
    \\
    & \apprle \| \varphi_t \|_{H^1} \big( 1/N + |z|/N^2 + |z|/N \| \varphi_t
    \|_{H^2} \big).
  \end{align*}
  Therefore, substituting this into the above inequality, and applying
  Proposition \ref{p:reg1} (possibly after choosing a smaller $T$), we obtain
  \begin{align*}
    & \sup_{t \in [0,T]} \| N f_N V_N * |\varphi_t|^2 - 8 \pi a_0
    |\varphi_t|^2 \|_{L^3} \\
    & \apprle \| fV \|_{L^1} \| \varphi_t \|_{H^1} \Big( \frac{1}{N} +
    \frac{R}{N^2} + \frac{R}{N} \sup_{t \in [0,T]} \| \varphi_t \|_{H^2} \Big)
    \le \frac{C}{N}(1 + \| \varphi \|_{H^2}) \le \frac{2C}{N}\| \varphi
    \|_{H^2},
  \end{align*}
  where $C$ is a constant that depends only on $\| fV \|_{L^1}$, $\text{supp
  }V$, $\| \varphi^{(N)} \|_{H^1}$ and $\| \varphi \|_{H^1}$. In view of
  \eqref{endproof}, this completes the proof.
\end{proof}


\bibliographystyle{plain}
\bibliography{gross-pitaevskii}


\end{document}
