\documentclass[a4paper,11pt]{article}
\usepackage[a4paper,hmargin=2.8cm,vmargin=2.8cm]{geometry}
\usepackage{amsmath,amssymb,amsthm,dsfont,mathtools,wasysym,verbatim}
%\usepackage[bookmarksnumbered=true]{hyperref}
\usepackage[color,notref,notcite]{showkeys}
%\linespread{1.15}

% Environments
\newtheorem{prop}{Proposition}
\newtheorem{lem}{Lemma}

% Nicer marginpar
\let\oldmarginpar\marginpar
\renewcommand\marginpar[1]{\-\oldmarginpar[\raggedleft\footnotesize #1]%
  {\raggedright\footnotesize #1}}
% More redefined commands
\renewcommand{\Im}{\,{\rm Im}\,}
\renewcommand{\Re}{\,{\rm Re}\,}

% Aliases
\newcommand{\R}{\mathds{R}}
\newcommand{\N}{\mathcal{N}}


\begin{document}


\tableofcontents


\section{Remarks}


\begin{itemize}
  \item Section ``Operator T'' almost finished. Need to finish proof of Prop.
    \ref{p:psr}(v)(vi). Perhaps we deal with that in a coffee break\dots
  \item In the Section ``Propositions'', just did $T^* \N T$ and $T^* \N T$,
    but still have to make the notation $a(f_x)$ compatible with the main {\tt
    g-p.tex}
\end{itemize}


\section{Zero-energy scattering equation}


Consider $V \in C_c^\infty(\R^3)$, and let $f$ be the solution of the
zero-energy scattering equation
\begin{displaymath}
  \left( -\Delta + \frac{1}{2} V \right) f = 0
\end{displaymath}
with normalization $\lim_{|x|\to\infty} f(x) = 1$. We will write
\begin{displaymath}
  f = 1 - w
\end{displaymath}
with $\lim_{|x|\to\infty} w(x) = 0$. The scattering length $a_0$ of $V$ is
defined as
\begin{displaymath}
  a_0 = \lim_{|x| \to \infty} w(x)|x|.
\end{displaymath}
Since $V$ has compact support, we have
\begin{displaymath}
  f(x) = 1 - \frac{a_0}{|x|} \qquad \text{for } |x| \ge R,
\end{displaymath}
where $R$ is such that $\text{supp }V \subset \{ x \in \R^3 \; | \;\; |x| \le
R \}$. From the zero-energy scattering equation, we also have the identity
\begin{displaymath}
  \int dx \, V(x) f(x) = 8 \pi a_0.
\end{displaymath}
By scaling, the scattering length of the potential $V_N(x) = N^2 V(Nx)$ is
$a_N = a_0/N$, and the zero-energy scattering equation for the potential
$V_N$ is
\begin{displaymath}
  \left( -\Delta + \frac{1}{2} V_N \right) f_N = 0,
\end{displaymath}
where $f_N(x) = 1 - w_N(x)$ with $w_N(x) = w(Nx)$. Note that $w_N(x) =
a_N/|x|$ for $|x| \ge R/N$.


\begin{lem}[\cite{ESY}]
  \label{l:w}
  Let $V \in C_c^\infty(\R^3)$ with $V \ge 0$, and suppose that $V$ is
  spherical symmetric with scattering length $a_0$. Let
  \begin{displaymath}
    \rho = \sup_{r \ge 0} r^2 V(r) + \int_0^\infty dr \, r V(r),
  \end{displaymath}
  and consider the solution $f_N = 1-w_N$ of the zero-energy scattering
  equation with potential $V_N$. Then, the following hold with constants
  uniform in $N$.

  (i) There exists a constant $C_0 > 0$, which depends on the unscaled
  potential $V$, such that
  \begin{displaymath}
    C_0 \le 1 - w_N(x) \le 1 \qquad \text{for all } x \in \R^3.
  \end{displaymath}

  (ii) Let $R$ be such that $\text{supp }V \subset \{ x \in \R^3 \; | \;\; |x|
  \le R \}$. Then,
  \begin{displaymath}
    w_N(x) = \frac{a_0}{N|x|} \qquad \text{for all } x \in \R^3 \text{ with }
    |x| > R/N.
  \end{displaymath}

  \marginpar{part (iii) may be cleaned up since we are not using all of them}
  (iii) There exist constants $C_1$ and $C_2$, depending only on $V$, such
  that
  \begin{displaymath}
    |\nabla w_N(x)| \le C_1 N \qquad \text{and} \qquad | \Delta w_N(x)| \le
    C_2 N^2 \qquad \text{for all } x \in \R^3.
  \end{displaymath}
  Moreover, there exists a universal constant $C$ such that
  \begin{displaymath}
    |\nabla w_N(x)| \le \frac{C a_0}{N |x|^2}, \qquad |\nabla w_N(x)| \le
    \frac{C \rho}{|x|}, \qquad |\Delta w_N(x)| \le \frac{C \rho}{N^2 |x|^2}
  \end{displaymath}
  for all $x \in \R^3$.
\end{lem}


\section{Operator $T$}


Let
\begin{displaymath}
  T = \exp \left( \frac{1}{2} \int dxdy \, k(x,y) a_x^* a_y^* - \frac{1}{2}
  \int dxdy \, \overline{k(x,y)} a_x a_y \right).
\end{displaymath}
Then,
\begin{align*}
  T^* a_x T & = a(c(x,\,\cdot\,)) + a^*(s(x,\,\cdot\,)) \\
  & = a_x + a(p(x,\,\cdot\,)) + a^*(s(x,\,\cdot\,))
\end{align*}
with
\begin{align*}
  c & = \delta + p, \\
  p & = \sum_{n=1}^\infty \frac{1}{(2n)!} \big( k \overline{k} \big)^n, \\
  s & = k + r, \\
  r & = \sum_{n=1}^\infty \frac{1}{(2n+1)!} \big( k \overline{k} \big)^n k,
\end{align*}
where $\delta$ is the Dirac delta, and the product of functions in the power
series is the convolution of integral kernels, that is, $fg(x,y) = \int dz \,
f(x,z) g(z,y)$.


\begin{prop}
  \label{p:psr}
  Let $\varphi \in H^2(\R^3)$, and let $f_N=1-w_N$ be the solution of the
  zero-energy scattering equation as in Lemma \ref{l:w}. Let
  \begin{displaymath}
    k(x,y) = - N w_N(x-y) \varphi(x) \varphi(y).
  \end{displaymath}
  Then, for $N \ge 1$ and $n \ge 1$,
  \begin{align}
    \| k \|_{L^2} & \apprle \| \varphi \|_{H^1}^2, \label{k} \tag{i} \\
    \| \nabla_1 k \|_{L^2} & \apprle \| \varphi \|_{H^2}^2 \sqrt{N},
    \label{gradk} \tag{ii} \\
    \| (k \overline{k})^n \|_{L^2} & \le C^n \| \varphi \|_{H^1}^{4n},
    \tag{iii} \\
    \| (\nabla_1 k \overline{k})^n \|_{L^2} & \le C^n \| \varphi
    \|_{H^1}^{4n}, \tag{iv}
  \end{align}
  where $C$ is a constant (that depends only on $a_0$ and $\rho$ from Lemma
  \ref{l:w}). Consequently,
  \begin{equation}
    \begin{alignedat}{2}
      \| p \|_{L^2} & \le C_1, \qquad & \| \nabla_1 p \|_{L^2} & \le C_1, \\
      \| r \|_{L^2} & \le C_1, \qquad & \| \nabla_1 r \|_{L^2} & \le C_1,
    \end{alignedat}
    \tag{v}
  \end{equation}
  and
  \begin{equation}
    \| s \|_{L^2} \le C_1, \qquad \| \nabla_1 s \|_{L^2} \apprle \| \varphi
    \|_{H^2}^2 \sqrt{N}, \tag{vi}
  \end{equation}
  with $C_1 = e^{C \| \varphi \|_{H^1}^6}$.
\end{prop}


\begin{proof}
  (i) By parts (i) and (ii) of Lemma \ref{l:w}, it follows that $w_N(x) \le
  \max\{a_0,R\} /(N|x|)$ for all $x \in \R^3$. Hence, using the operator
  inequality $-4 \Delta \ge |x|^{-2}$ on $L^2(\R^3)$, we find that
  \begin{equation}
    \begin{aligned}
      \int dy \, N^2 w_N(x-\,\cdot\,)^2 |\varphi(y)|^2 & \apprle \int dy \,
      |x-y|^{-2} |\varphi(y)|^2 \\
      & = \langle \varphi(\,\cdot\,+x), |\,\cdot\,|^{-2} \varphi(\,\cdot\,+x)
      \rangle_{L^2} \\
      & \le -4 \langle \varphi(\,\cdot\,+x), \Delta \varphi(\,\cdot\,+x)
      \rangle_{L^2} = 4 \| \nabla \varphi \|_{L^2}^2.
    \end{aligned}
    \label{w}
  \end{equation}
  Thus,
  \begin{displaymath}
    \| k \|_{L^2}^2 = \int dx \, |\varphi(x)|^2 \int dy \, N^2
    w_N(x-\,\cdot\,)^2 |\varphi(y)|^2 \apprle \| \varphi \|_{L^2}^2 \| \nabla
    \varphi \|_{L^2}^2 \apprle \| \varphi \|_{H^1}^4.
  \end{displaymath}


  (ii) By the Leibniz rule and triangle inequality, there are two terms to be
  estimated:
  \begin{displaymath}
    \| \nabla_1 k \|_{L^2} \le \| N w_N(x-y) \nabla \varphi(x) \varphi(y)
    \|_{L^2} + \| N \nabla w_N(x-y) \varphi(x) \varphi(y) \|_{L^2}.
  \end{displaymath}
  For the first term, similarly as in part (i), we have
  \begin{displaymath}
    \| N w_N(x-y) \nabla \varphi(x) \varphi(y) \|_{L^2} \apprle \| \nabla
    \varphi \|_{L^2}^2 \apprle \| \varphi \|_{H^1}^2.
  \end{displaymath}
  To bound the second term, we first compute
  \begin{align*}
    & \int dx \, |\varphi(x)|^2 \int dy \, N^2 \nabla w_N(x-\,\cdot\,)^2
    |\varphi(y)|^2 \\
    & = \int dx \, |\varphi(x)|^2 \int dy \, N^2 w_N(x-y) ( \nabla w_N(x-y)
    \nabla_y |\varphi(y)|^2 - \Delta w_N(x-y) |\varphi(y)|^2 ).
  \end{align*}
  By parts (i) and (iii) of Lemma \ref{l:w}, and then similarly as in the
  proof of \eqref{w},
  \begin{align*}
    & \int dx \, |\varphi(x)|^2 \int dy \, N^2 w_N(x-y) \nabla w_N(x-y)
    \nabla_y |\varphi(y)|^2 \\
    & \apprle N \int dx \, |\varphi(x)|^2 \Re \int |\nabla \varphi(y)| \,
    |x-y|^{-2} \varphi(y) \\
    & \apprle N \| \varphi \|_{L^2}^2 ( \| \nabla \varphi \|_{L^2}^2 + \|
    \Delta \varphi \|_{L^2}^2) \apprle N \| \varphi \|_{H^2}^4.
  \end{align*}
  Now, using the zero-energy scattering equation, and by `Young's inequality,
  Lemma \ref{l:w}(i), and Sobolev's inequality, \begin{align*}
    & - \int dx \, |\varphi(x)|^2 \int dy \, N^2 w_N(x-y) \Delta w_N(x-y)
    |\varphi(y)|^2 \\
    & = - N \int dx \, |\varphi(x)|^2 \int dy \, \big( w_N N (1-w_N) V_N
    \big)(x-y) |\varphi(y)|^2 \\
    & \le N \| \varphi^2 \|_{L^2}^2 \| w_N N (1-w_N) V_N \|_{L^1} \le N \|
    \varphi \|_{L^4}^4 \| w_N \|_{L^\infty} \| (1-w)V \|_{L^1} \\
    & \apprle N \| \varphi \|_{H^1}^4.
  \end{align*}
  Therefore, combining the last three expressions,
  \begin{displaymath}
    \| N \nabla w_N(x-y) \varphi(x) \varphi(y) \|_{L^2}^2 \apprle N \| \varphi
    \|_{H^2}^4 + N \| \varphi \|_{H^1}^4 \apprle N \| \varphi \|_{H^2}^4,
  \end{displaymath}
  and consequently
  \begin{displaymath}
    \| \nabla_1 k \|_{L^2} \apprle \| \varphi \|_{H^1}^2 + \sqrt{N} \| \varphi
    \|_{H^2}^2 \apprle \sqrt{N} \| \varphi \|_{H^2}^2.
  \end{displaymath}


  (iii) The proof is by induction on $n \ge 1$. We first prove the estimate
  for $n=1$. We begin observing that, by H\"older's inequality and \eqref{w},
  \begin{equation}
    \begin{aligned}
      & \int dz \, N^2 w_N(x-z) w_N(z-y) |\varphi(z)|^2 \\
      & \le \left( \int dz \, N^2 w_N(x-z)^2 |\varphi(z)|^2 \right)^{1/2} \left(
      \int dz \, N^2 w_N(z-y)^2 |\varphi(z)|^2 \right)^{1/2} \apprle \| \nabla
      \varphi \|_{L^2}^2.
    \end{aligned}
    \label{ww}
  \end{equation}
  Hence,
  \begin{align*}
    \| k \overline{k} \|_{L^2}^2 & = \int dx \, |\varphi(x)|^2 \int dy \,
    |\varphi(y)|^2 \left| \int dz \, N^2 w_N(x-z) w_N(z-y) |\varphi(z)|^2
    \right|^2 \le C^2 \| \varphi \|_{H^1}^8.
  \end{align*}
  Now, suppose that $\| (k \overline{k})^n \|_{L^2} \le C^{2n} \| \varphi
  \|_{H^1}^{4n}$. Then, interchanging the integration order, and applying
  H\"older's inequality twice,
  \begin{align*}
    & \| (k \overline{k})^{n+1} \|_{L^2}^2 = \int dz_1 dz_2 \int dx \,
    k\overline{k}(x,z_1) \overline{ k \overline{k}}(x,z_2) \int dy \,
    (k\overline{k})^n(z_1,y) \overline{ (k \overline{k})^n}(z_2,y) \\
    & \le \int dz_1 \, \left( \int dx \, | k\overline{k}(x,z_1)|^2
    \right)^{1/2} \left( \int dy \, | (k\overline{k})^n(z_1,y)|^2
    \right)^{1/2} \\
    & \qquad \times \int dz_2 \left( \int dx \, | k\overline{k}(x,z_2)|^2
    \right)^{1/2} \left( \int dy \, | (k\overline{k})^n(z_2,y)|^2
    \right)^{1/2} \\
    & \le \| k \overline{k} \|_{L^2}^2 \| (k \overline{k})^n \|_{L_2}^2 \le
    C^{2(n+1)} \| \varphi \|_{H^1}^{8(n+1)}.
  \end{align*}
  This completes the induction step and proves the desired estimate.


  (iv) We first prove the estimate for $n=1$. By the Leibniz rule and triangle
  inequality,
  \begin{displaymath}
    \| \nabla_1 k \overline{k} \|_{L^2} \le \| I \|_{L^2} + \| J \|_{L^2},
  \end{displaymath}
  where, using \eqref{ww},
  \begin{align*}
    \| I \|_{L^2}^2 & = \int dx \, |\nabla \varphi(x)|^2 \int dy \,
    |\varphi(y)|^2 \left| \int dz \, N^2 w_N(x-z) w_N(z-y) |\varphi(z)|^2
    \right|^2 \apprle \| \varphi \|_{H^1}^8,
  \end{align*}
  and by Lemma \ref{l:w}, and estimates \eqref{w} and \eqref{ww},
  \begin{align*}
    \| J \|_{L^2}^2 & = \int dx \, |\varphi(x)|^2 \int dy \, |\varphi(y)|^2
    \left| \int dz \, N^2 \nabla w_N(x-z) w_N(z-y) |\varphi(z)|^2 \right|^2 \\
    & \apprle \int dx \, |\varphi(x)|^2 \int dz_1
    \frac{|\varphi(z_1)|^2}{|x-z_1|^2} \int dz_2
    \frac{|\varphi(z_2)|^2}{|x-z_2|^2} \int dy \,
    \frac{|\varphi(y)|^2}{|z_1-y| \, |z_2-y|} \\
    & \apprle \| \varphi \|_{L^2}^2 \| \nabla \varphi \|_{L^2}^6 \apprle \|
    \varphi \|_{H^1}^8.
  \end{align*}
  Thus,
  \begin{displaymath}
    \| \nabla_1 k \overline{k} \|_{L^2} \le \| I \|_{L^2} + \| J \|_{L^2}
    \le C^2 \| \varphi \|_{H^1}^4.
  \end{displaymath}
  Consequently, by H\"older's inequality, similarly as in the proof of part
  (iii), and then applying part (iii),
  \begin{displaymath}
    \| \nabla_1 (k \overline{k})^n \|_{L^2} = \| (\nabla_1 k \overline{k}) (k
    \overline{k})^{n-1} \|_{L^2} \le \| \nabla_1 k \overline{k} \|_{L^2} \| (k
    \overline{k})^{n-1} \|_{L_2} \le C^{2n} \| \varphi \|_{H_1}^{8n}.
  \end{displaymath}
  This is the desired inequality.


  \marginpar{Unfinished}
  (v) By H\"older's inequality (as in the proof of (iii)) and then by (i) and
  (iii) we have that
  \begin{displaymath}
    \sum_{n=1}^\infty \left \| \frac{(k \overline{k})^n k}{(1+2n)!} \right
    \|_{L^2}^2 \le \sum_{n=1}^\infty \frac{C^n \| \varphi
    \|_{H^1}^{4n+2}}{(1+2n)!} \le e^{C \| \varphi \|_{H^1}^6}.
  \end{displaymath}
  Hence, by the dominated convergence theorem (or something easier ?) we find
  that $r \in L^2$ and $\| r \|_{L^2} \le e^{C \| \varphi \|_{H^1}^6}$.
  Similarly for $p$ \dots. Furthermore, since \dots \dots, we can interchange
  differentiation with summation to obtain $\nabla_1 p \in L^2$, \dots \dots,
  with $\| \nabla_1 p \|_{L^2} \le \dots$ \dots
\end{proof}


\section{Propositions}


Let $f(x,y)$ be a function of two variables. To simplify the notation we write
\begin{displaymath}
  f_x(y) = f(x,y).
\end{displaymath}
Consider two operators $A$ and $B$ on the Fock space $\mathcal{F}$. The
notation
\begin{displaymath}
  A \le B
\end{displaymath}
means that $\langle \psi, A \psi \rangle \le \langle \psi, B \psi \rangle$ for
all $\psi \in \mathcal{F}$.


\begin{prop}
  Let $p, s \in L^2(\R^3 \times \R^3)$. Then,
  \label{p:TNT}
  \begin{align}
    T^* \N T & \apprle C (\N+1), \label{TNT} \tag{i} \\
    T^* \N^2 T & \apprle C^2 (\N+1)^2, \label{TN2T} \tag{ii}
  \end{align}
  where $C = 1 + \| p \|_{L^2}^2 + \| s \|_{L^2}^2$.
\end{prop}


\begin{proof}
  (i) Write
  \begin{displaymath}
    \langle \psi, T^* \N T \psi \rangle = \int dx \, \langle T^* a_x T \psi,
    T^* a_x T \psi \rangle = \int dx \, \| (a_x + a(p_x) + a^*(s_x)) \psi
    \|^2.
  \end{displaymath}
  Then, by Cauchy-Schwarz inequality, and Lemma R-S 2.1,
  \begin{align*}
    \langle \psi, T^* \N T \psi \rangle & \le 5 \int dx \, \| a_x \psi \|^2 +
    5 \int dx \, \| a(p_x) \psi \|^2 + 5 \int dx \, \| a^*(s_x) \psi \|^2 \\
    & \le 5 \langle \psi, \N \psi \rangle + 5 \int dx \, \| p_x \|_{L^2}^2
    \langle \psi, \N \psi \rangle + 5 \int dx \, \| s_x \|_{L^2}^2 \langle
    \psi, (\N+1) \psi \rangle \\
    & \le 5 (1 + \| p \|_{L^2}^2 + \| s \|_{L^2}^2) \langle \psi, (\N+1) \psi
    \rangle.
  \end{align*}


  (ii) Write
  \begin{align*}
    & \langle \psi, T^* \N^2 T \psi \rangle \\
    & = \int dxdy \, \langle \psi, T^* a_x^* a_x a_y^* a_y T \psi \rangle \\
    & = \int dx \, \langle \psi, T^* a_x^* \N a_x T \psi \rangle + \langle
    \psi, T^* \N T \psi \rangle \\
    & = \int dx \, \langle \psi, (a_x^* + a^*(p_x) + a(s_x)) T^* \N T (a_x +
    a(p_x) + a^*(s_x)) \psi \rangle + \langle \psi, T^* \N T \psi \rangle.
  \end{align*}
  Then, by part (i) and Cauchy-Schwarz inequality, using that $a_x \N^{1/2} =
  (\N+1)^{1/2} a_x$, and by Lemma R-S 2.1,
  \begin{align*}
    & \langle \psi, T^* \N^2 T \psi \rangle \\
    & \le 5C \int dx \, \| (\N+1)^{1/2} (a_x + a(p_x) + a^*(s_x)) \psi \|^2 + C
    \langle \psi, (\N+1) \psi \rangle \\
    & \le 25C \int dx \, ( \langle a_x^* a_x \psi, \N \psi \rangle + \| a(p_x)
    \N^{1/2} \psi \|^2 + \| a^*(s_x) (\N+2)^{1/2} \psi \|^2 ) + C \langle
    \psi, (\N+1) \psi \rangle \\
    & \le (5C)^2 \langle \psi, (\N+2)^2 \psi \rangle,
  \end{align*}
  where $C = 1 + \| p \|_{L^2}^2 + \| s \|_{L^2}^2$.
\end{proof}


\begin{thebibliography}{99}
  \bibitem{ESY} L. Erd\"os, B. Schlein and H.-T. Yau, \emph{Derivation of the
    Gross-Pitaevskii equation for the dynamics of Bose-Einstein condensate},
    Annals of Math. {\bf 172} (2010) 291-370.
\end{thebibliography}


\end{document}
